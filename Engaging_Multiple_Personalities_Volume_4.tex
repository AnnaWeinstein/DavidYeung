\documentclass[]{book}
\usepackage{lmodern}
\usepackage{amssymb,amsmath}
\usepackage{ifxetex,ifluatex}
\usepackage{fixltx2e} % provides \textsubscript
\ifnum 0\ifxetex 1\fi\ifluatex 1\fi=0 % if pdftex
  \usepackage[T1]{fontenc}
  \usepackage[utf8]{inputenc}
\else % if luatex or xelatex
  \ifxetex
    \usepackage{mathspec}
  \else
    \usepackage{fontspec}
  \fi
  \defaultfontfeatures{Ligatures=TeX,Scale=MatchLowercase}
\fi
% use upquote if available, for straight quotes in verbatim environments
\IfFileExists{upquote.sty}{\usepackage{upquote}}{}
% use microtype if available
\IfFileExists{microtype.sty}{%
\usepackage{microtype}
\UseMicrotypeSet[protrusion]{basicmath} % disable protrusion for tt fonts
}{}
\usepackage[margin=1in]{geometry}
\usepackage{hyperref}
\hypersetup{unicode=true,
            pdftitle={Engaging Multiple Personalities},
            pdfauthor={David Yeung},
            pdfborder={0 0 0},
            breaklinks=true}
\urlstyle{same}  % don't use monospace font for urls
\usepackage{natbib}
\bibliographystyle{apalike}
\usepackage{longtable,booktabs}
\usepackage{graphicx,grffile}
\makeatletter
\def\maxwidth{\ifdim\Gin@nat@width>\linewidth\linewidth\else\Gin@nat@width\fi}
\def\maxheight{\ifdim\Gin@nat@height>\textheight\textheight\else\Gin@nat@height\fi}
\makeatother
% Scale images if necessary, so that they will not overflow the page
% margins by default, and it is still possible to overwrite the defaults
% using explicit options in \includegraphics[width, height, ...]{}
\setkeys{Gin}{width=\maxwidth,height=\maxheight,keepaspectratio}
\IfFileExists{parskip.sty}{%
\usepackage{parskip}
}{% else
\setlength{\parindent}{0pt}
\setlength{\parskip}{6pt plus 2pt minus 1pt}
}
\setlength{\emergencystretch}{3em}  % prevent overfull lines
\providecommand{\tightlist}{%
  \setlength{\itemsep}{0pt}\setlength{\parskip}{0pt}}
\setcounter{secnumdepth}{5}
% Redefines (sub)paragraphs to behave more like sections
\ifx\paragraph\undefined\else
\let\oldparagraph\paragraph
\renewcommand{\paragraph}[1]{\oldparagraph{#1}\mbox{}}
\fi
\ifx\subparagraph\undefined\else
\let\oldsubparagraph\subparagraph
\renewcommand{\subparagraph}[1]{\oldsubparagraph{#1}\mbox{}}
\fi

%%% Use protect on footnotes to avoid problems with footnotes in titles
\let\rmarkdownfootnote\footnote%
\def\footnote{\protect\rmarkdownfootnote}

%%% Change title format to be more compact
\usepackage{titling}

% Create subtitle command for use in maketitle
\providecommand{\subtitle}[1]{
  \posttitle{
    \begin{center}\large#1\end{center}
    }
}

\setlength{\droptitle}{-2em}

  \title{Engaging Multiple Personalities}
    \pretitle{\vspace{\droptitle}\centering\huge}
  \posttitle{\par}
  \subtitle{Volume 4: The Collected Blog Posts}
  \author{David Yeung}
    \preauthor{\centering\large\emph}
  \postauthor{\par}
      \predate{\centering\large\emph}
  \postdate{\par}
    \date{2020-03-31}

\usepackage{booktabs}

\begin{document}
\maketitle

{
\setcounter{tocdepth}{1}
\tableofcontents
}
\hypertarget{copyright}{%
\chapter*{Copyright}\label{copyright}}
\addcontentsline{toc}{chapter}{Copyright}

Copyright © 2020
by David Yeung

All rights reserved

ISBN ???????

BISAC: Psychology / Psychopathology / Dissociative Identity Disorder

\begin{quote}
Acknowledge the drama

but

Heal the trauma.
\end{quote}

\hypertarget{dedication}{%
\chapter*{Dedication}\label{dedication}}
\addcontentsline{toc}{chapter}{Dedication}

This collection of the \href{https://www.engagingmultiples.com/blog/}{Engagingmultiples.com/blog} posts is dedicated to the community of those with Dissociative Identity Disorder and their supporters, from whom I have learned so much.

\hypertarget{warning}{%
\chapter*{Warning}\label{warning}}
\addcontentsline{toc}{chapter}{Warning}

If you know or suspect that you have experienced childhood abuse, please make sure you have a good network of support to turn to when exploring such a past. Please stop and ask for help if you feel any emotional turmoil arising when reading this book. While every effort has been made to omit materials that might trigger traumatic memory, the best protection is to have the support of a competent therapist to help process any such turmoil.

\hypertarget{introduction-for-volume-4-the-collected-blog-posts}{%
\chapter*{Introduction for Volume 4 The Collected Blog Posts}\label{introduction-for-volume-4-the-collected-blog-posts}}
\addcontentsline{toc}{chapter}{Introduction for Volume 4 The Collected Blog Posts}

After the publication of the first in what turned out to be a series of books on dissociative Identity disorder, Engaging Multiple Personalities Volume 1 Contextual Case Histories, I started writing a blog in response to readers' comments and questions. Having retired as a psychiatrist whose practice focused on these issues for several decades, this was a way for me to continue to support the greater DID community.
The posts cover many issues relating to the broad field of early childhood trauma and dissociative disorders. These experiences, unfortunately, are not uncommon experiences. But, they are often overlooked by psychiatrists in their work with patients searching for healing.

The blog topics tend to cover practical issues for patients and therapists, including patients waiting to find therapists, significant others who crave information on how to live with their DID partners, and therapists who find it difficult to identify and treat dissociative disorders when they lack past experience. It is hard to make the first step in working with dissociative clients - until one experiences the power of simply engaging in direct communication with the presenting alters.

My blog posts are available on the Internet for free. This ebook of the collected posts is also available for free. I am releasing the posts in this way as it allows for interested individuals to have all the posts available in one easily searchable digital book. It also allows me to update the collection for anyone wishing to update their personal copy.

Many individuals with DID lack the resources to purchase my books. This has always bothered me. While I offered copies of my books to various libraries as a donation, there was no real response. Because the books address many of the same topics, often with a somewhat different focus and level of detail, by publishing this as a free download, it is a way to further support those members of the DID community.

Many of the blog topics are not covered in general journal articles or standard text books. For example, first hand information on disclosure of DID, first hand experience on how to engage a new alter who appears the first time at 3 am in the morning, angry alters, and alters in despair etc. These are common place practical issues that call for genuine empathic and kind responses by the person who happens on site, at the most unexpected hours of the day.

Other issues are discussed in these posts. A glance of the Table of Contents will show readers what to expect from reading these collected posts. None of the material I publish should be taken as therapy. As always, read potentially triggering material in a safe place and in small doses to avoid retraumatization.

The purpose of my books and blog is to demystify DID as well as its treatment as it is a common phenomenon that has been neglected for far too long.

As always, I must thank my former patients who have taught me more about dissociative disorders than the several prestigious institutions where I was trained at the beginning of my career, or any of the continuing medical education programs I attended over the course of my almost 50 years practicing psychiatry.

\emph{David Yeung}

\hypertarget{the-purpose-of-this-blog-collection}{%
\chapter{The Purpose of this Blog Collection}\label{the-purpose-of-this-blog-collection}}

\hypertarget{a-wonderful-use-of-this-blog-and-engaging-multiple-personalities-volume-1-and-2}{%
\section{A Wonderful Use of This Blog and Engaging Multiple Personalities Volume 1 and 2}\label{a-wonderful-use-of-this-blog-and-engaging-multiple-personalities-volume-1-and-2}}

\emph{Posted on October 10, 2016}

I received a personal message giving me permission to discuss how one member of a DID Facebook group used my books, Engaging Multiple Personalities. With great joy and appreciation, this is the message I received, lightly edited for clarity and anonymity:

\emph{``Yes of course you can have permission to use my words as you see fit. If it wasn't for your blogs, I very much doubt I would be helping mental health {[}workers{]} in my tiny area make small changes. On Sept 16th 2016 we managed to get a training day on DID for all who work in mental health in our rural sleepy little town in the {[}UK{]}. Until we appeared in this little place, the psychiatrist tells me they never had a case of DID!? I suggested that they have but didn't see them, misdiagnosed them or they are hiding still out of fear, fear they will lose their children, fear we will get that wrong label and be forced to take all sorts of unhelpfully unpleasant drugs. We weren't accepted easily though. We were taken away from our family put on a section. We were forced to go through a forensic evaluation to assess the risk we were to the public and our youngest child, he is 14. They failed to see he is the last child at home of 6 who was never abused or made to witness our self-harm. We passed the core assessment and forensic evaluation 14 months ago but were only given the right to be alone with our child 2 days ago. We committed no crime, we hurt no one. We were just brave enough to tell our psychiatrist that we have DID. But things are changing {[}here now{]}. Another 5 clients have stepped forward to reveal their DID but was in the local {[}mental health{]} system far longer than me. It does make us smile now that every person from mental health services we have seen since the training day now knows about DID. We are kept busy with appointments to speak to more CPN'S, social workers, therapists, crisis team nurses to help them in their education about what DID looks like, sounds like and to share our experiences with them. If we didn't stumble on your books none of this would be possible. So, if we can give a tiny bit back to you to show our appreciation we are more than willing. Thank you from all 17 of us.''}

I commend this individual for her bravery and strength in first dealing with the difficulties of her local mental health system for herself and for then helping that same mental health therapist group learn about DID. I am delighted that my blog and books continue to help individuals and mental health workers far from my home! My guess is that with the DID education of the therapists, those additional 5 clients felt safe enough to then disclose their DID. This is how the DID community's strength helps each other to heal, transforms therapists' understanding of DID, and can continue to do so.

It was very kind for this system to want to give back to me, to show appreciation. But, truly, this is appreciation for the hard work my own DID patients put into helping me understand how to work with DID. In many ways, my books and blogs are their messengers -- their gift of healing to others with DID.

\hypertarget{on-using-the-3-engaging-multiple-personalities-volumes}{%
\section{On Using the 3 Engaging Multiple Personalities Volumes}\label{on-using-the-3-engaging-multiple-personalities-volumes}}

\emph{Posted on July 10, 2018}

I want to express my thoughts on how best to use the 3 Volumes of the Engaging Multiple Personalities Series.

For those with DID, If you have a therapist, then Volume 1 may be very helpful to clarify issues you might be working on with your therapist. Given that DID manifests in many ways, some of the case histories might be useful by way of saying ``this is somewhat similar to my experience'' or ``this is not what I experience.'' The therapeutic keys can also be good points to bring forward with your therapist to the extent they ring true to your experience.

For therapists, and for individuals with DID who may have found a therapist willing to work with them but who has little to no experience with DID, Volume 2 will be most helpful for the therapist while Volume 1 can be a bridge through which you can work toward a positive therapeutic journey.

For those with DID who do not yet have a therapist, Volume 3 was written specifically for you. It can help you understand that there is a definite context to your experience. That in fact, dissociation is a critical response to enable you to survive abuse rather than something crazy. Dissociation is not insanity, far from it. While Volume 3 is not self-therapy, it may give you a strong foundation, self-empowerment if you will, upon which you can build a therapeutic alliance that will work for both you and a therapist in the future.

I find it very interesting that while the books get a very positive response from those with DID as well as from therapists that have read them as a result of patients' suggestions. I find it painful to have to repeat so often that the mainstream psychiatric community, and most therapists, still do not appreciate the impact of early childhood abuse that results in DID.
There are very few reviews on Amazon, where the series is sold. So, the outreach for these volumes is limited to those in the DID Facebook groups in which I post, and to which those members share. If people do find the different volumes helpful, and you feel safe enough to do so, please post a review on Amazon. I think it is likely best to do it anonymously or under a pseudonym. In that way, perhaps a wider audience of therapists, and those with DID that do not connect with the Facebook groups, may encounter the books.

Finally, I also learned recently that one of the libraries that purchased Volume 1 no longer has it on the shelves. Why? Because it has been read so much that it has fallen apart. If you contact your local library, perhaps they will purchase a hard copy which would then be a resource in that community. But, at the same time, I know that the ebook version will not fall apart when it is used -- no matter how many times! So, I am happy to donate the ebook to any library that wishes to have a copy regardless of whether or not they purchase a hard copy. There are certainly more libraries in the world than I can afford to do this with, but I am happy to start with 100. If your library is interested, please have them email me at \href{mailto:engagingmultiples@gmail.com}{\nolinkurl{engagingmultiples@gmail.com}}.

\hypertarget{post-stroke-thoughts}{%
\section{Post Stroke Thoughts}\label{post-stroke-thoughts}}

\emph{Posted on August 5, 2015}

I apologize for not updating my blog or participating in any of the DID Facebook groups for awhile.

I am recovering from a small stroke. While the recovery is going well, such events are always an important opportunity to take stock of one's life, conduct and aspirations. As you know, I wrote Engaging Multiple Personalities Volumes 1 and 2 last year in order to pass on the extraordinary knowledge and insight I received from my DID patients. Prior to my stroke, I was doing a bit of traveling but each evening I kept coming back to recollections of my patients. In hindsight, before I actually became aware of my lack of understanding, it is clear that I missed several DID cases.

In fact, early in my career there were a number of cases where I believe I fell into the traps I warn about in my books, diagnosing patients as bipolar or borderline. Like other psychiatrists of my generation (even up to now), I had been taught the DID was simply so rare that it was highly unlikely that I would ever see even one case. The result was that I did not pay attention to alters that showed up to see if I was trustworthy and open to their presence. Unfortunately, for some of my patients, out of my own ignorance, I missed the correct diagnosis/therapeutic path.

I hope that my books will guide other therapists to avoid making those same mistakes. I will continue to blog and participate in supporting the DID community as best I can during my recovery.

\hypertarget{surprising-responses-to-engaging-multiple-personalities}{%
\section{Surprising Responses to Engaging Multiple Personalities}\label{surprising-responses-to-engaging-multiple-personalities}}

\emph{Posted on November 13, 2016}

It has been about two years since the publication of Volume 1 of Engaging Multiple Personalities. While I have received numerous and important responses from individuals with DID and at least a few therapists, I have solicited responses from other readers from whom I had somewhat surprising feedback. I am putting up this post as it highlights some of the obstacles facing those with DID. Forewarned is forearmed, so I offer this as something to help prepare individuals with DID to deal with mistaken views on the part of therapists who should know better -- and others they may encounter.

1.\emph{``Trauma happened decades ago, surely patients can forget and move forward.''} This was also expressed as \emph{``They should stop dwelling on the past and focus on the future.''} This is the most common response to my book by both general (non-DID individuals) readers as well as highly learned or qualified people, including two professors in Medicine, one church minister and headmaster. I am flabbergasted! I thought that by now it would be general knowledge that after some trauma, the memory is stuck in the body, and that one cannot wipe it clear based on the strength of one's will. The saying is ``The body keeps the score.'' (Van der Kolk.)

General Dallaire of the Canadian Forces peace keeping soldiers Rwanda wrote a moving account of a flashback he had that was triggered by seeing a person chopping open a coconut shell with a cleaver. Simply seeing that image, he immediately began to re-experience watching people being killed with machetes. His ability to intervene and rescue anyone, to stop the slaughter, was blocked by the UN mandate prohibiting any intervention by him or his men. He re-experienced the trauma of seeing what was going on, as if he was there once again.

That is the way flashbacks work, it is not a question of choice. They come back faster than a rocket, by-passing the conceptual process. They take over your mind and your body through the autonomic nervous and motor system before coming to one's awareness. They take over your perceptions so that you are no longer grounded in the present, rather the past reaches out its hands to pull you back. People with PTSD all experience that. DID survivors commonly experience that kind of flashback regarding early childhood trauma that might have happened decades ago.

\begin{enumerate}
\def\labelenumi{\arabic{enumi}.}
\setcounter{enumi}{1}
\tightlist
\item
  Another frequent question was, \emph{``Do they really appear like that, as a 4 year old child in the body of a 50 year old woman?''} Rather than commenting on the depth of abuse that must have occurred to generate the protective mechanism of dissociation, this is the topic that generated interest. General readers, again referring to those without DID, sometimes get sidetracked by the dramatic aspect of the DID presentation, of an alter suddenly appearing. In doing so, they fail to grasp the impact of the trauma, the fear and suffering experienced the individual experienced in the past or in the present moment of a flashback, and consequent loss of function.
\end{enumerate}

This is worse than unfortunate! In general, people do not want to face the ugly facts of childhood trauma. Because of how terrible the trauma must have been, people cut off their own empathy -- perhaps afraid that they themselves will be overwhelmed just contemplating it. Instead, they often refer back to their own experience of a mild loss of details of events from their own childhood. But those references are to what life was like when they were 4 years old rather than imagining the trauma someone else experienced at that age that results in dissociation. It is safer for non-DID individuals to get carried away by the drama, and avoid the trauma.

\begin{enumerate}
\def\labelenumi{\arabic{enumi}.}
\setcounter{enumi}{2}
\tightlist
\item
  \emph{The general reader (and society in general) simply does not grasp the immensity of the problem, the number of individuals affected, and how horrific their experience must have been.} It impacts an enormous number of psychiatric patients who are looking for therapists to help treat their trauma and dissociation. It may be that this will change as the impact of foreign conflicts involving large numbers of traumatized children, just as it was not until the tidal wave of PTSD impacting military personnel returning from Vietnam forced society to at least acknowledge that it was there. And just as with the returning servicemen, the impact of the wartime trauma on children in foreign conflicts will take decades to truly unfold.
\end{enumerate}

Certainly toward the end of my psychiatric practice, I repeatedly received confirmation from patients I meet suffering from depression that they were prescribed antidepressants without questions even being asked about their possible adverse childhood experience. I am well aware that even when such questions are asked, they may not yield the correct answer in the first place -- which may correctly be yes or no. However, when patients are not even given the chance to offer any information on past trauma, the therapist has failed in a fundamental way.

I encourage you to have confidence in your own experience as you proceed on your healing journey rather than be subject to the confusion and ignorance of even professionals. Find therapists who do understand DID, or train decent therapists, who simply don't have experience, through the honesty of your journey.

\hypertarget{the-foundations-of-hope}{%
\chapter{The Foundations of Hope}\label{the-foundations-of-hope}}

\hypertarget{the-importance-of-hope}{%
\section{The Importance of Hope}\label{the-importance-of-hope}}

\emph{Posted on March 6, 2015}

As a retired psychiatrist reflecting on a life of treating broken bodies, spirits and souls, I have had the extraordinary privilege to learn from my past experience, both successes and failures, and identify the most basic fundamental ingredients essential to helping people heal.
They boil down to:

\begin{enumerate}
\def\labelenumi{\arabic{enumi}.}
\item
  Establishing a genuine therapeutic alliance, which necessarily involves congruence and empathic understanding on the part of the therapist.
\item
  Installing (or restoring) faith and hope in the client.
\end{enumerate}

In all the cases of successful suicide by patients that I am aware of, the common threads were the client being overwhelmed by loss of hope, and the failure of the therapist to instill or restore hope in the client. And all too often, when a patient successfully committed suicide, it was clear that they felt that their therapist had lost hope in their recovery too. It is a great sadness that therapists can and do lost hope in just that way.

We must do better as therapists, and it is possible to do so. I believe the key point is to understand that hopelessness, manifesting as depression, suicidal ideation or suicide attempts does not happen in a vacuum. Serotonin alone will not eliminate the risk of suicide if the underlying cause is not addressed. That underlying cause, in cases of abuse, is overwhelming fear. The dyad of hope and fear must be clearly understood.

In cases of Complex PTSD, the trauma is overwhelmingly powerful, leaving the client terrified. Being terrified, without any safe haven from the abuser, leads to hopelessness which must be recognized and addressed. For those suffering from Complex PTSD, the hopelessness is intimately tied to and a product of that fear. For abuse survivors, the fear is often tied to the direct inflicting of pain, physical, sexual, emotional, coupled with the repeated assertion that no one will believe that the survivor has been abused.

The patient hopes the abuse will stop, they fear it will not. They hope that someone will believe them, they fear no one will. They hope that if they act is whatever way the abuser demands, that they will be spared and they are not. Fear is the flip side of hope.

While the psychiatrist assesses the patient, the patient assesses the psychiatrist. The patient hopes the psychiatrist will understand, and fears that they won't. When those with complex PTSD have a long history of ineffective and somewhat destructive relationships with the mental health system, they fear -- often correctly -- that everything they had been programmed to believe about no one believing them is true. In this way, the dichotomy of hope and fear is brought into the therapeutic relationship from the very beginning.

To combat this and strengthen the therapeutic alliance, the psychiatrist must effectively communicate that the therapeutic journey will undermine that foundation of fear. To avoid scaring the patient, one must encourage them that taking the smallest steps toward healing are the safest -- particularly at the start of therapy. Each time any fear is undermined, a glimmer of hope emerges. That is the nature of the relationship of hope and fear to communicate to the patient.

Time and time again in my own practise, I was reminded that little gestures are the crucial building blocks of healing. Healing does not come from grand breakthrough of revelations or enlightenment. It is built on small building blocks even at the level of regaining the control of one comfortable breath.

Offer hope by helping the patient make tiny, achievable goals with each therapeutic encounter. Each session with the patient that enables them to exert some control, even in a very limited way, over the the runaway flashback symptoms is a critical ``baby step'' in healing.

As related in Chapter 1 of my book ``Engaging Multiple Personalities'', I told Joan in our first session that my aim was to help her feel just a little better each session. According to her, this was a most powerful suggestion that propelled her toward healing when she was in the darkest period of her life, having almost given up as a result of the total dis-empowerment of PTSD.

In another case, my last patient of the day calmly told me that she was going to kill herself after seeing me. There was no doubt in my mind that she was simply stating her intention, and that it was not an empty threat or desire for attention. There was literally only one hour to intervene.

I related to the angry part of her, understanding that the source of the anger was the deep hurt of past trauma. I helped her connect to the anger as a source of valuable energy that could be redirected to her healing. I gave her hope that she could turn around the anger, the hate, and see that the best revenge was to overcome the trauma inflicted by the abuser by showing that the abuser had not succeeded in destroying her.

The best revenge is indeed to show the abusers that they failed to destroy the child. Many survivors of childhood abuse carry this sense of hope, of mission, to survive to tell the world that such abuse did happen. To stay alive, to fight for the future so that one could bear witness to such horrendous crimes. We need to change the world so that every child grows up nurtured, loved and protected from abuse.

\hypertarget{on-being-a-supportive-spousepartner}{%
\section{On being a supportive spouse/partner}\label{on-being-a-supportive-spousepartner}}

\emph{Posted on February 16, 2015}

\textbf{This is a lightly edited response to a question posed by a spouse about alters coming out far more at night than during the day:}

In Volume 1 of Engaging Multiple Personalities, I discuss one of my patients who was similarly having alters, particularly highly traumatized young alters, come out at night. Her spouse had similar difficulties due to him being unable to go to sleep until the alters expressed what they needed to express -- and yes, not going to sleep until 2 or 3 am night after night. For that patient, there seemed to be two reasons for the evening appearances of alters, both equally important: 1) they came out at the time of night when the abuse generally occurred, and 2) the alters were feeling safe enough to come out with the spouse and express what they needed to express as part of their therapeutic journey.

The spouse came up with some quite innovative approaches to helping the alters, giving them space and comfort as well as the recognition that they were with the spouse in a time and place where the abuser never was. These are also discussed in Volume 1. Check in with the therapist working with your spouse on any approach you wish to take. Certainly, the therapist should know about the alters coming out each night and what is happening. I do not encourage spouses to be therapists, but when alters come out, you do need to be kind, empathic and know what to do.

\textbf{Please take care of your own health while doing this.} To provide the support your spouse needs, you MUST maintain your health, your balance and your empathy. Volume 2 includes a section on self care for therapists, and the warnings I give there might be applicable to spouses that are meeting with traumatized alters at home late at night.

Know your own limits, and know when they are being reached. You cannot expect that traumatized alters will see the strain on your health, and they often do not have the capacity to stop once the flashbacks start. Set time limits with the alters so that you can help them the next evening as well, for example, rather than burning yourself out. You might try some of the grounding exercises with them that I have written about in my books as well as on my blog, but again, check with the therapist.

\hypertarget{the-power-of-dissociation}{%
\section{The Power of Dissociation}\label{the-power-of-dissociation}}

\emph{Posted on November 17, 2015}

Without in any way trivializing the trauma that is the core of early childhood abuse, there is a fascinating aspect of MPD that is deserving of further exploration. The fact is that dissociation allowed the abused child to survive. That, in itself, is cause for appreciation of the power of the dissociative response. It is the habituation to dissociation as a response to triggers and unprocessed trauma arising that causes such tremendous difficulties for the patient including amnestic barriers and internal conflict. For some, dissociation can produce unexpected hosts of achievements as part and parcel of the impact of the disorder. In therapy, there is often an over-emphasis on the damage that has been done without a concurrent expression of how genuine healing is possible -- that there is hope.

Among those with DID that I have treated as well as those I have encountered after my retirement, some have accomplished extraordinary things both in recovery and in the world. While I discussed this aspect briefly in Engaging Multiple Personalities Volume 2, I believe it is worthwhile to go deeper into this aspect of DID.

It is clear to me that I failed to diagnose certain patients as DID in a timely fashion because of their external accomplishments. I was misdirected by my own admiration for them. I will not identify those patients for obvious privacy reasons but they included people in the top tier of their various professions, in both business and academia.

The first point to make is that for anyone to survive the intensity of trauma that gives rise to DID, they must of necessity be extraordinarily brave, strong and resilient. Anyone coping with and surviving ongoing abuse as a child crafts strategies on a survival level that successfully deal with vicious adult abusers. Some abusers are hiding in plain sight as valued members of the family and/or community. Some abusers are individuals that frighten law enforcement, other adult family members and other adults in the community. Consider the pressure a child is under dealing with abusers which the outside world either cheers as a valued individual or fears as a dangerous individual. For the child, there is no hope of escape, nowhere to run, no refuge.

Dissociation is a most brilliant survival strategy for such a small child. Fundamentally, that is the point I have tried to make in both volumes of Engaging Multiple Personalities as well as on my blog. To both therapists and those with DID, I say please do not turn away from the alters. However angry, mean, sad, or panicked they may be, it is the alters that were the means of surviving the abuse. The difficulties that DID individuals have is dealing with the aftereffects of habituating the use of such a radical means; the only means available to them as children.

Alters arise holding pieces of trauma as well as their own habitual modes of interacting with the world. The ability to dissociate provides a tremendous opportunity for an alter to completely focus when they are in control of the body. The single mindedness allowed survival as a child by focusing away from the trauma as it happened. As an adult, the dissociation via triggers can be an ongoing trap of retraumatization. Alternatively, it can be used to successfully accomplish things in the outside world. On a very basic level, dissociation allows DID individuals to go to work, take care of themselves and others such as their children, while holding the unprocessed trauma temporarily at bay until the system is overwhelmed.

There are those with MPD who may excel in multiple disciplines. For these individuals, each dissociative part, each alter, can develop their focused interest in a topic without distraction. Any scientist, scholar or artist, has this ability of total concentration when working to the exclusion of other distractions. With the ability to dissociate somewhat completely at will, the result of such total concentration can be excelling in a field. If one part is an academic, another an artist, and still another an athlete, how interesting that might be.

Individuals who have publicly disclosed their DID have often been ignored or had their DID denied. However, there are a few individuals whose standing in their respective communities allowed them to disclose their DID without quite the same level of disparagement as others have experienced or rightly may fear. This is not to say that such individuals experienced no negativity following their disclosure. However, because of their stature, they gave pause to the deniers of DID. Indeed, they created the opportunity for non-DID individuals to begin to see DID in a less perjorative light.

Robert Oxnam is an academic who revealed his MPD in his autobiographic A Fractured Mind (2004). Robert is a scholar of Asian studies, having taught in US universities as well as having lectured in Beijing University -- in Chinese. His most famous role was to lead a cultural tour of China for the likes of Bill Gate, Warren Buffet and president HW Bush. He also was a China expert advising the former US presidents. He has authored several books and served as the head of the Asia Society in New York. However, apart from the focus on Asia, he plays the cello, and is now a prominent sculptural artist. Beyond that, at different stages of his life, he was a competitive archer, an accomplished cyclist and a prominent, in some circles, rollerblader.

Another MPD autobiographer is Herschel Walker (author of Breaking Free {[}2008{]}). He was his high school valedictorian and a Heisman trophy winning athlete. He was an NFL player, and then excelled as a world class bobsledder, sprinter and mixed martial artist. He is a successful businessman in the food industry. In his autobiography, he mentions that his ability to dissociate allowed him to be apparently untouched by pain in the midst of crushing blows from opponents -- to their utter consternation.

Going back to the earliest days of psychiatry, Anna O. is believed to be the first MPD patient whose case history was described in detail. Her case is found in Freud's book---- Studies on Hysteria (1895). Freud missed the diagnosis, or, to be more accurate, there was not an applicable diagnostic category at the time other then the general one of hysteria. Even in missing the diagnosis, he did note her concern about ``time-loss'' and having ``two selves.'' Both of these are primary and often the first indicators of a potential DID diagnosis. At different times, Anna O would speak different languages and refuse to believe, for example, that she actually knew others. There are several other points that would lead one to consider a DID diagnosis that are clearly laid out in the case history.

Anna O (real name Bertha Pappenheim) was at one time a patient of Breuer (a colleague of Freud and co-author of Studies on Hysteria). He stopped treating her as she was becoming progressively worse and had to be institutionalized for a period of time. Breuer told Freud that she was deranged; he hoped she would die to end her suffering. One can imagine the depth of her depression through Breuer's comment. However, she later achieved renown for her social work, such that the West German government issued a postage stamp in honour of her contributions to that field. She was an author of several novellas, poems and plays. In addition, she was a translator and a writer of several important pieces attacking the trafficking of women in eastern Europe and the Orient. Her focus on helping others who were sexually traumatized is not uncommon in the DID world. In my own practice, I saw clear examples of this practical application of empathy by DID patients in dealing with children and other at-risk individuals.

Unfortunately, the term MPD has trivialized the concept of dissociation into parts, offering endless possibilities of theatrical materials for movies and TV series. They tend to emphasize the histrionic parts of the multiple facets of a single patient. This trivializes the pain of the original trauma that caused the dissociation as a defense to protect the fragile ego. It somewhat makes light of the damage done to the growing individual and the possible ill effects impacting the next generations as well as the untold misery affecting many people involved.

Psychiatry struggles to find a better name of the affliction, changing it from MPD to DID in 1989. I wonder if this change has made any difference. Die-hard disbelievers still cling to the pseudo logical argument that if a person can have more than one identity, then two persons hold the same passport or one person can have multiple passports -- completely missing the point of the disorder. The book by Schrieber reawakened interest on this issue but some professionals got distracted by a fascination with the multiplicities of the ``personalities''.

Because of the word personality or identity in the diagnostic label, many psychiatrists cannot make the paradigm shift to accept the concept of DID, nor accept DID as a genuine psychiatric disorder. Some serious academics still deny DID as a mental disorder, declaring it to be a condition that is produced iatrogenically, or otherwise non-existent. This mistaken view is much to the detriment of the welfare of DID sufferers trying to find a therapist. Even worse, it teaches new psychiatrists something that is simply wrong. Out of their ignorance, they will then perpetuate the same mistaken view and impact an even wider circle of patients.

By studying the successes that individuals with DID have had in healing, in worldly activities and in displaying great empathy helping others, psychiatrists and other therapists can learn quite a bit about trauma, its treatment and the possibility of truly leading those with the disorder into health. The successes can be used to give hope, the critical element in working with those trapped in retraumatization cycles, that healing is possible, that joy is possible and that their very survival as a child is a mark of how creative, strong and successful they were as a child all the way through to this present moment.

\hypertarget{spirituality-and-the-healing-of-traumatic-wounds}{%
\section{Spirituality and the Healing of Traumatic Wounds}\label{spirituality-and-the-healing-of-traumatic-wounds}}

\emph{Posted on January 14, 2016}

I have thought about this topic a great deal but have hesitated to write much about it. Communicating about spirituality in the treatment of early childhood trauma is difficult because, for many, religion as well as pseudo-religious imagery played an integral part in the trauma. On the other hand, spirituality has also, for many, been a key foundation upon which healing has arisen.

In my experience, spirituality, whether with or without formal doctrinal religious faith, played a powerful role for many of my patients in their becoming successful survivors. I have seen patients rise up from the depths of despair despite horrendous and prolonged backgrounds of traumatic experience. For those, spirituality gave them the hope and strength necessary to overcome the tremendous difficulties that resulted from their history of early childhood abuse.

Again in my experience, formal religion and religious doctrine can be a source of hope for patients as well as the source of their trauma. Therefore, as a therapist, one must be able to help a patient connect with the spiritual aspect of their life in a way that avoids the risk of re-traumatization. This means that one must be flexible enough to support the patient's religious faith when it differs from one's own as well as be able to invite a patient's spirituality in the complete absence of or antipathy toward religious doctrine if that is the path of safety for them.

A therapist friend of mine, a Buddhist, once sat with one of his patients in a church. The patient knew the therapist was a Buddhist and was surprised at the suggestion. But the patient had already made it clear to the therapist that religion was important to him and that he felt safe sitting in a pew in a church. My friend pointed out that if the goal was to make the patient feel safe as a way to set the ground for genuine therapeutic communication, why not do it in the place the patient felt most safe. To have taken him to a meditation center and asked him to sit on a cushion with legs crossed would have made him feel quite uncomfortable, if not completely unsafe. This is an example of the benefit to patients of therapists not being too stuck in their own personal religious view.

We tend to think of ourselves as amalgamations of mind, body and spirit. We have a general idea of the meaning of mind and body, but sometimes we don't pay much attention to what we mean by spirit. Often, we simply assert spirit to be something that is mystical, with no presence or relevance in the everyday aspects of life. My understanding of spirituality is that it refers to that which is both of and beyond the material world. It is more than a weekly visit to a church, synagogue, mosque or meditation center. However, spirituality does not need to be tied up with religious dogmas, rituals, heavens or hells.

I define spirituality as the framework of how we face our existence, how we face our selves. It is a fundamental understanding of how we might be kind to ourselves in both body and mind. It suffuses our awareness, leading us to be more in touch with our inner core.

Trauma is a fact of life in the natural world -- as when a tiger chases down a deer. Both therapists and patients have to accept this as part of the human condition as well, and each of us needs to find our own way to handle it. We need spiritual strength in our own life journey but we also need to cultivate, protect and enhance our spiritual strength when we try to guide someone on their healing path.

In 1968, Joseph Campbell said, ``In India, two amazing figures are used to characterize the two principal types of religious attitudes. One is `the way of the kitten; the other, `the way of the monkey.' When a kitten cries `Miaow,' it's mother coming, takes it by the scruff and carries it to safety; but as anyone who has ever traveled in India will have observed, when a band of monkeys come scampering down from a tree and across the road , the babies riding on their mothers' backs are hanging on by themselves.

Accordingly, with reference to the two attitudes: the first is that of the person who prays, `O Lord, O Lord, come save me' and of the second of one who, without such prayers or cries, goes to work on himself.'' In China and in Japan, the two attitudes are termed, ``outside strength'' and ``one's own strength.'' No matter which religion one pursues, or for that matter, or spirituality in the absence of a religious tradition, these approaches need not be contradictory. I respectfully request the indulgence of those literal dogmatists in any particular religious affiliations to accept that it is an individual matter to choose either of these approaches or a mix of the two, no matter if you are a Buddhist, Muslim, Christian, Jew or none of the foregoing.

You have to find a reason to fight to overcome the tremendous obstacles of an abusive childhood. One good example of a reason to overcome the obstacles of abuse is to defy the abuser's threats, to make yourself whole despite and against all odds, surviving the trauma and betrayal. If you subscribe to a personal deity, prayers asking for specific help and guidance can give you strength to overcome those obstacles. Like escaping from a deep well, you may need the sense of an external power to throw a rope for you to grasp and pull you out. On the other hand, if you do not engage a religious tradition, simply touching the power of the earth or feeling the warmth of the sunlight, those fundamental connections of life that are beyond you, may be enough to chase away the dark clouds and overcome past trauma.

In short, you need to have hope. Many of my atheist/agnostic patients relied on AA, NA or church fellowship for support in their difficult journey of healing, there is no need to fight alone.

Accessing genuine spirituality requires intention, practice and experience -- rather than just wishful thinking. Spiritual practice within a formal religions tradition is usually quite clear within that tradition. One can see spiritual practice outside of religious strictures as keeping still and paying attention to the now, the present moment of existence.

Be still. Within that, learn to be kind to your own mind. Start doing one-breath meditation. The gradually advance to more than one breath, then to 5, 10 or even 20 minutes. Move toward being non-judgmental. Slowly learn to love yourself, without evaluating that thought as being good or bad. Do not worry about closing your eyes, you may let them open if you so choose. Breathe each breath slowly, be alert and be stable in your sitting position.

You might try something like this:

Breathe in God and breathe out darkness
\emph{or}
Breathe in love and breathe out fear.

The need for a spiritual component to one's life applies equally to therapists who, day in and day out, listen with deep empathy. Listening in that way to the horrendous tales of their patients' extreme past and often present sufferings, therapists are in need of strength to purge such toxic material that is capable of inflicting vicarious trauma on them. I have suggested extracurricular activities such as physical workouts as well as creative hobbies of music, sculpture, pottery etc. These remind you that there is something wholesome, beautiful and noble in this world, that it is not simply filled up with ugliness, betrayal and negativity.

In their uncertainty, people tend to grasp hold of dogmas to anchor their sense of security. They tend to gravitate to an extreme end of some belief, unable to see compromise as healthy in their dogmatic system. But, kindness transcends dogma. It is the secret and quite magical ingredient for healing. Always be kind.

\hypertarget{co-consciousness}{%
\section{Co-consciousness}\label{co-consciousness}}

\emph{Posted on March 20, 2016}

Among DID individuals, the sharing of conscious awareness between alters exists in varying degrees. I have seen cases where there has appeared to be no amnestic barriers between individual alters, where the host and alters appeared to be fully cognizant of each other. On the other hand, I have seen cases where the host was absolutely unaware of any alters despite clear evidence of their presence. In those cases, while the host was not aware of the alters, there were alters with an awareness of the host as well as having some limited awareness of at least a few other alters. So, according to my experience, there is a spectrum of shared consciousness in DID patients. From a therapeutic point of view, while treatment of patients without amnestic barriers differs in some ways from treatment of those with such barriers, the fundamental goal of therapy is the same: to support the healing of the early childhood trauma that gave rise to the dissociation and its attendant alters.

Good DID therapy involves promoting co-consciousness. With co-consciousness, it is possible to begin teaching the patient's system the value of cooperation among the alters. Enjoin them to emulate the spirit of a champion football team, with each member utilizing their full potential and working together to achieve a common goal.

Returning to the patients that seemed to lack amnestic barriers, it is important to understand that such co-consciousness did not mean that the host and alters were well-coordinated or living in harmony. If they were all in harmony, there would be no ``disease.'' There would be little likelihood of a need or even desire for psychiatric intervention. It is when there is conflict between the host and/or among alters that treatment is needed.

Conflict in DID patients is usually quite evident. A system full of clashes is usually playing out a power struggle internally that is manifesting externally. In some cases, dictatorial alters may have an ironclad control over the information flow and behavior of the system as a whole. In addition, because of their strong individualistic feelings, some alters may appear to behave in a callous or selfish way with no regard for the needs of the host and the other alters. This can result in one alter hijacking control of the body for a time, short or extended, for the pleasure/intentions/wishes of that one alter alone. The result is usually the host's experience of time-loss, one of the key markers for DID and one of the primary causes for seeking therapy.

Such a hijacking alter may think, ``I don't care that others in the system are tired and may need to sleep. It is my time, and I want to go out. I want to have a good time at the bar.'' For example, I had a patient with one alter who regularly took off to have fun cruising around with motorcycle gangs. She totally disregarded the safety of the system, the boundaries of her support network, and the host's appointments to see me. I would note that this kind of conflict can occur even in the context of a patient with some level of co-consciousness.

It is not uncommon for a patient's host or front to vehemently deny the presence of alters despite clear evidence in diaries, letters, and even recorded messages of alters talking. This is the opposite of co-consciousness -- at least with respect to the host. It must be terrifying, not merely disconcerting, for an individual to realize that an alter, another inside part of that same individual, can so completely take over the executive functions of the system to the point that they establish a functioning separate life in the outside world. In fact, I have had patients whose alters would, on occasion, establish a completely separate existence for a few months at a time using that alter's name. In one case, the alter established her own residence in a different apartment, connected with a different social milieu, and, in that case, earned money as a sex trade worker.

I know hosts who have staunchly fought against such recognition of alters and even the idea of co-consciousness. One cannot blame them. The fear is so intense that I sometimes had patients leave therapy rather than work with the recognition. As a result of those experiences, I learned to sometimes withhold revealing or confirming a DID diagnosis so as to avoid scaring the patient into abruptly terminating therapy. In my judgment, it was occasionally justified to delay confirmation of the diagnosis at least until the foundation of a genuine therapeutic alliance was established.

With respect to cases where the alters are completely hidden from one another, one must tread gently. When the presence of alters is pointed out, some DID individuals may take a long time to be convinced that there are indeed other alters coexisting in that same physical body. As a therapist, do not push the point about alters or co-consciousness as a path to healing. It is not a debate to win or lose. Again, if there weren't problems, then the patient wouldn't be in therapy.

The question for the therapist is how to gently promote co-consciousness. First, one must prepare the patient to hear the news that there are alters inside. You must wait until you have confidence that the message is not going to create uncontrollable panic in the host. Establishing a therapeutic alliance with the host is absolutely critical to this. As the therapist, you may or may not have met some or all of the alters directly. Establishing a therapeutic alliance with alters that you have met, or with whom you can otherwise communicate, can strengthen the host's ability to hear the news.

Remember that the amnestic barriers arose for very good reasons. Breaking them down without permission invites further trauma. So, make sure the news is given in a way that makes clear that as a therapist you can help bridge the amnestic barriers when the different parts are ready. One helpful analogy for promoting co-consciousness might be to note that you, the therapist, might be frightened to walk down a dark street alone but would feel much safer walking that same street if you had a friend (or two or three) with you. Even if that friend was also scared, the companionship would be helpful to both you and your friend. Please use the analogy with respect to your own fears about walking down dark streets, not theirs. They will understand the point.

This analogy was quite helpful to some of my patients that had very young alters with similar but not identical trauma memories. The point was not to encourage or even suggest integration. Instead it was to allow each of those alters to know that they were not alone, that there were others inside that could truly understand. That can be the beginning of a friendship within the community of alters. Once that first companionship among alters arises, it can be referenced when talking to other alters that are stil blocked by amnestic barriers.

In this way, you can encourage the direct experience of feeling safer through the experience of co-consciousness. It is a step-by-step process. The patient may feel like they are treading on thin ice in terms of their fear and panic. The simple answer is to encourage them to go very slowly, just as you would when walking on thin ice. When you walk on thin ice, you do not know for certain if it is strong enough to hold you. You go inch by inch, testing and seeing what happens. It is the same here.

A not uncommon experience of one alter starting to consider the possibility of companionship with another alter (though not yet safety) is when they become aware that their traumas had strong similarities to the traumas of another alter or perhaps several alters. This is akin to the analogy of walking a dark street together rather than alone. In addition, when there is the experience of a frightened alter witnessing the emergence of a protective alter in the outside world, both alters can begin to appreciate their respective roles in the system. In this case, it might be the frightened alter identifying a danger and the protective alter reacting to that identification and fulfilling its function. This is again a prelude to developing a sense of safety and can occur more easily when the alters begin to become aware of each other. When this takes place without re-traumatization, this is the beginning of seeing the possibility of healthy teamwork that is a mark of healing.

\hypertarget{christianity-and-forgiveness-part-1}{%
\section{Christianity and Forgiveness -- Part 1}\label{christianity-and-forgiveness-part-1}}

\emph{Posted on February 8, 2016}

This post is a follow up to the short post entitled \href{https://www.engagingmultiples.com/trap-forgiveness/}{``The Trap of Forgiveness''}. It was written following feedback and questions from some readers that are very focused on the Christian notion of forgiveness as part of their healing. It is directed primarily to Christian patients and therapists whose therapeutic work is based on forgiveness as one of the central teachings of Jesus.

Some patients and therapists with deep roots in Christianity see forgiveness as the confirmation of healing. It is sometimes their view that being able to forgive is the ultimate expression of being healed. It is my experience that one needs to be extremely wary of how forgiveness is defined in the context of treating survivors of early childhood abuse.

For example, it is not uncommon for a typical female patient who has survived early childhood abuse by her father to face a spiritual crisis when that father, late in life and perhaps with failing health, asserts his dependency on her as a command. Now insisting on a ``normal'' father-daughter relationship, he may be conveniently ignoring, making light of, or rationalizing his abusive behavior. The patient may then struggle with this: Should she abandon her abusive father or perform her duty as a Christian daughter forgiving him his past sins?

I cannot emphasize enough the importance of defining forgiveness. Depending on the definition you choose, it is either a path to further healing or a path to further retraumatization. In the absence of clearly defining forgiveness, it is a dangerous goal to set for a DID patient.

Therapy must be practical. It must take into account the trauma that patients must process in order to heal. One must consider the likelihood of success, as the goals in therapy must be within the grasp of the patient. Positing conventional notions of forgiveness as the path, goal or indication of success in therapy seems to set both the therapist and the patient up for failure. Setting an unattainable goal will only reinforce the patient's negative self-image engendered by the abuse.

We must be clear that forgiving a living abuser is not like forgiving someone who stole an extra cookie when your back was turned, nor is it like forgiving someone when they are no longer able to harm you -- such as someone who has already died. Promoting or attempting forgiveness can be very dangerous if it involves an abuser who has brutally harmed the patient in the past and is still capable of inflicting further deep wounds and retraumatization through physical or psychological means.

A colleague of mine listened to a woman speak of her sexually abusive father, explaining that he really loved her and the abuse was simply his confusion about how to express it. This seemed to be her conventional version of having forgiven her abusive father for his conduct -- having an explanation she thought she could live with. My colleague told her in no uncertain terms that her father did not love her, that calling his molestation ``love'' was a psychological tactic common used by many abusers -- particularly paternal or older male abusers, and that until she understood that power dynamic she would not be free of the abuse, not healed. She reacted as if he had thrown a bucket of ice water on her; causing her to reconsider the import of what she herself had said.

It later came out that her father had continued to abuse other children -- including her toddler aged daughter. When this was discovered, she and her family moved within a week to another country to escape him. Had she not ``forgiven'' her father in that conventional sense, she would likely have been more on guard against him and thereby protected her own daughter as well as others. I use this real example to demonstrate that conventional notions of forgiveness can hold ongoing danger to the patient and others.

Most trauma that leads to DID is so overwhelming that ordinary individuals cannot truly imagine or comprehend the experience. To presume that one will eventually be able to forgive their abuser and, as a result, have an ongoing positive or at least neutral relationship, as a general rule, is a fantasy. From the Christian theological view, Jesus was able to forgive all their trespasses and sins. From that point of view, one can take joy in Jesus' power to forgive and leave that level of forgiving to Him. However, this is not something within the capacity of ordinary people whether they are DID or not, so do not push that as a therapeutic path or goal.

If you are bitten by a poisonous snake, you can forgive that snake its poisonous venom and understand that it was simply defending itself when you accidentally stepped on it in the jungle. Having venom is in the nature of being a poisonous snake. To forgive the abuser and engage him as if there was no current danger, would be like forgiving that poisonous snake and deciding to carry it back home with you in your pocket to prove your forgiveness. Don't do that!
One must work with forgiveness in a way that is not predicated on continuing to put oneself or others in danger of further abuse. The risk of retraumatization is too great to permit a patient to confuse conventional forgiveness so as to blur the boundaries of their personal safety.

It must also be understood that the critical sense of safety a patient is developing in therapy is the key. Forgiveness, from a therapeutic point of view, must be understood to be an internal process that does not require endangering proof of accomplishment. There are many important reasons to protect the patient from the danger of retraumatization. There is absolutely no need to test the depth of one's forgiveness by engaging an abuser as an expression of forgiveness to him. Patients can be encouraged to simply check their own hearts. Neither from a spiritual nor psychological point of view does forgiving an abuser in your heart mean that one presents the abuser with another opportunity to harm you.

I set out some realistic therapeutic goals for this kind of case in Engaging Multiple Personalities Volume 1 and 2 as well as some practical exercises for establishing safe boundaries in those volumes and in my blog posts. Hopefully, they will prove helpful to readers.

I have yet to define forgiveness in this piece, in part because there are many aspects and understandings of this in Christianity. However, before any notions of forgiveness can arise, it is important in DID therapy to understand and make sure the patient understands that it was often the angry and protective alters that enabled the patient to survive the abuse. So, while I consider that it is a healthy aspiration to forgive others, meaning letting go of bitterness and hatred that is rooted in the past, in therapy one must be very careful to allow that to come to its own fruition. Introducing or promoting forgiveness is denying the insight and role of the angry alters. It will be counterproductive to the therapeutic alliance and the overall healing path.

In my view, being unwilling to forgive means holding on a hateful feelings and bitterness which results in further suffering and prevents healing. My definition of forgiveness does not mean that you go have coffee with your abuser and chat about current events in the world. My definition of forgiveness means letting go of the hatred in your heart. That should happen as a by-product of therapy, maturing in its own time as the system's sense of safety permits. Forgiveness like that, with the warmth and lightness in the heart that results, is an indicator of the final stages of therapeutic success.

\hypertarget{christianity-and-forgiveness-part-2}{%
\section{Christianity and Forgiveness -- Part 2}\label{christianity-and-forgiveness-part-2}}

\emph{Posted on February 29, 2016}

Forgiveness, Christian or otherwise, does not mean condoning or giving excuses to wrongdoing. Sanity may be defined as the ability to tell right from wrong. So here it is: Sexual abuse is wrong. Traumatizing young children is wrong. There is no way to twist logic that makes such abhorrent conduct acceptable. But it is important to remember that the prerequisite to genuine forgiveness is that the victim no longer feels the pain, that the past ceases to intrude into the present.

There are two aspects to an abuser's wrongdoing: his intention and his action. In other words, he might perform despicable acts based on self-serving so-called ``reasoning.'' Many child molesters proceed with rationales they know to be false such as, ``It is really quite harmless. She is only 2 years old. She will not remember this when she grows up. After all, I don't remember what happened to me when I was 2 years old.''

It is likely that with the addition of alcohol and/or rage, the abuser may think that he was justified in his conduct or have forgotten it because it was not a particularly significant event to him. If the victim believes that the original infliction of the trauma is unintentional, they may believe that it will be easier to forgive. In fact, abusers may play on that but it reeks of shifting the blame to the patient along the lines of ``It never would have happened if you weren't such a bad child'' or ``I was drunk so I am not really responsible.'' With respect to the latter, I have colleagues that have studied the Bible and wonder how Lot's daughter's might feel about being blamed for their father's incestuous conduct.

One cannot advise a patient to forgive beyond their own heart if there is even the remotest possibility that the abuser might get a feeling of pathological pleasure, knowing that what he once did decades ago continues having a powerful effect on his victim. The therapist's task is to lead the patient to understanding that holding on to the bitterness about this past experience continues the entrapment by the abuser. The patient's task in therapy is to work through this, to process this part of their past experience so as to be liberated from the retraumatization power of the past.

If you are holding something tightly in your hand, it will fall as soon as you loosen your grip. It is the same with processing trauma. Letting go of a painful memory's strength is possible after you genuinely feel you have shared the experience with a significant person, like your therapist, and that you have finished the task of bearing witness to the crime -- the series of childhood traumas. This process of successful therapy is often accomplished by deep listening and empathetic sharing of the pain on the part of the therapist.

Know that forgiveness does not mean forgetting. You need to remember it as part of your experience in life. You need to maintain a certain vigilance, not hyper-vigilance but still vigilant awareness, to make sure you are not preyed upon in the future. If and after you forgive, you have a choice as to whether or not to include the past abuser in your life.

By forgiving, you are accepting the reality of what happened and are able to free yourself from the past's interference with your current life. This is a gradual process---and it doesn't necessarily have to include the abuser. Forgiveness isn't something you do for the person who wronged you; it's something you do for yourself.

As I and others have said many times, the trauma that leads to DID is so overwhelming that ordinary individuals cannot truly imagine the experience. To presume that one will eventually be able to forgive their abuser in any conventional understanding of forgiveness is, in my opinion and for practical purposes, a fantasy. The aim of treatment should focus on the task at hand, teaching the patient to experience and hold on to the safety of the present. It is to teach the patient that skill so that they can experience the safety of the present when memories of the past arise. When memories are just memories, and are no longer the involuntary reliving of pain, that is what it means by healing.

Here are some therapeutic goals I consider to be realistic for patients. They are practical applications of forgiveness in one's own heart.

\begin{enumerate}
\def\labelenumi{\arabic{enumi}.}
\item
  On the social side, measures that limit and circumscribe interactions with the abuser must be monitored. For example, patients may not be able to say ``no'' in daily life if they are still in contact with the abuser. Therapeutically, the first step is to establish a firm base of a pain-free and safe present. The patient needs to learn the real meaning of the present, which is the immediate experience of breathing this very breath. Forgiveness in this context is being non-judgmental towards oneself. There are usually alters that are in conflict and angry with others who participate in any way, shape, or form with an abuser. Introducing each conflicted alter to the possibility of forgiving alters with a different point of view is a very positive start. It is not telling them to go along with that other alter's view. Rather, it is explaining how that other alter feels. In essence, it is teaching the foundation of empathy. This is not easy, nor is it something that happens quickly. In my experience, it is best introduced talking about how the alter might wish to comfort a confused child -- not by yelling but by holding them with warmth. Then, within that warmth, clarifying the present danger rather than re-working the past.
\item
  In order to forgive oneself, a therapist introduces exercises that teach the patient how to find a physical/psychological safe place in the present. Patients are taught how to put put themselves in a physically relaxed and psychologically comfortable state. The immediate goal is for the patient to make sure that he/she is in a safe distance from the abuser. Within that experience of safety, one can develop the understanding that abusers are both dangerous and usually survivors of abuse themselves. In other words, through the physical and psychological experience of safety in the present, one can remain vigilantly awake, without being hyper-vigilant, and see that abusers are likely acting out the impact of their own history of having been abused. This is training on extending forgiveness without permitting further abuse.
\item
  Teach the patient to go back and process the past trauma in a titrated/controlled manner. In that way, the patient can eventually experience the arising of that memory without their present consciousness being flooded with sympathetic fight-flight-freeze reactions. Various techniques such as ``the 5\% rule'' have already been explained elsewhere. See: \url{https://www.engagingmultiples.com/the-5-rule/}
\item
  Eventually the patient will develop the ability to separate the emotions associated with past trauma from the present recall of that past in a manner which avoids retraumatization. A commonly observed sign of progress is the patient's increasing ability to spontaneously bring back some detail of the past trauma with less panic and more ease. She will speak in a calm voice, without being entrapped in fear or horror. This is usually accompanied with a sense of sadness -- which is completely appropriate. That sadness is another gateway to developing further forgiveness towards oneself and the alters in conflict about the abuser.
\item
  Sometimes there is a wish to understand why the abuse happened. There is the hope that if one can understand the why, then forgiveness will follow because there is a context for the abuse. As a therapist, one must be very clear that there is no acceptable context that permits abuse. One can understand what drives an abuser may have, but that does not grant the abuser permission to abuse. Sometimes there is a ready understanding of abuse -- such as a clear trans-generational abuse pattern. It is important that such a connection is discovered spontaneously by the patient. This is not something to be brought up by the therapist. The patient may show the beginning of understanding by replacing fear or anger with sadness. This means that the patient is developing empathy that is being extended to the abuser. Whether or not genuine forgiveness flows out of this should be left to the natural course of events for that patient. I think it is risking an inappropriate imposition of one's own religious ideas on the patient to bring up forgiveness to the patient as applied to the abuser. It is positive to encourage internally generating forgiveness by the patient for the patient. But, forgiveness is a heavily loaded term in Christian dogma. One must be extremely careful so that the burden of that loaded term is not imposed, intentionally or unintentionally, on the patient.
\item
  There may come practical real life situations that are difficult, such as whether the patient is obliged to visit, support, help, or nurse the abusive parent who may or may not be incapacitated but desires the patient's help in one form or another. My view is that a biological parent, having abused the patient, forfeits their parental status. He has disqualified himself as a parent just as a a physician can be struck off the registry because of misconduct involving a patient. The patient has no obligation towards the abuser as a parent, just as a physician is no longer a physician when his conduct has been found to be unbecoming of that position.
\end{enumerate}

If the patient insists on offering forgiveness, complete or otherwise, then the prerequisite should be that he/she is healed and recovered from the ill-effects of that abusive experience, to the point that they are truly no longer subject to retraumatization. The way he/she speaks of the past abusive experience will make it quite clear whether or not full recovery has been effected. While engaging the abuser as part of one's expression of forgiveness may be seen as a laudable goal from a religious point of view, for an abused individual it is unrealistic. It is not the appropriate goal for DID therapy.

\hypertarget{instilling-hope}{%
\section{Instilling Hope}\label{instilling-hope}}

\emph{Posted on September 16, 2016}

A decade after retirement, I remain preoccupied with some basic issues pertaining to psychotherapy. I believe it is important to express some of the misgivings I have about the general training and preparation of therapists, based on the experience I gained over 40 years as a psychiatrist.

After one graduates with a basic medical degree, the training to become a psychiatrist lasts for several more years. There are usually pre-medical school studies of basic science or humanities that one takes before embarking on subjects such as anatomy, biochemistry, physiology and psychology. But, somehow, the positive factors relating to healing and restoring individuals to wholeness are not discussed. They may be implied but they are not specifically engaged. Factors that directly influence the work of a therapist are usually not mentioned, the two key ones being hope and compassion. Perhaps they are regarded as self-evident and therefore not in need of exploration but, by failing to focus on them, therapists are not guided to consider their importance or trained in how to put them into action.

The fundamental message of compassion exists in every religious tradition I have encountered. It is an essential practice for many Saints in the Christian tradition; it is one of the principal teachings of the Buddha; and it is the most used word in the Koran. Clearly, the importance of hope and compassion transcends sectarian differences. In the absence of religious traditions, most individuals express their common humanity through kindness and compassion.

It is kindness, the active component of compassion, the instills hope. Hope offers a path back to a sense of possibility in our lives when all, or almost all, seems lost. It's about relief and restoration. There is a Chinese proverb that says, ``Beyond the dark willows and bright flowers, there is another village.'' A western proverb says, ``A dark cloud has a silver-lining.'' These can give sustenance to us going forward, strength to continue putting one foot in front of the other. They communicate the opposite of despair, the opposite of a ``dead-end street.''

As a therapist, it is worth considering a few questions concerning hope: how important it is to instill or invoke hope it in your client; how does one engender and nourish hope; what might undermine hope in a patient; what does it feel like for you, as a therapist, to hope; and, crucially, what does it feel like when you, as a human being, lose hope -- even briefly. While everyone's answers are different, asking the questions are critical for one's own understanding of the role hope plays in your work and life, as well as specifically they might apply to individual patients.

We all should, or are presumed to learn, these positive attributes of hope and compassion though the love and nurturing we receive from our primary caregivers. Generally speaking, they are learned from our parents, or perhaps our teachers in kindergarten and/or Sunday Schools. But, this is less and less the case in modern life. For patients, those positive attributes may not be accessible following trauma -- particularly repeated early childhood trauma caused by primary caregivers.

All of therapy is built on a foundation of hope. Hope that things can change: habits, behaviours, emotions, outlooks, relationships and even people themselves.

For those who do not find inspiration from religious texts whether it be the Bible, the Bhagavad Gita, the Koran, the Buddhist sutras or others, let me point out that hope is associated with life itself. The organism knows best. Just as plants grow towards the light, the human organism intuitively knows a healing path back to well-being. A good therapist can point out the light to a patient, but part of therapy is getting to what is blocking this intuitive understanding. Perhaps it is our chaotic day-to-day struggles, perhaps it is confusion that is the result of early traumatic experiences.

To properly provide a therapeutic container, a place where the light can shine on a patient, the therapist must be clear about their own internal obstructions. Therapists are prone to depression and negative mind-sets, just as their patients may be. Many therapists, unconsciously, are drawn to the profession as a way to work out their own psychological issues. Some may simply become overwhelmed by the intensity of their patients' suffering. Others may survive by becoming inured to it.

A depressed therapist tends to be bogged down by the client's problems perhaps because they are wearing glasses with that same tint. A therapist may also become depressed as a result of vicarious re-traumatization though their empathic listening. Trauma-fatigue is common for the therapists who have neglected their own mental health in the past, and/or fail to maintain it under the stress of their profession. My training was primarily in British institutions, where professionals are expected to keep a stiff upper lip and maintain one's dignity as a professional regardless of any internal turmoil. The risks of vicarious trauma and trauma fatigue were never mentioned in my training.

Looking through the case files of successful suicides, I have come to the conclusion that the common element was that hope was missing. There was a failure by the therapist in that critical goal of instilling a sense of hope in the patient. Hope is the predicate to reversing the suicidal path. Sometimes the right medication, or even electro-convulsive treatment, was able to slow down and perhaps reverse the progress towards self-destruction, but not always and certainly not in a majority of the cases I reviewed. If a therapist is honest in their self-reflection, consider the possibility that one if one gives a subtle signal of giving up on the patient, that can and often will be seen by the patient as a message of ``permission'' to end their life.

To put this in a practical context, when a therapist faces a patient who is imminently suicidal, the first response is to determine, by knowing the patient's personal circumstances and/or through truly deep listening, how serious the risk is for that patient to act out on their suicidal ideation.

In my previous post on the \href{https://www.engagingmultiples.com/the-importance-of-hope/}{importance of hope}, I discussed briefly a particular patient who was suicidal. She was the last appointment in my day's schedule. I know that many therapists would decide that immediate hospitalization would have been the correct response to this situation, and in many cases, if not most, they might be correct. However, this was a DID patient and I did not see hospital admission as likely being helpful to reverse that decision.

Hospitals can be a negative experience for the patient, especially when the treatment team or the ward milieu is not suitable for DID patients. One must remember some mental health professionals do not even acknowledge DID as a legitimate diagnostic mental disorder regardless of its inclusion in the DSM. Hospitalization in this particular case would have meant a cop-out for me as the therapist as it would not address the actual triggering issue or the loss of hope. So, I decided the only way to approach this was to see if I could actively instill hope in her.

The key was that I took her words of hopelessness as a simple direct statement rather than a threat of any kind; empty or genuine. Her decision to end her life was averted once hope was instilled in her. I am confident that hope was what saved her.

\hypertarget{imagery-and-imagination-in-healing}{%
\section{Imagery and Imagination in Healing}\label{imagery-and-imagination-in-healing}}

\emph{Posted on March 26, 2016}

This is posted in response to a question from a reader. I think it is an important question with much relevance to the individuals with DID as well as others with trauma in their personal history. Again, as always, I am retired and cannot give therapeutic advice for individual cases. My thoughts on this and other topics are intended to be suggestions that are generally applicable and something to perhaps discuss with your therapists.

In therapy for healing past trauma, it is often suggested that one use ``imagery'' or ``imagination'' to create a safe place ``inside'' for the alters. These can be visual, auditory (hearing), tactile (touch), temperature and kinesthetic (sensations that inform us of our position in space). In my experience the most effective cues leading to relaxation are using the temperature, touch, and kinesthetic modalities. In most cases, the least effective is the visual modality. Generally, we do not need hearing in imagination because we can produce sound, such as raindrops falling on a rooftop, from a music player -- whether it comes from a record, CD or audiotape.

Instead of imagining oneself laying on the beach of a tropical island safely enjoying a protected holiday by utilizing visual cues of the white sand and the distant sails in the horizon, I would suggest that the person to imagine lying on the warm sandy beach (feeling the warmth on one's back), and feeling the heaviness in the limbs and the backside as one is lying down after a long swim. It is my view that in seeing, one places oneself in the position of an observer watching something happening to another person. In concentrating on the sensation of touch, you become the person who is experiencing it.

I often sought to fully utilize the kinesthetic sense in imagery to produce better result. One way is to imagine oneself lying on a mattress which made of a huge bag full of little balloons. Imagine that the balloons are full of helium which is lighter than air, so that it is gently lifting you up in the air. Then imagine your legs are gently bending, flexing and extending all while being supported by the mattress holding you up. In your expansive imagination, you now are capable of doing simple yoga postures in the air because you are lighter than air, floating in the air. When you are imaging that, it is pretty difficult to remain tense. Fully using your imagination, give yourself the magical power to do whatever acrobatics in the air that you wish.

One can consider that two different aspects to the sense of touch: external and internal. Externally, we have our touch through our skin. Internally, through our muscles and joints, we can tell if our legs are straight or bent, if we are bending backwards or curling forwards into a ball. We do not need our sight to tell us that.

So, I suggest that you fully utilize your sense of touch/temperature/kinesthics with imagery; whether it is imagining that your body is feeling cool in a hot day under the shade of a tree or floating in the air. Pay attention to the bodily sensations of your breathing, the feel of the air moving in and out of your nose, the rising and falling of your chest. Bringing in a sense of relaxation, and most important of all, a sense of being in a safe place, you can re-learn what is it like feeling safe and enjoying the present moment.

\hypertarget{do-young-alters-need-to-age}{%
\section{Do Young Alters Need to Age?}\label{do-young-alters-need-to-age}}

\emph{Posted on July 9, 2017}

A reader of my blog asked me a private question about whether or not alters age (or remain stuck mentally at their age) even though it appears that they are able to do some adult tasks. This is an interesting and important question that I did not address in either Volume 1 or Volume 2 of Engaging Multiple Personalities.

I did not address this question in those volumes because my experience in DID therapy was focused on treating the alters as they presented. My recommendation is always to address the issues that are being presented by the alter or alters that are presenting them. I did not, nor do I recommend, that therapist try to ``dig'' into the background of a DID patient. In other words, I did not treat each alter as an individual for in-depth psychotherapy. If an alter's problem was panic with hyper-vigilance, then that was the problem to be treated.

The age of an alter, like the color of a client's hair, is not a feature we need to focus on. There is no therapeutic advantage in seeking to convert an 4 year old alter into a mature woman of the system's chronological age, say 40, because the age of the alter is not the problem.

Given that I never sought to help an alter ``age'' or ``mature'', my thoughts on this question are somewhat speculative.

First and foremost, it is quite clear that alters arise as part of the dissociative process in order to allow the system to survive early childhood trauma. However an alter arises, it is tied to that particular trauma. I don't see why there would be any need for such an alter to age, given that is served and may continue to serve a protective function should the system perceive the same or similar trauma environment. I see every reason for the system to permit the alter to remain as they arose in order to have that mode of dissociative protection available if and when needed.

I do not say that the alters don't change. It is my experience is that they do change. However, for my patient's, the alters didn't change their age. What did change was their ability, with therapy, to remain ever more grounded in the present so as to more properly distinguish danger from the ordinary ups and downs one encounters in life.

In other words, the hyper-vigilance was tamped down. I did not encourage eliminating vigilance as there remain dangers both ordinary and trauma related. It is the hyper-vigilance that was interfering with their life.

Second, I think it is a conceit of those who do not have DID, that have a unitary ego structure, to think that the ``correct'' or ``healthy'' result of therapy is that the alters age to the system's chronological age. It would be far better for therapists to appreciate the brilliance of dissociation as a protective mechanism that arises in the fiery cauldron of early childhood trauma. Knowing its roots in that horrific early childhood trauma, one can have a much deeper appreciation for the strength of the system that enabled survival through dissociation and the consequent alters.

Third, my approach to therapy was always to encourage cooperation among the alters. I think this occurs to some extent all the time, but in times of stress, when unprocessed trauma simply erupts through the appearance of one or more alters, therapy is critical. In DID therapy, we seek to eliminate the internal conflicts that prevent such cooperation. Eliminating the conflicts allows for greater cooperation and a more clear experience of co-consciousness. This limits the hold that the past trauma has on one's present existence.

In fact, I often encouraged systems to designate alters within that could comfort each other when no therapist is available, to provide an empathetic ear to listen to frightened or angry alters, and to help communicate across amnestic barriers. As cooperation was enhanced, the systems were generally able to use that cooperation as a way to strengthen their ability to remain grounded and healed.''

As their lives become more peaceful (less roller coaster like, and less stormy) the system's need for different alters to violently seize control, as opposed to cooperatively working with each other, will diminish. We are definitely not asking them to disappear, but they seem to lose the need to insistently take charge. Instead, they will begin to behave in a non-disruptive way.

\hypertarget{progress-in-the-did-community-part-1-of-2}{%
\section{Progress in the DID Community -- Part 1 of 2}\label{progress-in-the-did-community-part-1-of-2}}

\emph{Posted on January 27, 2020}

Since the publication of Volume 1 of Engaging Multiple Personalities, followed by Volumes 2 and 3, many members of the DID community have written to me expressing appreciation for those books. They have said often, directly and in Facebook groups, that the material has been helpful to them as well as to members of their support network. In fact, many have brought copies of my books and blog posts to their therapists to help communicate their needs as a patient with DID. This feedback from the DID community allows me to continue to push forward to communicate the importance of correct diagnosis and correct therapeutic support.

I have even received some notes directly from therapists about how helpful the volumes have been in their own work with DID patients. That is the good, actually wonderful, news.

Almost 6 years have passed since Volume 1 of Engaging Multiple Personalities was published. Volume 1 reviewed patients identified as having experienced early childhood trauma and dissociation. Some of these had been treated successfully with psychotherapy as their antidepressants were simultaneously tapered off and discontinued. I tried to identify the reasons why some were treated successfully while others were not. From members of the DID community, there were expressions of relief both that their difficulties had a context and that healing was possible. A year later, Volume 2 was released which specifically focused on guidance for therapists.

Unfortunately, it seems that the psychiatric community still remains, for the most part, fundamentally unchanged in its view of DID. Copies I sent to colleagues failed to cause even a ripple in their consideration of DID and early childhood trauma. In my naivete, I expected them to be at least disturbed enough to re-examine their prejudice against DID diagnoses. I hoped to raise their index of suspicion when meeting patients with depression, self-harm and dissociative presentations to at least consider the possibility.

From colleagues, I got the uncomfortable feeling that Volume 1 in particular was treated as a book of curiosities. They were not so interested in the other Volumes either. Because my peers had not identified any such cases in their decades of practice, they ignored my suggestion that perhaps they had simply missed them.

Nevertheless, I was confident in this explanation. Why? Because in the many patient referrals I received, their files included notes identifying them as having dissociative tendencies and presentations \emph{without a primary or even secondary diagnosis of a dissociative disorder}.

I am confident that psychiatrists see many dissociative patients in their daily practice. They don't identify them as such because they are not expecting to see them. This is based on their own incorrect training mischaracterizing DID as extremely rare, Therefore, their index of suspicion is very low. Further, therapists are routinely distracted from the dissociative symptoms by their search for symptoms of depression. Why? It is because their index of suspicion is geared towards symptoms that will justify and support the prescribing of medication; i.e.~antidepressants.

It is of ongoing concern to me that psychiatrists and other therapists are so stuck in their habitual way of looking at patients that they are not able to raise their index of suspicion to include dissociative disorders, despite overwhelming evidence.

Many years ago, a friend told me that he took a course on how to identify edible wild mushrooms. As soon as he completed the course, he suddenly started to notice just how many wild mushrooms were all along his daily jogging path. If only that kind of change had happened in the psychiatric community after we published the book(s). We can still work toward that.

Is it worthwhile to repeat this \emph{ad nauseam}? I think yes. Why? Because of the response noted above from the DID community itself.

\hypertarget{progress-in-the-did-community-part-2-of-2}{%
\section{Progress in the DID Community -- Part 2 of 2}\label{progress-in-the-did-community-part-2-of-2}}

\emph{Posted on January 28, 2020}

In today's psychiatry, medication has become the de facto treatment plan. Many colleagues are no longer even pretending to do psychotherapy. They are being trained by representatives of the pharmaceutical industry to see all mental disorders as brain diseases. While there are definitely therapeutic uses for psycho-active medications, using them as the wholesale solution to all mental health issues gives modern psychiatry a false air of scientific credibility.

This is a disaster in the 21st century. There is massive early childhood trauma throughout the world. This includes violence and sexual trauma in war torn regions as well as in refugee camps, not just in so-called healthy societies. There is no treatment focused specifically on those children because of the overwhelming nature of warfare and its consequences.

Make no mistake about it, the trauma is there and will become a massive problem that will show up in a few decades for those children whether we acknowledge it now or not. This is on top of the ongoing early childhood trauma that arises in the absence of war but in the realm of our own somewhat hidden and somewhat exposed plague of abuse.

The foundations of psychiatry include Pierre Janet's classic papers on PTSD at the end of the 19th century. It also includes Freud's original assertion in 1895 that incest was the root cause of several of his patients' difficulties. That initial assertion was withdrawn by him following a withering attack from the medical community of the time. They were insulted at even the idea that professionals, men of wealth and power, or that men in general, would do such a thing. Although perhaps Viennese society was not ready to look at its own dark side, that initial assertion was likely quite correct.

The early leaders in psychiatry pulled back from identifying early childhood sexual trauma for what it was. We should not do that, nor should we countenance others doing that. DID is a specific consequence of early childhood PTSD. We can be honest about that. That is the path forward. We can also use the acknowledgment of DID as a special sub-classification of PTSD to move forward the conversation and treatment of DID.

The common understanding of PTSD in soldiers was acknowledged throughout human written history. Physicians characterized it as ``nostalgia'' as early as the 1600s, ``soldiers heart'' in the US Civil War, shell-shock in World War I, ``battle fatigue'' in World War II, and PTSD in the DSM-III. By the end of the Viet Nam War, PTSD was being seen correctly as not a failure of will or defect of personality, but a product of trauma.

It is this understanding of the wartime foundation of PTSD which is the key, in my opinion, to bringing awareness of DID to the professional community. They accept PTSD. We should use this acceptance to highlight and identify DID as the product of (early childhood) trauma which it is, just as (battlefield) trauma results in soldiers with PTSD.

If you are someone with early childhood trauma in your background, or speak with someone who does, you will know that the analogy of battlefield trauma is spot-on. Any child who is being or has been traumatized early in their life on an ongoing basis experiences life as a battlefield. They live surrounded by potentially overwhelming adversaries seeking to harm them again and again and again.

Please continue to use whatever of my books and blog posts you think will help educate your own therapists to help you on your personal journey of healing.

I continue to hope that the small contribution the books and blog posts have made to support those with DID will ultimately produce a sea change in psychiatry away from automatic pharmaceutical intervention. I hope that they lead to the return of proper psychotherapy for the benefit and protection of those that were abused as children who are trying to heal now as adults.

\hypertarget{the-therapeutic-alliance-part-1-our-fundamental-humanity}{%
\section{The Therapeutic Alliance -- Part 1: Our Fundamental Humanity}\label{the-therapeutic-alliance-part-1-our-fundamental-humanity}}

\emph{Posted on February 13, 2020}

When two people meet, be it just for a handshake or making a deeper connection, there is much more that takes place than what meets the eye. There is a kind of transaction that takes place, whether it is a mere acknowledgement or a meeting of kindred spirits. With this reference point, Eric Berne developed the concept and paradigm of transactional analysis. In that paradigm, all social engagements are seen as transactions between people in their parent-like, child-like, or adult-like ego states.

For me, the ideal state is when 2 people meet in what Martin Buber's referred to as an ``I-Thou'' relationship for a soul to soul encounter. I imagine that was the experience when Martin Luther King met Thích Nhất Hạnh or Father Thomas Merton. Although they were from very different backgrounds, they related as brothers, soul to soul. What they didn't bring into the experience was an ego defence barrier. They were completely open to one another without asserting or subsuming their background in the engagement.

Well, we are not all able to do that but we can see that as a most positive aspiration. But, let us find out what we ordinary human beings can do, as therapists and clients in a healing relationship.

I cite the religious and philosophical thinkers above because in being practitioners of science based medicine, we somehow often forget we have souls. In that forgetting, we turn away from our fundamental humanity. But, that fundamental humanity matters most when we are dealing with crisis, trauma and healing. It is the key to a successful therapeutic alliance between therapist and patient.

In looking at books regarding different approaches used in psychotherapy, I find that little time is spent discussing the very basic fundamentals of psychotherapy. Perhaps they are so self-evident that authors and teachers assume it is unnecessary to restate them. I disagree. It is always necessary to return to and remind ourselves of the foundation of therapy, empathy, which is integral to our humanity.

I respectfully request that you not let my use of a word like soul, or to speak of religious thinkers, discourage you. I use these words for that which I am unable to communicate using conventional language. Most of our speech refers to concrete things such as a chair or a pound of butter. While such materialistic terms are necessary to our everyday existence, they are inadequate in communicating the wider sense of our experience.

When we use words like compassion, empathy, and understanding, we cannot use materialistic terms. You cannot scientifically measure the love you have with your significant other. You cannot scientifically measure the pain you experience when someone close to you passes away. So, we start to use words differently. For anything beyond the materialistic world, we need to use poetic language and often metaphors.

So please allow me to use words like love, humanity and God in that way. The way I use the word God, God is not an object I can pinpoint or describe in any literal way. For example, I cannot say, ``God is here, not there.'' I cannot place God in a location. As with love, God has no color, no size or weight. What I reference as God is not confined by time and space. It is in that same context that I speak about ``soul'' as in ``meeting of souls.''

Carl Rogers emphasized the critical nature of a person-to-person relationship between therapist and client/patient. That is the environment which can provide the patient with genuineness (openness), acceptance (being seen with unconditional positive regard), and empathy (being listened to and understood). To me, this is the necessary and basic requirement for the foundation of a proper therapeutic alliance, from which healing is possible.

Time and time again we are shocked when we encounter practitioners of healing professions that are lacking in such qualities. I mentioned the meeting of souls as the highest ideal. It is one that we seldom achieve. But, that is the lodestone we aspire to in seeking to create the person to person therapeutic alliance so critical to healing.

These days, patients need to make a great deal of effort to see a therapist. One must spend the time and energy getting a primary referral from one's doctor. And then, one must phone for an appointment and often experience an automated recorded voice saying something like:

``Please hold\ldots{}\ldots{}If you wish to speak in English, press 1. If you are a new patient calling for an appointment, press 2, if this is \ldots{}\ldots{}. press 3. If \ldots{}\ldots{}\ldots{}\ldots{}..press 4.'' After days, weeks and sometimes months on the waiting list, you finally arrive at the psychiatrist's office. Then when you see the psychiatrist, all he does is acknowledge that you are not doing so well. It can be along the lines of: ``You are depressed. You need to take this medication. It usually takes up to a few weeks to work, so be patient. We will start with this dose and then see from there.'' Does it sound like a therapeutic relationship, a space in which healing will take place?

All the years of learning in university are irrelevant unless we keep in mind that we absolutely must connect with a patient. Without that connection, we will be unable to help them work with their trauma. Expertise can make a difference to the outcome but if, and only if, it is leavened with empathy, with compassion, and with openness.

That is why I believe the most fundamental issue is to create a field for healing to take place. It is like sowing seeds, some fall on to rocks, and others to concrete ground. Only those seeds falling onto suitable ground, with the right amount of moisture, the right kind of soil mix, and the right amount of sunlight that will enable the seed to mature into a plant. This suitable ground for healing, this fertile milieu, is what we call a \emph{therapeutic alliance}.

It is the foundation of a relationship between a healthcare professional and a client (or patient), hoping that their engagement will effect beneficial change in the client. This relationship is the milieu, the soil from which will facilitate the sprouting of a seed that will grow into a healthy plant reaching down into the earth for nourishment and up toward the sun to flower.

\hypertarget{the-therapeutic-alliance-part-2-genuineness-acceptance-and-empathy}{%
\section{The Therapeutic Alliance -- Part 2: Genuineness, Acceptance, And Empathy}\label{the-therapeutic-alliance-part-2-genuineness-acceptance-and-empathy}}

\emph{Posted on February 13, 2020}

With a proper therapeutic alliance, the emphasis of the therapist on one side and the client on the other changes. It is shifted to the medium in which the communication occurs. It focuses not so much on what the therapist can do, but whether a milieu has been nurtured such that the client feels safe and trusting enough to take the hand offered, as it were, to get out of a difficult situation.

As therapists, we must bear in mind that we are not the only ones doing the assessment in an assessment interview. Your patient, after waiting for a few months to see you, has more invested in this venture than you. They are assessing whether they can trust you enough to share their innermost vulnerabilities, their most private concerns. If you don't establish the ground for that therapeutic alliance, if you have not engendered the feeling of safety, space and time to open up, then it will not happen.

It is in this context that the effectiveness of CBT (cognitive behavioural therapy,) EMDR (Eye movement desensitization and reprocessing and DBT (dialectical behavioral therapy), all recommended therapeutic approaches for cases of trauma, dissociation, and borderline personality disorder, must be evaluated. Without considering the importance of a therapeutic alliance, it is misleading to say that CBT, EMDR, DBT or any other therapeutic model are the treatments of choice.

When a psychiatrist offers you an antidepressant pill for your depression, in the absence of the correct therapeutic alliance -- even if it is an appropriate prescription -- you will remain locked in your belief system. That will counteract the psycho-active effect of the drug. It is noteworthy that most people understand the placebo effect; the beneficial effect that cannot be attributed to the pharmaceutical properties of the drug itself and must therefore be due to the patient's belief in that treatment. People are less aware of the nocebo effect, which is the opposite. The nocebo effect is that the patient's disbelief in the treatment lowers the positive pharmaceutical impact of that drug.

Therapists take heed: One's therapeutic effectiveness is directed related to quality of the therapeutic alliance we create in each and every one-to-one therapeutic session. It depends on how you say ``hello'' or even months prior to that, when and how the appointment was made.

\emph{The genuineness of a therapeutic alliance often explains how a history of early trauma is sometimes given to one therapist in the first visit, while other psychiatrists may have spent years with that same patient and still missed it.}

This is the mechanism by which some world known professors and heads of major universities as, well as the chief editor of a major national journal of psychiatry, erroneously declare that dissociative identity disorder is non-existent, is a fake disorder or created by over-enthusiastic therapists. They assert it is impossible because they never have encountered one such case. In fact, the odds are that they simply failed to recognize those cases. Instead, they decided upon common misdiagnoses such as treatment-resistant depression (which should usually be more correctly identified as drug resistant depression), bipolar disorder(s) and borderline personality disorder.

With self-reflective insight, they might come to understand that they have never given the time, space and safety for their patients to show them their innermost pain and suffering. The fact is that when a patient feels safe enough during therapy, spontaneous catharsis happens without asking. Our duty as therapists in such an event is to protect the patient from inadvertent re-traumatization throughout the cathartic process.

Focus on the process of healing, not the detail of the trauma. As always, the right therapeutic alliance guides the therapist to be sensitive to the need of the patient.

Being a brilliant scientist is of no use if one forgets that one is basically human. Religion is not about arguing whether or not God exists, whether or not God has this or that quality. The men I have quoted in Part 1 of this post are all from different religious backgrounds. Ultimately, it is about being reminded that we are human.

I offer and will repeat to offer the following guidance of Carl Rogers' emphasis of a person-to-person relationship between the therapist and their client, one that is characterized by genuineness, acceptance, and empathy. That emphasis is worth more than any diploma on your office wall.

\hypertarget{using-a-card-for-communication}{%
\section{Using a Card for Communication}\label{using-a-card-for-communication}}

\emph{Posted on February 25, 2020}

I received an inquiry from a DID FB group participant asking if I could suggest a card that might be carried by someone with DID. The idea is that it could be used to explain what they needed when dealing with a difficult public situation, like waiting too long in a doctor's crowded office. In short, something that could be used in that kind of situation to let whoever you are dealing with know what you need without having to explain in detail.

The analogy that came to mind was the cards some deaf individuals use to alert people that they are deaf and so lip read or use sign language to communicate. For those who are deaf, it is something along the lines of ``I speak in sign language. If you don't, in order to help with communication, I also read lips so please look directly at me and speak normally.'' This alerts others that there is an issue in communication for which there is a simple clear solution.

Here, we are talking about a communication card to do the same thing for those with trauma issues in public situations. The card language I suggest below identifies the issue, which is anxiety and panic, and the solution. It is \textbf{not} necessary to identify oneself as having DID or other dissociative disorders. (I have cautioned in other blog posts concerning the risks involved in that.) In any event, your DID diagnosis is more information than is needed in most ordinary interactions, like at a doctor's office or for a meeting at a government agency administration office.

Perhaps something like:
\textbf{\emph{``I have a problem with anxiety and panic. It can be triggered by being in a crowded or enclosed space as well as having to wait for appointments too long, even in a comfortable waiting room. If I have to wait in this room for longer than 15 minutes, it will be difficult for me. It is easier for me to slowly walk around the block while waiting. I will not be more than 10 minutes away. Please call me on my cellphone: \_\_\_\_\_\_\_\_\_\_\_\_ with a 15 minute warning and I will return immediately. Thank you for your understanding.''}}

Keeping it short (this will fit on a business card) and simple, avoids the need for detailed explanations. Most questions in those kinds of social situations begin with asking for identification information. I would not have that information on this card -- just your cellphone. I suggest you have your driver's license and Social security information separately ready to hand to a receptionist, for example, as needed. This again limits the need for you to speak if you are worried about being triggered in that environment.

Keep your verbal responses to a simple yes or no. Perhaps have a pad and pen if it is easier to write a short answer rather than speak out loud. This is the kind of accommodation that is made for many difficulties. Once you identify yourself as having an anxiety problem, I think it unlikely that people will suddenly conclude that you have DID and proceed based on their confused understanding of dissociative disorders.

The general public is well aware of anxiety issues and the idea of a panic attack. This may be a way to meet them where they are comfortable, in their understanding of anxiety, so that they can help you feel safe navigating the situation.

I never thought to suggest this to my DID patients when I was practicing psychiatry. In retrospect, I likely would have suggested it as something to try. I am happy to say that I continue to learn from the DID community. I hope this is helpful. If other members of the DID community have further suggestions, or perhaps better language, please do share that.

\hypertarget{the-importance-of-grounding}{%
\chapter{The Importance of Grounding}\label{the-importance-of-grounding}}

\hypertarget{grounding-exercises-and-working-with-flashbacks}{%
\section{Grounding Exercises and Working with Flashbacks}\label{grounding-exercises-and-working-with-flashbacks}}

\emph{Posted on February 9, 2015}

\textbf{All my postings are general comments, and must not be construed as treatment instructions for any specific individual. Should you choose to consider any of my suggestions, you must first consult your own physicians before exploring them.}

I do not find hypnosis useful while the client is having a panic attack or a flashback. However, hypnotic trance practice can be useful as a grounding exercise.

If you have had an experience of being in a hypnotic trance, you can easily go back to a trance by just remembering what it was like, how you felt, the last time you were in a trance. In other words, there is some of truth in the saying that ``all hypnosis is self-hypnosis.''

People who have severe PTSD or Panic Attacks all have forgotten what it feels like to feel comfort in their body. It is good to remind them so that they can set that up as a goal in their practice.

\textbf{Grounding Exercise:} Do it while you are \textbf{not} having a panic attack or a flashback. The more you practise grounding, the better you are equipped for coping with the next flashback. One useful self-hypnotic practice is choose a private place, such as in your room. After locking your door sit down, close your eyes, and imaging walking down a flight of stairs. Slowly and with each breath, go one step down. At the end of ten steps, you will find a door, through which you can enter a state of deep relaxation. You can imagine, for example, being suspended in mid-air resting on air-cushions. If you wish to be more relaxed, go down another flight of stairs. The most important thing to remember is to take your time, walk down slowly, with each exhalation. \textbf{GO SLOW!} Once there, stay and enjoy the peace and quietness.

\textbf{Flashbacks:} One aspect of DID is the PTSD suffered by some of the alters. PTSD is similar to Panic Attacks in that once turned on, the anxiety is fed into a vicious cycle. It gets energized and stuck in a closed feedback loop. Forget trying to relax, it does not work to try to force relaxation. To get out of the cycle, one has to take over the control of one's body. For example, start doing 10 push-ups. What happens when you do 10 push-ups? Your heart rate probably goes up beyond the heart rate produced by you panic. If not, you do another 10. Then sit down and notice your body settling down to a comfortable state. \textbf{It will on its own.}

If you have had an experience of a hypnotic trance, it will make it easier if you remember what it feels like to be in a comfortable and relaxed situation. Notice I don't say try to relax, but just notice the \emph{inevitable} relaxation after 10 push-ups.
Alternatively, instead of doing 10 push-ups, you can go have a shower. You can have a hot shower, then turn off the hot water and let the cold water shock you a little. In Finland people go into a hot sauna and run out to roll in the snow, so why not?

Instead of a shower, you can go out and walk briskly or run around the block. Take the initiative to find a way that is uniquely helpful to you. The natural state of calmness and ease that results from these grounding exercises will show you that you are master of your body. This mastery helps you to overcome the flashbacks and panic attacks.

\hypertarget{mindfulness-meditation-and-did}{%
\section{Mindfulness Meditation and DID}\label{mindfulness-meditation-and-did}}

\emph{Posted on October 6, 2015}

There was a posting on a DID facebook group that expressed some real difficulties with a mindfulness meditation based therapy the DID individual was trying. This individual was not alone in having difficulties as a DID person trying to do mindfulness based therapy. I discussed the issue with a friend of mine who is also a long time Buddhist meditation instructor. He did not want to criticize the facilitator of the group because no doubt they were trying to be helpful and hopefully were for most people. However, he said quite definitely that if you are afraid to close your eyes, then don't. It is not necessary and usually not advisable to do so anyway when practicing mindfulness. His point was that if you are trying to be ``here'' mindfully, then why would you close your eyes or imagine a stream? The practice is to just be where you are.

He suggested being very simple about it. A traditional technique is to start with a good posture (a straight back), comfortable sitting position, relax your jaw, eyes looking gently ahead angled slightly down and so that your gaze is falling to the ground about 6-8 feet in front of you. Allow yourself to settle and then simply count your outbreaths up to 10. Don't try to manipulate your breathing, just go with how it is happening. If you lose count, just start again with 1. Do not criticize yourself if you lose count, do not praise yourself if you get all the way to 10. Either way, it is no big deal. If you get to 10, start again with 1. Do it for a short time, especially when beginning to become familiar with mindfulness. Even just 1-2 minutes is good or you can try just 5 or 10 breaths, however long that may take. If you can do it even just a little each day, that is great.

For someone with DID, it is critical to experience feeling safe, so don't do anything that is going to frighten you or any parts -- such as closing your eyes (or scanning your body which is a technique in some mindfulness therapies) if that is a problem. Try just sitting and counting outbreaths in a safe physical space of your own choosing. The first experience of most people starting mindfulness practice is that they become aware of just how many thoughts they really have. This is because there is so much more space for thoughts to appear when you are quieter than usual and not focusing so much on external tasks.

But for someone with DID, those many thoughts can be quite scary. Individual parts may see that open quiet space as their chance to be out and carrying all their traumatic memories. The thoughts may be coming from many different alters so quickly it seems that they are all happening at the same time. With that intensity of traumatic memory and seeming chaos, it is not surprising that dissociation would occur right away. So, DID individuals must go very carefully with mindfulness meditation so that the open space doesn't trigger the fears of all the parts at the same time and result in retraumatization instead of healing.

But, if you can do it for only a few minutes or even just a few breaths, that starts you on the road to having confidence that you can indeed feel safe -- even if just for 1, 2 or 3 breaths at a time!

Experiencing safety starts with that one first breath. Make a decision before you start about how long you will do the counting. Try to do it for that long but once you reach your goal for the session, gently stop. That way you start to get the habit of being able to create a time-defined safe space which is a great habit to engender.

If and when you become more comfortable with the practice of mindfulness, you can increase it by just one or two more minutes or a just few more breaths. If you dissociate, no problem. When you recognize that you have dissociated, just go back to counting breaths without praise or blame directed to you or any alter. Encourage whoever is out during the dissociation to please try to continue to do the counting of the breaths while they are out. If they will do so, great. If not, don't worry. You can always gently (always gently) invite them next time. You can express that encouragement to the parts before you start, so they are acknowledged and even a bit prepared.

Slowly, there will likely be some benefit to the host, to the alters that participate, and also to those that watch without participating. Even a small benefit will encourage other alters to start to watch, maybe even participate, and to share that taste of safety in the breath. In fact, inviting the ones that appear when you dissociate is a very kind way to empower those alters, to show them that they too can be mindful of the ``here and now'' also -- safely and without struggle.

Remember, keep it short -- especially at the beginning -- and always safe . Later, if the practice is helpful, keep it safe and extend it for just a bit longer. The critical point is experiencing safety in the here and now.

\hypertarget{breathing}{%
\section{Breathing}\label{breathing}}

\emph{Posted on November 24, 2016}

It seems a bit silly to tell someone how to breathe. After all, everyone breathes. But,''take a deep breath'' is something people have been saying to others for generations when trying to help them calm down. What is the connection between breathing and healing, or potential healing in connection with DID?

Therapists have clued in on yoga and meditation as having some benefits in the psychological healing and restoration process. Many therapists encourage patients to engage in breathing exercises and ``controlled'' breathing. They usually connect this to mindfulness; meditation stripped of its spiritual/religious context. Previously, \href{https://www.engagingmultiples.com/mindfulness-meditation-and-did/}{I posted on the risks of mindfulness meditation with DID}. In my experience with patients, the benefits are not so automatic as some people say. And, there are risks.

Breathing is governed by the brain's respiratory center. While you can control your breathing as a voluntary act, your brain will continue to instruct your body to breathe without you having to continually give conscious instructions to do so. Beyond that, even though one can willfully hold one's breath, the eventual lack of oxygen and accumulation of carbon dioxide in the lungs causes the brain's respiratory center to overwhelm your will: You breathe again whether you want to or not. The point here is that breathing is governed both involuntarily as well as voluntarily.

Breathing patterns change when a person is under stress. A person under stress, any person, is likely to automatically to hold his breath or to breathe laboriously. In a genuine panic attack, one often feels as if one has to catch their breath, or as if one cannot breathe at all. A person in a panic usually takes very shallow and very fast breaths. This comes back to the common message we all get -- to ``take a deep breath'' in order to calm down.

In my opinion, ``take a deep breath'' is the wrong phrase to use. I preferred to tell my patients to slow down their breathing. Why not tell them to take a deep breath? There are two reasons. First, I want to avoid encouraging already panicking patients to continue with their rapid breathing, just try to grab more air as you pant.. That doesn't address the panic -- the patient continues to panic. As a therapist, you are then encouraging them to experience their panic attack like a drowning person trying to suck in all the air they can. Second, and most important as a therapeutic intervention, panic attacks are about experiencing a loss of control. Fear is coupled with that sensation of the loss of control. Therapeutic intervention needs to address the physiological connection between breathing, fear and the loss of control.

It is my experience that enabling a patient to reclaim control over their physiological response to fear gives them a tool for coming back to the present moment. It is often a very effective tool that opens a door to escaping the entrapment of the triggered physiological panic response. Go back to the fact that breathing is a blend of voluntary and involuntary control by the brain: It is within your power to slow down your breathing. One can start by slowing the breathing down just a little bit. With practice, a patient can utilize their voluntary control of breathing in a gentle way to slowly bring themselves back to a place of psychological safety. That is enabling a patient to connect with and rely upon their own fundamental strength. That is empowerment -- the opposite of loss of control.

Slowing down the breathing, rather than fighting the gasping for air head-on, allows the brain to blend back the voluntary control back into the involuntary panic driven breathing. Going from 100\% involuntary control of panicked breathing to 98\% involuntary control and 2\% voluntary control can happen without inducing further panic responses. Then one can slowly increase the percentage of voluntary control, perhaps to 5\% voluntary control and then maybe 10\%, and so on. So, for me and my patients, the key phrase was ``slow down the breath.''

Once the patient is willing to try to slow down the breathing, even asserting only 2\% voluntary control, they then have the direct experience of being able to assert some level of control over their breathing. That small level of control is the experience of having power over the impact of the past trauma -- even just that little bit. The patient discovers the fact that it is possible, through one's own breathing, to regain control over their body -- which is taking control away from the past trauma.

Slow breathing is always associated with a sense of ``equanimity and tranquility.'' In slowly breathing out, one activates the parasympathetic nervous system and engenders in the body a trophotropic state -- a state where the body rests and recovers its energy. It is a physical sensation that enables the distressed person to discover such feelings in the midst of chaos and fear. This is the way to redirect one's attention from the impact of past outside trauma to the genuine sensation of inner well-being.

Phrases like ``take a deep breath'' or '' controlled breathing'' are action-oriented. I choose to use more laid-back expressions that suggest lack of confrontation, expressions that call to mind receptivity and awareness. Encouraging a patient to slow down their breathing a little bit at a time lacks any harsh quality of an external command by the therapist. It remains as it should, a suggestion that we can tap into our strength safely. Putting it simply, all is not lost when you are still breathing.

In recent years, there is strong scientific evidence for the benefits of mindful breathing. Mindful breathing is a spiritual practice thousands of years old that is used in many religious traditions. But traditional instructions on mindful breathing are not about control, they are about letting go of the thoughts that tend to take one away from the present moment. The instructions are, effectively, to ride the breath, in and out, as the vehicle to do that letting go.

In the West , as Allen Watts pointed out, it is difficult to understand the concept of ``being'' as distinct from ``doing.'' Broadly speaking, in the Judeo-Christian Western world it is uncomfortable to ``just be.'' It seems that one has to be doing something at all times. The story used to explain this involves a group of villagers debating what a man, a distant figure far away, is doing. When they approached the man, asking him exactly that. He replied, '' I was just standing here.'' That's it. One does not have to be doing something in the active sense to justify one's existence. One can ``just be.'' Arriving in a new place, one can just absorb the experience of sound, sight, smell and taste of the land he is visiting.

While numerous neurological studies have concluded that prolonged practice of meditation can actually change brain structures and alter its way of reacting to stress, DID individuals must approach it slowly, gently and with protections in place. Again, in my experience, start by slowing down the breath when you panic. Practice that slow control mechanism until it becomes a habit. In that way, you are always enhancing your ability to protect your connection to safety in the present moment.

\hypertarget{the-meaning-of-forgiveness-part-1}{%
\section{The Meaning of Forgiveness -- Part 1}\label{the-meaning-of-forgiveness-part-1}}

\emph{Posted on February 25, 2018}

During the past few months, the question of forgiveness has repeatedly come to mind. I think this is somewhat both a haunting and daunting topic that usually arises at some stage of everyone's healing process.

In the context of DID, the question is why should anyone even consider forgiving the person who abused them? It is not sufficient to do so just because one is told to believe that it is the right thing to do. Even if one is told to forgive as a matter of religious doctrine, one still needs to understand the connection between the doctrine and the forgiving of such a crime.

It is often instructive to understand the origin of a word that is used so often and that can be so loaded. In the context of Christianity, the Greek word translated as ``forgiveness'' in the King James Bible literally means ``to let go,'' as when a person foregoes demanding payment of a debt. In his parable of the unmerciful slave, Jesus equated forgiveness with canceling a debt. (Matthew 18:23-35.)

The word translated as forgiveness is used to convey the state of mind we have when we let go of resentment, for when we give up any claim to be compensated for the hurt or loss we have suffered. But forgiveness of debt doesn't mean the debt never existed. It doesn't mean you have to loan more money to the debtor.

We must be honest with ourselves and with others: There is a big difference in letting go of a debt of few dollars as compared to letting go of the pain and anger connected with the abusive perpetrator of our early childhood trauma. If we lose some money because someone has failed to repay us, or someone has dealt with us in not such a good way, we can usually figure out how to proceed the next day or the rest of life because our core being has not been ruptured and split apart. An abuser has adversely affected one's entire adult life, and we cannot simply go about our business. There is no way to give a clean slate back to the child who has been psychologically pulled apart by trauma. There is no do-over.

To put it in another way, if one's whole life is ruined because of an early abusive relationship with the perpetrator, it is a different story than simply forgiving a debt of money or a minor inconvenience that one has the capacity to simply ``let go.''

Forgiveness also means pardon. Is it possible to pardon a perpetrator if the perpetrator does not even own up to the damage he has caused? Does forgiveness means somewhat condoning the evil act and/or allowing it to continue, possibly hurting future victims? What are the options for forgiveness?

First, one must be honest. Letting go does \textbf{\emph{not}} mean denying the damage that has been done. Letting go does \textbf{\emph{not}} mean nothing ever happened. Letting go does \textbf{\emph{not}} mean condoning the evil act.

Second, one must protect oneself. Most early childhood abuse is based on the most fundamental betrayals imaginable. Letting go does \textbf{\emph{not}} mean allowing a perpetrator to ever get close enough to harm you again.

Third is perhaps the most difficult. If you let go of the pain and anger, you might be able to understand that most people who abuse others were themselves abused. This does \textbf{\emph{not}} in any way shape or form undermine the critical important of the second point about protecting yourself. It simply means that you can understand the abuser was or is in pain, is confused, and is likely driven by their own trauma.

Again, that does \textbf{\emph{not}} mean you let them anywhere near you ever. It does \textbf{\emph{not}} mean you let your child or other children anywhere near them. Instead, it means that you can let go enough to wish that they are able to process their own pain and trauma. Letting to in that way is forgiveness enough so long as you remember that it is \textbf{\emph{not}} your obligation to help them process anything. It is their obligation, and it is theirs alone.

In this way, you can be very clear why you might forgive them, while at the same time remaining absolutely firm that your letting go does \textbf{\emph{not}} permit them to come anywhere near you, ever. If they wish to make amends, they can turn themselves in and confess to the authorities. If they wish to do something beneficial in penance for their evil deeds, they can anonomously donate all of their money to a charity devoted solely to protecting children from abuse. Why anonomously? Because that prevents them from evey being seen as an angelic benefactor for abused children. Whatever they may choose to do, or not to do, is their choice, their problem, their concern.

\emph{The original meaning of forgiveness requires nothing from you other than letting go of what you hold onto.}

It is a dangerously false assertion, religious or otherwise, to presume that forgiveness means giving someone a clean slate, to presume that it demands you ever share space with an abuser. You have nothing to prove to anyone about your forgiveness. Please be extremely clear and firm about that. Forgiveness is solely about your letting go, \textbf{\emph{not}} what happens to or with anyone else.

In short, forgiveness does \textbf{\emph{not}} mean forgetting what has happened and pretending that from now on, one can have a ``real'' relationship with the perpetrator, as if nothing pathologically evil had ever happened. It does \textbf{\emph{not}} mean that with forgiveness, one can ``be friends'' with the perpetrator. It does \textbf{\emph{not}} mean, in the case of incest, that one can have a normal father-daughter or sibling relationship with the abuser. Such things are \textbf{\emph{not}} possible. To hold them out as a goal to strive for will prevent healing rather than foster it.

\hypertarget{the-meaning-of-forgiveness-part-2}{%
\section{The Meaning of Forgiveness -- Part 2}\label{the-meaning-of-forgiveness-part-2}}

\emph{Posted on February 25, 2018}

The reason to consider forgiveness in the way described in Part 1 of this 2 part post, to consider letting go, is that non-forgiveness carries its own deep penalties. Intense and completely appropriate deep resentment, the deep sense of betrayal, and the other conflicted emotions that all go along with those are harmful to your own well-being, both spiritually and physically. But, do not ever forget that it was those same intense emotions that saved you as a young child.

Those same intense emotions may manifest as alters that you have difficulties with because of their intensity. By engaging those alters and acknowledging the truth of both your pain and their protective intentions, you can transform the intensity from conflict with alters to mutual cooperative support. In that way, you can forgive but not forget. In that way you honor those alters and your own survival. In that way, you protect yourself from falling into the trap of mistaken conventional understandings of forgiveness.

Persistent anger and resentment, feeling oppressed and being hyper-vigilant are mental states that are harmful to the those who do not forgive in a safe and protective way. It tinges their way of seeing the world. They are quick to look for, project out and only see the faults of others. They color their direct perceptions and adopt a negative way of seeing the world around around them. They are likely to miss the birds singing or the sun shining. They miss all the good stuff of being alive.

Their hyper-vigilance makes them paranoid and mistrustful. They handicap themselves and put up roadblocks to all potentially healthy relationships. In the extreme cases, they are chronically depressed, often drowning themselves with chemical addictions. They miss out on so many of the good things in life. So work on dialing down the hyper-vigilance. Let go into ordinary appropriate vigilance. It is safer and respectful to your protectors. They are still and will always be needed.

Physically, failing to let go results in chronically raised levels of cortisol, the so-called ``stress hormone.'' Scientists have clearly determined that elevated cortisol levels interfere with learning and \textbf{memory}, lower immune function and bone density, increase weight gain, blood pressure, cholesterol, heart disease. Letting go allows us to care enough about our bodies to get rid of negative hormones circulating in our system.

Again, we must be honest. Healing is a journey, a path. It is difficult to let go and forgive. One must pay attention to the part (whether it is a fragment of the person that appears momentarily, or an alter with the capacity for ongoing executive function) who is too hurt to let go of that anger and pain. One has to pay attention to these parts. One cannot just rush in and tell a part to forget the past and move on, so to speak. If you are a DID, ask who cannot or is unwilling to forgive, then gently allow that part to process getting over the negative experience he/she is stuck with. It will help to reinforce that the goal in forgiving does not, absolutely does not, include forgetting. It does not, absolutely does not, include allowing a perpetrator close once again.

Notice how much hurt the ``alter'' is still feeling. Work on consoling that part. If you are a partner of the one with DID, you can still work on a part that is unwilling and unable to let go of the hurt. Treat that part as deeply real as that part perceives itself to be. Work on consoling it to allow healing from the wound.

It takes time, but the goal is eventually arriving at the stage that you will be safely released from the negative emotions of anger and bitterness. Learn to be kind to yourself, or that specific part of yourself. Then you can truly be free.

\hypertarget{comments-on-depression-and-integration-from-healing-together-conference-presentation}{%
\section{Comments on Depression and Integration from Healing Together Conference presentation}\label{comments-on-depression-and-integration-from-healing-together-conference-presentation}}

\emph{Posted on February 6, 2015}

I just returned from a Conference on DID in Orlando Florida (January 30th to Feb 1) and posted on Facebook about how positive and supportive the conference was for all participants, DID, therapists, supporters and speakers. This was a great meeting, well organized by the group called ``An Infinite Mind.'' The conference is called ``Healing Together.''

The Keynote speaker was Robert Oxnam, author of A Fractured Mind, who gave a most inspiring and affirming talk on the positive aspects of DID. I was honored to be one of the speakers in a breakout session in the afternoon.

This post is taken from my presentation at the conference and is drawn primarily from Volume 1 of Engaging Multiple Personalities.

In chapter 5, I discuss my patient Ruth. She was experiencing unrelenting flashbacks, and self-destructive behavior, so much so that she was hospitalized 20 times by the time she was 28. Her children were taken away by her family in preparation for adoption out. She was diagnosed with a single diagnosis of Depression, plus a personality disorder. The treatment she had received, exclusively pharmaceuticals and electro-convulsive treatment, failed to alleviate her depression.

Under the circumstances, any reasonable person should understand that her depression was a normal and appropriate response to the reality of her circumstances. With the correct diagnosis of DID and appropriate treatment, the ``depression'' quickly disappeared.

After 2 and half years of intense psychotherapy, she was fully recovered and fully functional, without need to have further therapy or medications. Following up19 years later, upon receiving correspondence from her, showed her to be fully recovered. \textbf{The depression was both a misdiagnosis and a smoke screen. It was covering up the DID which the doctors never saw or even suspected.}

Ruth had brought up her children and was living a highly functional and creative life. She is engaged in helping other survivors of traumatized individuals through running a website as well as writing and publishing a book for survivors of abuse.

What can we learn from this clinical example? \textbf{Depression is too often mistaken as a stand-alone disorder, when the doctor will too quickly reach out for an antidepressant and ignore the core issue. Depression can be a normal and even healthy emotional response to life's circumstances, or as a co-morbid condition with another psychiatric illness.}

Further, the fact that she had numerous alters, over 400 even after her recovery, means that a therapist should not be obsessed in pushing for integration as the final goal. Really, so long as the alters are cooperating with each other and not fighting, there is no real problem.

While some people may scoff at anyone having so many alters, when I reviewed her old letters from my file with the card signed by an enormous number of her alters, the handwriting of each alter that signed the card matched with that alter's handwriting 19 years earlier.

I hope this message can help therapists not be blinded by the word depression, or be obsessed with the notion of integration. The fields of past trauma and dissociation, and the DID patients in particular, are waiting for the current generation of therapists to step in to help those suffering from something that is more than depression. Groups such as An Infinite Mind and Ivory Garden (that put on another incredibly supportive conference for DID survivors, therapists and supports on the West Coast) are doing a great job in supporting individuals with DID. Their work extends to the important task of raising the public's awareness of the need for correct diagnosis and treatment.

I shall post something from my talk and my books on DID on this website from time to time, hoping to continue my effort to improve the well-being of those individuals with DID, for those who are supporters of DID individuals, and for therapists.

\hypertarget{avoid-retraumatization}{%
\section{Avoid Retraumatization}\label{avoid-retraumatization}}

\emph{Posted on February 24, 2015}

The sad and terrible truth is that people prefer to simply ignore the depth of horror that the abuse of children entails. They find it easier to dismiss stories, memories, and writings than to look directly at the evil of child abuse and confront it. It makes hiding the abuse easy for the perpetrator and places an overwhelming burden on children to deal with it, often decades later. People who have survived traumatic abuse know all too well the truth of such evil. The evil is compounded when, as an adult, they continue to have their traumatic history denigrated and dismissed.

PTSD and complex PTSD always involves the loss of control, whether it be in a courtroom at the hands of a defense attorney on cross examination or at the hands of a celebrity ``therapist'' seeking ratings and phony closure within the 45 minute segment they are promoting. If asked by a patient about the advisability of suing an abuser, I always cautioned about the risk of retraumatization. While it was always the patient's decision, when asked, I always urged careful consideration of this risk. Then, I supported whatever choice they made.

Re-learning the experience of safety is not easy when safety is taken from you at an early age, but relearning it is the key to healing. Therapy is often and correctly focused in many ways on protecting yourself from retraumatization so that you can genuinely experience being safe in the present moment. You are not obligated to relate to abusers, past or present, whether they be family members or celebrities, regardless of their demands and promises. They will likely have their own agenda, which is not necessarily your healing. Your healing must be the priority, and remember that you are \emph{never} obligated to retraumatize yourself for anyone's entertainment.

\hypertarget{the-trap-of-forgiveness}{%
\section{The Trap of Forgiveness}\label{the-trap-of-forgiveness}}

\emph{Posted on February 24, 2015}

\textbf{In Volume 2 of Engaging Multiple Personalities, there is a discussion entitled ``The Trap of Forgiveness'' which is directed to therapists counseling DID patients:}

``Therapy must be practical. It must take into account the trauma that the patient must process. Setting an unattainable goal will only reinforce the patient's negative self-image engendered by the abuse. One must consider the likelihood of success, so set goals in therapy that are within the grasp of the patient.

Some therapists, particularly those with a religious background, see the goal in healing as forgiveness. It is the view that being able to forgive is the ultimate expression of being healed. While it is a fine aspiration and appropriate in religious contexts, it is a dangerous goal to set for a patient.

Most trauma that leads to DID is so overwhelming that ordinary individuals cannot truly imagine the experience. To presume that one will eventually be able to forgive their abuser is, for practical purposes, a fantasy. Focus on the task at hand, teaching the patient to experience and hold on to the safety of the present. Teach the patient that skill so that they can experience the safety of the present when memories of the past arise. When memories are just memories, no longer the involuntary reliving of pain, that is what it means to heal.''

\hypertarget{working-with-flashbacks}{%
\section{Working with flashbacks}\label{working-with-flashbacks}}

\emph{Posted on November 19, 2014}

Therapy comes down to one simple goal: \emph{helping patients make friends with their own mind}.

\textbf{I often encouraged my patients to engage in physical exercise during a flashback.} By reclaiming the present moment in the body through exercise, my patients became more able to self-titrate exposure to their memories in a safe way. Teaching control and empowerment through the body experience of exercise in the present moment diminished the power of the past abuse in that present moment. Titrating exposure to the trauma in this way doesn't deny or negate the flashback or its content. It is not telling the alter that the memory isn't important. It is communicating that the system cannot safely handle the entire memory quite yet.

Over time, it makes the memory more accessible as it becomes less frightening and overwhelming. Learning to deal with one flashback in this way strengthens the system's ability to deal with others.

\hypertarget{panic-attacks}{%
\section{Panic Attacks}\label{panic-attacks}}

\emph{Posted on August 19, 2015}

There are always many postings about DID and other PTSD patients being overwhelmed by panic attacks. I have discussed this in both volumes of Engaging Multiple Personalities in some detail but I want to emphasize that there is a path to dealing with panic attacks.

The essence of any panic attack is the complete loss of control. First, one must engage in preparation -- learning what you need to do to avoid that loss of control. For anyone, DID or otherwise, panic attacks are terrifying. They appear to come out of the blue, they arise with immediacy and always involve the sensation of a loss of control. It is very helpful to practice how to work with your mind and body at a time when you are \emph{not} in the midst of a panic attack.

For all human beings, there is a completely intimate connection between mind and body. If your mind panics, your body will follow. If your body panics, your mind will follow. Triggers, therefore, can come in many ways -- events that trigger either one's mind or body. While this can be confusing to some patients and therapists, it is quite straightforward.

The mind following the body and the body following the mind is actually quite good news. If the mind starts to panic, by calming the body the mind will settle down. If the body panics, by calming the mind the body will settle down. While it is sometimes too hard to settle a panicking mind with thought, you can often settle the body with exercise. Get the body in exertion mode (such as a brisk walk, push-ups or even dancing), exerting just a bit more than the bodily arousal produced by the panic. Then, as soon as you stop the exercise, the body will automatically settle down. As the body slows down, the mind goes along with it.

Alternatively, sometimes it is the body initiating the panicking. While it may be too hard to settle it down with exertion, by settling the mind through mindful breathing, the body will often follow.

Some patients ask what mindful breathing is. It is simply paying attention to a specific aspect of your breathing. There are many techniques, but the simplest is to count each outbreath up to 10 -- and then repeat. You can also pay attention to the feeling of the air moving out through your nose on the outbreath and in through your nose on the inbreath.

You can be mindful of when the shift from outbreath to inbreath happens, and vice versa, just as you can be mindful of when you are holding your breath. When you see that you are tensely holding your breath, you can then control its release by intentionally breathing out. When you release the breath, the air goes out along with the tension that subtly caused you to hold your breath.

Whatever method you use, understand that using a method rather than simply being carried along by the panic is an indication that you are re-asserting control. This is very positive.

Practice is what is called for, usually a lot of it done regularly. In a quiet safe space, you can intentionally allow a very small slightly negative thought to arise -- something that is an ordinary everyday irritant and not a deep trauma. This is something which you are controlling, that is key. Remember, starting with baby steps is extremely important.

As your heartbeat increases, choose to do some exercise or some mindful breathing. Then, after a few minutes when you choose to stop the exercise or mindful breathing, you will see that your mind and/or body has settled. Making a choice about the technique and trying it is a second indication/assertion that you are in control.

The point of the exercises is to re-empowerment. It is to create new habits in both your mind and body so that when you actually are hit by a panic attack, you have already created new pathways to react to it -- all of which are marked by being in control.

Do not try to generate thoughts of trauma and try to work them out this way. That is dangerous and will not be helpful. Please work directly with a therapist on the trauma material.

When in the midst of a panic attack, first try to remember what you have practiced, and second, try to do it. Don't worry if you cannot quell the panic right now. Do not be angry with yourself if you remain terrified and panicked. That is not an indication of failure, it is simply an indication that you need to practice more in a safe place. Remember, this is a path that needs to be trod step-by-step. Even remembering that you are panicking more than you had hoped you would is an indication that you have retained some level of control in the midst of the attack.

While medication can support you in working with panic attacks, genuine healing occurs only when the disempowerment experience of the trauma is overcome. Re-empowerment is the goal. It is much more important than simply wrestling the agitated mind into submission again and again by a chemical which will have limited ongoing impact on the panic.

You have that power for re-empowerment in the present moment. Practicing before a panic attack, again and again, enables you to access that present moment power when you need it.

\hypertarget{self-soothing-techniques-for-those-unable-to-locate-a-did-therapist-part-1-of-3-background}{%
\section{Self-Soothing Techniques for Those Unable to Locate a DID Therapist -- Part 1 of 3: Background}\label{self-soothing-techniques-for-those-unable-to-locate-a-did-therapist-part-1-of-3-background}}

\emph{Posted on December 14, 2015}

This post is to encourage the development of self-soothing skills. It is not psychiatric advice, as I am retired, no longer have patients, and cannot give therapeutic counsel. I am posting these thoughts and recommendations based on approaches I took with some of my patients that had positive results. If you do not have a therapist at the moment, please make sure that you remain safe as you consider or try developing self-soothing skills. If these seem like they might be helpful to you, and you do have a therapist, please discuss them with your therapist before trying any of them.

I have posted this because it is common knowledge that there is a dire unmet need for competent DID therapists. This is true all over the world. Even if one gets past the barrier of being able to find a therapist who acknowledges the validity of DID as a diagnosis in accordance with the DSM, one still has to find a therapist within that group who is willing to work with DID patients, and who has the time as well as the training to do so. These obstacles can sometimes appear to be insurmountable, at least in the short term.

The clear problem facing DID individuals then is what to do in terms of self-care if circumstances dictate a long waiting period to find a therapist. However, we can start with the understanding that even in therapy, self-soothing techniques are complimentary to basic one-on-one psychotherapy. Just as Olympic athletes in training needs to do daily weight-lifting and stretching exercise routines, self-soothing practices should be part of the routine for DID individuals.

The fundamental point of any self-soothing practice is learning to be kind to yourself. In general, DID individuals are in conflict and pain -- often both internally and externally. They generally experience being trapped in a haze of confusion, sometimes with and sometimes without an ongoing conscious awareness of their DID circumstances. They are struggling with the consequence of dissociation. This can show up in the conflicts between the host and some alters, between alters, and with others they encounter in society. There is the ongoing suffering from the pain of early childhood trauma, whether it was physical and/or sexual assault or lack of emotional attachment to the primary care taker.

With DID, just as with any other form of PTSD, one is easily triggered into flashbacks. In a flashback, your body is behaving out of the host's control. On an ongoing basis, there is likely an accompanying self-destructive behaviour such as substance abuse, eating disorders, and/or attacking one's own body.

Substance abuse is related to taking a short-cut, using chemicals for self-soothing as are eating disorders. Repeating self-destructive behaviors has a similar impact and consequence. Unfortunately, these do not fundamentally do anything for your healing. It simply provides a short term impact that creates an ever increasing need for more of whatever substance or conduct is being abused. Relying on this kind of external and negative source of comfort falls short of processing the basic trauma, because it does not empower you.

Without processing the trauma and gaining the self-empowerment that goes along with that processing, one continues to feel empty, weak and passive. There is a loss of personal power, or dis-empowerment, that began with the original early abuse. DID has that component of PTSD which robs the individual of his or her innate basic confidence because the nature of abuse-based dis-empowerment trains you to believe that you will always be a victim, no matter what. This fundamental dis-empowerment needs to be exposed for the lie that it is, a lie told by abusers to further subjugate the abused.

The basic therapeutic approach to correct this destructive imprint involves re-empowering the DID individual. Positive conduct that promotes the personal power and confidence of someone with DID would be a most beneficial adjunct to the therapeutic goal of processing the trauma.

Is there some basic principle to follow? The answer is yes; definitely yes. Learn to make friends with yourself. This is not a platitude, it is an actual thing to practice. You must learn to be kind to all the parts. That can only happen when you are open to understanding why the different parts may seem to have competing attitudes, agendas, and demands.

Do practices that strengthen the system as a whole. You are all in that one body together so stay connected and learn to function as a team. Visualize you are like an Olympic team with a distinct common goal in mind. As an Olympic team, you have a target and a purpose, which is to score goals. The target and goal here is to be kind to each other.

\hypertarget{self-soothing-techniques-for-those-unable-to-locate-a-did-therapist-part-2-of-3-practical-suggestions}{%
\section{Self-Soothing Techniques for Those Unable to Locate a DID Therapist -- Part 2 of 3: Practical Suggestions}\label{self-soothing-techniques-for-those-unable-to-locate-a-did-therapist-part-2-of-3-practical-suggestions}}

\emph{Posted on December 14, 2015}

Here are some suggestions for self-care in practice:

\hypertarget{a.-create-imagery-for-yourself-that-is-a-sanctuary-a-place-of-refuge.}{%
\subsection*{A. Create Imagery For Yourself That Is A Sanctuary, A Place Of Refuge.}\label{a.-create-imagery-for-yourself-that-is-a-sanctuary-a-place-of-refuge.}}
\addcontentsline{toc}{subsection}{A. Create Imagery For Yourself That Is A Sanctuary, A Place Of Refuge.}

You can easily make your mouth water simply by imagining sucking on a slice of lemon. If you can do that so easily, have confidence that, similarly, you can create a mental image of a safe place where you can rest and recuperate. Begin to heal your wounds by creating that place of refuge where you can allow healing to take place. Do not underestimate the power of suggestion. Here, we are using that power of suggestion to heal ourselves. It is the exact opposite of what abusers do, which is use the power of suggestion coupled with abuse so as to try to deny you this innate ability we all have to heal.

If one breaks a bone, the doctor puts the broken pieces as close together, and immobilizes the injury in a cast. Now the fracture is stabilized with the bones fragments held in place. This allows for the body to go through its non-conceptual and completely natural healing process. The cast is the safe environment which allows your bones to heal together while protecting the injury from further disturbance.

You don't have to give instructions to each part of the bone to grow a little this way, a little that way, now join with this other piece and that other piece, and now all of you grow together\ldots{} The knowledge of that healing is already available to you as a result of having a human body. The job of the doctor is to make sure the bones are close enough together that they will knit strongly and quickly, and that the injured area is protected from breaking again due to external forces during the healing process.

For someone with DID, the same kind of process can be put in place. The parts are brought together in an environment in which they can become close rather than in conflict. Within the visualized place of refuge, they can start to knit together. In that visualized place of refuge, they are protected from re-traumatization, which is the equivalent of a bone breaking again in the previous analogy.

When a child is hurt outside of an abuse context, a protective adult holds the child, soothing her with soft words and reassurance. That nurturing kind of remedy is love in action, highly creative and healing. So, within the place of refuge you have established through imagery, when the protective parts are close enough to hold the frightened ones, the injured ones, the ones that continue to feel torment, self-soothing and healing can take place.

Healing is best visualized in kinesthetic (sense of touch) terms. Through the sense of touch, one can connect with warmth and security through the imagery of being enclosed and protected in a cocoon. Caterpillars transform into butterflies while protected in the cocoon. Your place of refuge can serve you in that same way.

There are many DID individuals who have expressed positive experiences using a healing blanket, one which is weighted that they feel safe under. To me, this is reminiscent of the circumstances of a fetus in the womb. Before birth, one is protected by the tremendously strong uterine muscles of the mother's body, floating gently in the warm liquid of the amniotic sac, protected without effort.

There are both religious and secular imageries that can be used. One should strive for a kinesthetic imagery that creates a physical sensation that is beneficial for the hurt individual or part seeking relief. For patients of mine that were devout Christians, I borrowed the imagery from Jean Vanier that ``Prayer is rest; it is to be still, to abide in the presence and in the arms of God, knowing that we are loved just as we are; we are held and safe.'' I would literally ask the patient to feel the sense of gentle pressure one experiences while being hugged. For patients of mine that were atheist or agnostic, a similar imagery was used without an anthropomorphic God (God in human form).

For one patient, the imagery that she found most helpful, i.e.~most safe, was to be alone on a tropical island with a white sand beach that was so warm and comforting in the sun while all the while a large thick tropical forest, which started at the edge of the sand, kept anyone else from finding her. She could feel the very fine sand warm against her skin warming her from below and the sun warming her from above. She could smell the ocean and feel its warm breeze.

Use the imagery that is kinesthetic and safe. Religion vs secularism is not the point. Healing the sick is the purpose of psychotherapy so find the safest, most acceptable and effective way for you to re-learn the empowerment of experiencing safety in a place of refuge so that you can heal.

Traumatized individuals often have forgotten what it is like to feel comfortable and secure. So, small step by small step, explore ways to establish the sensorial feeling of comfort and security. There is comfort and pleasure in simply eating a piece of warm buttered toast when you have a cold, or drinking a glass of water when you are parched. In some mindfulness groups, the teacher starts their instruction in class by handing everyone a raisin. Participants are instructed to appreciate the simple sense perceptions connected to that raisin: how it looks, how it feels to the finger tips holding it and the teeth biting it, and how it tastes when it is in your mouth.

Comfort is usually accessible as we encounter ordinary objects in our everyday life, but we have forgotten about it, or are in such a hurry that we bypass the experience. We need to allow ourselves to re-experience it. I suggest the following simple ways you could try: When you go to sleep, feel the comfort of a warm heavy blanket enveloping you. Re-create the primal environment of the baby floating in the womb. Explore the foetal position when you are in bed and see how comfortable it is when you curl up in that position under the blanket. Don't tell yourself about it or guess at what it might be like. Instead, actually feel the sensation.

Experiment with physical comfort. A security blanket, literally, is one that is heavy, warm and protective. There is a direct sensation of protection and comfort that happens when you are all nicely wrapped up and tucked in.

Though your own effort, imagine you are on a beach, a castle at the top of a mountain or in some other place of refuge that you choose. Find and define your safe place wherever you want to nurse your wounds. In that place, re-learn the sense of comfort and security which can be generated in and through your body. You have the power to generate the feeling of comfort and security. Make the time and space to practice doing so.

\hypertarget{b.-stay-connected-to-your-body}{%
\subsection*{B. Stay Connected To Your Body}\label{b.-stay-connected-to-your-body}}
\addcontentsline{toc}{subsection}{B. Stay Connected To Your Body}

\begin{enumerate}
\def\labelenumi{\arabic{enumi}.}
\item
  Sunlight -- bright light increases the production of serotonin in the body. Spending time in the sunlight can absolutely improve your mood and also soothe muscle aches. Full spectrum lighting can be helpful if you live in areas where there is little sun.
\item
  Massage -- physical contact from working your muscles stimulates the release of endorphins. Massaging your own scalp and using shower massagers can provide an affordable alternative to expensive treatments. Massage therapy can feel wonderful.
\item
  Meditation -- meditation helps the nervous system operate at its best. There has been quite a lot of research done to confirm its benefit. There is more about this later.
\item
  Physical Exercise -- one of the best natural ways to produce serotonin, dopamine and endorphins. Vigorous exercise is best because the stronger the physical demand you place on your body, the greater the release of endorphins. You should try weight training as well as high and low intensity exercises. Work out only for so long as you can based on your capacity at the time. You want exercise to gradually strengthen your body, not to overwhelm it. Engage in regular physical exercise in muscle building, cardiovascular aerobic exercises, and stretching exercises. Learning and practicing yoga and taichi can be very supportive of both the mind and body.
\item
  Music -- music is powerful and can move you emotionally. That is why you can tell what is going to happen in a movie scene based on the music. Good music can absolutely help your mood and get you positively grounded again. Try and listen to mostly upbeat music. Try dancing to it in the safety and privacy of your own home -- combining the music with joyful exercise.
\item
  Laughter -- savor the feeling of laughter with friends (or with other alters you might connect with) or watch a good comedy movie.
\item
  Sex -- is a powerful producer of endorphins. One must be very cautious as it comes with responsibility, obligations and is often connected with dangerous triggers for retraumatization. I may be castigated for suggesting this but, as I suggested in Engaging Multiple Personalities, if sex is important for you, and particularly if you are unattached, the safest sex for healing and grounding may be masturbation.
\item
  Acupuncture -- increases circulation and stimulates the release of endorphins. Of course, one must find a well-trained and capable acupuncturist just like when you look for any other professional.
\item
  Nourishing Teas -- in the absence of diabetes, a warm ginger, honey and lemon tea can make you feel quite nice.
\end{enumerate}

Remember the general principle that you can gently retrain the body and mind so as to correct the feelings of ``I am a powerless victim'', feelings which are inherent in the process leading to DID. A gentle transition through kind and inviting body connections is therapeutic. Do not seek an easy way out that is simply a repetition of the experience of dis-empowerment -- such as self medication through drugs, alcohol or other compulsive behaviors. If you feel better physically, through exercise and connectedness, you will gradually enlarge your capacity to work with all the parts as a team, in harmony. Keeping the mind in a creative mode through art music communing with nature and the like are foundations for improving and healing the wounds of DID.

\hypertarget{c.-stay-connected-with-others.}{%
\subsection*{C. Stay Connected with Others.}\label{c.-stay-connected-with-others.}}
\addcontentsline{toc}{subsection}{C. Stay Connected with Others.}

Close friends for support are essential in healing. Join an online support network so long as the administrators are properly protective of the members, on guard for individuals who are not there for the purpose of supporting others seeking to heal their DID. Online groups can have a truly positive impact. Active groups usually have people online 24/7 so that if you need to communicate with someone supportive in the middle of the night, it can actually happen. Make sure when joining such as a group, that they require warnings to be posted before writing anything that might be triggering.

Join a choir if you like music and singing. Join a photography club, a drawing or pottery class if you are artistically inclined. Join a hiking club. Well-defined interest groups are safer and more functional than other social clubs. These amateur groups are usually filled with enthusiastic members and they offer valuable support within the specific interest that can help you build a creative hobby. Connecting with people in such clubs can fill your life with warm memories.

Altruistic volunteer groups of people who are willing to contribute their spare time for the welfare of others can enrich you life in very meaningful ways. There is nothing more rewarding than to devote time to turn your kindness towards the less fortunate.

In this vein, remember that spending time with animals can also establish a sense of well-being and non-judgmental connectedness. This is discussed in more detail later on. In short, if you don't have a pet or cannot afford one, there are always opportunities to help at an animal shelter. Supporting an abandoned or traumatized dog or cat is another way to nurture the strength of your own compassion. Training in that way can also lead to establishing roots of internally focused kindness -- toward alters that can help the amnestic barriers slowly and safely begin to dissolve.

Note that I have not included traditional support groups in this category. That is not to say that they do not have value, and often tremendously positive value. However, one must be careful to keep to the specific purpose of such support groups. Alcoholics Anonymous, Gamblers Anonymous, Narcotics Anonymous and the like all have long and important histories of making real positive differences in the lives of people with those addictions. The very nature of such groups is that the focus is the addiction. Here, I am suggesting connecting with groups where the focus is quite different -- not about dealing with a deep problem but instead about singing, art, hiking and so on.

Please do go to and continue to participate in AA, GA and NA meetings as much and as often as is helpful. Nevertheless, there are predatory individuals that attend such meetings so keep the boundaries quite firm.
Just as you need to maintain firm boundaries when you might encounter individuals that are triggering, understand that there are reasons protective alters emerge. Respect their intentions always. By maintaining firm boundaries, you let them know that you are giving credence to their assessments. Having done so, invite them to re-assess the individuals periodically. This is a way to gently allow them to moderate the hypervigilance common to protective alters while allowing them to fulfill their protective function.

\hypertarget{d.-being-with-animals.}{%
\subsection*{D. Being With Animals.}\label{d.-being-with-animals.}}
\addcontentsline{toc}{subsection}{D. Being With Animals.}

Pet therapy has been extended to help individuals in many ways with many different kinds of difficulties. For example, there are now courts that permit service dogs to support child witnesses testifying about being abused. There are service dogs for the emotionally disabled, just like service dogs for the blind. It is obvious to all, when a service dog, or almost any dog or cat, is brought into a nursing home or old folks' home, it immediately gently energizes the atmosphere, and brings joy to the residents. Horses have also been incorporated into PTSD therapy.

Pet-Therapy is an encouraging trend.

I have a few colleagues, and am aware of other therapists, that have dogs in their office. One in particular has a three legged dog he rescued from the SPCA. The impact of having that dog in his office has been incredibly effective in communicating to patients that his office is a safe place. Were I to be starting out as a new psychiatrist, rather than being retired as I now am, I would consider having a dog or cat around for my patients.

If you are emotionally traumatized, consider having a service dog. There are substantial costs to get a trained service but tremendous potential benefits. As an alternative, you can go to the local SPCA and claim a rejected and/or traumatized dog. He/she will understand how you feel and will give you years of companionship. It can be a tremendous healing experience.

A dog is usually quite in tune with how its owner feels. When a stranger appears at the door, the dog will sense how the owner feels about that stranger and behave accordingly, either aggressively defensive or behaving in a warm and friendly way. For those with DID and the deep experience of betrayal trauma, a dog is far more reliable assessor of both your state of mind and that of the other person. Further, from the point of view of protective alters, a dog is far less likely than the host, or another person, to be deceived into betrayal by someone's surface smile.

\hypertarget{self-soothing-techniques-for-those-unable-to-locate-a-did-therapist-part-3-of-3-practical-suggestions-continued-and-conclusion}{%
\section{Self-Soothing Techniques for Those Unable to Locate a DID Therapist -- Part 3 of 3: Practical Suggestions Continued and Conclusion}\label{self-soothing-techniques-for-those-unable-to-locate-a-did-therapist-part-3-of-3-practical-suggestions-continued-and-conclusion}}

\emph{Posted on December 14, 2015}

Part 2 of 3 discusses practical suggestions for self-care. This Part 3 of 3 continues that discussion.

\hypertarget{e.-slowly-engage-the-practice-of-mindfulness-including-walking-meditation}{%
\subsection*{E. Slowly Engage The Practice Of Mindfulness -- Including Walking Meditation}\label{e.-slowly-engage-the-practice-of-mindfulness-including-walking-meditation}}
\addcontentsline{toc}{subsection}{E. Slowly Engage The Practice Of Mindfulness -- Including Walking Meditation}

This kind of practice is allowing your mind to become more stable. You begin by holding your spine as straight as you can. You train in focusing on the here and now. The most important thing is to accept yourself and simply start taking one breath at a time. Do not congratulate yourself when your mind seems calm just as you shouldn't get annoyed and scold yourself if you drift off course. It is the nature of mind that we keep drifting off the course in meditation. The practice is to always come back to the here an now when you notice that drift.

\emph{Begin with very short sessions}. Do not aim for even 10 minutes to start with. Aim for doing it during the time that you are taking just one breath. Then do another breath. You can just start with 1 breath as the entire session until you feel at ease.

Ordinarily, our mind is always chattering and full of distractions. When you can stop this chattering, even for a split second, or the time it takes to breath in, you are into the practice of meditation -- paying attention to the reality of the now. This is no mean feat. It may seem like a drop in the bucket, but the ocean itself is made up of water droplets.

I suggested very short sessions for a reason, a warning. As always, one must be aware of the very real risks of re-traumatization. For individuals with DID, sometimes creating that little space results in the alters seeing it as the opportunity to emerge uncontrollably, flooding you with their many separate agendas. These usually include retraumatizing flashbacks.

While taking one breath alone is unlikely to provoke an immediate flood, please check yourself. If you begin to feel the flooding of a flashback starting, stop the mindfulness practice by moving your body. Stand up from your seat. Stretch your arms fully. Straighten your legs completely. Identify the room you are in right now. Perhaps start your journaling ritual (see below) and allow some communication to happen in that way.

Go back to the mindfulness practice the next day, but don't try to just jump back in trying to extend the duration of the practice. Always check your sense of safety first. Take this approach until your mindfulness practice is stable enough to allow thoughts to arise without the retraumatizing flooding of flashbacks.

When your mind begins to stabilize, you start to be aware earlier on and ever earlier on in the flashback cycle. The sooner you see the cycle start, the easier it is to ground yourself and avoid retraumatization. Consider how much easier it is to stop a car going 5 miles an hour than a car going 100 miles an hour. In that same way, grounding yourself at an ever-earlier stage of a flashback cycle is far easier than trying to put the brakes on a full-blown flashback.

Remember to take baby steps: Connecting to the safety of the here and now for even a fraction of an in-breath is better than just digging into flashbacks and being trapped in the retraumatization cycle.

\hypertarget{f.-establish-empowering-rituals.}{%
\subsection*{F. Establish Empowering Rituals.}\label{f.-establish-empowering-rituals.}}
\addcontentsline{toc}{subsection}{F. Establish Empowering Rituals.}

We can make a positive ritual out of a simple sequence of thought and/or conduct so that it is turned into a daily habit. It only takes repetition to build a habit and a routine -- good or bad. So, take the steps necessary to build a positive empowering habit.

We all already have a routine when we get up in the morning and one before retiring at night. We have already ritualized and habituated ourselves to these routines. So, we do not have to struggle thinking about them. Build into this existing habit the focus of learning to feel safe and secure.

For example when you wash, at the sink or in the shower, imagine that you are not just washing the day's dirt off your hands and face, but that you are washing down the drain the feelings you might have of having been dirtied by abuse. When you wash your hair in the shower, as you rinse out the shampoo, imagine that all of the physical and psychological dirt along with the sense of being soiled, simply goes down the drain. Imagine that you leave the shower both physically and, even just a little bit, psychologically cleaner than when you entered. You can extend this into brushing your teeth and other ordinary cleaning activities.

I often encouraged my DID patients to establish a clear ritual for safe communication with and between alters by ongoing journaling. In essence, it is creating a form that is empowering because it is within your control. Pick a book to write in that is only for this purpose. Establish a place and regular time to journal. It can be used for meetings of all the parts, it can be used for parts to leave messages for other parts, it can be one of your places of refuge. Always begin with some grounding exercise(s), open the journal, allow everyone inside that wishes to say something to do so by writing in the journal. In that way, communications from different parts can be shared with the host and other alters.

A critical point of this approach is to authorize the closing of the journal if things become triggering. In such circumstances, close the journal in accordance with the ritual you have established, with the express intention of allowing what has been raised to be processed. Include the promise of allowing further journaling on that triggering issue as soon as the system is able to process it. Then, and most important, following closing of the journal and always putting it away in its designated place, do a closing grounding exercise.

Often, the best grounding following journaling is to go for a walk outside. When walking, keep your senses as open as possible to the air that you breathe, to the trees you walk by, to the stability of the earth that you walk on. The earth, in particular, has the capacity to ground the energy the journaling has generated, in the same way that when you connect a lightning rod to the earth, the lightning's electricity is safely absorbed.

\textbf{Concluding Remarks}

All healing that is effective has to come through one's own effort. So, consider working on self-soothing practices before you have a therapist. The more you participate in such practices, the more effective and self-empowering is the healing. This way, when you are able to connect with a therapist, you will have already started to build a strong foundation for the therapist to support your continuing healing journey.

All the above may be used as complementary tasks for healing even after you have found a therapist, but make sure you tell the therapist what you have been doing in terms of self-care. It is an opportunity to assess the therapist and for the therapist to assess you -- and for the therapist to give you further direct guidance for self-care.

None of these self-soothing approaches are a panacea, a cure-all. They are merely, but potentially powerfully, supportive of the overall healing process. Remember that DID is not the pathology, it is the resultant display of extreme trauma. Its manifestation in alters is the message, the instant emoticon you could say, that there is deep unprocessed trauma. In my opinion, the problem is not the alters. It is the amnestic barriers and the resulting internal conflicts, which get played out both internally and externally, that are the problem.

Above all, understand that healing is possible and is within your capacity.

\hypertarget{anxiety-and-panic-disorders}{%
\section{Anxiety and Panic Disorders}\label{anxiety-and-panic-disorders}}

\emph{Posted on June 27, 2016}

If we suddenly encounter a danger or a threat, we will fight, try to get away or be in such fear that we are immobilized and freeze. The fight, flight or freeze responses are daily experiences in the animal world. A gazelle lives its life grazing in the field and propagating for species survival, while simultaneously being on the alert for predators. Anxiety is an alarm system to keep an animal on its toes, to maintain a look-out for possible life threatening danger. The nervous system is fine-tuned to anticipate danger or threat so that there is time to escape danger.

These responses are normal in the human condition. Something may trigger our alarm system and we are thrown into the emergency alert mode. If the internal alarm goes off when there is no obvious danger or threat, how does one handle this internal warning? You really cannot completely ignore it. You will try to find an explanation to account for it. Your mind may start building up a scenario to account for such fear and anxiety. It may be a subliminal flashback of memory that is the trigger.

More fear will feed on that initial intangible fear, and perhaps a bodily sensation gets misinterpreted. The alarm system will convince you that something is wrong, that there is still danger. And then, you get into a full response mode of fight, flight or freeze. Even if we are getting a clearly false signal of impending danger, we may have already set into motion those patterns of getting ready to fight, running away, or becoming frozen with fear. This is a primitive reaction that is in our genes. It is a reaction cycle that kept our ancestors alive for tens of thousands of years.

The problem is that this kind of response behavior is \emph{usually} no longer adaptive for survival in modern life. In most cases we do not have a natural predator lurking behind the tall bushes in the park to prey on us. However, as is clear from the statistics on early childhood abuse, there are predators out there, sometimes in the child's own home. In later life, if some past trauma for which our body has been keeping the score raises its ugly head as a fragment of implicit memory, we receive the same alarm signal warning us of possible life-threatening predatory danger.

Traumatic memory does not function like narrative memory in our ordinary life, like remembering coffee yesterday with a friend. Traumatic memory is often cued by sights, smells, tastes and the feeling tone in an environment. The memory often arises in a pre-verbal way. So, not conceptually remembering the specific trauma doesn't mean that we have not experienced it, nor does it mean that we don't carry that trauma in our mindstream.

Therapists in clinical practice see that anxiety comes in all forms. The purest form is anxiety that emerges seemingly out of the blue, without an identifiable reason. When a person reacts to a small triggering sensation, often without even identifying the sensation, the associated traumatic memory of fear itself will emerge quickly into a full blown panic. The sensation can be as small as the tinge of an odor similar to one that was experienced in trauma, or the passing twinge of a painful sensation. The mind is brought back to a danger of the past. The entire body shifts into ``battle station'' mode. It is not that one is not afraid of something unknown, rather one is on the lookout for something familiarly frightening.

It is very instructive for a therapist to watch anxiety developing right in front of them. I have had the experience of watching a patient developing a panic attack right in front of me in a hospital when I was the psychiatrist on call one night. While remembering that any physical discomfort or symptom such as chest pain may actually have a real pathological rather than psychological basis -- which will be left to the Emergency staff physician to handle -- but with respect to a possible psychologically based anxiety attack, there are a series of steps for the therapist to take.

\begin{enumerate}
\def\labelenumi{\arabic{enumi}.}
\item
  The first thing is to convey to the patient of your empathic understanding of the magnitude of fear the patient is experiencing. The worst thing is to make light of the patient's panic, saying that there is really nothing to worry about.
\item
  Once you have their confidence, you will have to ascertain that the condition is really a panic disorder, not some physical problem that mimics a panic attack.
\item
  The preferred treatment depends on the orientation of the physician as well as the time available. In my experience, the efficacy of medication is uncertain. I believe the effects are often largely a matter of how much the patient trusts the therapist. In purely relying on the pharmaceutical effect, one runs into the danger of having to use a colossal dose to suppress the physiological arousal of a panic attack. At best, medication is useful as a short-term temporary intervention.
\end{enumerate}

Panic disorders are related to the patient feeling loss of control over his/her bodily power. I characterize it as a disorder of ``dis-empowerment.'' The patient is thinking, ``Why is my heart racing so fast when I am sitting down, not even walking.'' He/she does not realize it is the response reaction that has spun out of control, with the mind and body setting itself up in preparation for dealing with some as yet unseen but monumental threat. Whether that threat is a present danger or artifact from the past, the physical response is entirely understandable and beyond self-control.

The test of the therapist's skill is how to suggest or assist the patient in reversing that escalating panic response. Stopping something when it is already in motion is very hard. For example, if a car is moving at 100 miles per hour, and the driver's foot is pressed down on the accelerator, it is exceedingly difficult to stop the car. The first thing to do is to let him/her regain confidence that the car is still controllable. It is easier to let the person continue speeding while gently steering it in a different direction, perhaps up a hill, rather than insisting to the person that they get their foot off of the accelerator and stop short. The patient is already overwhelmed by the intensity of the panic, it is impossible for them to stop doing whatever their body response dictates.

Remember the analogy of heading the speeding car uphill. Highways that run through mountains have special lanes for runaway trucks with failed brakes -- they exit from the main road and head up a hill so that gravity, that invisible hand, acts as an environmental brake. How can a therapist use this analogy? Redirect the patient's energy rather than confront it. Focus attention onto something for the patient to do that is not connected with denying the panic. There is the well known ``Brown Paper Bag'' method. This invites the patient to breathe in and out of a brown paper bag. I know of many patients who have successfully used this method. In fact, some carry a brown paper bag with them in case the panic returns.

The paper bag method is so simple. It is not asserting anything about the panic being correct or imaginary, therefore there is usually no obstacle to doing it. When someone is in a panic, the natural tendency is to ``do something.'' Just as with grounding exercises, this method fits into that protocol every well.

The reason it works is psychological, not physiological. Blowing into a paper bag is a simple task. The mind and body are engaged in a task. Through that engagement, the mind and body energies are redirected rather than suppressed.

There is another reason I like this method. It is because it is something the patient does which leads out of the panic. That is what counts. The best treatment is one that patients can do on their own, which engenders the confidence that they can control their bodily functions. This is re-empowerment.

Most psychiatrists advocate relaxation as the central focus in psychotherapy. This is difficult to apply, and generally not possible in the midst of a panic attack. To ask a patient to try to relax during a panic attack is like saying to a drowning man, ``Relax, your body will naturally float.'' It doesn't work.

There is a proper time and place for discussion of the patient's fears, whether they are seen as rational or irrational, but it is not during an attack. The cognitive or rational-emotive approach is appropriate only later, in the context of a supportive therapeutic relationship and environment. For example, a behavioral approach emphasizing graduated exposure to panic-inducing situations is only appropriate after the patient is taught methods of regaining self-control, that he is again the master of his body.

I do not have confidence in the long-term benefits of the text book treatment of panic attack such as:

\begin{itemize}
\item
  Carrying items such as medication, water or a cell phone
\item
  Having a companion (e.g.~a family member or friend) accompany them places
\item
  Avoiding physical activities (e.g.~exercising, sex) that might trigger panic-like feelings
\item
  Avoiding certain foods (e.g.~spicy dishes) or beverages (e.g.~caffeine, alcohol) because they might trigger panic-like symptoms
\item
  Sitting near exits of a room.
\end{itemize}

All of these may be helpful short-term supports but they generally involve increasing the dependency of the patients, confirming that they are helpless and remain unprepared for the next onslaught of panic. These methods are not based on, nor will they result in, re-empowering the patients.

I have practiced slow breathing long enough to be able to hold my breath for about 2 minutes. Given that, I was able to show and reassure my patients that it is quite safe to not breathe for 15 seconds. Then all I asked them to do is to slow down their breaths to say 4 times a minute. Once they were willing to try to slow down their breathing, even just by counting to 10 between each inhalation and exhalation, their panic dissipated.

No one can sustain panic when the breath is slowed down. The usual difficulty is convincing a patient to slow down their breath because they all feel they are struggling for air. By having them breathe along with me, they can see that they are able to work with their own breath. Then, they do it themselves. Once this is accomplished, the panic will usually not return in that intensity, and the patient will not become dependent on medication for anxiety.

After the panic is under control, find out what else needs attention. Is there past trauma? Is the current life-situation full of difficulties? Tell your therapist. In the absence of a therapist, or if you have yet to establish a safe therapeutic relationship, tell yourself by writing into your diary. Putting your troubles into words is always better than just stewing about it. In writing, it becomes something tangible with boundaries that can be worked with. Too much thinking often becomes a fruitless exercise -- like a dog chasing its own tail.

Panic disorders are not something that you need to find a magic pill to cure. Even if there is such a pill, it will only work temporarily. I am generally against giving pills for this because on the one hand, they may not work and on the other hand, they most certainly will not re-empower you. Grounding exercises are critical for a patient's re-empowerment. Practice them regularly before a panic attack arises so that you develop a personal panic toolbox to keep you centered in the present moment.

Panic attacks are self-perpetuating, tail-chasing, vicious cycles that distract us. What do they distract us from? Usually, they keep us from getting near a deep unhealed wound. A bacterial infection needs an antibiotic for healing, but panic attacks are not caused by an external agent like a bacteria. To eliminate a panic attack, one needs an inoculation of the present moment's safety. Grounding is that specific inoculation.

Panic may be your body telling you that there is danger or that something needs to be fixed. Take heed of its warning---use your time and energy to deal with the real issue, rather than seeking a medication to suppress the alarm signal. If you cannot yet find the reason for your fear, through grounding, you have at least found a way to control your body, to re-own it again.

A famous psychotherapist in the mid 20th century, Frieda Fromm-Reichmann, wrote about a man, probably not her patient, who was suffering from severe anxiety. He underwent in-depth psychoanalysis. In this case, there was a real yet seemingly unrecognized reason for the anxiety, even though he was then at the peak of his wealth/fame/family bliss. Soon after he was ``cured'', the Nazis took over and he was taken to the concentration camp to be exterminated.

There is an important lesson in this: Anxiety, like depression, is not always a symptom to be eliminated. Don't limit your focus in therapy to turning off the alarm. Check to make sure whether or not the alarm signal is correctly assessing a present danger.

\hypertarget{when-you-dont-want-to-leave-your-therapist}{%
\section{When You Don't Want to Leave Your Therapist}\label{when-you-dont-want-to-leave-your-therapist}}

\emph{Posted on December 18, 2016}

In general, therapists and clients have an extraordinary relationship. While therapists make their living by providing therapy services, their relationship with clients must be genuine, congruent and empathic to be effective. It is not the same as having a conventional or ordinary close friendship because the trust and power dynamics are neither conventional nor ordinary.

Naturally, some alters, particularly the young ones, want to cling on to the relationship with their therapist after termination of therapy. This is true whether it is the end of a single session or the end of therapy completely. This is an important issue in the therapeutic relationship. Expressing the confidence and willingness to be there for the next session is something a therapist commonly does to encourage and support the client in ongoing therapy. It is saying that the relationship is not over -- just the session. This is quite different from ending the therapeutic relationship. This is something I dealt with in preparing my patients for my retirement a decade ago. I took a year to help prepare them for that transition.

Clients often see their therapist as a particular \emph{kind} of a close friend: One willing to communicate confidentially about the client's personal history for the sole purpose of helping them heal from trauma. It is someone with whom clients can talk about issues and histories that are not so safe for them to communicate about outside of the therapeutic environment. So, even thought the therapist/client relationship is based on payment for services, it is also like the best part of a friend who gives you their undivided attention. They give that undivided attention for an hour every week or 2 weeks. This is different than an ordinary friendship, no matter how genuine.

In normal personal relationships, you choose your friends based on certain qualities that appeal to you, whether he/she is funny, handsome or smart, etc. There is an expectation of sharing information about one's life, more or less deeply depending on the depth of the friendship. Change and growth are implicitly expected in any relationship, but in a therapeutic relationship that expectation is solely about the change, growth and healing of the client.

Therapists don't choose patients in the same way they choose their friends, such as common interests, social circles, and the like. They treat the individuals that come into their office, whether or not they have friends, hobbies or other things in common. Unless there are exceptional circumstances, the therapist takes whoever comes into their office needing his/her service. It is important to understand that your therapist-friend has problems of his or her own, but, unlike a conventional friendship, he does not share them with you. He maintains this boundary in order to ensure that he is there solely for your needs, not his own. Your therapist has to keep his problems to himself in order to properly be there for you.

Deep down, the therapist treating DID is often providing a corrective parenting experience offered in the safety of a therapeutic relationship to support the client processing past trauma. What is a corrective parenting experience? It is being there for someone when they are hurt, reassuring them of their basic goodness and helping them feel better. For example, when a child falls down and scrapes their knee, a proper parental response would be picking up the child, looking at the injury, assessing it and either getting the child medical treatment or reassuring them that the injury will heal without much of a problem. In other words, providing comfort and safety. A traumatizing parental response would be something belittling, mean. It would be using the incident as an excuse to further crush the child's self-esteem and sense of safety by eliminating the idea that the parent will ever serve as the child's adult protector in the world.

There is a risk that the young alters in particular will not understand that the therapist providing a corrective parental experience is \emph{not} the same as the therapist becoming a replacement parent. This is something the therapist must gently and consistently clarify for the client.

After termination of therapy, there is no legal requirement that there be a complete cessation of contact. However, for ethical and genuinely therapeutic reasons, it is risky and inappropriate for the therapist to engage in a direct relationship of friendship with a former client. The power dynamic inherent in the original relationship will not disappear. Further, and critically important, is the fact that should the client need therapeutic assistance in the future, a direct relationship of friendship will cut off that possibility.

It might be OK if the client wishes to send their former therapist an occasional greeting card. Sometimes that may be done \emph{by the client} in order to leave open the possibility of returning to therapy with the original therapist. In fact, one of my patients continued for years to send the occasional brief letter to my secretary to maintain some continuity with my office. in my experience, maintaining that boundary is important for the well-being of the (former) client. It is my view that when the client finishes with therapy, it is important that he/she feels the improvement is based on their own efforts, rather than something to be credited to me. It is their successful processing of the past trauma, their survival, that is the point -- not my achievement. If they can move on in life without further therapy, it is all for the better.

This is very difficult for some of the alters to understand, particularly when they remain infantile or of very young age. This does highlight another question, whether or not young alters grow and mature in their age during therapy. As is shown in some of my other writings, integration is not necessarily the goal. If the alters integrate, and their age approximates that of the body's chronological age, that is fine. But, in my opinion, the most important mark of successful DID therapy is that the conflicts among alters are resolved so that they are working together rather than at cross-purposes based on unprocessed trauma.

This answer will not satisfy all, especially those who remain having young alters in their system. There are really no comforting words that are guaranteed to reassure a group of children (in DID -- the young alters) when we take away their caregiver (in DID -- the therapist) and say everything is OK. It doesn't work for a child traumatized and separated from their loving parent as a result of worldly circumstances like illness or war, and it doesn't work for a DID system traumatized in the past and now separating from their therapist.

However, one can give them all confidence in their ability to continue on their healing journey. That is part of the preparation work, prior to termination of therapy, that I tried to do for all of my patients. Perhaps some of the adult alters in the DID system can take over some parenting function transferred from the therapist. Perhaps the alters can become really good friends inside, supporting and mentoring each other. Perhaps the system can become more firmly established in their self-care and grounding exercises. The best reminder for the system is that all the parts are there for a reason so be kind to everyone inside, always be kind.

\hypertarget{anxiety-part-1-symptom-and-message}{%
\section{Anxiety -- Part 1: Symptom and Message}\label{anxiety-part-1-symptom-and-message}}

\emph{Posted on March 29, 2017}

In a psychiatric practice, anxiety is the most common complaint among patients. But consider how common it is that a psychiatrist facing an anxious patient immediately concludes that the patient is suffering from ``Anxiety Disorder'' and simply prescribes a pill for the anxiety. The same holds true with patients complaining of depression. No wonder that a consensus is slowly building everywhere but in the pharmaceutical industry that there is an alarming number of North Americans -- men, women and children -- are over-medicated for pain, anxiety, or depression.

In an ordinary medical practice, pain is the most common complaint among patients seeing their family doctor. For example, when hearing a complaint of pain in the stomach area, the doctor first tries to find out a little more about the pain before making a diagnosis. The doctor will ask if the pain is acute, chronic, triggered by a particular movement or food, whether it hurts when it is pressed here rather than there, and so on. Only after the analysis concerning the source of the pain is made would a diagnosis be made and an analgesic (a pain killer) be prescribed.
Just as many cases of pain can be traced to bad posture, lack of exercise, and lack of mobility in the elderly, many cases of anxiety and depression can be traced to very real experiences of deep trauma.

Life is filled with mixtures of joy and sadness, carefree laughter and deep worries. Joy and laughter are seldom experienced as a problem. But when something in the environment triggers your internal alarm system, you will start worrying. Worrying is not per se a bad thing. It can be helpful in deciding to focus your energy in preparation for a task at hand, for problem solving or securing a level of certainty. However, when worry becomes one's normal state of being, it becomes difficult to control. It can result in persistent anxiety, loss of sleep and/or raising blood pressure. When worry becomes ongoing anxiety, inappropriate or disproportionate to the object of concern, it is no longer helpful.

Clearly, not all worrying is pathological. For instance, if your teenage son is going out for a casual ride but you suspect that the driver has had a couple of beers, then your worry is perfectly justified. It is an alarm bell going off that is to be taken seriously. But once we have done the necessary scrutinizing of a situation, and ensured that reasonable actions have been taken, worrying is a waste of energy. If that worrying continues to the point of paralysis, it then fits into the psychiatric category of an Anxiety Disorder.

The next question needs to be asked though, ``Does that mean that drugs are necessarily the best treatment?'' One argument against pharmacological treatment is that while drugs can ease your mental tension, they may also take away the ability to encourage yourself to practice self-regulation while potentially leading you down the path of chemical dependency. Being trained to deal with tension via a quick chemical fix is not particularly that far from the entryway of addiction. In the long run, is this beneficial for you? It is my view that as therapists, we should be encouraging patients to engage in correcting and refining the balance of their internal alarm system through therapy that may include medication as an adjunct but not the sole treatment.

Life is full of obstacles past, present and future. One must beware of relying solely on drugs to protect you. Relying on a drug that helps, without embarking on the necessary internal re-calibration work of psychotherapy, is a mistake. Why? Because you have not used the situation to learn about the root causes of your difficulties in dealing with the obstacles you face. This leads to the ongoing undermining of your own sense of self-empowerment.

If you visit your family doctor because of a pain in your right shoulder, I certainly hope the doctor does not say, ``We will open up your shoulder and take out whatever is bothering you.'' No, you want the doctor to ask more questions, to further examine the shoulder, and order some tests to find out the real pathology. Shoulder pain can be what is called ``referred pain'' which can indicate potential pathologies as diverse as abdominal, pelvic, heart and lung problems as well as, of course, a strain or tear in the shoulder muscles. In a similar way, simply treating anxiety or depression without identifying the actual illness or circumstances causing those symptoms is at best lazy medicine and, unfortunately, has the capacity to be far worse.

\hypertarget{anxiety-part-2-patterns-and-recommendations}{%
\section{Anxiety -- Part 2: Patterns and Recommendations}\label{anxiety-part-2-patterns-and-recommendations}}

\emph{Posted on March 29, 2017}

In psychiatry, just as in any other branch of medicine, the real cause of a symptom may be hidden from the patient, the doctor or both. In psychiatry, the root causes of disorders are often unintentionally hidden because of internal conflicts that are submerged below one's consciousness. We all have experienced worrying about one thing only to eventually discover that the real issue is something quite different.

\begin{enumerate}
\def\labelenumi{\arabic{enumi}.}
\item
  Anxiety may be caused by a taboo subject we simply cannot or do not wish to face.For example, a young woman married to an abusive man may not consider divorce an option, will not even see it as a possibility. Sometimes that is because there are children involved, sometimes it is because of financial circumstances or religious teachings. Such a person coming to a therapist for help presenting a variety of anxieties and depressive symptoms may never mention the domestic violence -- even though it is the core issue. It behooves the therapist to exercise their sensitive radar to clue in because direct questioning will often elicit a simple negative answer resulting in everyone involved heading down the garden path of deception.
\item
  Anxiety often creates other symptoms in order to create a distraction that can lead both the patient and therapist on a wild-goose chase. For example, a sore back, a stiff neck or a splitting headache or compulsively cleaning at a specific time late each evening may be the complaint. Focusing on the distractions soon turns the distraction into a real problem that even more strongly leads away from the root issues.
\item
  Anxiety is characterized by a state of helplessness, of complete loss of self-control. Typical examples include, ``I cannot breathe'' in a panic attack as well as ``I can't sleep'' in insomnia. Normally breathing and sleeping are part of the effortless behavior pattern of a living being. The anxiety is interrupting the basic human operating systems.
\item
  Anxiety leads to cognitive distortions. Even doctors have anxiety/depression. I know of a doctor, a specialist, who in the depth of his depression, without any reason, worried that no one would come to see him in his practice when in reality, the usual waiting time to see him was 6 months.
\end{enumerate}

\textbf{Key Remedies to Consider:}

\hypertarget{come-back-to-the-present.}{%
\subsection*{1. Come back to the present.}\label{come-back-to-the-present.}}
\addcontentsline{toc}{subsection}{1. Come back to the present.}

For runaway anxiety such as panic attacks, the first thing to do is to bring oneself back to the present moment. That is empowerment. In other words, don't fight the panic in your mind: Reassert control over your physiological response to the thoughts. How to do this? Well, the most common complaint in anxiety attacks is ``I cannot breathe.'' Under the circumstances, until the patient is able to do this alone, the therapist will have to strongly take over and direct the patient to breathe SLOWLY. I would reassure my patients that it is safe to regain the control by holding the breath for ten seconds. I would tell them that even at my age, I could still hold my breath for one minute without causing any harm to my body. That usually caught their attention. By seeing that it was harmless for me to hold my breath for 1 minute, they were reassured that holding their breath for 10 seconds would not hurt them. Once they had intentionally slowed down their breathing, they generally felt re-empowered, back in control of their most basic body function. This was simple, immediate, effective and \emph{within their capacity}. The fact that I was willing to throw myself into their fearful experience reassured them that I was taking their concern seriously, not casually dismissing their complaints as trivial.

\hypertarget{learn-how-to-worry-constructively}{%
\subsection*{2. Learn how to worry constructively}\label{learn-how-to-worry-constructively}}
\addcontentsline{toc}{subsection}{2. Learn how to worry constructively}

Some worries are necessary while others are not. Let us call one group ``constructive'' and the other ``destructive.'' First, we need to learn to distinguish between the two. The reason we usually cannot separate one from the other, leading to the paralysis of anxiety disorder, is that we are too emotionally involved with the subject of our worry. Effectively, we are talking about correcting cognitive distortion.
Destructive worrying is worry that entraps one onto a vicious self-perpetuating cycle. The resultant worry feeds into whatever is causing the worrying making it worse, creating a sense of loss of control or ``dis-empowerment.'' There are usually two internally opposing voices at work. One tells you that your worrying is justified. The other says that you are just worrying too much, that you should not be worrying. Constructive worrying enables you to consider the issues without entrapping yourself in a vicious self-perpetuating cycle.

How can we determine whether or not our worry is constructive? A patient came to see me once for anxiety issues. I applied the paradoxical intention concept from Victor Frankl's Logotherapy approach. I asked my patient to sit still for a few minutes to prepare himself, organize his thoughts and review what was bothering him, before telling me all the things that were bothering him. In other words, I was asking him to worry without trying to fight it.

As a result of my request, he gave himself permission to worry. He sat down and focused on the internal turmoil without worrying about his worrying. When he was sitting and reviewing what he was supposed to be worrying about without that self-criticism, he was able to process his situation without further outside input. After a few minutes, to my surprise, he opened his eyes and said that what he was worrying turned out to be of no real significance. He did not even need to tell me! His body language confirmed it.

Be very clear about your non-verbal language as a therapist because it carries a lot of weight -- often more weight than the spoken word. It was clear to my patient both verbally and non-verbally that I took his complaint seriously and respectfully, rather than being dismissive of his worries.

\hypertarget{anxiety-part-3-meditating-on-anxiety}{%
\section{Anxiety -- Part 3 Meditating on Anxiety}\label{anxiety-part-3-meditating-on-anxiety}}

\emph{Posted on March 29, 2017}

Sanjay Gupta, the CNN Chief Medical Correspondent who is also a neurosurgeon, \href{http://www.cnn.com/2017/02/15/health/sanjay-gupta-dalai-lama-meditation/?ref=yfp}{documented the personal instructions on meditation that Dalai Lama gave to him}. The Dalai Lama taught him to meditate by focusing on his worry while imagining isolating it in a bubble. He did this as a daily practice of meditation for a few months and vouched for it having changed his life in a significant way. By putting his worries in a bubble, he allowed himself to worry, but in a way that did not generate a vicious self-perpetuating cycle. With this approach, the worry does not get worse. It can either stay the same or get better. If the worry is inappropriate or disproportionate, it will not only get better by having you see it in a more reality based perspective.

One of the worst aspects of anxiety is that it has an all-pervasive quality. So-called ``free-floating anxiety'' is everywhere which means that one can find it hard to pin-point anything about it. What I sometimes asked my patients to do was to use their imagination to put the anxiety onto some part of their body, such as the chest, or their abdomen. It is like finding a location or a point that enabled them to keep the anxiety in one place. I suggested that they continue, gently but steadfastly, to focus on this point, to look at it and feel it. I encouraged them to keep returning to it if their mind wandered off. I sometimes asked them to move the anxiety just a bit with their breathing; an inch here or there.

Once you've localized some anxiety, just breathe and feel it. Treat it like your friend, stay with it through thick or thin. You can even begin to play with it!

Keep breathing, and keep practicing just accepting this feeling of anxiety -- kind of like accepting the feeling of the weight of your pillow on your abdomen. If the sensation of anxiety starts to fade, bring it back and accept it. Simply stay with the anxiety instead of trying to get rid of it or fight with it. Treat it as your friend. Eventually, almost everyone ends up having difficulty maintaining the anxiety.

Call it whatever name you like. Some call it ``paradoxical intention'', some call it ``reverse psychology, some call it ``mindfulness practice''. Whatever you call it, it is a way to empower yourself to work with anxiety rather than being disabled by it.

I do not rule out that in special circumstances, medicine may still be justified. However, it must not be given without considering all the alternatives -- including using medication as a support for the kinds of self-empowering practices described previously in Part 1 and 2. It is far better to begin to learn how to heal yourself rather than giving up the autonomy which is your own power, and blindly trusting chemicals alone.

\hypertarget{how-to-worry-constructively-part-1}{%
\section{How To Worry Constructively -- Part 1}\label{how-to-worry-constructively-part-1}}

\emph{Posted on April 13, 2017}

In life, one experiences alternating and sometimes mixed emotions of joy and sadness as well as of carefree laughter and deep frowning concerns. While joy and laughter are seldom experienced as problems, when something in your environment triggers an internal alarm, you will start to worry. Worrying can encourage you to focus energy for a task at hand, in problem solving, or in securing a needed level of certainty. However, when worry gets out of control and results in persistent anxiety, loss of sleep and/or ongoing increased blood pressure, it is no longer helpful. At such a point, it no longer helps you achieve a goal. Instead, it self-perpetuates in a seemingly endless loop of stress.

Nowadays, there is a tendency to see every ordinary emotional up and down as a kind of pathological disorder in need of treatment. If you present concerns about your emotions to a busy doctor, it is likely that a magic pill will be prescribed ``to take the edge off'' whatever worries you.

In DID, worrying can transform into an almost perpetual anxiety, which may then get taken to the extreme of crippling hyper-vigilance. Psycho-active medications can be very helpful when worrying has become incapacitating anxiety. Nevertheless, thoughtlessly employing medication may mean that we miss out on opportunities to mature, to grow stronger and to become more self-reliant. My recommendation was always to include psychotherapy as part of any regimen that included psycho-pharmaceuticals.

But, are all kinds of worrying pathological? No.

If your teenager is going out with friends, you may worry and automatically look for specific danger signals. If the driver has had a couple of beers, then your worry crystallizes -- you identified a true predictor of the risk of catastrophe. You may need to actively intervene as a result of this clear and present danger.

Your worry in that example is clearly not pathological. It is a sane response to evaluate known risks of danger. Once we have scrutinized the situation, and ensured that the driver is not drinking then the worry has resulted in reasonable actions of protection. At that point, continuing to worry is a waste of energy.

Having taken reasonable precautions, remaining paralyzed with disabling worries fits into the psychiatric category of ``Anxiety Disorder.'' If that is the case, then yes, you have an Anxiety Disorder. This leads to follow-up questions, ``Are drugs the only treatment? Are they necessary?''

When necessary, I sometimes prescribed medication \emph{but only as an adjunct to psychotherapy}. A principal argument against pharmacological treatment alone is that while drugs can ease your mental tension for a short period of time, they take away your autonomy and the possibility of self-empowerment. Taken alone, medications can lead you down the path of chemical dependency.

Training yourself to deal with tension solely with medication does not allow for correcting and refining the balance of your innate alarm system or for modulating your responses to those alarms as appropriate. For example, if you hear a car horn honk some distance away, you notice the sound and scan to identify what is going on. You do not jump and run in a panic. If you are jay-walking and you hear a car horn honk very close, then the appropriate response may indeed be to jump and run out of the way.

Life is full of obstacles for everyone. You will likely meet situations in the future that are similar to what is triggering the anxiety. Relying solely on medication without embarking on the necessary internal re-calibration work of psychotherapy, brings only an even stronger reliance on the drug! It is like starting on sleeping pills to put you back to normal sleeping pattern. Once you find them helpful, you start worrying that you will need them again. But this time the worry is truly justified! Why? Because you have not used the situation to learn about and process the root causes of the obstacle you face. This is the task, this is the essence of the healing journey.

\hypertarget{how-to-worry-constructively-part-2}{%
\section{How To Worry Constructively -- Part 2}\label{how-to-worry-constructively-part-2}}

\emph{Posted on April 13, 2017}

It is clear that some worries are helpful while others are not. Let's call the first group ``constructive'' and the second ``destructive.'' When worrying is helpful, justified in the circumstances, it is constructive. When worrying is not helpful, not justified in the circumstances, it is destructive.

We often cannot separate one from the other. But, we need to distinguish them for our own well-being. So, it is important to learn how to determine whether or not our worry is constructive.

When a patient came to see me about his paralyzing anxiety, I applied the paradoxical intention concept as used in logotherapy, developed by Victor Frankl. I asked him to sit still for a few minutes before telling me all the things that were bothering him. I asked him to prepare himself by organizing his thoughts and reviewing what was bothering him. In other words, I was asking him to allow his worrying to come out without contesting it.
When a patient came to see me about his paralyzing anxiety, I applied the paradoxical intention concept as used in logotherapy, developed by Victor Frankl. I asked him to sit still for a few minutes before telling me all the things that were bothering him. I asked him to prepare himself by organizing his thoughts and reviewing what was bothering him. In other words, I was asking him to allow his worrying to come out without contesting it.
As a result of my request, he gave himself permission to worry. First, appearing somewhat subdued and dull, he sat down. Then, he focused on the issues of concern without the internal turmoil of worrying about the fact that he was worrying. After a few minutes, he opened his eyes and said that what he was worrying about turned out to be of no real significance. To my surprise, he did not even think he needed to tell me about it further! His body language clearly indicated that his mind was lighter, unburdened by the anxiety. Simply allowing his worry to come out in a safe environment had empowered him to solve his difficulty with no other assistance.

Mind you, there were important factors at work in the interaction. Remember, the therapist's communication in non-verbal language way can carry as much weight, if not more, than the spoken word. So, my body language and verbal communication were in accord with each other -- entirely respectful, rather than being dismissive of his concern. It was clear that I took his complaint seriously.

My interpretation of this event was that when he was encouraged to go ahead and explore his concerns in a safe and nurturing environment, he was free from the paralysis of being unable to determine whether he should or should not worry. Without that internal tug of war, without that self-criticism, he was able to focus on and process his concerns without getting in his own way.

\hypertarget{working-with-self-victimization}{%
\section{Working with Self-Victimization}\label{working-with-self-victimization}}

\emph{Posted on April 17, 2017}

I received a request from a reader about a major issue encountered by many with DID, self-victimization. In my experience treating DID patients, self-victimization and self-harm seemed to be the rule rather than the exception.

Survivors of early childhood abuse often get attached to successive abusive partners, one after another. It is one of the most unfortunate aspects of the abuse cycle that one goes out to search for an abuser to complete the abuser-victim transaction sketched out in the Karpman Triangle (also known as the Drama Triangle) of the abuser-victim-rescuer socio-gram. Survivors also have similar dynamics within their systems, with different alters playing out similar roles on the inside.

This happens for a variety of reasons, with many different excuses and rationalizations. But rather than respond to the internal logic (or lack of understandable external logic) in the self-harming conduct with arguments, the key point to understand is that healing arises from connecting directly with the alters who are willing to engage, and warmly/gently/kindly inviting others to try to engage.

What does that actually mean? Whether you are seeing a therapist or you have an alter within the system helping out, patiently connecting with a punitive alter (or alters) who wants to inflict pain is critical. Honestly generating empathy for that angry punishing later is key -- faking empathy will make things worse. You can generate real empathy by remembering that such alters arose for the same reasons that every other one did -- dealing with the direct impact and aftereffects of horrific trauma.
Begin by telling him/her that you understand the feelings of anger, desperation and pain. Be honest if you don't understand the self-victimization. In that case, you can ask for guidance on how you can better understand his/her view. Gently suggest that the trauma is in the past, but do so by encouraging the alter's connections with the present. It can be done by identifying the current date. If you are not in the area where the abuse occurred, remind them of where they are right now. My office window overlooked a bridge so I often suggested to my patients that they look at the window and identify that bridge so they could note that the abuse did not occur in this particular city, in this present time.

Physical cues are also helpful. By asking him/her to take a deep breath with you, he/she may then experience even a moment of distance from that traumatic experience. Again, it is encouraging the realization that the trauma can be viewed from a distance in time and space. In other words, it is inviting the experience of safety in the present. It is putting the flashback where it belongs -- in the past. It is a memory, not a current experience even though it seems like it is happening now because your body is having the same response of fear, panic, hyper-vigilance and a pounding heart. This is the therapeutic process of disengaging from a flash back.

It is just as important to connect with the alters that are being harmed inside -- not just the ones doing the harm. Remember that while some alters seek to do harm, others feel the need to be harmed. It may be phrased by those alters that they deserve to be harmed or that there is no other option. Both need empathy, compassion and understanding as well. Within an environment of understanding, empathy and compassion, the suffering alters will be able to make a shift.

Help those alters in conflict find a safe place to process and heal the wounds. Encourage others in the system to help modulate the conduct of the angry harmful alters, to become their friends. Encourage those that are being victimized inside to connect with other parts of the system that can act as their protectors -- not from the point of view of fighting with the other angry alters but rather as a bridge to understanding each alters particular suffering, generating empathy for each other and a spirit of healing teamwork.

For example, the soft and comforting voice of the therapist, or your self-appointed internal therapist, can shift the alter who is still stuck to the past traumatic experience back into the present moment of safety. The pathology is that the particular alter will remain tenaciously stuck to the past trauma. The therapist, no matter from inside or outside, has to persevere to achieve a small, step-by-step separation from the past trauma into the present. It is difficult but this is often the very central practice in DID therapy.

I want to be quite clear that in this I am directly addressing an alter, treating him/her as my client and giving them my full attention. Some therapists prefer to speak only to the host. In fact, colleagues have criticized my approach as reinforcing the splitting aspect of DID pathology. As I have written elsewhere, fish swim, they are not taught to swim. In that same way, DID individuals dissociate, they are not taught to dissociate. For fish, to swim is instinctual. For DID individuals, to dissociate is similarly instinctual -- it is their instinct for self-preservation in the context of massive early childhood trauma.

In my experience, a large part of DID therapy is one-on-one psychotherapy directed to individual alters. If you have an internally appointed self-therapist, working as an adjunct to your therapist or alone because you are unable to find a therapist, one can encourage the same approach of communication between the inner therapist and the self-abuser. Knowing that the self-abuser is acting out unprocessed trauma allows one to communicate always with respect and always with kindness.

\emph{The Karpman Triangle is found on page 126 of Engaging Multiple Personalities Vol 1. It is also displayed by Joan, in Chapter 1, who struggled with this almost every night for months.}

\hypertarget{weighted-blankets-and-the-sensation-of-safety}{%
\section{Weighted Blankets and the Sensation of Safety}\label{weighted-blankets-and-the-sensation-of-safety}}

\emph{Posted on June 25, 2019}

There is a general bias in identifying a pharmaceutical agent as a better, effective, or more scientific way to overcome a symptom than a natural method. In addition, it is extremely difficult to apply for a research grant to prove that such a natural method may be better than a pill for insomnia. Certainly not from the pharmaceutical industry! So, I doubt if grants are available to study how effective a weighted blanket is compared to a pharmaceutical agent for insomnia. On the other hand, there are now many anecdotal reports on the benefits of weighted blankets.

When I first read of the use of weighted blankets as an aid for sleep, decades ago, I thought it would make sense if it helped people with insomnia. I seldom have insomnia, so the idea of purchasing one never came to me.

Recently, I read some postings in DID Facebook groups that reawakened my interest. I also found out that weighted blankets have become a common commodity. I just stopped into a shop that specialized in things that promote sleep and bought a weighted blanket. It is 15 pounds; the recommended weight for someone my size. I wanted to explore the first hand experience of using one. Here is my report after using it for 3 weeks.

In the beginning, I felt mildly resentful because I felt restricted in my movements due to the weight. It seemed to be a hindrance to moving around in bed. I quickly realized that what I was seeing as an impediment was a mistaken understanding. It was an obstacle to manifesting my agitation physcially in bed. But when I stopped fighting that sensation, it seemed that what it was actually generating was the sensation of safety a baby might experience being ``tucked-in.'' So, I imagined I was a baby in a ``wrap.'' Sleep came over me soon. True to my expectations, I did experience a positive feeling as if I were being held and hugged while under the heavy blanket.

Three weeks have gone by. My sleep, in general, feels deeper. The number of times I wake up to use the toilet has decreased. In the evening, I look forward to the experience of going under the heavy blanket. I feel more refreshed in the morning. I cannot rule out that it might just be a placebo effect. Placebo or otherwise, with the blanket, I quickly settle into sleep regardless of any sense of resentment at the restriction caused by the weight.

\textbf{Conclusion}

\begin{enumerate}
\def\labelenumi{\arabic{enumi}.}
\item
  The simplest way for me to fall asleep has always been to be still, to stop tossing and turning. This heavy blanket is like a gentle reminder to me just to keep still. I believe ``tossing and turning'' is the most commonest reason people who have difficulty falling asleep get into a pattern of insomnia. It is like the phenomenon of scratching an itch: The more one scratches, the more itchy one feels. Tossing and turning make people more restless, which makes them toss and turn even more.
\item
  I can feel the sensation of being tucked-in and held. There is an immediate shift from the physical sensation to the emotional.
\item
  Some people may resent the sensation of restriction of the blanket stopping you from freely tossing around, as I did initially and still do to a lesser extent. It is easy to get over this, because the sensation of being safely tucked-in, that good feeling, quickly takes over.
\end{enumerate}

Considering this from the viewpoint of someone with DID, how might this be helpful? For someone with early childhood trauma, the sensation of being touched is often frightening -- the opposite of being safely tucked in. This early trauma impacts those with DID for decades into the future. So, a question in therapy is how can the therapist help a patient re-learn the experience of that sensation of safety without touching the patient? Further, how can the patient experience safety at home, perhaps at night when it might be most needed? Knowing the difficulties DID patients often experience with close body contact with their partners, how can a partner help engender that experience of safety in a way that eliminates any trigger of sexuality? And finally, how can someone with DID that doesn't have a partner experience that sense of safety alone in their home.

I want to be clear that I do not believe there is \emph{any} approach to DID that will successfully address the problems of all individuals with DID. However, given that different approaches can be helpful to different patients, I thought it was both interesting and possibly important enough to post these thoughts. If readers of this post have read my books on DID, the Engaging Multiple Personalities Series, you will know that I believe that re-learning the experience of safety in the here and now, the experiential sensation of being safe, is a critical component in the healing journey. The question for therapists is how to deliver that sensation.

I have written in my books about using a large bolster cushion to push lightly against a patient's chest to have them experience the sensation of a grounding safe touch without me actually touching them directly. There was always the large bolster between us both for ethical concerns and for avoiding potential triggers. The weighted blanket seems to serve a similar function, but can be used at home, alone or with your partner, and is in your control which has the additional benefit of self-empowerment.

I told the spouse of one of my former patients that I wanted to experiment with this. He wrote back letting me know they had purchased one already. He wrote that the first night of use his spouse was irritated with the weight of the blanket. She commented that she really didn't see any difference and didn't like it particularly. Nevertheless, without prompting, she has used it every night since. No longer getting up repeatedly to use the washroom, no longer waiting for the spouse to fall asleep first before drifting off herself to sleep, and other positive results have happened.

So, in short, I think it may be of benefit. There are many versions and at different price points. I am not well versed enough to comment on which weighted blanket might be better than some other one. If people have tried this and had a negative reaction or no positive reaction, please post that as well. It may be helpful for others in the DID community to get a sense of whether or not it is something that has a likelihood of being helpful to them.

I do have to add a disclaimer: I do NOT hold any commercial interest associated with sleep promoting products whatsoever. To my knowledge, neither my former patient nor her spouse has any economic interest in weighted blanket companies.

\hypertarget{the-sensation-of-touch}{%
\section{The Sensation of Touch}\label{the-sensation-of-touch}}

\emph{Posted on June 30, 2019}

In the previous post regarding weighted blankets, I spoke about the importance of experiencing safe touch and that the use of a weighted blanket may, for some people, be a safe way to re-learn that experience. Most people usually think of touch as a pleasant or painful sensation, but rarely a highlight of life in the absence of the heightened experience of touch related to sexuality or pain. Touch, more than merely the interface between our bodies and the outside world, is probably the most misunderstood sensation.

The sensation of touch may be differentiated into several categories: light touch, deep sense of pressure, temperature and pain, vibration, proprioception or sense of position in space (such as joint sense ) and others. It is said that there are 20 different types of highly specialised receptors associated with touch. These are sensory neurons found in all parts of the body except the brain. They vary in density and sensitivity to stimuli.

Touch is a fundamental part of our daily experience. The sense of touch gathers information about our surroundings as well as being a means of establishing trust and social bonds with others, both people and animals. It is crucial to creating our unique human experience. No wonder we use phrases such as calling something a ``touching experience'', saying someone is ``touchy'', or feeling ``soft-hearted.'' We often use the word ``feeling'' to reference emotional states rather than solely the sense perception.

The human brain has two distinct but parallel pathways for processing touch information.

\begin{enumerate}
\def\labelenumi{\arabic{enumi}.}
\item
  The first pathway is sensory, which conveys some dry facts: vibration, pressure, location and fine texture. It can tell you if someone is stroking you, up or down your arm. That part of the sensory pathway is a brain region called the primary somatosensory cortex, which is the first region to be triggered by the experience of touch.
\item
  The second pathway processes social and emotional information. This pathway identifies the emotional content of mostly interpersonal touch using different sensors in the skin. This pathway activates brain regions associated with social/psychological bonding, pleasure and pain centers.
\end{enumerate}

Touch is critical for child development. This is something researched for many decades, such as Harlow's experiments on monkeys. We know a parent's touch, whether positive or negative, is a crucial factor in a child's development.

Most people wouldn't have difficulty distinguishing between a friendly touch of social support and a touch involving sexual suggestion or seduction. An arm around the shoulder coming from a person will change the way you experience that touch based on your relationship with that person. Our brain processes the sensorial experience with information about the social context from other parts of the brain. Usually the social context enables us to tell whether the gesture is genuine or insincere, whether it is straightforward or perhaps cloaks a hidden agenda. Identifying a gesture as insincere or containing a hidden agenda then appropriately triggers the need for further investigation.

\hypertarget{therapeutic-touch}{%
\subsection*{Therapeutic Touch}\label{therapeutic-touch}}
\addcontentsline{toc}{subsection}{Therapeutic Touch}

A large body of research suggests that therapeutic massage can be helpful for a number of physical and mental ailments. These include pain relief and addiction recovery, as well as maintaining emotional equilibrium, cognitive function and mobility among an aging population. Others have also suggested that massage may be an effective way to treat anxiety, insomnia, headaches and digestive problems.

The popularity of therapeutic touch has not received its due recognition. This is partly due to the dominant status of the pharmaceutical approach and partly due to the lack of vigorous scientific proof of its efficacy. However, those with trauma in their background must remain aware that it may be difficult to identify the very moment that touch may change from a healing procedure into a sexual transgression. Always be as clear as possible about your personal boundaries and in maintaining them. It may be a good idea to tell your massage therapist at the very beginning that you have very strict boundaries that must not be crossed. Doing this may avert the danger of the massage therapist inadvertently or even intentionally blurring/crossing the ethical boundary between the therapist and the client being touched.

I am especially interested in establishing healthy self-soothing practices in our daily routines. Self-soothing happens when we need to be soothed. We use all kinds of methods to be soothed -- some are safe, some are unsafe and some are neutral, depending on how they are applied. Given the general phenomenon of addiction related to nicotine and alcohol along with the uncontrolled use of comfort foods (especially sweetened foods), and their attendant unhealthy consequences of obesity, diabetes, high blood pressure etc, safe self-soothing options are critical. Advertising for alcohol, nicotine and processed foods are designed to seduce us without concern as to their negative attributes.

In my experience with DID patients, grounding and self-soothing are part and parcel in healing those who were subjected to early childhood trauma. Helping them re-experience the literal sensation of ``safe'' was critical to therapy. Learning to self-sooth, the self-empowering process of being able to generate that sensate of safety is an essential part in healing.

It is in this context that I wrote an earlier post on the therapeutic potential of weighted blankets. I think any self-soothing practice that does not cause harm is worth very serious consideration. Consider comparing the negative aspects of self-soothing with drugs or alcohol (or any other addictive behavior) with simply resting under a weighted blanket. If the weighted blanket works for you then there is no contest.

\hypertarget{countering-the-far-reaching-effects-of-humiliation-part-1-disempowerment}{%
\section{Countering the Far Reaching Effects of Humiliation Part 1 -- Disempowerment}\label{countering-the-far-reaching-effects-of-humiliation-part-1-disempowerment}}

\emph{Posted on September 19, 2019}

To humiliate someone is to make them feel ashamed or stupid, to make them feel like they have lost, and are undeserving of, the respect of other people. Humiliation is a common tactic abusers employ to subjugate a child. It is accomplished by debasing a child's status, through breaking their spirit and pushing down their ego to the point of abject submission.

Humiliation is always connected with a power imbalance. In short, an abuser is communicating that ``I am stronger than you/I have authority over you/I will overpower you. Therefore, you will submit to me/my wishes/my demands.''

The result of humiliation is dis-empowerment. And, as with any experience of dis-empowerment, the consequences ongoing traumatic humiliation are far reaching.

Michael J. Fox said, ``One's dignity may be assaulted, vandalized and cruelly mocked, but it can never be taken away unless it is surrendered.'' For an adult, there may be a choice to not surrender, or to surrender only in part. An adult can decide to fight it out, or to maintain a spirit of rebellion when there is no possibility of physically fighting back. In other words, an adult can choose not to surrender (in spirit) or to physically fight back, rather than to silently accept defeat and fully accept the role of being a victim. An adult may, perhaps, blame him or herself. They may feel guilty and deserving of the humiliation, but they have adult tools to fight back with or to work with those feelings.

Now imagine the gross confusion and bewilderment that arises when humiliation is heaped upon a child. Imagine, as a child, being forced to participate in the obscene act of being sexually violated, being physically beaten, or being otherwise abused, and to accept that humiliation in a spirit of submission. Escape through dissociation is likely the only logical or even possible outcome.

For a young child being abused by an adult or older child, surrender is not a choice but is rather an inescapable outcome. For very young children, survival through submission is the only option. When early childhood abuse and humiliation is a repetitive experience, dissociation becomes the default response -- regardless of age.

Keep in mind that humiliation is not the sole goal of the abuser. Crushing the spirit of the victim is part and parcel of establishing the power dynamic that permits the abuse of that child in the future when and as the abuser may wish while limiting the possibility of genuine push-back from, or exposure by, the child. Humiliation enables this by the piecemeal or violent erasure of personal boundaries by the abuser, normalizing the humiliating conduct and eroding further a child's sense of having any place in the world.

Many of the difficulties people encounter in daily life can be traced to experiences of humiliation; in adults, in children and in society. The impact of humiliation seems to include a baseline, different for each individual, beyond which a person will be unable to pull back from its clutches. When a person's spirit is crushed in this way, unremitting depression sets in. The choiceless acceptance of humiliation is often followed by powerless rage.

Consider the intensity of that choicelessness coupled with powerless rage. In so doing, one might get a sense, a glimmer, of the importance of angry alters. They keep open both the chance of survival and the healing potential of the system. The trapping of rage in those angry alters keeps the system alert to identifying potential dangers, albeit often hyper-alert with its attendant difficult consequences. At the same time, trapping that rage in the angry alters allows dissociated submission that enables survival while experiencing abuse.

Humiliation can often be a non-physical assault. Some people may think that because it is not necessarily physical, it therefore is not so big a deal. However, looking forward, it can have the extraordinarily destructive result of setting up an abused child for ongoing future assaults both psychological and physical.

\hypertarget{countering-the-far-reaching-effects-of-humiliation-part-2-elements-of-humiliation}{%
\section{Countering the Far Reaching Effects of Humiliation Part 2 -- Elements of Humiliation}\label{countering-the-far-reaching-effects-of-humiliation-part-2-elements-of-humiliation}}

\emph{Posted on September 22, 2019}

Humiliation is not a defined term in the DSM 5, although it is used in a few isolated instances. This presumes a common understanding of humiliation which is unlikely to be as common as people may think. But, like other forms of trauma, particularly those lacking an initial specific motivation occurring out of callousness or lack of empathy, the way humiliation is experienced by the one harmed dictates their future pathological responses.

Generally speaking, there are 3 elements to humiliation in the abuse context:

\begin{enumerate}
\def\labelenumi{\arabic{enumi}.}
\item
  Denying the status of the victim through a subjugation that undermines the pride, humanity or dignity of that person.
\item
  Reducing the victim to passivity as the method for rendering them powerless. It uses a gross power imbalance in subjugating the will of the victim, of even their experience of self-hood.
\item
  Violently destroying the personal boundaries of the abused, leaving a damaged psyche. The end result is the decimation of their self-confidence. The person is dis-empowered, often with life-long disabling consequences.
\end{enumerate}

Stepping back, consider the role humiliation plays in the case of corporal punishment. Caning in schools, or in the family, immediately establishes a hierarchy of physical power, with the one who administers the punishment over the one being punished. For the child, if inwardly rebellious and able to silently remain angry, their buttocks may be bruised or scarred, but damage to their psyche is mostly spared. In those circumstances, humiliation is countered by refusal to identify as a powerless victim. Such a child faces punishment with a fighting spirit, rather than surrender.

But contrast that with a critical factor in DID etiology -- that the abuse occurs at an extremely early age. For those with DID, humiliation in connection with abuse (physical, sexual and/or emotional) often occurs before the child is old enough to have established a psychological structure with enough stability to even envision fighting their abuser. It is here that the real damage to the child happens, when there is such an intense subjugation as to prevent the child from establishing the foundation for any sense of safety in life. At the same time, the dissociative response often enables the arising of angry alters whose importance to healing is critical, as is discussed in Part 4 of this extended blog post.

\hypertarget{countering-the-far-reaching-effects-of-humiliation-part-3-dignity-humiliation-respect}{%
\section{Countering the Far Reaching Effects of Humiliation -- Part 3 Dignity, Humiliation, Respect}\label{countering-the-far-reaching-effects-of-humiliation-part-3-dignity-humiliation-respect}}

\emph{Posted on September 22, 2019}

Dignity is the state or quality of being worthy of honor or respect. It is the inherent right of people to be treated with dignity. From a religious point of view, dignity may be seen as God's gift to each individual. From a secular point of view, dignity can be seen as one's human right to act and have their own agency in the world based on the simple fact of their human existence.

Dignity is displayed in a calm and controlled demeanor. But, it can be harmed through a humiliating experience or crushed through repeated humiliations. Dignity is a sense of pride in oneself, of self-respect. The polar opposite of dignity is humiliation.

Humiliation is the crushing of dignity by an outside agency -- in the case of DID etiology, by an abuser attacking a young child. Unfortunately, dignity and humiliation are usually outside the language spoken by psychiatrists, or mentioned in diagnostic formulations.

Humiliation is mentioned a few times in the DSM 5, but not in the context of DID etiology. But, it remains undefined in the DSM so far as I have been able to determine. It is used (on page 703) in this way: ``Sustained feelings of shame or humiliation and the attendant self-criticism may be associated with social withdrawal, depressed mood, and persistent depressive disorder (dysthymia) or major depressive disorder.''

This quote from the DSM infers that shame and humiliation are interchangeable terms, but that is not the case. They are not identical. Nevertheless, it should be instructive to clinicians that the listed symptoms resulting from sustained (or one might say repeated) such feelings are often presented by individuals with DID. Despite this, such symptoms are instead usually seen as pathology markers on their own rather than in the context of a potential DID diagnosis.

While humiliation and shame both make a person feel bad about himself, it is important to distinguish between them. Humiliation is always provoked by someone else, while shame is connected to one's own actions or simply chance circumstances. With shame, people mostly focus on themselves and how others might perceive them. With humiliation, there is the added traumatic factor that the other person is intentionally causing them harm. Abusers often seek to instill shame in those abused by blaming them for the abuse itself. This is true in early childhood abuse, in spousal abuse and in other abusive circumstances.

In short, we bring embarrassment upon ourselves and may feel ashamed as a result. But, humiliation is brought upon us by others. From a therapeutic point of view, it is clearly an abuser's assertion of power over a child that cuts far deeper, leaving both scars and open wounds which show up as triggers in the future. Because humiliation is traumatic, it is kept hidden by the one humiliated while being simultaneously used as a weapon by the humiliating abuser. Fundamentally, humiliation involves abasement of pride and dignity, along with a loss of status both personally and socially.

Respect is something earned through one's actions. Self-respect is a state of mind that is founded upon pride and confidence in oneself. It is a feeling that expresses itself through behaving with honour and dignity. Self-respect means proper esteem or regard for the dignity of one's own character. This self-respect is part of both the path and a marker of healing.

We can easily trace many negative character traits to their origin in a loss in dignity, in those with DID and others whose negative conduct does not rise to the level of being pathological. This can happen when a child is under assault in the form of bullying or massively disproportionate and severe punishment. The result may be perpetually defending oneself even when one is not under attack, in excessive one-up-man-ship. It may show up in excessive social competitiveness, aggressive or even abrasive personality or social phobia and excessively passivity. Mistrust and paranoia can often be linked to pronounced early childhood humiliating experiences.

Alternatively, it can result in a child developing an overwhelming passivity. In the face of ongoing humiliation, a child may internalize the message of the abuser that the child has no ability to defend themselves -- even internally. In effect, such a child may end up adopting an abuser's weapon of humiliation as an adaptation of survival. By giving up any fight, the child survives another day with the abuser.

As noted in Part 1 of this extended sequence of posts, because humiliating experiences are not necessarily physically overwhelming, they may not be seen by an outside person -- therapist or other adults -- as being genuinely traumatic. This is a tremendous mistake. This kind of humiliation, this abusive power dynamic, is often no less damaging than physical trauma. But, it is easier for a therapist or other adult to ignore because the evidence does not show up externally -- at least not immediately -- the way one can see a broken arm or the bruise from a punch.

One must consider instead the fact that the damage may show up in the future as violence directed inwardly as self-harm or outwardly against others. When it appears primarily as a psychiatric morbidity, as depression for example, therapists as well as patients may miss the possibility of humiliation as a causative agent. The result may then be medication to suppress the depression rather than helping the patient process the early-childhood psychological trauma through therapy.

When anti-depressant medications don't work, it may lead to a diagnosis of ``treatment resistant depression.'' I don't consider treatment resistant depression to be an accurate categorization. Rather, in the current environment of prescribing antidepressants as the primary method of treating depression, it should be seen as drug resistant depression. When medication doesn't treat the cause, and instead solely treats the a symptom, the cause remains intact. If the cause remains intact, it will continue to manifest in some way, shape or form despite the medication. The unfortunate result can often be over-medicating patients to the point that the medication causes dysfunction separate and apart from the cause of the depression.

With unresolved early childhood trauma, the antidepressants may have limited benefit but that does NOT mean that you should just stop taking them. Instead, work with your doctor to have meaningful psychotherapy with the antidepressant as an adjunct to therapy rather than the principle method. With proper psychotherapy, as you heal from the trauma, the medication should be able to be successfully and safely reduced. Because suddenly stopping a psychoactive medication has potentially quite a bit of risk, if you are on antidepressants, only stop taking them under your doctor's guidance.

\hypertarget{countering-the-far-reaching-effects-of-humiliation-part-4-the-power-of-angry-alters}{%
\section{Countering the Far Reaching Effects of Humiliation Part 4 -- The Power of Angry Alters}\label{countering-the-far-reaching-effects-of-humiliation-part-4-the-power-of-angry-alters}}

\emph{Posted on September 24, 2019}

The best indicator of a positive prognosis for those with DID is found in those with defiant angry alters. In effect, it is those parts that say ``You have no right to humiliate me. I will not surrender to your will. You have not subjugated me. I will fight you always.'' The implication for those treating DID patients is to remind those patients that their angry alters are generally the ones that refused to simply surrender to their abuser. Even though they may only have the initial capacity to express their anger in ways that are frightening to others both internally and externally, a path forward to healing can be found through engaging with them in therapy.

Engaging the angry alters is the opportunity to access that positive defiance, that refusal to accept humiliation as defining them. Appreciating their strength and insight is a genuine method to develop support within the system so that alters can begin to work in concert rather than in conflict. In my practice, the patients with the best prognoses were those that were able to connect with their anger -- which often meant engaging with those angry alters again and again. By ongoing engagement in that way, one invites their assessment and potential trust in the therapy.

A common and negative outcome for the victim is submitting to the punishment without harboring some internal rage. In short, succumbing to the abuser's humiliation of them.

There are several possible changes in the personalities of people who emerge from significant childhood humiliation experiences. They range from inability to relate to others which may appear as awkward socialization to severe psychopathic behaviour.

In the worst scenarios, the victims of humiliation -- in the case of DID it may be one or more alters -- over-compensate. As noted in the previous post, they may develop a powerful urge to gain personal power to control all social interaction. The drive to control can be so strong that it eliminates any sense of sympathy and compassion for others -- including other alters -- as well as when interacting with other people. Extreme levels of self-protection take over. It is this manifestation that makes angry alters both powerful and difficult. But, please don't see those alters as identical or inextricable with their difficulties. To do that will mean that you miss their capacities as keys to healing.

For many with DID who have had their spirit seemingly crushed through humiliation, instead of acting out for revenge externally, that rage against powerlessness is turned inward. Chronic depression may be coupled with generalized fear with the loss of self-confidence as the outcome. Social relationships, including familial and marital, are compromised because of deep inherent mistrust. To heal this, and it is possible to heal this, calls for powerful transformational changes.

Once again, humiliation crushes the child's spirit. It is intended to undermine any possibility of self-confidence and to infuse the child with fear. It impacts the child by giving them a twisted perspective of human relationships. The result is often to eliminate the capacity for genuine intimacy. It attacks that human capacity for intimacy by convincing them to distrust all relationships that might appear safe.

It does so by convincing them that they will never be safe, certainly never from the abuser. It seeks to convince them that ``Anyone trying to convince you that they are safe is only presenting an appearance of safety because safety doesn't exist.''

This is a critical barrier for therapists to be aware of and to overcome in dealing with anyone with DID. In my practice, the only path to overcoming that barrier was to respectfully engage the alters including the angry ones and to always gently invite them all to participate in therapy even by simply listening in as I engaged others in the system.

\hypertarget{countering-the-far-reaching-effects-of-humiliation-part-5-healing-from-early-childhood-humiliation}{%
\section{Countering the Far Reaching Effects of Humiliation Part 5 -- Healing From Early Childhood Humiliation}\label{countering-the-far-reaching-effects-of-humiliation-part-5-healing-from-early-childhood-humiliation}}

\emph{Posted on September 24, 2019}

Alice Miller wrote:

\begin{quote}
As long as they are loved, children can recover from abuse and even the horror of war.
\end{quote}

Humiliation is a form of severe child abuse when the child experiences it on a repeated or ongoing basis beginning in their childhood. The path to recovery from humiliation is through love. Love starts from the ability to accept love, from oneself as well as from others. However, it is extremely difficult to practice self-love in the absence of love from others.

The characteristics an individual displays depends upon whether a person is given love, protection, tenderness and understanding or experiences rejection, coldness, indifference and cruelty in the early formative years. These characteristics are not innate but rather are dependent on what stimulus a child experiences. For example, the stimulus indispensable for developing the capacity for empathy is the experience of loving care.

When a child is forced to grow up neglected, emotionally starved and subjected to physical abuse, this innate capacity will fail to develop or its development will be stunted. It is important to appreciate that while the negative experiences of children from infancy to early childhood explain their later behaviour, subsequent positive influences can be effective agents for change.

Alice Miller also wrote that if a traumatized or neglected child can later come to know what she calls an ``enlightened or knowing witness,'' he or she can process the effects of childhood trauma with positive results.

While remaining open to the opportunity to experience love, or positive influence, one should continue to pay attention to one's boundaries and protectors. At the same time, pay attention to the following in sequence:

\emph{1. Become aware of the connective link between your styles of engaging with others and your childhood experiences of humiliation.}

Repeatedly noticing, and paying attention to the causal connections, is the beginning of making changes. The more you pay attention to this, the more you come to realize that you are not alone. You will see that such experience (of humiliation) is a human drama played out unfortunately and repeatedly every day in so many situations for so many people. Looking at it this way, one begins to transcend the isolating aspect of humiliation's personal pain and hurt -- you are not alone. Eliminating that isolation is another foundation of healing.

One can cultivate this through cognitive restructuring. Reminding yourself of this in a daily quiet time. You can set up a regular time to do this, such as going out for a walk in the morning on a definite consistent schedule.

And again, remember, you are not alone.

\emph{2. Learn to distinguish the past from the present.} If you are standing on the bank of the mighty Amazon river and take two pictures a minute apart, each photo shows different water. The water in the first picture has already moved on towards the Atlantic, replaced by entirely different water -- even if it looks pretty much the same. In just that way, we are not exactly the same person as we were a minute ago.

This shows that the future is not exactly the same as the past. Use that truth to healthily correct the hangover of feeling humiliated in the past. You can do this by training to focus on the present moment. As Tolstoy wrote, ``Remember that there is only one important time and it is Now. The present moment is the only time over which we have dominion.''

By recognizing the impact of one's difficult personal history and bringing one's awareness to focus on the now, we can begin to wipe out the negative influences of the past.

\emph{3. Protect others from humiliation, particularly children.} Should you come across a child being humiliated, or perhaps on the edge of being humiliated, step in and say exactly what you would have wanted to hear so many years ago. At the same time, be protective of yourself as well and, if necessary, call the police or child protective services to the situation. For just that moment, be the protector for a child in the present that you needed in the past.

Learning that you can protect a child from humiliation is a path to healing for yourself. While you cannot travel back in time to when you were humiliated to undo the impact of the humiliation, you don't have to. If you see a child being humiliated or abused, you can help that child right then and there. You will be letting that child know they are not alone, that there is protection in the world. You will also be giving that very same message to the child you were years ago.

\hypertarget{engaging-alters-in-conflict}{%
\chapter{Engaging Alters in Conflict}\label{engaging-alters-in-conflict}}

\hypertarget{working-with-angry-alters}{%
\section{Working with Angry Alters}\label{working-with-angry-alters}}

\emph{Posted on April 23, 2015}

This is in response to several postings on Facebook about potentially dangerous alters. These are angry alters that may harm the body or harm others. This touches on some very basic and frequently misunderstood issues pertaining to DID therapy.

The original function of the angry alter is protection. It is an ingenious defense mechanism for an abused child to establish a self-protective function when they are faced with repetitive abuse that often extends over years. Without that protective function, it is unlikely that a child could survive such impossibly difficult situations. It has the aspect of asserting power, that the child is not solely a victim. There is at least one part that is still fighting the abuser.

The angry alter is \emph{not} the enemy. These alters arise from a deep survival instinct, filled with power and energy. Without these alters, the trauma would likely overwhelm the child -- or, later in life, overwhelm the DID system. These alters keep the system alive within the context of and following the trauma. Without connecting to that energy, the prognosis in therapy is not good. The likelihood of the system simply giving up increases tremendously. In my practice, I had patients who were unable to access the energy of those alters and therapy was, fundamentally, a failure. The key point is to work with the energy, with the alter, rather than seeking to eliminate it. Far from being the enemy, these alters are potential partners in healing the system.

As a result of the hypervigilance that results from ongoing trauma, the anger that arises as that alter is often directed towards other alters or the host. This is despite the fact that the dissociation and resultant alters arose because there was no other way to survive the abuse. They usually blame the host or other alters for ``allowing'' the abuse to take place. This mistake in perception by the angry alter can lead to debilitating internal conflict. That same anger can also be turned on anyone outside that the angry alter might see, presume or experience as threatening -- including the therapist.

The therapist must be sensitive to the presence of the angry alter(s). An alter's subtle but definite show of power in a threatening manner is often discernible to the alert therapist -- just as it would be in treating any non-DID patient.

Early in therapy, as soon as I had confidence that DID was the correct diagnosis for a patient, and regardless of whether or not I communicated the diagnosis to the patient at that time, I stated aloud that in order to proceed with therapy I needed the patient to understand and agree that they could not seek to frighten or threaten me. Without that agreement, one cannot proceed with therapy. This is because a proper therapeutic alliance cannot be established if the therapist has concerns about their own safety.

I would inform the DID system that if I felt unsafe, I simply could not be an effective therapist. I would make that statement while concurrently expressing appreciation for the protective function the alter was fulfilling. This is an honest approach that was much appreciated by my DID clients -- particularly when that message was coupled with the message that you appreciated -- and all the other alters should appreciate -- that the function of the angry alter was to enable the system to survive at the time of the original trauma(s).

Following that, whenever I sensed that an angry alter was around, I would seek to engage that alter directly. This is a priority. Genuinely, always genuinely, I would thank the alter for having protected the system in the past. I let the alter know that it is good that they are keeping an eye on me, the therapist. Acknowledging this -- because it is true -- is telling the alter that it is no longer necessary to try to instill fear in me as a protective shield. This was because their function, along with the DID system in general, was now safely in the open. The system and all the alters within it were within the container of compassionate therapy. That was further assuring the alter that between the two of us, therapy could be conducted in a safe and secure manner.

I would invite the angry alters out if they were willing to engage me, but I would \emph{never} provoke them to come out. I would point out that they needed to remain vigilant to continue to protect the system -- definitely encouraging them to keep their watchful eye on me -- but that being hypervigilant was not so helpful. Being watchful without being hypervigilant was the healthy quality of their protectiveness. It was something to be maintained and applied to the other alters as well as to people they might encounter in their daily life. In this way, they were invited to reclaim their original role as a guardian.

Generally speaking, prior to DID therapy, alters have not been recognized, acknowledged or appreciated. When directly engaged in communication, they have the capacity to change. Like any patient, they appreciate the experience of being treated with kindness and dignity. In most cases, over time, they understand and change their protective view from one of hypervigilance to appropriate vigilance.

Unfortunately, many therapists take the opposite approach. There is a general reluctance to engage alters for various reasons, especially angry alters, including fear and the consequent denial of alters. It is the therapist's fear that cuts off communication and solidifies the mistaken view that the angry alter is the enemy. They are potentially potent collaborators in healing. On looking back on my decades of experience treating DID, I still find cases where I wish I had taken a more direct approach to engaging the alters, particularly the angry ones, in therapy.

Alters behave like other patients in therapy. They get relief when encouraged to express themselves and feel reassured when they are understood. Once the hypervigilance is transformed into vigilance, they respond to reason and very often make appropriate changes.

DID patients can heal, even after years of neglect and/or abuse. I hope that DID individuals read this so as to gain confidence in the importance of making friends with all their parts. I also deeply wish that therapists consider these points so that they may overcome their reluctance to engage and learn from alters.

I have written Engaging Multiple Personalities Volumes 1 and 2, and continue to write this blog, in retirement. It is my opportunity to reflect back, to acknowledge my past mistakes in my practice, and to offer my painfully learned experience to others so that DID individuals and their therapists can further and quicken the healing process.

\hypertarget{working-with-despair-and-anger}{%
\section{Working with Despair and Anger}\label{working-with-despair-and-anger}}

\emph{Posted on March 21, 2015}

My patients who connected with their anger safely were the ones that made the strongest and safest recoveries. Those unable to connect with their anger had more difficult journeys. This is clear in the case histories discussed in Engaging Multiple Personalities Volume 1.

Despair arises because it seems that there is no way out of that depression and fatigue. But there is: It is to work with the anger. Angry alters can often be converted to protectors in DID therapy because they usually arose originally in a protective function. It is getting back to that basic protective energy so therapists take note: We don't get rid of the angry alter. He/she can be a highly valuable co-therapist or protector in the healing of a DID client.

It is very common for survivors of trauma and dissociation to feel tired, depressed and hopeless. Energy has been and continues to be drained away dealing with the pain of the past. Colin Ross clearly explained, in the chapter ``The Healing Power of Feeling: Anger and Grief'' in his book Trauma Model Therapy, that ``Anger and depression are psycho-physiologically incompatible states.'' The polar opposite of depression is ``anger, (which) is energy, arousal, adrenalin, good posture, aggression, and the fight response.'' He continues, ``Assisting clients to step into their anger leads to stepping out of depression. That is partly because of the state switches to an energized, activated state, and partly because it takes considerable energy to repress all that anger.''

It is much preferable that you have a therapist who assists you to step into that anger. If you are doing it by yourself, through journaling or otherwise, be extremely careful and following these guidelines:

\begin{enumerate}
\def\labelenumi{\arabic{enumi}.}
\item
  Go into it slowly. Instead of trying to do it all in one sitting, be prepared to do it over weeks or over as long a time as is needed to do it safely!
\item
  Take baby-steps. The first step is to learn how to stop, and to be able to go for a walk to ensure that you are establishing safety and control of the anger. It is like when I first learned how to drive a car -- I made sure I knew how to step on the brake correctly to stop the (slowly) moving car first before I went driving around on the real roads. Control is the key.
\item
  Be kind to yourself -- to every part of your dissociative self. The usual mistake is going too fast. Old anger, when it is first released, tends to go overboard. The risk of getting in touch of your anger is that it may become destructive, such as getting physical and breaking furniture. So, I do not recommend doing this alone without the strong support of a significant other or supervision by a therapist.
\end{enumerate}

Every survivor of abuse has the right to be angry. They were abused -- often by people that should have protected them but instead betrayed the relationship in the most vile ways imaginable. Get in touch with that anger SLOWLY AND IN A CONTROLLED SAFE MANNER. It will generally lead you out of depression and fatigue.

\hypertarget{when-alters-despair}{%
\section{When Alters Despair}\label{when-alters-despair}}

\emph{Posted on November 24, 2015}

A question came up from one of the readers of Engaging Multiple Personalities who is DID. As it seemed to be a topic that was relevant perhaps to many DID individuals, I thought I would share some of my thoughts. As always, it is important to understand that I am retired and cannot offer therapeutic advice to anyone. Please do work with your therapist and know that healing is possible.

The basic question was not about angry alters, rather it was about alters that hold so much depression and trauma that the only solution they see is to die. They don't want to harm any of the other alters or the host, they just see no exit from their pain.

I can tell you that my life experience (I am almost 80 years old at this point) is that no one wants to suffer. Whether they are DID, have PhDs, are poor, are wealthy, are young or old, no one wants to be in pain. Much of our lives are spent simply trying to avoid pain and seek comfort.

My DID patients usually had severely depressed alters that would present their logic for why they (and everyone else in the system) would be better served if they were gone. With only the experience of holding traumatic memories -- and walled off by the DID from any experience other alters might have of laughter, enjoying food, and a warm glance from a genuine dear friend -- their desire to give up on life is understandable.

It was always difficult to establish a bridge of communication to help those alters shift their perspective. But, when that bridge was established they were able to begin to shift their perspective, if only for a moment and if only just a little bit. Once that happens, it is as if the clouds are starting to soften and maybe even part after a huge storm. The thick black clouds begin to get a touch of grey. The sun may not yet be fully visible, but at least there is more confidence that it is up there somewhere.

Alters stuck in their despair understand that each alter shares the body with all the other alters, including the host, so a peaceful death of one alter without affecting the body simply doesn't work. They don't wish to harm anyone, therefore they are looking for something other than suicide. Like other alters who are holding the most difficult trauma, they simply don't see a way out. That is because they arose in response to an abuser psychologically hammering into them the belief that there was no escape from the pain then nor would there be in the future.

The fact is that these alters have taken on an incredible amount of pain so that the system can survive and function. In fact, I had patients with alters that had sequentially arisen to take care of a certain level of trauma and then, within and during the same traumatic event, when the pain increased too much for that alter, another alter would arise to take on that increased level of pain -- and so on. These alters were taught that there is no relief that will ever be available to them. Nevertheless, relief is what they want and the only solution they see, because of amnestic barriers, is to die.

I would sometimes give my patients, and specifically those alters, an analogy to their experience that some of them found helpful: I would point out that for one person to lift a 500 pound weight is generally impossible. But, if 500 people share the lifting of that weight then each one is only taking on 1 pound -- easy to do. In this case, the intensity of the pain held by the alter in despair is the 500 pound weight. Clearly they can never lift it alone, and therefore see no escape from the pain.

However, the more other alters engage, befriend and share with that alter, the burden being borne solely by that alter eases a bit. Maybe at first from 500 pounds down to 499 pounds. That is still too much for any one to bear, but with each engagement, if a few pounds are shifted, then the path to relief starts to become clear -- even for the severely depressed alters. That is the point when the black skies lighten just a bit. Sharing their burden is not easy for the despairing alters to do, and is also something that many of the other alters don't wish to try. After all, holding that despair walled off within an amnestic barrier is why that alter was likely created. So, effort needs to be made to encourage the non-despairing alters to take on just a small touch of the pain.

The corollary to this is that when you share joy, it increases. Kind of like when everyone is watching a movie and laughs at a funny line. It is experienced as much funnier than one experiences the line watching that same movie alone. So, the guidance is to try to share the burden and share the joy. Both of these are difficult for an alter in despair to try because it goes against the imprinting by the abuser that no relief is possible.

Without pressuring them, make sure the alters that want to die begin to listen to the joy that some alters feel. They don't have to immediately experience it if they don't want to or cannot, but it is kind of like inviting them to at least listen at the door until they feel safe enough to come join in -- even just enough to stick their toe inside. In short, this is how you and that alter who really wants to put an end to the suffering without harming the commonly shared body can proceed.

The above suggestions are methods to work in the mind(s) of the system. How this might be done in the body is the next issue.

Sometimes, I suggested to my patients that they might invite the despairing alter to go for a restful sleep. This is not to tell them they are unwelcome. In fact, it is just the opposite. It is like when someone is ill, you want to bundle them up safe in a warm bed with warm honey tea and buttered toast. They can stay in bed resting while you guard the door, so to speak.

I would suggest for this alter to go on a short ``retreat.'' Let me elaborate further by giving an example. An ideal retreat for me is going to a place that has a quiet garden, eating very simple food and spending some time walking in a forest, smelling the wild plants, listening to the birds and maybe the noise of a small babbling brook. If you can do this even for a few minutes, for a few hours, or for a few days, without books, radio, e-readers, i-Pods etc., it can be very healing.

Allow nature -- here meaning the outside world and your body's interaction with it through the senses -- to bring you back to the complete ``here and now.'' Try to bring a complete attention to the present, the present breath, the gentle tired feelings in your legs and the slight hunger in the stomach before a nourishing simple meal. This is a way to use the sense experience of the body to comfort the distressed alter.

Invite the troubled alter specifically to join you in that simple retreat, without being heavy handed. If they say they don't want to come, no problem. When you go, they will automatically be there with you so long as you leave the door open for them through your good heart. Again, it is like inviting someone waiting just outside the door to listen in, perhaps they are too frightened to enter but they want to hear what is happening.

I know this suggestion may be met with resistance by some individuals who do not like its possible religious overtones, but the Earth holds all of us. Being undistracted in nature, allowing our senses to engage it, is not particularly religious.
If it is too difficult to get to a forest or garden space, my other suggestion is simply giving that alter a period of therapeutic sleep. Invite him/her to go for a long weekend of therapeutic sleep. Let them know that this kind of sleep is to allow a period of deep safe rest and healing rather than simply a time of avoidance. When you do this, make sure that as part of the invitation, they know that when the therapeutic sleep is over, you will have a meeting with them, speak directly to them, invite them to be your friend and share your experience of peace and safety of the garden/forest walk. At the meeting, listen to them speak of what is on their mind without judgment. As with any alter, they need to process their trauma safely.

Healing does not depend only on talking and thinking, it also requires rest and re-organization. It is like setting a fractured bone. You put the fracture in good alignment with the main bone and keep it in a plaster of Paris cast. The cast is a safe, protected place where the pieces can grow back together. The good alignment is kindness. As best you can, always be kind.

\hypertarget{engaging-with-many-voices}{%
\section{Engaging with Many Voices}\label{engaging-with-many-voices}}

\emph{Posted on March 10, 2018}

Letters from readers applying the information contained in my books and blog are a rich reward in my retirement. Trauma and dissociation is widespread and, unfortunately, so often dismissed by professionals in the field of mental health. So, nothing is more satisfying than to learn that my humble writing efforts reach around the Globe and offer some help to individuals with DID. With blogs and social media support groups, there are now additional vehicles to bring comfort to many who continue to suffer from trauma and dissociation.

I received some kind words about my books, along with a question, from a reader who is both a DID therapist and patient. While he may post a review, the more important message was in his question. I felt his message included something in particular, a way to communicate the experience of DID, to those in the mental health field as well as to those outside of it. I think this reader nailed perfectly his experience of dealing with many voices.

The reader described his experience as being ``like an uncontrollable, undisciplined meeting where everybody is speaking at the same time'' or ``like listening to 15 radio stations at the same time, but being unable to understand or sort out what they say.''

Part of educating non-DID individuals as to the experience and truth of DID is figuring out how to communicate the DID experience. The analogy given by the correspondent, that it is like turning on 15 radios set on different stations simultaneously, is accurate and instructive to non-DID individuals. If your friends or therapist doesn't understand or appreciate the experience, bringing in 15 radios tuned to different stations might be incredibly helpful. Perhaps this is something that should be done at each and every meeting of therapists who deny DID because they simply cannot connect with the experience.

The questions had to do with dealing with that experience of so many voices clamoring for attention at the same time. I suggested to begin journaling. My patient Ruth, presented in Chapter 5 , volume 1, had hundreds of alters all trying to communicate at the same time. We sorted that out fairly quickly. I asked her to invite the alters who wished to introduce themselves and write about their grievances to allow Ruth to bring in a few pages of messages to therapy sessions, according to degree of urgency or severity. The alters very quickly realized they all had a chance to get their problems addressed. They became very cooperative and took turns to be ``heard.''

Reading out the messages and responding to the specific alters directly, it became a method to quickly engage them individually. The approach was problem orientated, rather than alter orientated.

Journal writing can be very effective, both as a form of self-therapy or incorporated into the therapy session. Writing has a calming effect in organizing what appears to be chaotic and confusing thoughts.

As a child, I learned that most of the time, if I could put my problem or question clearly in words, I have often come close to the answer myself. That is because the process of writing helps organize ideas and thoughts. If, having organized the ideas and thoughts by writing, with your therapist's support you can begin to come close to the answer -- which is the beginning of processing the trauma. Once Ruth's alters were assured that they were being taken seriously, they would take turns to present their concerns. I did not try to go through a checklist of all the alters but only listened to those with the most urgent messages. Not all the alters needed to go through individual therapy. I encouraged communication among the alters, and the breaking down of memory barriers. In that way, when one was helped, others felt that their problems were helped as well because many do have similar problems and issues.

As readers of my books and blog know, I think it is very important to connect to one's sense perceptions in order to understand the here and now experience. Grounding exercises, using the 5\% rule and one-breath meditation are all techniques that can be helpful to address the concerns of all parts of the system.

\hypertarget{communicating-with-alters-that-dont-speak}{%
\section{Communicating With Alters That Don't Speak}\label{communicating-with-alters-that-dont-speak}}

\emph{Posted on March 15, 2018}

A reader wrote to me asking my thoughts about a problem that affects many DID individuals. The question was about working with alters who are mute, perhaps too young to speak, or, in general, uncommunicative. This raised a common concern: how can we communicate when they do not speak?

The foundation of this is the understanding that communication with alters in need is essential for healing. Ignoring alters will simply make matters worse.

When we consider the possibilities of communication, it can take place directly or indirectly, verbally or non-verbally. Direct verbal communication is usually, though not always, somewhat straightforward. Indirect verbal communication can refer messages routed through a 3rd party. In the case of DID, this 3rd party routing can be very useful. I had patients that established one or more alters as the spokesperson and/or message deliverer between me and alters that for whatever reason did not wish to communicate directly. Sometimes, the communication gateways were alters that knew the silent ones inside enough to approach them, or be approached by them, to pass messages in both directions.

With respect to young alters that were pre-verbal, these had usually arisen at the time of early abuse that took place when the host was pre-verbal. For those, it seemed that alters who were just a bit older and already verbal were the best at communicating to those very young alters, and could facilitate communications.

In short, I suggest that encouraging alters to take on that role can be very helpful. Some may be willing to do so, some not. Inviting alters to try, even if they don't succeed, is a positive step forward -- like cracking open a door that has been long closed. The door won't readily swing on hinges that have been frozen in place after so long, but the first little opening enables a second to take place and a third until eventually the hinge begins to swing more easily.

Sometimes, among all people, communication involves messages that say one thing on the surface but make another, sometimes different statement, at the same time. In DID, the internal conflict can play out in that kind of communication. For example, the communication might come from a very angry protector saying ``I hate you -- so stay away or else\ldots{}'' But that same message may be commingled or be an overlay of a message from a frightened very young alter testing whether or not the therapeutic alliance is genuine or just another opportunity for betrayal.

Non-verbal communication takes place all the time and is a very important way of communication. Take the example of communication between species, we all know for example a dog owner and his dog can be in deep communication without use of words. Most of us have heard of dolphins being trapped in a fish net, who express their gratitude after being cut loose by a diver. I believe all this is true--- we humans just get a little carried away by over-dependence on the use of words.

Non-verbal communication is powerful and often overlooked. We all have experienced hunches and ``6th sense'' warnings, alerting us of danger or conveying respect and positive regard from total strangers speaking in an unfamiliar language.

How does this relate to individuals with DID? Often the body language will be the communication -- unadorned and straightforward. This is true whether it is rage, fear or laughter. Again, one can use that body language as a way to open another long-closed door. For example, an alter (male) of one of my patients became angry at home one afternoon and just started banging her head really hard against the floor as she grunted. It was quite frightening for the spouse.

The intensity of the anger and the head-banging didn't make any sense. There hadn't been any argument but something had triggered this reaction. Taking the approach of trying to engage what was obviously an alter, the spouse said that he didn't understand why she was banging her head against the floor but really wanted to understand because it was obviously important to know the ``why.''

Taking that body language and grunting as communication rather than as psychosis, allowed the spouse to ask that genuine question. The spouse asked for help to understand what the head-banging meant. Because a genuine question was asked respectfully, and because the spouse was genuinely trying to engage, the 5 year old alter answered in words that it was how he protected the system from the abuser. This didn't make much sense to the spouse. How was it protective to be smashing your head against the floor?

The alter first glared at the spouse -- pretty much indicating that the spouse was obviously too slow-witted to get it. But then, again because the spouse was genuinely trying to engage on the alter's own terms, the alter was quite explicit that he did the head-banging because he knew it would frighten the abuser. If he frightened the abuser then he was the one in control -- not the abuser. He explained that if he hurt the body's head, they (meaning the host and the family abuser) would end up at the hospital. The abuser didn't want that because then he would have to explain how the head injury occurred to the police or doctors at the hospital.

When the spouse remarked that it was incredibly brave and insightful to have come up with that on the spot, the alter straightened up his body -- head up, shoulders back, the chest swelling with pride. Why? Because the spouse understood and appreciated the hidden message. The spouse understood that banging the head against the floor was a brilliant and sane thing to do by a five year old under the circumstances. It was \emph{not} something crazy. As the alter swelled with pride, the spouse started laughing and then the alter started grinning -- a gigantic grin.

A bridge was built on the spot in that way. That was the beginning of the spouse being able to successfully invite a shift in an angry protective alter and turn that alter into a support in healing.

That same patient had alters that would come out at night. Having already developed a relationship with the alter described above, the spouse was told that these were really really little ones. They would come out crying in fetal positions, wracked with sobbing. With the information from the 5 year old alter that these were likely infants, the spouse had them lay their heads on his chest so they could feel his heartbeat, his slow breathing, and his arms softly holding their back, protecting them. In holding them the way a parent would hold an injured infant, these alters would cry for a while and then leave when they had been held enough for that moment. After a few months, they wouldn't come out sobbing but rather would come out crying just a bit and finally would sometimes fall asleep on the spouse's chest. Their appearance became increasingly rare until they no longer seemed to need to come out. Nothing else really needed to be done but to be there for them, acknowledging that they had been hurt and needed comforting.

I think we can strive toward communication with alters who do not use verbal communication. Try to be sensitive to their needs, their unprocessed trauma. By being there, being genuine and sometimes simply being still, conveying the willingness to listen and to understand, we may be able to help those with DID accomplish quite a bit of healing.

\hypertarget{when-alters-attack-inside}{%
\section{When Alters Attack Inside}\label{when-alters-attack-inside}}

\emph{Posted on April 7, 2018}

\textbf{TRIGGER WARNING}

\textbf{The following is in response to an enquiry which I think may have a general relevance to our readers. As the question involves violence within a DID multiplicity system, please note that this post comes with a trigger warning.}

A reader with DID spoke to her therapist about an alter who was attacking small alters inside, including sexually. The therapist told her to get over it because it didn't really happen. I believe this is a mistaken approach to therapy that will undermine the possibility of a genuine therapeutic alliance with those alters. In my view, it perpetuates the belief in the system that no one believes them about their trauma. This experience is not particularly different from when one angry alter does physical harm to another alter, like cutting or cigarette burning. The alters experience it in the same way -- they are under attack, they cannot defend themselves, and they are not being believed.

In establishing a therapeutic relationship with DID individuals, the therapist has to get over their conventional view, their own ties to the logic of a unitary consciousness. They have to accept how the alter that is communication genuinely feels rather than impose their own logic on to his/her patient. To an outsider, an individual cutting himself is hurting himself. It is visible to the therapists eyes. In the context of a DID system, this is often seen as one alter trying to cause harm and injury to another alter. When the damage is not visible to the therapist, that doesn't mean it isn't happening.

I remember the case of Ruth in Chapter 5 of my book. Ruth was hospitalized against her wish to keep her from bleeding to death, because of her continuing attempts to cut herself. She was forcibly kept in a general hospital for 5 continuous months. She was discharged with the case-note indicating that she was still alive. Despite the clear dissociative symptoms, she was not given a dissociative diagnosis.

So, how was she able to survive and heal? Most important, she wanted to heal. She interviewed me as a potential psychiatrist to help her. Treatment was quickly instigated through weekly psycho-therapeutic sessions, and by inviting her alters to air their complaints. For Ruth, it took the form of therapy through journaling and discussing the written material she brought to the therapy sessions. While she never responded to anti-depressant medication, involving years on heavy dosages including in the hospital, her ``depression'' responded to psychotherapy. Her cutting was quickly reduced as a result and did not pose any more danger to her life. So long as she felt hopeful, I never worried that she would succeed in killing herself.

Therapists have to get over the hurdle of understanding that the experience of a DID individual is based on understanding the context in which alters engage each other and the outside world. The most effective way to do this, in my experience, was to engage the alters as they presented. In considering this seemingly ``illogical'' proposition of one alter sexually abusing another, it could be seen quite straightforwardly as one alter angry enough to want to cause physical and psychological harm to another alter.

The therapeutic task with the angry alter is then to engage that angry alter to understand what function the rage and conduct is serving, why they feel it is necessary to do this. It is no doubt related to that alter's own trauma and the seeds of healing will be found in that engagement. The therapeutic task with the abused alter is, as always, to engage that alter to allow them to process their trauma. While there is no ``one size fits all'' approach in helping alters process their trauma, engaging each alter as they present their feelings, their experience, opens the gate for healing. In my practice, I would often suggest that other alters engage with the angry alter, to listen to that one as well as to intervene as a friend just as I would suggest that other alters engage with the abused alter to listen as well as intervene as a friend.

It is not appropriate for an outsider, therapist or otherwise, to debate what they see as the impossibility of one alter abusing another, when they are sharing the same body. It misses the entire point of the dissociative response.

We, as therapists, have to accept how an alter feels, which is genuine and real, no matter how ``illogical'' this may appear to an outsider. Without that acceptance, a genuine therapeutic alliance simply will not take root.

\hypertarget{encouraging-empathy-within-did-systems}{%
\section{Encouraging Empathy Within DID Systems}\label{encouraging-empathy-within-did-systems}}

\emph{Posted on May 11, 2018}

Recently, I posted a two-part piece on the importance of cultivating and training therapists in empathy. I am confident that if a therapist has empathy, or even the seeds of empathy, that quality can be nourished, enhanced and cultivated which will necessarily increase their capabilities as a therapist. The distinction was made between sympathy and empathy in that piece, identifying them with compassion as critical components in therapy.

An equally important question for DID individuals is that if empathy can be taught among therapist-trainees, can we engender and help cultivate empathy in alters? In my experience, it is definitely possible and can be vital in DID therapy. Let's examine the possibility of suggesting to patients (and to others with DID) that alters can begin to connect with other alters inside in ways that are both kind and safe.

In practice, encouraging connections among alters needs to be done slowly, gently and over time. Remember, empathy requires the ability to place oneself in the position of the other. Alters are, very correctly, scared of this. After all, it was early childhood trauma on an ongoing basis that is the general origin of DID. For an alter to fully experience and express empathy for another traumatized alter is extremely difficult. Why? Because the system actually does know how terrible the trauma was -- it is not just projection and guesswork as it may be for a therapist.

Many alters are frightened of other alters, in particular those that act out internally and externally in extreme ways. They are often frightened of the intensity of the trauma other alters hold. After all, protecting the system's parts is the reason the dissociative response often produces amnestic barriers. So, disturbing the protection established very early on with the amnestic barriers is something to be done only with the agreement of the alters, which can be gently invited but never demanded.

Some alters are dismissive of other alters, denigrating them for a perceived weakness. Some alters are angry so as to keep their armor up and attuned to potential attack. It is important when you see the myriad of presentations inside a system, even your own, to know that it is not necessary to try to speak to each and every alter about the importance of empathy.

The fact is that beginning a connection between one alter with just one other can have a fundamentally powerful impact on the relationships among all the alters. Why? Because it demonstrates the possibility of safe interaction. It demonstrates the power of simple warmth along with the ability and benefit of gently dissolving some of the amnestic barriers.

Imagine a radiator. It will have a scary quality if your first experience of a radiator is burning yourself on its hot surface. You might never get close to a radiator again out of fear. But, if you are cold, and someone shows you that staying 5 feet away from the radiator will make you feel a little warmer but not too much warmer, you can learn that the radiator isn't always dangerous at that distance. Then, you can stay 3 feet away and see how much warmth you experience there. When it gets a little too warm, say at 1 foot away, then you have learned the boundary of safety in terms of that particular radiator.

The warm connection of empathy inside can be the same for the alter that is frightened to connect to another. That is true on both sides, the alter considering extending warmth -- who may not want to get too close to the trauma material of another alter -- as well as the alter considering accepting warmth -- who may not want to get too close to another for fear of betrayal or of retraumatization if they open up even a little bit. Encourage the alters to express and to feel the warmth a little at a time, like being 5 feet away from the radiator, or even 10 feet. It is the intention, the aspiration to connect, which opens the gate of and to support.

Don't suggest that any alter truly try to take on the trauma of another, or to go deep into their imagination of the trauma material held by another. The system knows what is and has gone on, even if individual alters only hold a piece of the memory. Just as with the approach I took with my patients, it is never necessary to pry into the trauma material, just be available to listen to and for what an alter might present. That is enough for empathy inside.
Pushing further increases the risk of retraumatization. So, go safely, small step by small step, while asking the protectors to watch over the process to ensure it doesn't go too fast. Even inside, \href{https://www.engagingmultiples.com/the-5-rule/}{the 5\% Rule} is a key protective mechanism to remember.

Many alters hold specific traumas or parts of trauma, and have done so since the trauma occurred. In so many ways that is their identity, their reason to exist. The trauma they hold was affixed to them in a dissociative experience, one that no doubt terrified the system. This resulted in the arising of that alter and perhaps others.

While memories may be walled off internally between alters, many alters know of the others, or at least some of the others. Many alters know which alters they want to stay far away from and which ones they might be willing to connect a bit closer to. You can start with encouraging an alter to simply be there to listen to another alter who may be crying, who may need the experience of a kind word inside, who may simply need the experience of not being alone. You are not trying to have one alter fix another, just to confirm a connection -- like catching someone's eye across the room and nodding to them. That connection can be a balm which sets healing in motion.

In my experience, once alters start helping one another, the rate of healing is tremendously accelerated. Encouraging an alter to explore the possibility of helping has to come very slowly and very skillfully, a subtly suggested invitation. The initial response is usually a big NO. Why? Because to suggest that alters help each other goes against the foundation of dissociation. It goes against the amnestic barriers that arose originally for protection and to minimize pain.

Even just the idea that there are alters inside that will befriend or at least listen to another alter -- that will listen to that sadness, anger, whatever -- is extraordinarily powerful. It establishes the sense, correctly, that there is the possibility of comfort and even help 24/7 -- right there within the system. It can become one of the pillars that allows for co-consciousness and for eliminating the sharp edges of internal conflict. This is self-empowerment.

\hypertarget{advice-to-clinicians}{%
\chapter{Advice to Clinicians}\label{advice-to-clinicians}}

\hypertarget{support-group-feedback}{%
\section{Support group feedback}\label{support-group-feedback}}

\emph{Posted on March 27, 2014}

When I received feedback from the founder of a DID support group letting me know that they wished to use the book for a study group, I was delighted. This is exactly my hope -- that members of the DID community would access the material for their own healing. To read the material in that kind of safe place -- with others in a support community -- is a wonderful way to protect oneself from re-traumatization while exploring the paths to healing that are presented in the book. This is something I had not previously considered. It is truly a great joy to hear about this and even more positive that the idea came from the DID community itself!

\hypertarget{the-therapeutic-window}{%
\section{The Therapeutic Window}\label{the-therapeutic-window}}

\emph{Posted on April 30, 2015}

The concept of ``therapeutic window'' is discussed in detail in Principles of Trauma Therapy: A Guide to Symptoms, Evaluation, and Treatment by John N. Briere and Catherine Scott. The therapeutic window, as presented by Briere and Scott, refers to a psychological midpoint between the inadequate and overwhelming activation of trauma-related emotion during treatment. It is a hypothetical target where therapeutic interventions can be most helpful. Psychotherapeutic interventions within the therapeutic window are neither so trivial that they provide inadequate memory exposure and processing; nor so intense that the client's balance between acceptable memory activation and overwhelming emotion is tipped towards the latter. In other words, interventions that take into consideration the therapeutic window are those that trigger trauma memory enough to promote processing but do not overwhelm internal protective systems such that untoward avoidance responses take over.

To put it another way, interventions that ``undershoot'' the therapeutic window are ineffective and a waste of time. Those that ``overshoot'' the window constitute re-traumatization. In the former, the client may avoid returning for further treatment because they feel nothing is being accomplished. In the latter, the client may avoid returning for treatment because, having been retraumatized, they are frightened.

The therapist must remain completely attuned to the client, their verbal, emotional and physical presentations give you the keys to see how far to go and how not to go farther. Each encounter between the therapist and client should be more powerful than the titrated exposure to the trauma within that therapeutic window. In that way, each time a client comes to therapy they will leave feeling just a little bit better than when they came in. This is critical to encouraging their hope for recovery and their further establishing of a therapeutic alliance with the therapist.

\hypertarget{david-yeung-comments-on-psychiatric-times-sra-article}{%
\section{David Yeung Comments on Psychiatric Times SRA article}\label{david-yeung-comments-on-psychiatric-times-sra-article}}

\emph{Posted on March 25, 2014}

\emph{This is in response to a recent article denying all aspects of Satanic Ritual Abuse.}

Clinicians should be wary of being drawn into this debate. The real issue is the impact of past trauma into present experience. One should not be sidetracked by a debate about the ontological aspect of SRA and therefore miss the point about Dissociative Identity Disorder.

War veterans reliving PTSD flashbacks are not interrogated as to the accuracy of their traumatic memory, while victims of child abuse are met with scepticism and disbelief. The purpose of psychiatry is to treat patients, not to cross-examine them as if their treatment sessions are taking place in a court of law.

Patients traumatized in childhood decades ago, by powerful abusers and under inescapable circumstances, have difficulty reconstructing their experience from non-declarative memory. In converting that experience into words, it is understandable that they will express it as a memory of satanic ritual abuse. After all, if someone who is conventionally seen as your protector violates you in unspeakable ways, that is close enough to an archetypal satanic experience to be seen, at least, as metaphorically true. There may be no other words to describe this indescribable experience of betrayal. The false memory debate pre-supposes that there is only one ``true'' memory -- one that can satisfy a judge and jury. Perhaps those who are fixated on denying SRA should watch Kurosawa's Rashomon again.

To call it false memory because it might not survive a court challenge will prevent a psychiatrist from helping patients to heal. People abused in childhood are already convinced by their abusers that no one will believe them, no one will take them seriously. For a clinician to do the same thing is to reinforce that abusive imprint.

After years of scientific and liberal education, psychiatrists should understand the difference between factual and metaphorical accounts. The conversation between the fox and the crow in Aesop's Fable surely did not take place in an historical provable event but it represents metaphorically a deep truth.

I would not dismiss the possibility of ritual abuse. The current unveiling, surely not complete yet, of international child pornography and child sex slave selling rings promoted throughout the Internet is not far removed from what a victim may say using language of SRA in an effort to communicate their experience.

\hypertarget{the-5-rule}{%
\section{The 5\% Rule}\label{the-5-rule}}

\emph{Posted on May 28, 2015}

I am glad that some readers are finding my books helpful, and that some therapists are open-minded enough to read them. For patients and their significant others, my wish is that they will find hope for healing in the material. For therapists, my wish is that whatever they find helpful will prove of benefit to their own patients.

I recently received a request for more detail on the 5\% rule I discussed with my patients. In Engaging Multiple Personalities Volume 1 page 31, I suggested to a patient that she could try to limit her experience of pain to 5\% of the actual memory of pain. In that way, she could begin to relate to the pain while remaining in control and avoid being overwhelmed by it. 5\% seemed to give the pain a boundary of tolerability, regardless of that boundary's illusory nature. Further, if 5\% was too much to handle, at that point she could decide to only take on 2\%. Again, the idea was to create a vehicle through which she could begin to process her trauma on her own terms -rather than being uncontrollably swept away by the memories.

The notion of 5\% is a way of pointing out that a difficult task can be divided into small bits, so that each part can be handled successfully without overwhelming the system. In concrete terms, if one has to climb a tall mountain, it might seem impossible at the beginning. By dividing it into 20 sections, each part is only 5\% of the whole. That small part appears on its own to be manageable. The next 5 \% will be likewise manageable too, and so on. When climbing Mount Everest, even the professional climbers acclimatize by spending time in a series of Base Camps that are each a bit higher in altitude than the prior camp before making the final ascent.

When it comes to pain, one has to use some imagination. I once treated a patient with severe snake phobia. I applied the 5\% rule in this way. I suggested that she could imagine 5\% of the fear to be like imagining a snake placed in a locked cage, in a locked room in a locked building, situated in the next city block over. The next step would be for her to imagine allowing the snake to be brought in the locked cage just outside the locked room, but still being kept in the locked building in the next city block. This amounted to her feeling a certain percentage of the fear of the snake without succumbing to panic. In this way, she was able to have some measure of control. Step by step she was able to regain control of her reactions to snakes.

The important suggestion is that one can use one's imagination to break down into fractions whatever it is that one is frightened of. As in all behavioural therapy, the key point is generating that sense of control in the hands of the survivor. With control, one is no longer a helpless victim. Rather than being a victim of an onslaught of debilitating memories, the patient (NOT the therapist) is then in charge of allowing whatever amount of the distress to come through for processing.

Even simply talking and planning such a technique with the patient in a secure milieu is in itself therapeutic. It is best to engage the patient to fully participate in the therapeutic procedure. The primary default response to a trauma is helplessness---a sense of loss of control. This approach gives the patient the tools to transform the default response of helplessness into a powerful controlled processing response.

\hypertarget{advice-for-novice-did-therapists}{%
\section{Advice for Novice DID Therapists}\label{advice-for-novice-did-therapists}}

\emph{Posted on August 12, 2015}

If you are a therapist who has never treated DID \ldots{}

Is there a point to deny, discount or argue with different alters in a DID patient? Bluntly speaking, the answer is no. While DID/MPD deniers will deny the existence of alters no matter what evidence or experience you present to them, I have seen and engaged many different alters in DID cases. Further external validation is unnecessary to proceed with treatment.

There are two key points to keep in mind when acknowledging the presence of different alters:

\begin{enumerate}
\def\labelenumi{\arabic{enumi}.}
\tightlist
\item
  Alters feel strongly about their individuality. To insist that they are just one identity or personality is going to push them away from the therapist and destroy any possibility of a therapeutic alliance. I accept their way of thinking of themselves as separate individuals. I will not impose my own ``unitary'' concept of personality and try to convince them that they are deluded or simply wrong. This multiplicity aspect of personality is prevalent in all of us. It is only a matter of degree. When I play tennis, I am acting and feeling like a teenager, trying to hit the ball to the other side so that my opponent cannot return it. It is the ``teenager'' in me that is playing the game. It is a conceit to think that the teenage quality of me playing tennis is not part of the continuum of experience that includes alters in DID patients.
\end{enumerate}

A therapist should never argue or try to convince a client that he/she does not have different alters. It would be akin to attempting to convince that a schizophrenic patient's voices are not ``real.'' However, common sense and appropriate therapeutic demands dictate that clients' alters should all work out a way to handle the practical aspect of day to day business. Alters should obviously find a way to live in a cooperative way because there is only one body -- one cannot go to a party and simultaneously rest at home.

\begin{enumerate}
\def\labelenumi{\arabic{enumi}.}
\setcounter{enumi}{1}
\tightlist
\item
  When treating a DID patient, unless a therapist acknowledges the presence of alters, treatment cannot even begin. Therapists cannot get anywhere if they insist on ignoring an alter because that means shutting down therapeutic communication. This is so basic but is one of the major obstacles in DID therapy for psychiatrists who have no experience in treating such patients. There is a strongly held but erroneous belief that if a therapist talks to an alter, it is going to make things worse. In fact, the opposite is true. Ignoring the alter(s) undermines the therapeutic alliance. The patient will close down this most important support and gateway for healing. It is the equivalent of telling a non-DID patient that the therapist does not want to hear what is really bothering him/her.
\end{enumerate}

Once these two points are understood and agreed upon by the therapist, treatment of a DID client is no different from treating patients with most other psychiatric diagnoses. In DID therapy, therapists should focus on processing past trauma, and bringing together the alters so that they learn to live together in harmony and mutual support, like a team of athletes with different strengths and skills all pulling together toward the common goal of healing.

Sometimes a therapist who has never treated DID will be open-minded and even read my books to get started in treating DID. Claiming to have no experience is not a good excuse because DID is not rare. Sooner or later the therapist will see or at least recognize another case of DID. No therapist should deprive themselves of the opportunity to learn to treat DID.

If you cannot find a psychiatrist, any psychotherapist from other disciplines can work with DID patients. Social workers, psychologists and others can equally engage in treating DID. Anyone trained in psychotherapy can treat DID if one follows the simple principle of processing trauma and bringing together the different alters to work as a team.

\hypertarget{treating-massive-multiplicity}{%
\section{Treating Massive Multiplicity}\label{treating-massive-multiplicity}}

\emph{Posted on June 2, 2015}

Chapter 5 in Engaging Multiple Personalities Volume 1 documents Ruth, who eventually told me that she had over 400 personalities. I treated the fact of such a large number of alters in a low-keyed way. In my approach, it didn't matter if she had 4 alters or 400. The important point is that the therapy was not dependent on the number of alters -- it was dependent on some key general guidelines regarding alters.

Alters usually demand to be treated as separate individuals. That is how they experience their own being. In therapy, integration was not the goal. The ultimate goal is to help all the alters to function as one cohesive unit, like a football team with a common aim of winning the game. I tried to teach them to aim for cooperation, communication, and coordination. The approach was for me to help them to respect each other and encourage them to help each other. This needed to be done without my denying their desire to assert their own individuality.

Initially, there are conflicts that are often expressed quite strongly by alters that are not interested in working with or befriending any other alters. There are usually those that are quite angry with other alters and wish to be violent toward them. This approach requires the therapist to proceed with sensitivity and tact. Without hurting their feelings or telling them to drop their individuality, I would point out to them the need to acknowledge that they should work for the common good because they are all sharing only the one body. This is one of the main tasks in DID therapy, to help alters come to terms with the idea that they have to work together, to sacrifice a small measure of their individual demands so as to be able to work together for the bigger and more powerfully functioning unit.

Eventually, after alters have processed their traumatic memory, the need to be separate individuals often diminishes. As a therapist, it is paramount to control one's curiosity and undue inquisitiveness as to the individual alters' personality and characteristics. I never said anything other than the truth that all the alters played a part in saving the system from destruction under the direct assault of the trauma and its after-effects. Therefore, all the parts needed to respect all the other parts.

One must conscientiously refrain from trying to treat each and every alter separately as if you have 400 separate patients. My contact with each of Ruth's alters was limited to whatever and whoever arose in the here and now of therapy. If one alter was suicidal, I would encourage that alter to come out to address that one's specific issues, \emph{without demanding that she come out}. It is a delicate balance of neither denying their individuality nor encouraging their separateness. As the trauma is processed, the individuality becomes more and more of a non-issue.

On the other hand, therapy for a suicidal alter must be straight to the point. One can explain to the suicidal alter that she is angry and fighting against herself, whereas the real anger should be directed towards the abuser. It was the pain inside that led her into wanting to hurt herself. I then pointed out different things she could do to soothe and ease that pain. Critically important was to point out that she could use the anger itself as a powerful force for healing and recovery.

Empathy from the therapist goes a very long way. One can seek to motivate other helpful alters to rally to the task of facilitating the healing. Other alters may be assigned the task as co-therapists, or at least to hold the hand of the sad or suicidal alter(s) to let them know they are not alone.

While therapy has to be flexible and dynamic, it needs to be goal-oriented and task-focused. I conscientiously avoided socializing with interesting, colourful and engaging alters. Therapy is not a chit-chatting social event with alters over a cup of tea.

In this way, even with several hundreds of alters, I managed to complete therapy in 2 and half years with Ruth. It is worthwhile to note that in a follow up instigated by Ruth, she reported that she had not needed support from the mental health system since therapy over 20 years ago at this point. This is a far cry from her history prior to psychotherapy that involved 20+ hospitalizations (one of which lasted 5 months), multiple psychiatric emergency visits and ongoing and unsuccessful pharmaceutical attempts at treating her depression.

This is not to say therapy can always be completed in such a short time. There are tremendous individual variations. Nevertheless, this is a confirmation that DID therapy does not have to drag on for years and years.

\hypertarget{misdiagnosing-did}{%
\section{Misdiagnosing DID}\label{misdiagnosing-did}}

\emph{Posted on September 9, 2015}

Psychiatric diagnoses are based on clinical features rather than laboratory tests as in organic pathologies. For example, a patient presenting with a fever might have a common cold, a kidney infection, pneumonia or many other illnesses. It is the task of the clinician to determine what disease the symptom, fever, is resulting from.

In psychiatry, it is necessary to maintain a proper index of suspicion in that same way. When a patient presents with depression, it could be that they are bipolar, it could be that they have borderline personality disorder, it could be that they are in the midst of a divorce, or perhaps they have DID. It is the job of the psychiatric clinician to be open to all of the possibilities as they commence the diagnostic investigation.

Psychiatric disorders according to DSM 5 are based on symptoms, just like fever in the case of Malaria. Treating malaria with fever lowering drugs is akin to treating the depression in DID, which are cases of trauma and dissociation, with prescriptions that do nothing to treat that underlying cause. Just as treating malaria solely by lowering the fever will not cure the malaria, treating depression with anti-depressants alone will not necessarily cure the underlying cause of the depression.

When a patient presents with depression, the answer must not always and simply be an anti-depressant. The therapist must understand, first of all, that depression may simply be a symptom, not a syndrome or disorder, and it is not always a pathology. It is often an appropriate human response to difficult situations.

In my practice, it was quite common to see patients referred to me by other psychiatrists with diagnoses of Bipolar Disorder, Borderline Personality Disorder, and Clinical Depression that have not responded to medication. For those patients, it was almost always because the diagnoses were in error. The referring psychiatrists had focused on one symptom rather than the patient's overall circumstances.

I believe this is due to a failure in training -- a failure that I was subject to for many years as a practicing psychiatrist. I had been taught that I would never see a case of DID because it was such a rare phenomenon. Most of my colleagues, all veteran psychiatrists, never had a case of DID either. Looking back over my 40 years of practicing psychiatry, I would correct that statement to say that most of my colleagues (including me at the time) had never recognized a case of DID. Indeed, many referrals I received included notations of dissociative features being displayed by the referred patient but a refusal to include dissociative disorders as a primary or even secondary diagnosis.

I still remember my encounter with a patient in an infectiously happy mood who came to see me for recurrent spells of depression. Applying the DSM 4 criterion applicable at the time, I could not have been more certain that I was encountering a case of Bipolar Disorder. No one could have convinced me otherwise. I had certainty in the diagnosis and felt greatly relieved. I knew exactly what to do, how to follow the well laid-out protocol of mood stabilizers and so on. With the diagnostic certainty, I was confident that my task was virtually accomplished.

As a result, I never even considered the possibility that what I was seeing in my office was an alter who appeared carefree and happy. A careful consideration of the patient's life history and early incestuous abuse should have alerted me to the possibility of quite a different diagnosis. In fact, after seeing her numerous times, she disclosed that she had been abused by her father. Rather than raising my index of suspicion, as should have occurred, I simply said ``Oh, that's part of your personal history.'' Although my response was in accord with the standard psychiatric practices at the time, it was an abject failure that I did not reconsider the bipolar diagnosis.

Thanks to the bravery of some of my DID patients, I can say with confidence that a more appropriate approach to the ``hypomanic'' part as a possible alter would have opened the door to healing. My conduct at the time, instead, confirmed for the patient that I didn't think the abuse history was all that important in the context of the bipolar diagnosis -- again this conformed to the standard practice of psychiatry at the time. This was ignorance, dangerous ignorance on my part, and a continuing regret.

I should have understood that her opening up that personal history to me was dangerous and frightening for her. It should have been met with gentle kindness and openness. It should have led me to reconsider the therapeutic approach. There existed an alter who could have connected me to the severe inner turmoil and complexity of a psyche suffering from complex PTSD. That would have further established and strengthened the therapeutic alliance. It would have enabled her healing to have proceeded in a safe, supported and appropriate way.

I believe that many therapists make the same kind of mistake as I did for the first many years as a practicing psychiatrist. One must always be aware of the possibility that by simply labeling a patient as Bipolar or Borderline may mistakenly lead the therapist toward concentrating on a pharmaceutical treatment that will only cover up the real pain of a badly traumatized individual. It is often worse than no diagnosis at all as the patient's difficulties are both compounded and hidden by the cascading effects of psycho-active medications.

What do I suggest to the current and future generations of therapists? Pay attention to the following research statistics:

The incidence of DID is not rare. According to Lowenstein1, DID may occur at a 1\% rate in the general population, which is close to that of Schizophrenia.

The prevalence rate for schizophrenia is approximately 1.1\% of the population over the age of 18 (source:NIMH) or, in other words, at any one time as many as 51 million people worldwide suffer from schizophrenia. There is an almost identical rate of prevalence for DID.

Physicians and other therapists are all aware that schizophrenia is not uncommon. Let us raise our index of suspicion about DID as it is just as common, statistically speaking. Just because high functioning individuals with DID exist and are well known, such as Dr.~Robert Oxnam, Herschel Walker and others, it is not a rare disorder. In my opinion, the diagnostic bias against DID is connected to the discomfort people have acknowledging the constellation of circumstances that give rise to it: early childhood abuse, dissociation, and betrayal.

\hypertarget{why-some-clinicians-refuse-to-acknowledge-did}{%
\section{Why some clinicians refuse to acknowledge DID}\label{why-some-clinicians-refuse-to-acknowledge-did}}

\emph{Posted on September 18, 2015}

I previously posted on the danger of diagnosing a mental disorder based on clinical symptomatology alone. In that post, I discussed my own failure to diagnose DID in a patient because she presented what appeared to be a classic case of Bipolar Disorder. DID is rooted in early childhood abuse. It seems that many psychotherapists, and others throughout society, prefer to avoid the issue of the rampant abuse/molestation by people across all economic, religious, social and cultural boundaries. To acknowledge DID is to acknowledge the epidemic level of abuse that occurs in one's own societal milieu.

The purpose of this post is to highlight other reasons for the failure to diagnose DID correctly and, in particular, why clinicians affirmatively choose other diagnoses over DID. During my years of practice, I received many referrals of patients that had multiple diagnoses, usually borderline personality disorder, bipolar disorder and schizo-affective disorders. In many, but not all, there were clear acknowledgments of dissociative qualities indications. Nevertheless, in the referral documentation dissociative disorders were simply not considered in the diagnoses.

I have long puzzled over the fact that there are deniers of DID even among seasoned psychotherapists. I think the crucial issue is that in the experience of many therapists, they have never encountered even one patient with DID. With that background it might be understandable why he/she would reject such a diagnosis. But that should \emph{not} be the end of the inquiry.

In my experience, it is not that therapists, certainly the vast majority of those that referred patients to me, never encounter DID; rather, they simply fail or refuse to recognize it.

Human beings have a predisposition to perceive things in a certain way. In psychological terms, this is known as a perceptual set or a perceptual expectancy Numerous studies confirm that perception is highly influenced by what one expects to perceive. For example, because we are highly attuned to hearing our own name, we recognize it even in a loud and chaotic environment. In a similar way, if we believe that our key has been stolen we will fail to see that key even if it is right in front of us.

Applying that same expectancy analysis to psychotherapists and DID, if a clinician believes that DID is rare, its presentation in a patient will be missed. This happened to me on many occasions before I came to realize that DID was no less common than many other disorders. I needed to modify my diagnostic index of suspicion to include DID as a possible diagnosis as likely as bipolar, borderline personality disorder or schizo-affective disorders.

Another common reason for missing DID is that the DID is hidden behind some other presenting symptom. For example, many patients come presenting with depression. Others may be presenting with sexual or other addictions. Still others may present with difficult so-called character flaw problems like pervasive anger. Therefore, it is important to examine the problem of basing diagnoses on mere symptomology without an appropriate index of suspicion.

By way of example, malaria and typhoid are two different diseases but sometimes physicians are unable to diagnose them properly due to certain symptoms they share in common. In the initial stages, both may present with the following clinical features indistinguishable from each other: high fever, abdominal pain and lethargy. Yet they are completely different in etiology and demand. Typhoid fever is caused by a gram negative bacteria named as salmonella typhi whereas malaria is a protozoal disease due to different species of Plasmodium invading the red blood cells, transmitted via mosquito. Treatment for malaria will not help a patient with typhoid, nor will treatment for typhoid help a patient with malaria. Fortunately, some simple lab tests can distinguish between the two. However, there are no such laboratory tests to distinguish between most psychiatric disorders, such as between schizophrenia and DID.

So, again, one must not end one's inquiry simply because one has seen what one expected. One can see bipolar disorder in mood swings, but the mood swings might also be different alters presenting themselves to the therapist. Depression may be a disorder, but it might also be an appropriate response to difficulties in life or it could be rooted in DID that is held by one or more alters.

Competent therapists need to examine their own index of suspicion. DID should be included in that index of suspicion when seeing patients with presenting symptoms that are found in common with other disorders, whether it be depression, addiction, schizo-affective disorders, bipolar or borderline personality disorder.

\hypertarget{alters-are-not-the-pathology}{%
\section{Alters are not the pathology}\label{alters-are-not-the-pathology}}

\emph{Posted on September 14, 2015}

Many years ago, a fellow psychiatrist courteously wrote and explained why he disagreed with my therapeutic approach of speaking to the alters. He clearly considered this an error that would lead to ``consolidating a pathology of dissociation.'' He was taking DID as a disease in which the 6 year old alter speaking in a 6 year old voice was seen as the illness rather than a symptom. Effectively, he saw the alter as the pathology that needed to be eliminated. Thus, he viewed engaging in dialogue with the ``voice'' (the alter) as clearly an unwise practice that would only consolidate the problem rather than eliminate it.

In fact, the correct analytic approach should be to consider that the unprocessed trauma is the pathology, not the alter. The alter needs to be brought back into harmony with the other parts because they are all pieces of the same psychological system.

The treatment of DID is to engage the patient's experience of having an alter as a separate part. Talking to an alter, acknowledging its presence, is a necessary step to draw that split-off piece of the self back in order to bring the whole system into a functioning unit instead of a group of perpetually conflicted and competing parts (alters).

This means that the therapist must be open to the fact that a 6 year old alter in the body of a 46 year old adult is not a symptom to be eliminated. Rather, it is a separated part of a wholeness to be healed, like a fragment of broken bone. With a broken bone, it is the fracture that is the pathology, not the bone fragment. And just as with a fractured bone, the broken piece that manifests as an alter is not garbage to be excised and thrown away. Treat the brokenness, which is the unprocessed trauma, don't denigrate it.

A fractured bone can become quite strong and functional once it is healed -- although never exactly identical to the bone that has never broken. It doesn't need to be. In that same way, once a DID system is healed, it can likewise become strong and functional -- although never exactly identical to a mind that has never been fractured in that way.

I learned over the course of 40 years of practicing psychiatry never to ignore or try to get rid of an alter. This is true however vicious an alter may initially present. Even the most angry and self-destructive alters can be seen as a repository of highly charged energy, worthy to be engaged and brought into harmony, not eliminated. Often they hold the keys to the knowledge of how the system protected itself under the pressure of the trauma as well as clarifying the path to healing.

All of the alters hold gems of insight. With a proper therapeutic alliance, they will show the therapist those gems without interrogation, prodding or challenge. Kindness and connection open the doors to healing. It is the task of the therapist to invite the patient through those doors. Just as setting a broken bone in its proper position will allow the fracture to heal, creating the proper invitation to the alters will allow that fracture the is DID to heal.

\hypertarget{did-and-schizophrenia-part-1}{%
\section{DID and Schizophrenia -- Part 1}\label{did-and-schizophrenia-part-1}}

\emph{Posted on October 1, 2015}

This is a short theoretical and philosophical discussion concerning whether or not there is any difference between DID and Schizophrenia in terms of classification, diagnosis or treatment. There are not necessarily any confirmed definite answers, but I believe there are guideposts to consider.

Schizophrenia is traditionally classified under a group of functional psychoses while DID belongs to a group of neuroses. In the traditional understanding of psychosis, the patient may lose touch with reality. In neurosis, the patient retains some acknowledgment of his illness. From this traditional perspective, Schizophrenia and DID are two entirely different kinds of mental disorders.

The term schizophrenia was conceptualized by Eugene Bleuler and further refined by Kurt Schneider (1959), a German psychiatrist whose delineation of ``first rank symptoms of schizophrenia'' remains widely adopted. Unfortunately, Schneider's primary criterion for schizophrenia is the experience of ``hearing voices.'' Hearing voices is how those with Dissociative Identity Disorder -- especially pre-diagnosis -- often describe their experience of alters expressing themselves internally. It is crucial to consider as an analogue the fact that having fever and abdominal pain are symptoms common in both malaria and typhoid. In other words, just as malaria and typhoid are two completely different physical illnesses with symptoms in common, Schizophrenia and DID are two distinctly different mental disorders with symptoms in common.

The first rank symptoms of schizophrenia are summarized in the following mnemonic of \textbf{ABCD}:

\begin{itemize}
\item
  Auditory hallucinations: hearing voices conversing with one another, voices heard commenting on one's actions;
\item
  Broadcasting of thought: a form of auditory hallucination in which the patient hears his/her thoughts spoken aloud;
\item
  Controlled thought (delusions of control);
\item
  Delusional perception.
\end{itemize}

Patients with dissociative identity disorder may report ``hearing voices'' even more commonly than patients with schizophrenia. If one is trained to presume that hearing voices is always an hallucination, then most therapists will jump to the conclusion that the correct diagnosis is schizophrenia. They will mistake the auditory manifestation of internal conflict between the alters to be an auditory hallucination that comes from nowhere, points to nothing understandable in any context, and is completely disconnected from reality.

Spiegel and Loewenstein have commented on the considerable overlapping of the symptoms of the DID and Schizophrenia. But, if we follow Schneider's diagnostic criteria with that presumption, we will have to come to the inclusion of DID within the group classification of schizophrenia. This is despite the fact that they are as different as apples and oranges in terms of classification (psychosis vs neurosis), diagnosis and treatment.
In my experience in treating both schizophrenic and DID patients, the hearing of voices in DID is quite distinguishable from the auditory hallucinations of a schizophrenic. This and other mistaken applications of the ABCD as applied to DID patients are discussed in Volume Two of Engaging Multiple Personalities.

A crucial difference between the two disorders is that schizophrenia usually causes the patient to be highly impaired in his/her thinking. Schizophrenic impairment is generally quite pronounced and leaves the individual severely dysfunctional. In the case of patients with DID, some can be extremely high functioning, while others can barely get along, but most have alters that are usually quite capable of relating to the outside world. Nevertheless, they may be impaired in other ways, such as having co-morbidity of drug addiction and/or alcoholism in one or more of the alters. As a side note, this may be why many DID individuals come to the realization that they may have DID in the course of addiction treatment -- whether at AA, NA or at addiction treatment facilities.

Generally, specific diagnostic criteria are followed in making a diagnosis, This is necessary for consistency and uniformity so that treatment guidelines can be applied correctly. It is a key tool for clinicians but like all tools, one must know when and how to use it. When one fails to recognize that there are many psycho-pathologies that display identical symptoms to DID on first, second or even third encounters, the clinician will have failed to use the tool of the DSM properly. This highlights the importance of maintaining a proper index of suspicion for all illnesses having common symptoms -- physical and/or psychological -- until one or another has been definitively excluded or confirmed.

Simply put, a patient presenting with ``hearing voices'' may be schizophrenic but, based on the percentage of incidence in the general population, may be equally likely to have DID. This highlights the limitations inherent in relying on one or two symptoms alone in making a diagnosis for mental disorders. One must examine the entire milieu of the presenting patient. This is completely analogous to the danger of diagnosing either malaria and typhoid based on fever and abdominal pain alone.

It is an inconsistency in logic to force a psychiatrist to choose whether to follow Schneider all the way and call DID a true schizophrenia with dissociative features, while understanding that in nosology (classification in medical science,) Schizophrenia is a form of psychosis while DID is a form of neurosis. At the moment, I am merely explaining the dilemma in psychiatry. While I have no definitive answer to that dilemma, I do have my experience of treating patients with both disorders that I relied upon in my practice.

I can say, definitively, that when the logical inconsistency is ignored, psychiatrists are more and more led down an incorrect path of treatment for individuals with DID. This has dire consequences that may take years to play out, investigate and correct. Unfortunately, for many patients, the dire consequences mean more trauma is inflicted in the attempt to heal as a result of the psycho-pharmaceutical blinders the incorrect diagnoses place on the therapists, in the patient files, and on the patient directly. Having a schizophrenic patient talk to the voices he hears will exacerbate his Schizophrenia. Having a DID patient engage in communicating with the voices of alters is part of the necessary treatment of his DID disorder. So, it is crucial to be able to distinguish the two in order to properly treat, and not harm, the patient.

\hypertarget{did-and-schizophrenia-part-2}{%
\section{DID and Schizophrenia -- Part 2}\label{did-and-schizophrenia-part-2}}

\emph{Posted on October 1, 2015}

Colin Ross, a pioneer and authority on DID, proposes to consider DID as a type of schizophrenia with dissociative features. He made this decision because ``two thirds of people with DID meet structured interview criteria for schizophrenia or schizo-affective disorders.'' (p.~131 Trauma Model Therapy, Ross and Halpern (2009).) While this approach enables one to conform to the DSM Criteria, in essence it is making a DID diagnosis more palatable to the general community of psychiatrists who are more comfortable identifying patients as schizophrenic than dissociative.

Despite my view presented in part 1 of this topic, of the logical inconsistency of merging a disorder that is classified as a neurosis (DID) with a disorder classified as a psychosis (Schizophrenia), there may be other tangible benefits to Ross's re-definition of DID as a schizophrenia subtype. Such an inclusion of DID as a subtype of schizophrenia may prove effective for heightening awareness of DID within the psychiatric community. As such, it may be very helpful to DID patients, so long as the therapy is correctly targeted to the DID rather than the conventional (and drug treatment related) approach to schizophrenic patients. Without that refinement in treatment understanding, this may prove difficult for practitioners to truly grasp and implement. Below, I have paraphrased Ross's explanation of this view, and as such, any error in the paraphrase and explanation is entirely my responsibility.

\begin{enumerate}
\def\labelenumi{\arabic{enumi}.}
\item
  ``A proposal of having a dissociative subtype of schizophrenia facilitates the technique of talking to the voices, otherwise therapists will never talk to the voices.'' This is a reasoning that may have wide benefits in the treatment of DID, if it enables psychiatrists to grant themselves the permission to indeed engage directly with alters.
\item
  ``A large number of schizophrenic or schizo-affective patients do not respond to conventional treatment using medication. The ethical burden or political barriers of talking to the voices are reduced when conventional treatment has not worked.'' This is a subcategory of 1 above with an important added benefit of a specific criteria indicating the need for directly talking to the voices -- that the medication that has been proven to work with schizophrenics has not worked for the patient in question.
\item
  ``Talking to the voices often works.'' As I said before, the proof is in the pudding. It seems to me the main purpose of including DID under the broad rubric of schizophrenia is to remove mainstream psychiatry's roadblocks to the technique of direct engagement with alters. It is my hope that the more psychiatrists experience the treatment benefit of speaking directly to alters, the more they will understand the efficacy of that approach in healing the trauma that is at the root of DID.
\end{enumerate}

Returning again to the ABCD of schizophrenic symptomatology, when speaking to the voice(s) respectfully, a genuine schizophrenic will likely respond with a statement that indicates a wide gap in his connection to reality while DID patients respond with contexts that make the content understandable in that specific context. The statement from the patient could be as simple as `` There is no way I can speak to you.'' A true schizophrenic may give an explanation along the lines of ``The clouds this morning were shaped like pumpkins so clearly I am unable to communicate with you.'' No matter how you go at that kind of response, there isn't a bridge to enable understanding. A DID patient would say something quite different that does indeed give a context that enables understanding.

This example comes from a DID patient that trusted me enough in our first meeting to tell me, unprompted, of her abuse history. Then, in a somewhat different voice, shd immediately said that she couldn't continue therapy because there was no way she could speak to me. This made no sense as she clearly had just spoken to me on an extremely deep level revealing core trauma issues. A few moments later, when I asked why she felt she couldn't speak to me, she gave the context: She had been abused by someone named David. Therefore, she (or one of the alters then presenting) simply could never trust me nor anyone else with that name. I immediately understood the issue and did not argue. Instead, I referred her to another therapist with a different first name.

Nevertheless, I could have mistakenly convinced myself of an ABCD analysis fairly easily. I could have presumed that the different sounding voice telling her she could never trust me was an auditory hallucination she was simply describing out loud, the non-trusting voice was broadcasting thought to the ``actual'' patient, the non-trusting voice was asserting control over the thoughts of the ``actual'' patient, and finally that the ``actual'' patient had the delusional perception that I was irrevocably related, solely through the link of my first name, to an abuser.

While this ABCD analysis may seem trivial or specious, I saw many such analyses in patients diagnosed as schizophrenic that were referred to me -- even as their files indicated strong dissociative features. The impact on such patients of the incorrect diagnosis followed by the impact of inappropriate medications -- often over long periods of time -- was incredibly harmful to the patients and their families.

I included a few examples of success using the approach of speaking directly to alters in Volume One of Engaging Multiple Personalities. I also included failures when that approach was not used. Without talking to the voices, the patients who succeeded in healing would not have stood a chance of any recovery. In Volume Two of Engaging Multiple Personalities, I make recommendations to therapists concerning implementing the technique of direct engagement with alters.

Again, it is my aspiration that more therapists will at least explore directly communicating with alters in patients with DID, or suspected cases of DID, so that they will have their own experience to consider. They can then make their own assessment as to ``the proof in the pudding.''

\hypertarget{characterizing-did-illness-or-injury}{%
\section{Characterizing DID: Illness or Injury?}\label{characterizing-did-illness-or-injury}}

\emph{Posted on October 19, 2015}

Language has power. Whether you examine it from the point of view of ordinary communication, advertising, or threats, words and how we use them have tremendous impact -- some of which is intentional and some of which isn't. This is because the words are chosen based on the experience of the speaker/writer while the impact of the words is based on the experience of the listener/reader. For those with DID, words are tied intimately to the body language of the abuser. For people without DID, they often fail to understand the power that words have to trigger retraumatization -- because they fail to understand the physicality, violence and/or threats of violence, that accompanied those words.

Given that a word may have one meaning to one person and be experienced quite differently by another, I want to look at the use of the terms illness and injury in DID. I had not thought about this before, but in a DID Facebook group, one member defined himself as injured, not ill. In considering the refusal to consider himself ill, going against most therapeutic models, he was quite clear: he had been injured. He advised me that this distinction came from a therapist at Del Amo's National Treatment (Trauma) Center. I believe this critical distinction is both accurate and subtle.

Illness and injury are often used as synonyms. Conventionally speaking, this is not usually a big deal but while they can sometimes be used interchangeably, they are not exactly the same. An illness is something that people understand to be bacterial, viral and, at least subconsciously in almost every circumstance, potentially infectious. An injury is something that is the result of some external force exerted on a person, whether deriving from a fall, a chemical, or something done by one person to another. This is not something that people, even subconsciously, generally view as potentially infectious.

The truth is that the trauma of child abuse is not an illness that arises due to a microscopic life form such as a bacteria or virus invading one's body. Those attack one at a cellular level. The body's defenses rise to fight the illness, sometimes successfully on its own as in a common cold, and sometimes successfully with medicines.

Child abuse is an external force -- physical, psychological or, often, both -- that attacks and injures the child as an entire individual. In situations of child abuse, there is no cellular defense that can rise to fight the abuser. In the case of trauma resulting in DID, the mind's defenses rise in the form of multiplicity to survive the external force of the abuser.

When someone breaks a child's arm, the broken arm is classified as an injury. If the bone protrudes from the break and becomes infected, the infection would be considered an illness but the broken bone would continue to be classified as an injury. In fact, the root of the illness (the infection) was the injury. We must keep this distinction in mind when examining the etiology and resultant manifestation of DID.

Characterizing DID as an injury, rather than an illness, has the potential to benefit those with DID as it is a more accurate classification of the source of DID. Thinking of DID as an illness implies, conventionally speaking, that one needs rest, medicine and homemade chicken soup. But, no patient with DID got it because someone sneezed near them in a crowded bus, or because they ate at a restaurant where the chef didn't wash his hands before cooking. No patient got DID because they stepped on a rusty nail. Patients manifest DID as a result of very real injuries that are unrelated to the microbial world.

This re-characterization may enable those with DID, and those without it that engage them -- whether therapists, family, friends -- to see them the same way one would see a person who has a broken leg. That person, perhaps with a cast, needs extra help. They need to be protected from anything or anyone banging into the broken leg intentionally (an abuser continuing the abuse) or by accident (a non-abuser unaware of interpersonal triggers). Just as a bone takes time to knit as part of the healing process, DID takes time to process the trauma as part of its healing process. Let us understand the injury so that we -- patient, therapist and supporters -- understand the importance of protecting both the mind and body during the course of healing.

\hypertarget{the-body-keeps-the-score}{%
\section{The Body Keeps The Score}\label{the-body-keeps-the-score}}

\emph{Posted on November 5, 2015}

If someone breaks a leg, is burned, or is otherwise physically injured, it is easy to see. It shows right there on the surface of their body. Maybe they are wearing a cast, or have a scar, or some other clear sign of damage. But when someone has been traumatized, it is not always so easy to see. Nevertheless, it is there -- locked in the body. Often you can see it in someone's posture, in the way they flinch when a sudden noise surprises them, or in the way they try to hide from a gaze.

We all lie to ourselves and to others, usually in small ways that are not a big deal like, ``I am walking out the door right now'' when we are still inside getting on our shoes. But we are capable of lying in ways that are quite dangerous, to ourselves and others. Why is this important to understand when dealing with trauma? It is important because lying is conceptual, it is manipulating thoughts and strategies. While the mind can do that quite easily, the body cannot. The body doesn't operate like that. When working with trauma, remember that words can be deceiving. Words can misdirect the attention both of the patient and the therapist. Instead, trust the body. The body doesn't have the capacity to lie. The truth is locked into the body, and the body will confirm the words that are true.

Memories of early childhood trauma usually do not come in logical, sequential verbal narratives. This is because it is mostly implicit memory rather than explicit, demonstrative memory. Explicit memory is simply not available when abuse happens in infancy, when it happens to you as a toddler, or when you were a young child. In other words, when the abuse occurred before a child's developmental unfolding of logic, of conceptual grasping of reality and thinking, one cannot expect to recall it as if describing last night's television show that you watched as an adult. Abusers count on this, knowing that the child will be unable to express a logical, sequential and, for the most part, fact-checkable explanation of their pain -- now or in the future.

As a psychiatrist, my primary concern was with treatment, with healing an injured patient. For both the patient and therapist, my advice is to refrain from searching for a logical, sequential, and verbal expression of the abuse experience. This is personal experiential stuff. If your body is telling you that you were abused, that is the foundational truth. Searching for confirmation of details is not nearly as important as trusting the truth held by your body.

How it happened, when and where it happened, are less relevant unless you are still in the physical orbit of the abuser. Trying to force the implicit bodily memory of abuse into an explicit narrative memory will likely cause further confusion and doubt. The body will allow access to the implicit memory when the patient feels safe enough to permit it, or when there is enough stress that the patient's ability to suppress the implicit memory is overwhelmed.

When the implicit memory arises, don't dissect or argue with it. It is true on its own foundational terms. Appreciate the wisdom of the body in keeping a record of the trauma, and the wisdom of the child to have survived the abuse. Allow the memory to be as it is, to be expressed as needed, but this time in the safety of the therapeutic environment. This enables the patient to start to experience the distinction between an explicit memory of the past and the present discharge of implicit memory.

Practice the ``here and now'' formula. In short, you are, at the time of this one breath in the therapist's office or in your protected place at home, safe and whole. In the midst of implicit memory, breathe. You are breathing anyway, so why not pay just a bit of attention to it. In the moment of this very breath, one can access a powerful feeling of stability. Practice just experiencing that feeling without trying to extend it, manipulate it or otherwise hold on to it. Why not try to hold on to it? Because it is now the next moment, the next fresh breath, the next opportunity to experience safety.

The more often you can experience the safety of a here and now moment, the more that experience -- on its own -- will leak into your everyday life. Work on creating the habit of noticing your breath when any past difficult memory starts to arise, implicit or explicit. Each time you connect with that safety in the breath during the remembering, whatever happened in the past begins to weaken its present grip on you.

It is a process of very small steps. The past will not suddenly lose its power, but it will begin to do so gradually. With processing the trauma gently, slowly and safely, the past will cease being so potent. It will become more and more like an ordinary memory, with limited impact on the present.

As the trauma is processed in therapy, the body will shift just a bit, letting go a little bit. What I said to my patients was that I wanted them to leave my office feeling just a little better than when they came in. In that way, there was no pressure to have a giant breakthrough with the attendant pitfalls of loading such pressure on them. Instead, my patients would make small gains without retraumatization. It was with gratitude that I could see a patient walk out of the office a little more gently, a little more erect, and feeling a little more safe inside, than when they entered.

This is not to say that patients were on a continuously uphill trajectory of healing. Everyone's life has ups and downs. This is true whether seen over the course of days, weeks, or months but also over the course of minutes, at times. So, each session with a patient was a new starting point -- how did they come in that very day and how did they leave.

The body keeps the score, and communicates it every moment. Be open to its messages.

\hypertarget{guidelines-for-therapists-on-first-encounters-with-a-did-alter}{%
\section{Guidelines for Therapists On First Encounters with a DID Alter}\label{guidelines-for-therapists-on-first-encounters-with-a-did-alter}}

\emph{Posted on December 9, 2015}

I am writing this because, in my psychiatric practice, I made many mistakes over the course of learning to work with DID. From the perspective of having been retired for the last 9 years, I have reviewed my patient histories so that others may learn from those mistakes. This is the core of my purpose in publishing Engaging Multiple Personalities Volumes 1 and 2.

The education and training I received as a psychiatrist gave me no clue as to how to identify and treat DID patients. In particular, there was no guidance or even discussion of how to relate to a DID alter that might appear in a client session. Because the first encounter with an alter is critical to establishing the necessary therapeutic alliance required for treatment, psychiatrists and other therapists need to be aware of the pitfalls of not being prepared for such an event as well as the benefits that can arise from proper preparation.

In general, DID is rarely diagnosed during the first many therapeutic sessions. According to various authors and studies, most DID patients are only diagnosed after cycling in and out of the mental health care system for several years. This is because, unlike other disorders, DID cannot be discovered through questioning or ``digging out'' information from the patient.

The foundation of therapy is understanding that the diagnostic procedure is a mutual process: The therapist is assessing the patient just as the patient is also assessing the therapist. Until the patient feels safe with the therapist, and thinks the therapist is or may possibly be trustworthy, the patient is not going to share their innermost secrets or confidential material with the therapist. The DID patient, in particular, due to amnestic barriers between the host and alters, will likely be barred from even being able to access that information. Through decades of experience interacting with people, alters are hyper-vigilant in evaluating who is likely or unlikely to understand their plight. They will not risk being ridiculed by someone unlikely to listen with empathy, although they may conduct themselves with aggression if they feel threatened by the therapist and/or therapeutic environment.

In the event that the DID system deems the therapist worthy of being shown an opening to those innermost secrets, an alter may suddenly ``jump out'' in the middle of a therapeutic session. In such a case, at least for me in every case in which this happened, the therapist will likely feel ``a shiver up the spine'' sensation. It is a somewhat indescribable experience. To see a little boy suddenly appear in the body of a 45 year old woman in a business suit, with a young boy's posture, manner of speech, and emotional presentation, amounts to more than a simple surprise. I developed a code of behavior for myself to follow when first knowingly encountering an alter.

These are the rules I established for myself (the therapist) in such a situation. I hope they will be of benefit to others.

\begin{enumerate}
\def\labelenumi{\arabic{enumi}.}
\item
  I shall remain stable in my own mind, calm and non-reactive.
\item
  I shall treat the alter with respect and appreciation that he/she is willing to be seen by me and to talk to me directly.
\item
  I shall contain my curiosity and refrain from asking for a complete personal history of the alter as that could be interpreted as an interrogation.
\item
  I shall just wait in a silence of empathy. The alter will likely tell me all he/she wants me to know, with minimal leading questions.
\item
  The ultimate guideline of decorum is that I behave as if I were being introduced to a new person at a social event: I metaphorically shake his/her hand and sincerely say, ``It is nice to meet you.''
\end{enumerate}

The most common mistake therapists make is based on the idea that getting rid of the alters is the prime goal of treatment. In fact, the therapist should realize that the appearance of an alter is a golden opportunity to access and clarify the confusion created by the dissociation. The alter is the main path, the highway so to speak, to access the information needed to enable the alter(s), and the system overall, to process the trauma. Because of amnestic barriers, in many cases the host is not even cognizant of the abuse history. The alters hold the keys to the mystery of what is hidden behind the compartmentalization of the alters, what is being blockaded by the amnestic barriers in the personality structure of the patient.

Avoid seeing the appearance of an alter as the pathology. The amnestic barrier of dissociation is the real pathology. The priority now is to get acquainted with that sequestered part, which is essential in the healing process. That part may hold much information about the abuse history. Be prepared that there may be an abreaction in detailing the abuse history. So, do not demand details of the trauma and do not provoke the patient to recount them. Letting the alters feel comfortable and secure enough to establish a proper therapeutic alliance is the best, quickest and safest approach to avoid retraumatization. With a proper therapeutic alliance, therapy can generate a positive cathartic experience. Without it, there is only retraumatization.

The first question to ask is not about the personal history of the presenting alter. Rather, it is to find out the age and function of this alter. The age is important so that you use language that is age appropriate to the alter. If the alter doesn't wish to say their age, then take your cue from how they are speaking to you in terms of how you respond to them. In my experience, the alter will usually tell you whether he/she is, for example, a protector, a persecutor, or perhaps a fearful and suffering child still holding the abuse memory so that the system can function in some capacity without the constant burden of the trauma. The alters generally are quite aware of their function, and sometimes can phrase it exactly in that way.

It is critical to understand that an improper reaction on the part of the therapist can lead to disastrous results and will probably close off any future communication. The unwary therapist taken by surprise may make inappropriate demands when a 45 year old patient starts behaving like a toddler, and blurt out an admonishment like, ``Don't play games with me. Act your age. Go back and sit on your chair.'' In other words, harshly denying the alter right in front of you! Such a reaction, spontaneous or otherwise, is negating the person-hood of what appears in front of you. There can be no therapeutic alliance if you deny that alter.

An alter sincerely considers him/herself a separate identity. Why not just accept the alter on his/her own terms, exactly the way you would when meeting any patient who first comes into your office? No therapist can dismissively brush off a client and expect to work with that same client in any genuine way. You are going to have to work with this alter so treat him/her respectfully. This is the first rule.

In chapter 1 of Engaging Multiple Personalities Volume 1, the suicidally-depressed woman patient in her business suit suddenly morphed into an arrogant and proud 5 year old boy, boasting about his bravery and dismissing me as an idiot. I reacted calmly, and respectfully thanked him for talking to me. I addressed him in a matter of fact way inquiring for his name and purpose in being there. The result was to establish a therapeutic alliance with that alter but also to all the other alters that were listening in and watching. It was the key to the system preparing to trust me.

Conventionally speaking, of course the boy is not a separate person. But we are not meeting the alter in a circumstance where a government ID is required for entry into our office. We are talking about meeting an alter in the context of psychotherapy. Psychiatrists should have the flexibility of mind to accept that if there has been severe ongoing early childhood trauma, DID and the consequent appearance of alters, is reasonable, logical and appropriate to the circumstances. In this context, it is ridiculous to hold on to some argument about whether or not alters truly ``exist'' at all!

Don't argue with an alter, trying to convince her that she is really the host. You may be legally correct but therapeutically it will be a disaster. We need to remain focused on what works as therapeutic intervention for healing from such trauma rather than trying to force our understanding of our own experience onto the patient.

Therapeutically, the boy alter should indeed be treated as a person in need of healing in his own right. For those therapists that simply cannot wrap their head/mind around the notion of an alter, perhaps an analogy would be helpful. To refuse to treat the DID patient because the alters are doing the talking rather than the host that you think you should be speaking with is ludicrous. It would be like saying you won't treat a mute patient because they can't tell you how they contracted their illness. You would not feel justified in denying treatment to a mute because someone else in their household told you they were running a fever, vomiting and sobbing all night long.

Please respect the bravery of the alters to come out and communicate directly. Anything other than that will ruin the therapeutic contact. Treat the alter as if he or she were a completely separate identity and the result is that you will benefit all the others, including the massively unhappy frightened host.

The other common mistake is to be anxious to learn the details of what is hidden. In the past, therapists were so focused on getting that information, on an almost gossip level, that injections of sodium amytal, hypnosis, or outright interrogations were used. Instead, by preparing the alter by letting him/her know that you are ready to listen, and providing a milieu of reassurance and support, a cathartic experience will naturally follow. You will get all the information you need to conduct psychotherapy.

There is no need to push. A cathartic experience is only therapeutically useful if done in a secure environment making sure the patient is not re-traumatized in the process. It should be done in a gentle way, as if the therapist is holding the hand of a child revisiting the trauma scene. The role of the therapist is bearing witness to a crime often committed decades ago, guiding and comforting the survivor as he/she goes through the journey once more, but this time with the critical difference being that he/she is no longer going through it alone.

Another task the therapist has to perform is to gently remind the survivor that the ``here and now'' is where safety is found. This has to be repeated many times until it hits home. There are all kinds of ways to convey this message. Mostly I would point this out through the ``touch sense'' (the kinesthetic sensation), reminding the alter that the traumatic experience happened in a different place, at a different time, and with people that are not now in the room. Often the alter is stuck in the past, usually decades past, and feels surrounded by the enemies that were the original abusers. Replacing the palpable fear of the past with a comfortable bodily sensation of warmth and relaxation, of heaviness in the limbs and so on, is often quite helpful. The therapist is now helping the system process the PTSD symptoms. This pointing out of ``the past and the present'' is essential. I would use all kinds of signs and clues to point out the passage of time and the difference in location.

Trust your own experience but be prepared so that you remain stable should an alter jump out to meet you. Know that it is a sign that the patient is showing his/her trust in you. The alter is giving you, the therapist, the chance to prove that you understand and will treat the alter with respect and acceptance, that you will not laugh at him/her, and that therapy will now take a positive turn. The alter is sharing with you a deep secret. Don't waste this golden opportunity for therapy. Do the right thing!

\hypertarget{when-patients-present-memories-of-abuse}{%
\section{When Patients Present Memories of Abuse}\label{when-patients-present-memories-of-abuse}}

\emph{Posted on December 29, 2015}

For most people, and for many therapists encountering DID patients, the first question that comes to mind is whether or not to take the reported memories of abuse to be truth or fantasy. But, there is an even more fundamental question that is at the heart of the matter: \emph{Why} is that the first question for so many people, whether they are trained as therapists or not?

In my experience, it is because most people simply don't want to believe that another human being would do something so evil to an infant, to a toddler or to any small child. People don't even want to believe such things when it is adults doing evil to adults. This is clearly shown by the disbelief during World War II of the initial reports of the concentration camps, of the genocide in Rwanda, and of the Cultural Revolution in China -- among other horrific events. And so, people continue to suspend belief, and such horrors continue without protest, until the evidence overwhelms the bias against looking at the evil of which human beings are capable. The same is true with child abuse.

The raw unvarnished truth is that the abuse of children, physical and sexual, happens. The raw unvarnished truth is that such evil has happened in the past, is happening in the present and in all likelihood will happen in the future. The terrible consequences echo throughout the life of the child with ramifications in future generations in that family and for all of society. This is clear for anyone to see, if they are willing to actually look at the abuse and its cascading effects.

Consider the inclusion of fantasy as part of that first question arising when one hears a tale of abuse. To the abused individual, the use of the word fantasy, whether it is said out loud or is expressed in the subtext of a therapist's body language, can only sound offensive and demeaning. But still worse, it is a confirmation of the ongoing fear ingrained by their abuser that no one will believe them that such things happened.

It usually takes months of waiting to see a specialist, after perhaps years of gathering the courage to tell a doctor one's innermost private and excruciating history of early sexual abuse. How would you feel if you were finally able to disclose even a hint of the trauma, and then consider how you would feel if the person you are looking to for healing and support, the person in authority evaluating your trauma history, is hesitating as they consider whether or not your memory is some kind of fantasy. It is important to know that they are generally not hesitating because they think you are lying. That is a second step. They hesitate because they simply don't believe that another human being, particularly a parent or close family friend, could or would do such a thing.

But, no therapist can establish a genuine therapeutic alliance with a patient if they cannot listen deeply to such trauma material, remaining present without judgment. This means keeping one's own mind stable without doing an on-the-spot calculus concerning the details of the patient's recounting of abuse. Forget the calculus, you will get the truth of the trauma far more directly and accurately by remaining fully present and grounded for the patient. In that way you can see the totality of the context, presented verbally as well as in body language. The assessment needs to be about whether or not there has been trauma is the point, not the details.

My advice to therapists is to sit still and project genuine empathy, empathy based on understanding that any individual talking about being abused has experienced trauma. As with any memory, traumatic memory does not need to be 100\% accurate in its detail because it will be accurate in its context.

Look at an ordinary memory, for example my memory of my childhood bedroom. I remember it as being quite large. There is no doubt that if I were to walk into that bedroom today, it would appear to be quite small. But no one would challenge my memory of that bedroom as being fantasy. It would be taken for granted that when I was a small child (the context of the memory), I would definitely have experienced it as much larger than I would experience it as an adult.

So, when listening to a patient's memory of trauma, particularly a flashback of trauma, don't be stuck on proving or disproving ``fantasy.'' To proceed with therapy, it is enough to know that there was trauma that is reaching into the present and trapping the patient in its past.

The use of the word ``fantasy'' can be traced back to the very beginning of psychoanalytic theory. When Freud formulated his theory of neurosis by the end of the 19th century in Vienna, he had already encountered many patients who talked about early sexual experiences with their fathers. He privately wrote to a friend that it was a highly significant discovery, like discovering the source of the Nile. The discovery suggested, for the first time, that there was a causal link between hysteria and early childhood sexual molestation.

When Freud delivered his first lecture on this causal connection, the academic and medical authorities were quite unreceptive to this discovery. He explicitly used the terms incest, rape and gross sexual abuse in describing the experiences related to him by his patients. Krafft-Ebing, then one of the most prominent physicians of the time who was senior to Freud both in age and professional stature, described Freud as ``spinning a fairy tale.''

Having felt the ice-cold response to his discovery, Freud then changed his theory and used the term ``fantasy'' to describe the recounted sexual experiences he heard from his patients. He then postulated that it was a kind of wishful thinking that infant girls had for their father.

There have been many explanations for this change in his view: Was it beyond his imagination to believe that these molestations in fact took place? Unlikely, as his original presentation was quite explicitly about molestation, not imagination. Did he change his words and his mind to ensure the survival of psychiatry in the harsh intransigent academic world of Heidelberg and Vienna which at that time was the center of science and medicine in the Western world, or perhaps as a way to preserve his own reputation in order to be able to continue his work? Possibly. Was he afraid to force a confrontation with leading lights of society whose daughters told him of having been abused, a confrontation he might easily lose? Quite possible given that this is something that continues to happen up to this very day, when people are terrified to confront abusers that are leading lights of today's society.

Regardless of why he changed the theory, and whether or not he then reverted to his original view, his use of the terms ``seduction'' and ``fantasy'' enabled society and the abusers to infer participatory intent in the abused children instead of forcing an acknowledgment that the abuse was exactly what it was -- rape, incest, and assault, just as he had originally characterized it.*

Later, much later, psychiatrists like Judith Hermann, in her extremely clear and invaluable book ``Father-Daughter Incest'' published in 1980, elaborated the truly sinister aspects of such early childhood sexual abuse experience.

Today, we should correctly appreciate Freud's discovery of the link between hysteria and the psychological experience of a patient's childhood. At that time, it was well beyond the imagination of others. People were then, and many still are, stuck on searching a biological root for the phenomenon which, in Freud's time, was called hysteria. There are still psychiatrists obsessively denying the impact of early childhood abuse on adult patients as they search for a biological cause of the mental phenomenon that results, whether it be deemed hysteria, DID, PTSD or other diagnoses.

Based on my clinical experience, the odds are that Freud's patients were indeed victims of incest, sexual assault and abuse. As the studies and news reports continue to highlight the ongoing patterns of molestation across religious, cultural and ethnic lines, it is a phenomenal disservice to patients to presume as a therapist that there is a burden of proof a patient must meet before the therapist is willing to try to establish a therapeutic alliance. The moment such a burden is placed on the patient, the ground for a therapeutic alliance is likely poisoned.

Sit still, be kind, project empathy. A patient will experience anything else as the therapist assigning himself the role of judge and jury. Remember that no memory is foolproof, no memory is incontrovertibly accurate in all details. But also remember that the heart of the matter, the energy of the memory, is accurate in context. Don't fear acknowledging the context; sit still and listen deeply.

\begin{itemize}
\tightlist
\item
  \emph{``Assault on Truth'' by Jeffrey Masson (1984) has some convincing alternative explanations of Freud's views on abuse as well as the development and possible repudiation of the seduction theory.}
\end{itemize}

\hypertarget{roots-of-psychiatry-the-reality-of-childhood-trauma}{%
\section{Roots of Psychiatry: The Reality of Childhood Trauma}\label{roots-of-psychiatry-the-reality-of-childhood-trauma}}

\emph{Posted on February 26, 2016}

The first thing that often comes to mind for a patient as well as for the therapist is whether memories of early childhood abuse are truth or fantasy. Often such memories are dismissed automatically as being untrue -- even by the adult who had been abused as a child. I believe that the reason for this is that people don't want to believe that horrific abuse of a child can or has happened -- to themselves or to others. This societal issue played out in the earliest history of Psychiatry. It may be helpful to examine the background for the use of the term ``fantasy''in psychiatry.

Human communication presupposes that people, in general, present themselves and are taken pretty much at face value. In ordinary conversation, one generally does not assume that what one hears is fantasy. The only time one considers something spoken to be fantasy is when it is explicitly stated to be so or when the content is simply beyond the belief of the listener. The crux of childhood trauma is connected most definitely to the latter.

The use of the word fantasy in psychiatry is tied to Freud's ``seduction theory''of hysteria. But it is important and instructive to note that at the beginning of his work, prior to propounding the seduction theory, he used various words interchangeably in an 1896 paper entitled ``The Aetiology of Hysteria'' to describe ``infantile sexual scenes'': Vergewaltigung (rape), Missbrauch (abuse), Verführung (seduction), Angriff (attack), Attentat (a French term, meaning an assault), Aggression, and Traumen (traumas).'' All these words explicitly characterize sexual violence directed against the child by an adult. The infantile sex scenes were not characterized as fantasy according to that original work.

Many of Freud's patients were suffering from what was then termed ``hysteria.'' Those working with DID, as patient or therapist, will recognize that the common denominator in all kinds of hysteria discussed at that time in psychiatry is dissociation. Freud's patients were often daughters of prominent men in society, or even of his colleagues. It may or may not have been beyond his imagination to believe that the sexual misconduct of his own upper class community was factual, although clearly in ``The Aetiology of Hysteria'' he did not doubt that the molestation memories his patients presented were truth. Nowhere in that paper does he raise the question of the memories being fantasy.

However, having formulated his theory of neurosis at the end of the 19th century in Vienna, he had to find an explanation for the sexual memories that was acceptable to his colleagues, the Viennese circle of eminent neurologists and neuropsychiatrists who dismissed his early work.

The result was the ``seduction theory.'' My understanding is that Freud used the word seduction to soften the tone describing sexual abuse. Using the word seduction implied a consent by the infant, that the infant consented to have sex in the context of seduction. Even that partial blaming of the infant was too close to accusing adults of abuse, so it was rejected by his peers. Freud ended up repudiating the seduction theory, characterizing the expressed memories of his patients as wishful thinking. In other words, there had been no actual sexual conduct. The fault was in the patient, having fantasized a sexual relationship with their father.

In this way, he made the expressing of memories of early sexual experiences with their fathers acceptable in the context of therapy because there was no actual accusation of molestation. In my opinion, the case histories described genuine examples of incest, rape and gross sexual abuse -- not fantasy. The explanation given by the seduction theory was that such molestation never actually happened but rather came from patients' wishful thinking. In this explanation, Freud chose to use the word seduction and fantasy instead of the explicitly violent terms ``sexual assault'' and ``rape'' that he used in 1896.

With Freud having characterized the memories as ``fantasy,'' the word became embedded in the roots of psychoanalytic thinking about early childhood sexual trauma that has dominated American psychiatry up to the 1950s and beyond. Here then, in the very earliest roots of psychiatry, is the repetition of society's historical shifting of blame onto the victim and away from the perpetrator. It is a consequence of refusing to consider even the possibility that such evil conduct can be perpetrated on a child -- particularly by well-to-do educated adults that are often at the head of the family or at the pinnacle of society. The critical impact of this repetition in psychiatry is that it gave a pseudo-scientific/pseudo-medical gloss to the denial and dismissal of molestation memories.

According to my clinical experience, incest and sexual abuse within a family is not uncommon but is often ignored and disbelieved. A 1988 Finnish study, carried out on 9000 15-year-old schoolgirls, had found the prevalence of incest to be .2\% with biological fathers and 3.7\% with step-fathers.1 Father-daughter incest is and was not as rare as many would like to believe, even today. In my experience, the rate of incest in certain communities is staggeringly high, such as in aboriginal communities suffering the aftermath of cultural genocide.

DID as the result of early childhood trauma is not uncommon and is almost completely ignored and disbelieved. I am confident that this kind of molestation is widespread. Being part of the upper strata of society, being of any particular religion or ethnic group does not impart any immunity to this. In short, I believe Freud's insight at the very beginning was correct. It seems far more plausible that Freud's patients were in fact victims of incest, sexual assault, and abuse.

Returning to the use of the word seduction, it is often misunderstood as not being part and parcel of violence. It infers that there is some form of participation by the child, or that there is a quality of love, as it is conventional understood, embedded in the seduction2. This is because seduction has a soft romantic connotation for most people. However, one must not forget that it has nefarious connotations in cases of fraud or of trapping people into sexual exploitation such as trafficking for example. There can be no ``consent'' by an infant or child to incest or other early childhood abuse -- sexual or otherwise.

Let's not continue any such misunderstanding. Considering the use of the term ``seduction'' when analyzing the relationship between an adult and an infant, toddler or other young child is wrong, dangerous and a critical warning that bad things are happening. Calling something seduction, when in the context of sexual contact with a child, whether it be an infant, toddler, or beyond is violent. While it does not necessarily physically injure the child, that is often the case. In all cases of which I am aware, it most definitely injures the child's psychosocial development -- at least through the first 5 stages as categorized by Erickson. Through seeing that injury in a child, one cannot avoid the conclusion that it was the result of violence.

Molestation in the guise of seduction is violence. Do not be deceived by it being dressed up in fancy clothes, fancy language, or accompanied by gifts. Seduction of a child is molestation. It is violence, full-stop.

I have gone into some detail because Freud's seduction theory and characterization of expressed memories of sexual abuse as wishful thinking had been embraced for a century as a fundamental truth. It is only with the more recent findings of the severity, prevalence and universality of incest and sexual abuse that it is being questioned.

To me, the rate of occurrence of incest and abuse has been -- and continues to be -- grossly underestimated. Taking sexual abuse as a myth to be dismissed re-traumatizes all those who have been abused. Even as society now begins to acknowledge the violence against young children, in particular young girls, one continues to see the societal prejudice against acknowledging abuse and its effects on boys.

With respect to the statistics on DID, there is reported to be a 6:1 ratio of DID diagnoses for women as compared to men. It is my clinical experience that women tend more often toward direct self-harm and thus are shepherded into the mental health system while men are more likely to engage in physical altercations with the result that they are shepherded into the criminal justice system.

The tension over the question of declaring the memories to be fantasy or reality continues. It prevents many trauma patients from receiving proper diagnoses as well as proper treatment. In my psychiatric practice, after gaining a few decades of experience, it was clear that the body doesn't lie3. Traumatic events may not be recalled with precision. Whether this is due to the age of the individual when abused or the intensity of the circumstances is irrelevant. The real tension should be understood as the difference between explicit and implicit memory. The body stores only implicit memory when conceptual faculties are not yet developed, or when they are overwhelmed at the time the trauma is inflicted.

Events that might have seemed phantasmagorical to Freud or to a currently practicing therapist may be explicit memory or it may simply be implicit memory being stored and subsequently expressed in archetypal forms. Simply because you cannot imagine the trauma does not mean that the trauma did not happen.

You know a large boat has passed in the ocean by its wake, you don't need to know what country the boat came from, or how many people were on it, in order to know that it passed by. For treatment, the fact that trauma has occurred is the point to work with. As a therapist, you see the wake of the trauma, you don't need to dig out the details. The details as expressed by the patient indicate the triggers and the impact of those triggers. They are not points for cross-examination by the therapist. I encourage therapists to avoid this as well as to redirect patients from cross-examining themselves in their internal dialogues.

\hypertarget{on-calling-out-alters}{%
\section{On Calling Out Alters}\label{on-calling-out-alters}}

\emph{Posted on March 24, 2016}

Controlling the appearance of alters, how they seem to be switched on or off in a system, is a complex matter. I do not claim to know all the ways this plays out, but I suspect there are many different considerations that govern the appearance of any particular alter. It is likely that specific triggers govern certain appearances, but the overall control is based on an internal system of vigilance that is constantly evaluating the total environment.

The appearance of any particular alter likely depends on the system's assessment as to whether an alter should come out to fulfill a function or, alternatively, is triggered to jump out in reaction to a situation. In the absence of specific triggers, there is sometimes an alter, often called the gatekeeper or having that specific function, with almost complete power to decide who may come out and when.

During the course of therapy, the therapist may eventually learn specific ways to invite out the appearance of particular alters. But, we should not take lightly the ability to ``press the button'' as it were, to call out an alter for therapy. This should not be done in the absence of extreme circumstances, such as the immediate risk of serious self-injury. Instead, let the system present the alters needing therapy in its own time based on its own assessment. Sometimes the presenting will be direct, as in an alter coming out and speaking to you about their issue in a session. Sometimes it will be indirect, when one alter starts talking about the difficulties of another alter or bringing in notes another alter has written down for the therapist to read.

I give an example of a mistaken approach I once took with the hope that other therapists will not repeat this mistake. I had a patient with one severely depressed alter. At the suggestion of the patient's very supportive husband, I wanted to bring out this alter for therapy. He said that the specific alter was triggered to come out by the touching of her hair. Because the suggestion and encouragement came from her genuinely caring husband, I thought there was an implicit consent to this by the patient. That was not a correct assumption. Looking back, the touching of her hair was likely experienced by the patient, specifically that alter, as a kind of violation. I learned from that mistake, but it was a bitter lesson.

It is far more preferable to allow the system to choose, at any particular moment during therapy, to self-initiate therapy with a specific alter. In other words, it is for the system to control the appearance of an alter in need, not the therapist. Giving that power back to the patient is consistent with good psycho-therapeutic practice for patients suffering from early trauma and dissociation. I learned later in my practice that empowering the patient is essential. It is a foundational approach in therapy for survivors of early childhood abuse.

\hypertarget{is-depression-just-a-chemical-imbalance}{%
\section{Is Depression Just a Chemical Imbalance?}\label{is-depression-just-a-chemical-imbalance}}

\emph{Posted on May 3, 2016}

For decades, in trying to persuade patients to take drugs for depression, psychiatrists have given them the rationale that the medication was to ``correct a chemical imbalance in the brain.''

What is the evidence supporting that rationale? It started many years ago, when Pfizer, manufacturer of the antidepressant \textbf{Sertraline (Zoloft)}, wrote that \emph{``while the cause {[}of depression{]} is unknown, depression may be related to an imbalance of natural chemicals between nerve cells in the brain. Prescription Zoloft works to correct this imbalance''}.

Because \textbf{Sertraline (Zoloft)} was known to be a serotonin re-uptake inhibitor, it was widely assumed that it worked by increasing the serotonin level in the synapses, or gaps, between neurons. This was predicated on the further assumption that depression was related to a low level of serotonin in this synaptic space. The term chemical imbalance then became a ``go to'' cliché in the psycho-pharmaceutical view of psychiatry. While this is presented as an assumption, in fact some patients genuinely responded in a positive way. But, not all do.

However, in the subsequent frantic race to produce other kinds of antidepressant, it was found that \textbf{Bupropion (Wellbutrin)} also works in the treatment of depression. This medication works by inhibiting nor-epinephrine and dopamine re-uptake. This antidepressant is devoid of clinically significant serotonergic effects. It has no direct effect on postsynaptic receptors as does sertraline. Again, some patients genuinely responded in a positive way. But, not all do.

The general idea is that a deficiency of certain neurotransmitters (chemical messengers) at synapses between neurons interferes with the transmission of nerve impulses, causing or contributing to depression. According to this view, it remains unclear whether either one or more of the monoamine neurotransmitters are responsible for depression.

The problem with this view is the failure to acknowledge the fact that while a drug reduces particular symptoms, that does not mean the symptom is caused by a chemical problem the drug corrects. Aspirin will bring down a fever, but it is too much a jump in logic to conclude that Aspirin is correcting a chemical imbalance in the body.

Similarly, one cannot loosely use the term chemical imbalance to explain a gonorrhea infection when the infection responds to a dose of penicillin. In fact, bacterial diseases such as gonorrhea develop resistance to medications. I point out the example of gonorrhea because some strains of that STD are known to be drug resistant. It is instructive to know that such drug resistance is not labelled ``treatment resistant.'' When anti-depressants fail to work, the depression is deemed treatment resistant. More helpful and more accurate would be to use that same label of the depression being ``drug resistant.'' Just as a drug resistant STD would send the physician looking for a different treatment, when a myriad of anti-depressants fail to alleviate depression the psychiatrist needs to see that their patient is not simply a chemical soup to experiment with. There are most likely other causes of depression for that patient that playing with chemistry will not overcome.

Further evidence throwing doubt to the hypothesis of depression as simply a chemical imbalance comes from the efficacy of a newly developed antidepressant, \textbf{Stablon (Tianeptine)}, which \textbf{decreases} levels of serotonin at synapses. The fact is that many depressed people simply are not helped by these serotonin re-uptake inhibitors. In a 2009 study, Michael Gitlin of the University of California, reported that one third of those treated with antidepressants do not improve. Further, he reported that a significant percentage of the balance get somewhat better but remain depressed. If a chemical imbalance is the underlying cause of depression, and antidepressants correct that chemical imbalance, all or most depressed people should get better after taking them.

Neuro-imaging studies have revealed that the amygdala, hypothalamus and anterior cingulate cortex (specific parts of the brain) are often less active in depressed people. Some areas of the prefrontal cortex also show diminished activity, whereas other regions display the opposite pattern. When someone is under recurrent stress, a hormone called cortisol is released into the bloodstream by the adrenal glands. Long-term elevated cortisol levels can harm some bodily systems. It is well known that in animals, excess cortisol reduces the volume of the hippocampus.

Smaller hippocampus volume is also associated with people with severe childhood trauma. In PTSD studies of pairs of twins (not focused on early childhood trauma), where one had been exposed to trauma and the other has not, there is a significantly smaller hippocampi in the twins with trauma exposure when compared to their twins without trauma exposure. It is noteworthy that depression is almost always present in those with severe childhood trauma and it is almost always a part of the Chronic PTSD picture.

Thus far, there has not been established a clear or direct cause-and-effect relation between brain chemistry and depression. Chemical Imbalance is just a vague term to suggest that there seems to be some chemical disturbance associated with depression, and that certain drugs are known to alleviate depression in some of these depressed patients. The explanation is speculative and the proof is far from conclusive. It is not known if the depression generated the chemistry or if the chemistry generated the depression. Depression almost certainly does not result from just one change in the brain chemistry. A focus on any one single piece of the depression puzzle---be it brain chemistry, neural networks or socio-psychological stress (for example a recent or remote past stressor) is gross simplification.

From a clinical point of view, depression as a symptom began to assume the status of a disease. It is akin to classifying a fever as a disease, rather than as a body reaction to a stressor. Internal medicine has not taken that step: We still limit ourselves to documenting fever for investigation to look for its root cause. In psychiatry, that limitation of distinguishing symptoms from disease has gradually eroded to the point where we are bending the diagnostic criteria for making diagnoses. We can now ``diagnose'' the illness as ``Major depression'' or ``Bipolar affective depression.'' In short, we have selected a bunch of symptoms, put them together and call it a syndrome, a disease.

The psychiatrist may be eager to find a disorder that comes with a textbook protocol of pharmaceutical remedies. In fact, to make a diagnosis of either major depression or bipolar, the symptoms have to satisfy a stringent list as laid down in DSM 5. Often anxiety and agitation may be interpreted as hyperactivity mimicking hypomania. Bipolar is easier to ``treat'', as there is a standardized algorithm to follow. Once diagnosed as bipolar, the main treatment approach is pharmaceutical.

Arriving at a DSM 5 psychiatric diagnosis does not and should not make therapists feel satisfied and over-confident to the point of ignoring other complicating and contributing factors influencing the clinical features. The danger today is the false confidence a therapist has once a bipolar label is established, the entire attention is focused on an exclusively pharmaceutical approach. One then has the protocol of waiting for the medication to work, which usually takes weeks. If the medication in adequate dosages fails to work after a few weeks, should one double the dose and wait again? That is certainly one part of the protocol promoted by the pharmaceutical companies' guidance.

If the patient starts self-destructive behavior, does it mean her depression is worse, or she is feeling hopeless. Perhaps her children are being taken away for adoption because she is considered to be an unfit mother. Would that not be a reasonable, non-chemical imbalance based cause to be depressed? I have seen numerous examples of cases where once the focus is placed on pharmaceutical treatment, it is as if all socio-psychological factors impinging on the life of the patient can be and are ignored.

We know quite little about depression on a molecular level. Given the multiple reasons for the etiology of depression, to call depression a chemical imbalance in the brain is reminiscent of the classic story in which a group of blind men each touch just one part of an elephant to learn what the animal looks like. If one man happens to have touched the tusk of an elephant, he would swear that the elephant is like a cylinder of polished hardwood while another touching the elephant's stomach would swear it was like a wall. The catchphrase ``chemical imbalance'' suggests a phenomenon associated with depression. But, association does not necessarily mean causation.

We really know very little about depression as a disorder. What we do know is that in patients with depression, less than half (roughly speaking) may have their symptoms alleviated by taking an antidepressant.

I am not against the use of antidepressants in treatment. I have witnessed effective and even dramatic responses to antidepressants in some patients. However, I am totally against mechanistically calling a symptom a disease and blindly prescribing a pill for that symptom -- especially when the symptom is often a normal emotional response to real life circumstances. Such a course of action can keep a person dysfunctional for years. With that mechanistic view, treatment will be fundamentally limited to finding the magical antidepressant that works, or, at best, one that produces the least harmful side-effects.

While common sense and the history of psychiatry dictates that psychotherapy should be the first line of treatment when someone displays mental health issues, in their eagerness to expedite recovery, psychiatrists starting treatment with pscho-active drugs may lead them to ignore psychological factors for depression, such as severe childhood or other trauma.

Ultimately, which patient should be prescribed drugs as a priority is a matter that should be determined by an experienced and compassionate psychiatrist. To understand the causes of depression, we have to see the entire person, rather than just looking for a chemical disorder called depression. We have to maintain a strong index of suspicion for hidden or affirmatively ignored childhood trauma. It is imperative that we therapists always look at the patient as a person, with mind, body and spirit. Only deep listening and empathy can help to bring to awareness, in both therapist and patient, those significant factors that can manifest as depression. We should not attend to just the brain chemistry in a patient with depression. Just as when a car is by the side of the road, we do not just assume that the battery has died. It may be that the driver has run out of gas or is taking a nap!

Anyone can practice medicine if all he does is to prescribe aspirin for fever, a broad spectrum antibiotic for infection, a pain-killer for pain and a steroid as an anti-inflammatory. These are standard non-specific medications for common symptoms in general practice. Such a practitioner will help some people with some illnesses that those non-specific medications can benefit. He will cause harm to virtually all others due to this lack of insight and lack of a proper index of suspicion for the many diseases that actually affect people.

Depression is a common symptom for almost all patients coming for the first time to see a psychiatrist. Prescribing an antidepressant as soon as one sees depression in patients is a cop-out that can have enormously bad consequences. Psychiatry must be on guard against the brain-washing influence of both the pharmaceutical industry and the insurance companies as the payees of the health care providers. We must not embrace ``chemical imbalance in the brain'' as the answer to the question of depression. Far too many working in the Mental Health field have fallen into that way of thinking. We need to wake up and re-examine our basic understanding of human beings again. The obligation to our patients is their well being. Our depressed patients are not just simply pools of chemicals that are not in balance!

\hypertarget{working-with-traumatic-memory-practically-speaking}{%
\section{Working with Traumatic Memory: Practically Speaking}\label{working-with-traumatic-memory-practically-speaking}}

\emph{Posted on May 11, 2016}

In psychiatry, and in fact for all kinds of counseling, all procedures start with collecting data from the patient. Starting with the individual's history, finding out what is happening with the patient and learning the psychological background as well as social context, one then attempts to comfort, counsel and heal. This information gathering involves asking some questions but more important is listening to the clients', and sometimes others', account of the current and the past situations. Often past trauma is an essential part of the history. Thus, understanding the dynamic of traumatic memory is fundamental to gathering history, just as it is fundamental to proper treatment.

All police officers, judges, counselors, therapists, clinical psychologists, and psychiatrists must at least have some basic knowledge of this dynamic. Without it, grave misunderstandings may arise. The individual's veracity may be questioned and incorrectly denied. Injustice may be the result based on misunderstanding of the dynamic and demanding a narrative of non-declarative memory. Such a demand simply won't work. Non-declarative traumatic memory is simply not expressed as a narrative. That doesn't imply that it is false. It simply means that one has to understand it without the crutch of a conventionally presented storyline.

Often, some past trauma is not remembered. Past trauma is not something anyone really wants to remember, especially if remembering it means, in one's body, that one re-experiences it.. However, eventually past trauma will resurface. Not too long ago, there was a great deal of furor debating on this topic. The question was posed as to whether or not such ``recovered'' memory, memories that eventually resurfaced, especially during psychotherapy, can be accepted at face value.

``Repressed memory'' is a Freudian term referring to memory that has been unconsciously blocked, due to the memory being associated with a high level of stress or trauma. The theory postulates that even though the individual cannot recall the memory, it may still be affecting them consciously.

A more neutral term, ``forgotten or lost memory'', is often used instead. Some studies have shown that forgotten memory can occur in victims of trauma, while others dispute it. According to some psychologists, forgotten memory can be recovered through therapy. Other psychologists argue that this is in fact rather a process through which false memories are created by blending actual memories and outside influences. According to the American Psychological Association, it is not possible to distinguish lost memories from false ones without corroborating evidence.

So, if a patient begins to remember traumatic memories during the process of therapy, how does one know if such memory is accurate or iatrogenic, meaning that the patient has been misled by the therapist into creating a false memory? In psychotherapy, recall of the traumatic past during the process of psychotherapy is commonplace. This includes ``dissociative amnesia,'' which is defined in the DSM as ``an inability to recall important personal information, usually of a traumatic or stressful nature, that is too extensive to be explained by ordinary forgetfulness.''

It is well-recognized that ``Traumatic Memory'' resulting from massive psychic trauma may be associated with amnesia, as well as, paradoxically, hypermnesia. Hypermnesia refers to the unusual power or enhancement of memory, typically under abnormal conditions such as trauma, hypnosis, or narcosis.

A person may be so overwhelmed by a traumatic experience that certain aspect of, or the whole experience may not be registered. For example, many former inmates of Nazi concentration camps could not remember anything of the first days of imprisonment because perception of reality was so overwhelming that it would lead to a mental chaos. (Read Krystal: Massive Psychic Trauma (1968)) At the same time, some part of the traumatic memory may be extremely vivid as if etched in the psyche. An example of this is when a rape victim may retain in great detail the pattern of the curtain behind the abuser at the time of the assault with only the haziest recollection of the appearance of the abuser.

Therapists, police officers and other professionals, unfamiliar with this paradoxical phenomenon, may question the veracity of the victim if the recall of the trauma contains both amnesia and hypermnesia. They presume that, ``If the woman was beaten and raped, surely she should remember correctly the color of the car that drove the assailant away.'' It is dangerous to use our own ability to access non-traumatic memories as a standard against which we judge a trauma victim's response.

Fundamentally, there are two kinds of memory: the narrative (explicit memory), and the non-declarative (procedural) memory. The former is involved in the straightforward narrative of an event while the latter is involved in memory that is often unconscious, sub-conscious or simply beyond verbalization. For example, this can refer to recalling an experience such as riding a bicycle (pertaining to motor skill), an emotional response, or a reflex action.

The conversion of the raw data of experience into memory is sifted through different neurological structures such as the amygdala and the hippocampus in the brain. Memory retention is often related to the valence of the emotion associated. Moderate to high activation of the amygdala enhances the long-term potentiation of narrative memory mediated by the hippocampus, while extreme overwhelming arousal disrupts hippocampal functioning, leaving the memories to be stored as affective states or sensori-motor modalities such as somatic sensations or visual images but as not narratives.

One tends to remember something very special, such as the phone number of a person with whom one is very much infatuated. But, in the immediate aftermath of a car accident, the color of the other vehicle may not be registered in one's narrative memory because of the psychological shock experienced at the time. This is where the therapist (or police officer), in taking a history related to extreme trauma, may find patches of amnesia. One must never jump to hasty conclusions declaring such memory as false just because it has amnestic holes in the narrative. The paradox is that due to the overwhelming arousal, what would ordinarily be stored as narrative memory is instead stored as non-declarative memory.

The above is a gross simplification of the activity of some of the neurological structures that relate to trauma and memory. Because many people are not able to understand or even recognize this complex phenomenon of the impact of trauma on memory, victims are often disbelieved. They are challenged based on their ``inadequate'' narrative memory of the traumatic experience. But the narrative component of traumatic memory is typically like Swiss cheese, full of holes. It is adding insult to injury to demand a survivor prove his/her case of having been abused in early childhood as a narrative, after they have finally pulled together the courage to come forward to bear witness to their abusive experience. The victim, and their non-declarative memory, are not to blame.

Practical guidelines to follow when one suspects a patient history that includes trauma:

\begin{enumerate}
\def\labelenumi{\arabic{enumi}.}
\item
  \textbf{Avoid obsessive digging at the past.} Do not interrogate a patient before a therapeutic relationship has been established. Even after establishing such a relationship, avoid demanding details. Remember that every question telegraph's the questioner's bias to the patient. By the choice of words and the affect associated with the question, one's bias is revealed in the tone of voice, in body language, etc. Limit your presentation of bias to the extent you can. It takes special effort to phrase a question -- including one's own body language -- in a neutral way. Make the effort. The goal is to permit the patient to allow traumatic memories, if they do exist, to arise in their own time and in their own manner of presentation. If you do this and such memories arise, they will arise with authenticity and be far more available to healing.
\item
  \textbf{The less interrogation, the easier to establish a therapeutic alliance.} In the absence of interrogation, in a container of stable warmth, it is far more likely that trust can be rapidly established. With that trust, trauma information will be forthcoming when and as needed. Usually, it is presented by the patient without any need for prodding by the therapist.
\item
  \textbf{It is not important to know all the details.} The task of the therapist is to help patients deal with the psychological and the physiological effects of past trauma. For example, is the patient able to bring her mind and body back to the here and now, or is she stuck in the past? Successful therapy doesn't mean the patient must learn and acknowledges all the details of the past trauma. Success is demonstrated when the patient is able to live in the present experiencing safety unencumbered by the past trauma. The patient's ability to control the disturbance of the memory of the past, to be able to come back to enjoy the present moment of safety and peace, is the hallmark of recovery. The patient will tell you what is important to work on.
\item
  \textbf{You are not preparing a police report.} The central issue is whether the patient is able to develop some detachment and objectivity of the experience. This means that the patient no longer experiences retraumatization, no longer becoming overwhelmed and re-living the trauma when the memory arises. As a therapist, the goal is healing -- not building a court case. Neither you as the therapist nor the patient needs to prove the dotting of every ``i'' and the crossing of every ``t''.
\item
  \textbf{Understand Traumatic Memory.} Traumatic memory consists of images, sensorial and affective states, and behaviors that are invariably consistent over time. These memories are highly state-dependent and cannot be evoked at will. They are not condensed to fit social expectations. Narrative memory is social and adaptable to the needs of both the narrator and the listener. As such, it can be expanded or contracted according to social demands.
\end{enumerate}

Survivors of early childhood trauma are usually left with non-declarative memories of horrific past experiences that are locked in somatic and sensorial memories. These are usually terrifying as they survivors lack a narrative memory to help conceptualize frightening visual imageries. It is common that people are unable to accept these thoughts and feelings.

Once people become conscious of the intrusive qualities of the trauma memory, they are likely to try to fill in the blanks and complete the picture. The stories that people tell about their trauma are as vulnerable to distortion as are people's stories about anything else. As a result, trauma history may be distorted when it is subjected to misguided leading questions from the therapist. However, just because trauma history may be distortable by its lack of narrative memory or by leading questions, does not mean that trauma did not occur. Let me reiterate the point -- human memories are simply not 100\% accurate. We are not computers or digital cameras playing back a recording.

\begin{enumerate}
\def\labelenumi{\arabic{enumi}.}
\setcounter{enumi}{5}
\tightlist
\item
  \textbf{Truth and Non-Declarative Memory.} With non-declarative memory, accuracy to a third party's conceptual (narrative) understanding of ``truth'' is not the point. Just as the host in a DID system may simply refuse to believe the truth of the non-declarative memory, that memory is accurate in its context. As I have mentioned repeatedly, the details are not necessary to the therapy. Once the therapist has determined that trauma did occur, let the patient assess the right time to disclose an abusive history in a form and context of their choosing. This is far more likely to produce benefits in therapy as compared to an interrogation based data collection that seeks to determine ``exactly'' what happened. For the therapist, it is preferable to simply accept the truth that when trauma occurs, details of the traumatic experience may not be recalled in exactly accurate narrative detail.
\end{enumerate}

It is more important for the therapist (and the patient) to know whether or not trauma did occur, rather than the details of who did what when and to whom. There are some specific instances where some of the details may be critical, for example when the abuser is a primary caretaker of the patient and remains in a position to further abuse the patient or others.

\begin{enumerate}
\def\labelenumi{\arabic{enumi}.}
\setcounter{enumi}{6}
\tightlist
\item
  \textbf{Memories Held by Alters.} Joan, my patient mentioned in Chapter 1 of my book Engaging Multiple Personalities, Volume 1, came to see me complaining of visual imageries and memories of her father abusing her -- even though she did not believe it had ever happened. She was afraid she was going out of her mind, that she might be locked up as a crazy person for having such thoughts. Such amnesia, which in this case included the refusal to accept that abuse had happened, is typical of abuse memories when they are being held and expressed only by an alter. The inaccessibility of such memories to the host is exactly the safety dynamic that enabled the individual to survive the abuse at the time it was happening.
\end{enumerate}

The function of such an alter is to spare other parts of the personality the burden/pain of the abuse. This is an example of true dissociated memory. Despite many papers which have argued against ``repressed memory,'' I have seen it vividly during direct interactions with patients. People who have been traumatized as young children will almost never be able to tell you about it when they first come to see you as your patient. Information gathered through some compulsory interrogation on the first patient's first visit must be viewed with caution.

\begin{enumerate}
\def\labelenumi{\arabic{enumi}.}
\setcounter{enumi}{7}
\tightlist
\item
  \textbf{Genuine Therapeutic Alliance is Key.} Those who deny repressed memory claim that to do otherwise invites false positives, abuse memories being presented because the patient thinks that is what you want to hear or that you have ``implanted'' such memories because of your own confused issues as a therapist. In other words, you have not established a genuine therapeutic alliance and therefore the idea of repressed memories is a vehicle for mutual delusion. The real issue to be concerned with is that one runs a far greater risk of getting false negatives because the patient simply cannot access the non-declarative memories in front of a stranger -- which is what you are until a genuine therapeutic alliance has been established.
\end{enumerate}

\emph{This post contains paraphrased material from Bessel A. van der Kolk's book} \textbf{Traumatic Stress} \emph{(1996)}

\hypertarget{should-integration-be-the-goal-in-therapy}{%
\section{Should Integration Be The Goal in Therapy?}\label{should-integration-be-the-goal-in-therapy}}

\emph{Posted on September 20, 2016}

Most standard texts consider a unitary personality, meaning the integration of alters into one single personality, as the ultimate goal and measurement of therapeutic success. I beg to differ. My criterion of success was and is measured in terms of social and relational function. If a DID individual is functioning with minimal internal conflict, like a well put together orchestra or football team, that is success in therapy.

It is not helpful to demand a unitary personality as the final criterion of success. After all, a DID individual is an expert in dissociation. For those with DID, Dissociation is strongly ingrained after being used for decades as a defense mechanism against overwhelming stress at the beginning of abuse, and potentially overwhelming stress one might encounter in the future. That habitual defense pattern will reappear as soon as the post-integration DID individual faces stress in the future that is greater than the strength of the integration.

I believe and maintain that the single personality ideal is a myth. In many non-DID individuals, although there are not amnestic barriers, there are clearly different parts that emerge when needed, whether it be the office personality, the romantic personality or perhaps the competitive athletic personality. So long as there is no undue internal friction, life can carry on even more in a colorful way with multiplicity.

There are many real life examples of highly successful DID individuals who are functioning in their multiplicity as a group of alters who have come to agreement of how to live together in the spirit of cooperation and collaboration within that one body.

\hypertarget{should-closure-be-a-goal-in-therapy}{%
\section{Should Closure Be A Goal in Therapy?}\label{should-closure-be-a-goal-in-therapy}}

\emph{Posted on September 29, 2016}

Like forgiveness, \href{https://www.engagingmultiples.com/trap-forgiveness/}{discussed in an earlier blog post}, the conventional understanding of closure is not necessarily a realistic goal in therapy. In my opinion, there should not be the presumption that is required for healing.

``Closure'' or ``Need for Closure'' (NFC), the latter being often used interchangeably with Need for Cognitive Closure (NFCC), are psychological terms that describe an individual's desire for a firm answer to a question and an aversion toward ambiguity. The term ``need'' denotes a motivated tendency to seek out information.

For my DID patients, the notion of closure was generally connected to seeking some outside confirmation that the abuse indeed happened exactly as remembered. In the therapeutic approach I took, the question of confirming the details of the abuse simply weren't all that important for therapy. It was clear that my DID patients had been terribly traumatized. It was clear that they were, in the present, subject to tremendous fear, anger and dissociation. They all had triggers they might encounter in the present that, when activated, at any given point in time would pull them back into past trauma. The point of therapy was to limit the impact of the past trauma on the present.

To focus on getting some kind of conventional outside confirmation of the details of the abuse misses the point. The details are not something to be healed. Horrible as they were, they are historical experiences. there is no magic wand or magic pill to make them undo them. They are, simply and brutally, the traumatic experiences that resulted in DID. The problem to be addressed, and the injury to be healed, is the past trauma still affecting the patient in the present. No therapy -- no closure -- is going to take away the fact that traumatic events occurred. What therapy \emph{can} do is support healing from the traumatic event(s) and reclaiming one's life in the present.

It is instructive that many concentration camp survivors -- even those that were liberated 70 years earlier -- continue to be impacted by the intensity of their experience. Consider that society in general does not discount their experiences. Indeed, they are now usually honored as survivors bearing witness to horror and holding a critical collective memory. Yet, whatever support they receive, the survivors of the Nazi concentration camps still carry their wounds. How they carry those wounds, and how it impacts their lives, may be instructive for treating survivors of child abuse -- whether or not they have DID.

Those that survived the camps seemed to be able to access a critical desire -- the desire to bear witness. This bearing witness can often be linked to the anger they experienced in being tortured, in being treated as if they were not even human. It is the drive to survive and bear witness that has genuine power, but it is not based on a need for closure. From a DID perspective, I would argue that this highlights the importance of the angry alter(s), who often see fighting for survival as necessary to be able to call out, at some point, the perpetrators.

Those from the camps that continue to speak out in their nineties do not appear to be concerned with anyone outside confirming whether or not their memory is true. They have the confidence that the events happened. There is documentary evidence showing that such things happened. If anything, whether it is seen as spiritual or moral, they perceive that their obligation is to warn humanity of the danger of dehumanizing one's perception of another person. This is quite different from conventional understandings of closure.

There are a few critical points that distinguish DID patients from concentration camp survivors. First, DID patients were usually assaulted as individuals by individuals close to them -- not by others from outside their immediate community. Concentration camp survivors could see that their horror was something they experienced communally -- no one denied their suffering in the camp.

Second, it was after the war, usually decades after the war, that holocaust deniers attacked survivors as liars. However, this was a minority that was confronted by the majority of outside powers. It is the opposite of the experience of DID patients where denial of their abuse history begins almost from the moment of the abuse. That denial comes from the abusers, from people they try to communicate to about it, and, based on the usually overwhelming positions of power of their abusers, the abused children themselves.

Third, concentration camp victims were of all ages while most DID patients were abused at an extremely young age -- before their ordinary ego structures coalesced, before they developed conceptual defenses and abilities to process their trauma experience. Most very young children brought to concentration camps were killed quickly, as they were too young to be worked to death. This was the case unless they were singled out for use in medical experiments by the horrific Dr.~Mengele.

In DID treatment, if one posits the therapeutic goal as closure, then the notion of closure must be framed as something attainable. It can be likened to survival and witnessing by the concentration camp survivors but in this case it is to warn humanity of the horrific danger and consequences of child molestation and other abuse. This appears to be happening, finally, as more and more victims of child abuse become willing to talk about their trauma history.

While there is likely no ``closure'' for the vast majority of DID patients in terms of external confirmation of the abuse, there is the very real possibility of hope, of joy, and of liberation in reclaiming one's life. This hope, joy and liberation is the best and genuine closure.

\hypertarget{resetting-the-nervous-system-after-trauma-part-1}{%
\section{Resetting The Nervous System after Trauma Part 1}\label{resetting-the-nervous-system-after-trauma-part-1}}

\emph{Posted on June 15, 2017}

I have paid close attention to the insights of Peter Levine ever since his book ``Waking the Tiger: Healing Trauma'' was published. He pointed out that while animals face being hunted down almost on a daily basis, they are virtually immune to traumatic symptoms. With that observation in mind, we have to take a fresh look at trauma healing in human survivors.

Healing involves understanding the role in healing played by bodily sensations, especially in the kinesthetic sensation. Through heightened awareness of these sensations, trauma can be healed. Levine's approach does not fall into the prevailing practice of over-emphasis on pharmaceuticals even though he considers the problem to be a physical and neurological one.

These are my thoughts concerning his ideas. If there is any confusion or accidental misrepresentation of his work in what I have written, the fault is entirely mine.

The task of the therapist is to help clients by offering them an ``island of safety.'' It doesn't have to be a big island, even a very small one is beneficial so long as it is safe. Then, give them a tool so that they can get to that island of safety by themselves. This is accomplished by teaching them self-soothing techniques.

The ``island of safety'' refers to a palpable, kinesthetic sense of comfort and security; an experience of safety and of being firmly anchored. This must be clearly communicated and demonstrated to the client in therapy so that the therapist conveys the experience, not just the idea.

As survivors of past trauma know, the consequence of the trauma is something that is deeply locked in the body. It is quite different from a conventional understanding of memory, which is narrative. It is unlike the memory of recalling what happened at the ball game yesterday. It is more like a computer that is stuck in a loop and must be re-booted.

``The body keeps the score,'' is a phrase emphasized and coined by another trauma specialist, Dr.~Bessel van der Kolk. Basically, the body has lost the ability to feel safe. The patient has lost even the memory of what ``safe'' feels like. This is something that comes up often in therapy.

Therefore, the task in therapy is to help the client to ``reset'' his or her hyper-alert response. Post trauma, patients are ``jumpy'' with easily triggered nervous system reactions. With therapy, the patient can realize that it is possible to re-learn and experience safety.

How is this possible? The path to healing is to teach the client that although terrible things did happen in the past, it is possible to experience feelings of safety and joy by living in the present. To a frightened or hurt child, pain and insecurity is felt in the body without narrative memory. The best remedy is to experience the mother (usually but not always) to ``physically cradle the child in her loving embrace.''

The cradling by any genuinely protective adult has power. This is how healing is possible. But, as therapists we cannot cradle a patient for reasons that, although obvious, bear repeating:

\begin{enumerate}
\def\labelenumi{\arabic{enumi}.}
\item
  It is prone to leading to transgressions of the therapeutic boundary.
\item
  Our client is not a child, even though the emotional feeling they may express is that of a helpless child.
\item
  The therapist, ideally, is teaching the client ``to fish'' rather than being given a fish as a temporary measure. One aims to teach the client self-healing rather than dependency on the therapist or a pill.
\end{enumerate}

The procedure is teaching the client to generate a sense of self-reliance, of learning what constitutes comfort, security and physical stability. The most immediate goal is to bring the client to the relatively comfortable and secure environment of the here and now through actual experience rather than through words or pharmaceutical intervention.

Words do not create bodily sensations and often run the risk of unknowing retraumatization. I have had patients whose abusers repeatedly used the word ``relax'' as a prelude to the abuse. For such patients, to suggest that they relax is one of the worst possible things to say. While retraumatizing words have immediate bodily sensations for the patient, they are the opposite of what you need to convey. They remember the feeling of trauma, they do not remember the feeling of safety. Therefore, mere words are often empty of meaning for patients.

There are medications that can temporarily quell emotional distress. However, there is a clear difference between the actual experience of safety and the experience of suppressing distress through medication. The actual experience is tangible safety. It nourishes the patient and undermines the power of the traumatic events. Quelling the emotional upheaval through medication that suppresses the distress is not the experience of safety. It is simply the apparent absence of upheaval. These are quite different things. I use the term ``absence of upheaval'' because the distress will re-appear the moment the medication's suppressive effect weakens.

We must be careful to ensure that our work will not be re-traumatizing, so pre-therapy contracting work is important here. The therapeutic alliance is basic and fundamental to the process. It needs to be cultivated and sustained through empathy, positive regard and congruence.

\hypertarget{resetting-the-nervous-system-after-trauma-part-2}{%
\section{Resetting the Nervous System after Trauma -- Part 2}\label{resetting-the-nervous-system-after-trauma-part-2}}

\emph{Posted on June 15, 2017}

Levine suggests two very simple physical procedures. One is to ask the client to put one hand in the opposite armpit, creating a physical sensation of containment of warmth and gentle pressure, directing one's attention to something immediate, tangible, and palpable in the body. Note that these words convey healing by simply directing the client's attention to feeling their own kinesthetic sense (sense of touch) in their body. Note how much more immediate and real the experience is of such instruction than if a therapist says the words: ``try to relax'' to the client.

Levine also suggests asking the client to put one hand on their forehead, the other on the chest. Feel the sensation. Or self-tapping their whole body. In general, these techniques help the client to define their bodily boundary which helps them gain a feeling of security.

The practice of tapping, as in the Emotional Freedom Technique, or gently tapping all over the body to provide physical stimulation and redefining physical boundary, creates a gently directed attention to the body in the ``here and now.'' This bypasses the emptiness of words when one is dealing with traumatic memory. Words sound hollow both to the therapist and the client when the therapist speaks to a highly agitated client.

Remember, traumatized clients are on permanent high-alert having been induced to a hyperkinetic state by the traumatic memory. The only way of reaching the psyche of such a client meaningfully is through bringing their attention back to their body.

Alternatively, the therapist could gently push a large cushion against the chest of a client and ask him/her to take a slow deep breath, hold the breath, close the eyes and slowly exhale. Use only a large cushion slowly and gently. This is to ensure that there is no physical miscommunication about the intent of the pushing.

Sometimes therapists overly rely on the use of words. I once used this method on a 59 year old man. It aroused an immediate emotional response and linking to a memory without any verbal intervention on my part. If you can assist a client to get in touch inwards, you don't have to do a lot of talking in psychotherapy!

If these kinds of exercises are done within a trusting therapeutic relationship, the client cannot escape the physical sensation of gentle pressure against the body. This bodily experience enables them to feel the comfort of relaxation. This is what is meant by grounding, or resetting the nervous system. It is teaching the client the beginnings of re-learning the experiential feelings of comfort and security. The client is induced to re-learn that long lost sensation of good feelings in the body.

I always relied heavily on the therapeutic alliance as a pre-requisite to any procedure in trauma therapy. Then, any instruction was more likely to reach the client physically, rather than merely through words. The simplest way for me to help the client to reach a state of grounding was through that kinesthetic sense, coupled with the attention to the breath.

One key point about breathing is that I never asked my clients to breathe deeply, it seemed too invasive. He/she might end up breathing in too deeply and beginning rapid hyperventilation! If the client was exhibiting agitation, breathing deeply was almost like asking them to breathe back in all the agitation they were putting out into the environment. In fact, they might mistakenly imagine them doing exactly that. The key guidance I chose to give was to ask my clients to breathe ``slowly.''

If a therapist is essentially repeating to the client that they need to try to let go and relax, well it simply is not going to work. Give them a path connected to physical sensations they can generate by themselves within their body and you have set them on the course of self-healing. You have given them an understanding of the experience of feeling safe once again -- an experience they likely thought would never exist for them again.

I have mentioned aspects of the healing process in various sections of my book series Engaging Multiple Personalities. I offer no apology for such repetition because the fundamentals of trauma healing, although quite simple, are very difficult to convey either to novice therapists or to clients.

We must reset the nervous system in relationship to the immediate physical experience of the living body. If we have been grasping a ball tightly, it is not so easy to simply let it go. We have to let go of our grip, muscle by muscle, making sure that we don't begin to re-grip the ball each time we move to the next muscle. This is what we must teach clients to do with the tight grip with which they are holding their traumatic material. Small step by small step, with ongoing empathy and support, the resetting happens.

\hypertarget{complex-ptsd-part-1}{%
\section{Complex PTSD -- Part 1}\label{complex-ptsd-part-1}}

\emph{Posted on December 13, 2017}

\hypertarget{understanding-ptsd-and-complex-ptsd}{%
\subsection*{Understanding PTSD and Complex PTSD}\label{understanding-ptsd-and-complex-ptsd}}
\addcontentsline{toc}{subsection}{Understanding PTSD and Complex PTSD}

These days, most people are familiar with the general concept of post-traumatic stress disorder. Usually, they are familiar with it as it applies to returning veterans, and to some extent as it applies to others who have experienced overwhelming distress for which they were unprepared, like earthquake survivors.

It was not until 1980 that mental health professionals seriously acknowledged the long term impact of these kinds of trauma by coining the diagnostic label of PTSD. PTSD is diagnosed after a person experiences symptoms for at least one month following a traumatic event. However, symptoms may not appear until several months or even years later.

PTSD is characterized by three main symptoms:

\begin{itemize}
\item
  Re-experiencing traumas through intrusive distressing recollections of the events, flashbacks, and nightmares.
\item
  Emotional numbness and avoidance of places, people, and activities that are reminders of the trauma.
\item
  Increased arousal such as difficulty in sleeping, difficulty in concentrating, feeling jumpy as well as being easily irritated and angered.
\end{itemize}

In 1992, Judith Herman pointed out that for the group of survivors of early child abuse, such as incest, the concept of PTSD does not adequately cover the injuries inflicted on these individuals. This inadequacy also covers cases of prolonged repetitive distress such as in cases of kidnap victims or political prisoners. Herman considers these cases as belonging to a special kind of PTSD that she proposed to call Complex-PTSD.

\hypertarget{complex-ptsd}{%
\subsection*{Complex-PTSD}\label{complex-ptsd}}
\addcontentsline{toc}{subsection}{Complex-PTSD}

Herman pointed out that these cases typically have a history of subjection to totalitarian control over a prolonged period (months to years). There are usually features of alterations in consciousness, including amnesia or hyper-amnesia related to traumatic events as well as other dissociative features.

The repetitive nature of the assaults inevitably deepens the effect of the trauma, making it an almost indelible imprint that is destined to be long-lasting. Deeper damage affects the person in the cognitive area. A common example is in the forming of almost delusional negative beliefs or expectations about oneself, others, or the world, e.g., ``I am bad,'' ``No one can ever be trusted,'' ``The world is completely dangerous at all times and in all directions''.

\hypertarget{prognosis-of-complex-ptsd}{%
\subsection*{Prognosis of Complex PTSD}\label{prognosis-of-complex-ptsd}}
\addcontentsline{toc}{subsection}{Prognosis of Complex PTSD}

Early childhood abuse is usually silent, hidden, ignored and/or being denied or dismissed, often even by professionals. They suffer alone. They are voiceless. Their primary fear is that their experiences are not believed. The implicit memory may be highly distorted or even forgotten. We have yet to develop a unified systematic approach in how to cope with such cases.

I have come across numerous instances among my peers that, despite identifying dissociative tendencies, have:

\begin{enumerate}
\def\labelenumi{\alph{enumi}.}
\item
  Expressed the attitude of ``So what, the past is passed;'' or ``What can you do about it.''
\item
  Failed to pursue an analysis to determine whether or not there was early childhood trauma because the emphasis is on the constitutional factors, such as how many siblings or uncles are suffering from bipolar or alcoholism etc., and diagnoses that are treatable with medication.
\item
  Based diagnoses on perfunctory information gathering by simply filling in the blanks, neither yields accurate answers nor leads to correct diagnoses.
\end{enumerate}

The prognosis for those patients from such colleagues is almost always bad. It is so bad that they avoided taking dissociative patients that were victims of early childhood abuse because there was no medication to prescribe. They just saw a long painful path of treatment failure and had no experience of positive outcomes. The solution for most was to note the dissociation, avoid the dissociative diagnosis and avoid working with the early childhood or other trauma material.

They often chose to diagnose such patients with Borderline Personality Disorder or Bipolar Disorder, diagnoses that would allow them to prescribe medications despite the fact that, though the failure to address the fundamental issues, the pharmacological treatment would fail. The result was often a diagnosis of treatment resistant depression -- identifying the patient's depression as the obstacle rather than their misdiagnosis and consequent erroneous treatment.

This kind of prognosis is a failure on the part of therapists. It becomes a self-fulfilling prophecy that further damages patients seeking to heal from early childhood trauma.

\hypertarget{complex-ptsd-part-2}{%
\section{Complex PTSD -- Part 2}\label{complex-ptsd-part-2}}

\emph{Posted on December 13, 2017}

\hypertarget{understanding-the-mind-body-connection-in-complex-ptsd}{%
\subsection*{Understanding the mind-body connection in Complex-PTSD}\label{understanding-the-mind-body-connection-in-complex-ptsd}}
\addcontentsline{toc}{subsection}{Understanding the mind-body connection in Complex-PTSD}

PTSD is more than a brain disease. Human beings are not simply a chemical stew that needs a little ``salt here or pepper there'' to fix them. Time and time again I have come across the knee-jerk reflex response of colleagues in trying to find the right medication as soon as they have identified a symptom, be it depression or anxiety. Their next thought is always trying to find the latest drug for depression or anxiety. We can do better than that. We must do better than that.

Treatment of Complex PTSD should include assessing the biological, psychological, social and spiritual aspects of a patient's life. In PTSD, the entire body-mind system has been overwhelmed by a tremendously potent destructive force. The result is that the individual is dis-empowered. The dis-empowerment often manifests as an almost complete loss of confidence in coping with the ordinary ups and downs of life. Therefore, we must look at the entire life of the patient to proactively assist in their re-empowerment.

The neurological system of individuals with PTSD has been damaged. It has been reset to hyper-vigilance, like an alarm system that has been accidentally set to be hypersensitive. It is as if you are in a house with a motion-sensor that will not stop setting off the alarm. There are only a few small corners in which you can move without setting it off, because it is ready to go off with the blink of an eye. Imagine how difficult your life would be if you had to constantly suppress blinking your eye, and how terrifying it would be to know, as you uncontrollably blink, that alarm is about to start screaming. One shouldn't be surprised at the speed with which flashbacks and re-traumatization occur.

Through the impact of trauma, PTSD is not only a brain (autonomic nervous system) disease. It is also a psychological disorder. Now in hyper-vigilant state, the tiniest cue will set off a huge autonomic storm. One's bodily reactions trigger the traumatic state of mind just as the memory of trauma triggers one's bodily reactions. Together, they mutually suppress, if not destroy, the memory of what it was like to feel secure and at peace.

The thought processes of the Complex PTSD individual are overwhelmingly preoccupied with fear, distrust and loss of confidence. The individual cannot perceive any hint that there is a pathway to even a moment of tranquility. The individual is constantly living in the past, poised to re-experience the trauma of being attacked, or facing another earthquake or explosion. The whole autonomic nervous system is now detoured toward re-traumatization.

Meanwhile, the fundamental question of what is the meaning of life and its purpose is thrown out of kilter. Questions arise with only angst, not answers, like: ``Why is this happening to me?'' or, in a sick family, `` Why was my sister spared when I was chosen to suffer?''

Given the depth of the impact of the trauma at the root of Complex PTSD, there is no reason to expect that a simple solution, such as deep brain stimulation, or a magical pill is going to heal such a condition. While these things may help, we need to accept that more is needed to bring about healing in the individual. We need a multi-pronged approach.

\hypertarget{complex-ptsd-part-3}{%
\section{Complex PTSD -- Part 3}\label{complex-ptsd-part-3}}

\emph{Posted on December 13, 2017}

\hypertarget{integrating-approaches-in-the-treatment-of-ptsd}{%
\subsection*{Integrating Approaches in the Treatment of PTSD}\label{integrating-approaches-in-the-treatment-of-ptsd}}
\addcontentsline{toc}{subsection}{Integrating Approaches in the Treatment of PTSD}

We have not advanced much from the old days of Descartes when the scientists started thinking in terms of either the material world or the non-material world. We still think in terms of ``either-or.'' We are still stuck to picking either the drug or talking cure.

There is a prevailing tendency to believe that talk-therapy is too slow to work or it is ineffective. We cannot prove that it works through a double blind control study to prove its efficacy. Yet in clinical experience we have come across cases where the right kind of listening and talking can achieve wonders. For example, when the diagnosis of DID is correctly identified, effective remedy is instituted, and patient's useless medication is discontinued, the patient achieves rapid improvement.

However, such cases are dismissively labeled ``anecdotal.'' They cannot be reproduced in a laboratory. Without double-blind studies paralleling pharmaceutical research, it is claimed that they do not validate any particular approach. This disparages the history of psychiatry from its inception until the advent of pharmaceutical industry control over psychiatric training programs, insurance reimbursement, and the consequent denial of the impact of the many schools of psycho-therapy from Freud, Jung, Frankl and others.

\hypertarget{an-integrated-approach-to-treating-ptsd}{%
\subsection*{An Integrated Approach to Treating PTSD}\label{an-integrated-approach-to-treating-ptsd}}
\addcontentsline{toc}{subsection}{An Integrated Approach to Treating PTSD}

\begin{enumerate}
\def\labelenumi{\arabic{enumi}.}
\tightlist
\item
  In PTSD, we identify the ``hyper-aroused nervous system.'' In that hyper-arousal, the individual is also robbed of his confidence. Why? It is because he is not in control of his own body, which suddenly and without warning transitions into a hyper-kinetic state.
\end{enumerate}

The first goal in treatment is for the body to relearn the experience of remaining in the here and now, appropriate to the reality in which it finds itself. Putting it simply, if you are running, your pulse rate should be high. If you are resting for a few minutes, your pulse rate should reflect a resting state of the body. In the PTSD experience, the pulse rate skyrockets even when the body is not running. In fact, the body starts to move in response to the messages it is being bombarded with -- message of danger. These messages come all the time, when there is no danger, when the individual is simply sitting at home.

Several of my earlier blog posts discuss grounding exercises to aid the individual to achieve this goal. Surely yoga and meditation make sense as a fundamental exercise to get in touch with the here and now body. Many experienced therapists advise their clients to do these exercises of yoga and mindfulness. The ``scientific'' therapists, perhaps more accurately the pharmaceutically trained therapists, shy away from such advice. Why? According to conversations I have had, it is for fear of being ridiculed since there is little peer-reviewed literature to support yoga and meditation as an adjunct in the treatment of PTSD.

\begin{enumerate}
\def\labelenumi{\arabic{enumi}.}
\setcounter{enumi}{1}
\tightlist
\item
  We need to open our eyes and listen with deep empathy. We should be openly waiting for our clients to tell us what their concern in life really is.
\end{enumerate}

I had a patient who was given ECT and kept in hospital for months. Her diagnosis was depression, accompanied by self-mutilating behaviour. No one seemed interested or inclined to listen to her story. It was as if her life of being abused by family and neighbors was irrelevant to her mental health.

Her children were taken away to be adopted out by relatives. She was trying hard to leave her abusive husband while her church insisted that she should try to reconcile with him. No wonder her depression never got remitted despite being seen by numerous doctors. No therapist ever came close to the Complex PTSD issue, not to mention the DID diagnosis. No one seemed to show any inclination to listen for the patient to communicate these problems in her life.

In short, we need to go back to square one. Therapists need to make sure they have a firm therapeutic alliance with the survivor, before they even begin to try to understand each and every case. There is no shortcut. Making a diagnosis of depression and prescribing an antidepressant is a far cry from thoroughly assessing and understanding the individual. The ability to write a prescription has little to do with learning about the nature of the trauma that is causing the disability.

\hypertarget{the-right-direction}{%
\subsection*{The right direction}\label{the-right-direction}}
\addcontentsline{toc}{subsection}{The right direction}

The National Institute for the Clinical Application of Behavioral Medicine {[}NICABM {]} offers a good program in training therapists to do trauma therapy. The works of Colin Ross, Judith Herman, B, Van de Kolk and others are very important. There are several self-help groups of DID survivors that have organized themselves that have a lot of good information and some training programs.

But, in short, the therapist who only offers pills will not get to the heart of the issue. Medication can be an important adjunct to psychotherapy, particularly in an immediate crisis. It is never a substitute. Complex PTSD can be healed through the efforts of the survivor and the support of competent therapists.

\hypertarget{repeating-myself-again}{%
\section{Repeating Myself Again}\label{repeating-myself-again}}

\emph{Posted on November 25, 2017}

I find that I am repeating myself more and more, perhaps because of my encroaching senility or perhaps because misconceptions die hard. I think it is because once the general public has been brainwashed, one has to repeat the truth again and again in order to undo the misconceptions.

\hypertarget{corrective-emotional-experience}{%
\subsection*{1. Corrective emotional experience}\label{corrective-emotional-experience}}
\addcontentsline{toc}{subsection}{1. Corrective emotional experience}

A genuine experience of kindness and love may convert a victim of cruelty to a kind loving person. Whether it is portrayed in fiction, such as in Les Miserable, or as displayed by Pope John Paul, long before he became Pope, carrying a concentration camp survivor on his back because she could no longer walk, kindness is an incredibly corrective experience that changes people.

For individuals with DID, it takes time for many of the alters to be convinced of the genuineness of kindness, but it can happen. Corrective emotional experiences cannot always be created elegantly with a swift and fundamental impact. But, little by little, it is like a river cutting a new path through the ground.

\hypertarget{the-time-factor-for-therapists}{%
\subsection*{2. The Time Factor for Therapists}\label{the-time-factor-for-therapists}}
\addcontentsline{toc}{subsection}{2. The Time Factor for Therapists}

I hear all the time that mental health workers have no time. This forces them to look for quick answers. The result, coupled with the intense marketing of pharmaceuticals, is prescribing pills for ``quick'' solutions. Psychiatric labelling and prescribing a drug instead of listening to their client is often the order of the day, ``You are depressed, take the antidepressant.'' Time constraint is not a good reason for looking for a short cut. Most of those short cuts have the grave risk of other problems that will obscure the root issue of trauma. Remember, we don't just take cough syrup if we have a chest infection.

As always, a reminder that looking for evidence of traumatic childhood abuse does not mean gathering of details like preparing a police report. The client will find \emph{their} right moment to offer clues of such trauma. The therapist must remain open to recognizing those clues, rather than trying to force them out. Healing starts when the survivor feels safe enough to express their experience, is believed, and is emotionally supported. When that happens, the isolation they have felt since their early trauma starts to dissolve.

\hypertarget{leaving-the-past-behind}{%
\subsection*{3. Leaving the Past Behind}\label{leaving-the-past-behind}}
\addcontentsline{toc}{subsection}{3. Leaving the Past Behind}

Time and again I hear mental health workers say that traumatic childhood experience is something we should leave in the past, that it isn't happening now, so move on to the future.

In the absence of processing the trauma, this is nonsense. I do not advocate dwelling in the past, but without healing, the past is like an old festering wound that refuses to go away. If we have an infection in our foot, we can limp along for quite awhile. We can do our best to avoid banging it against the curb when we cross the street. But when we accidentally hit the curb, we might scream in agony. We can ignore the infection for only so long until it spreads further and our whole being is under attack.

Leaving the past behind without healing the trauma is like that. We need to heal the past that is encroaching in the present.

\hypertarget{out-of-sight-out-of-mind}{%
\subsection*{4. Out of Sight, Out of Mind}\label{out-of-sight-out-of-mind}}
\addcontentsline{toc}{subsection}{4. Out of Sight, Out of Mind}

There are many phenomena that remain out of sight and, as a result, out of mind. We ignore them thinking that they do not exist, or that they are so rare that we simply will never run into them. Kind of like a ``no harm, no foul'' mentality.

The problem with that view is that in normal social interactions, we are primed to avoid the sordid and the painful that is not right in our faces.

Do not dismiss the evils of refugee displacement, gender inequality, abuse of power, cruelty to children, PTSD sufferers unable to heal, marginalization of the disabled and the disenfranchised. We can only ignore them to the extent that we believe they either do not exist at all, they are not in our sphere of experience, or are very rare.

We all have difficulty handling bad news and need to keep a sense of balance in order not to be overwhelmed by negativity. At the same time, we need to recognize that there is indeed pain and suffering constantly within, around and among us. These are not rare. Recognition of this needs to be accompanied by appreciating positive aspects of our lives. In other words, remain grounded with a balanced grasp of reality.

\hypertarget{practice-kindness}{%
\subsection*{5. Practice Kindness}\label{practice-kindness}}
\addcontentsline{toc}{subsection}{5. Practice Kindness}

We are surprised when we are informed that a respected person in a position of power gets caught for abusing over a hundred of victims under his care over a period of decades. We do not need to wonder why these cases usually take decades to get exposed. Victims are trapped so as to remain in their position by design of the abusers. Let us increase our awareness, not be surprised, and practice kindness with insight wherever we go towards ourselves and each other.

\hypertarget{diagnostic-labels}{%
\section{Diagnostic Labels}\label{diagnostic-labels}}

\emph{Posted on March 22, 2018}

A reader posted a question regarding the diagnostic label that might be applied to him. Apparently, his therapist read my Volume 1 of Engaging Multiple Personalities and decided the reader was not ``multiple'' but has ``dissociative parts''. Not surprisingly, the parts see that as a statement invalidating their existence and significance. In short, it was taken as making the parts appear to be less than real -- even though, as the reader put it, ``we feel pretty darn real''.

This issue may be something of interest for the general DID community and its support networks.

I am not sure why, after reading Volume 1, a therapist would take the position distinguishing dissociative parts from multiples in that way. In Volumes 1 and 2, I do distinguish between parts that have executive functioning capability and those that don't. But that distinction is useful only to identify which parts developed in ways that encouraged executive capacity and which parts developed for holding discrete pieces of trauma. This distinction has nothing to do with whether one part is more real than any other part -- or any less real than any other part. If the individual has dissociative parts that feel they are not being fully acknowledged because they are not seen as ``personalities'', but just as ``dissociative parts'', then I don't see how a true therapeutic alliance can fully form between the patient and therapist. If the parts feel they are separate individual personalities, who am I or any therapist to argue that they are not sufficiently distinct and separate to be given that classification? If you feel deeply about the sense that you are a personality, just like other alters, you should be acknowledged accordingly.

Diagnostic labels are just that -- labels. They are just words. They are labels used to organize ideas and facilitate communication of phenomenon or experience. They should be used to promote healing, not conflict.

For example, some readers have complained that I use the words ``multiple personalities'' in the title of the series. Given the change in the DSM from Multiple Personality Disorder to Dissociative Identity Disorder, why do I continue to do that? It is because many DID individuals, and certainly my patients when I was in practice, prefer the word personalities. They feel the term to be more appropriate to how they, including alters, feel. That was more important to me as a therapist than the views of many people outside the DID experience, including doctors or therapists, who vehemently object to the use of the words multiple and personality together, who insist there cannot be more than one person in one physical body. One could have a philosophical argument about that but will it help process any trauma? No.

I do not have any problem if my patients or anyone else prefer to use the word personality instead of identity. These are just words, so use any word that you feel applies to you that communicates your experience. Of course, you have to pay attention to your immediate circumstances in choosing the appropriate words for that context. There is no problem explaining that you have 7 personalities or identities while in a therapy session but there is no point in expecting an immigration officer at the border to understand that there are 7 of you as you show your passport at the border.

In therapy, the focus is to process the past trauma that keeps on intruding into the here and now. It is to facilitate internal cooperation, communication, coordination within the system. The idea is to minimize the conflict among the alters because that conflict prevents processing the trauma and prevents you from reclaiming your life.

It is important for the therapist to concentrate on helping the alters to feel respected, validated and taken seriously, as they individually appear, so that a genuine therapeutic alliance can be established. With that, an environment of healing can be created. Everything else is of minor importance. If you have a therapist you can work with, I would not waste time fighting about a diagnostic label. It is better to simply tell them what you need. If it is too difficult for someone to say out loud, then written messages from alters that can be delivered in a therapy session may be helpful.

Diagnostic labels are created by theorists trying to describe observed phenomena. In my psychiatric practice, the guiding principle was not theory but rather practicality -- how to help someone process trauma. Processing trauma is not theory. It is hard work. Its success is based on the efforts of the patient and the application by the therapist of psycho-therapy with kindness, compassion and empathy.

\hypertarget{inviting-alters-to-therapy}{%
\section{Inviting Alters to Therapy}\label{inviting-alters-to-therapy}}

\emph{Posted on April 12, 2018}

A reader asked about working with alters that were afraid to present themselves authentically in therapy, even though at least some of them viewed their therapist as amazing. It seems that because they were fearful and wanted to remain safe, they were prevented from presenting by what was likely a protector. At least part of the system was afraid of ``losing control.'' It seems that there was at least one part that ``filled with rage and seems to need to come out but can't.''

Internal conflicts like this are a common phenomenon. With any such internal conflict, it is important to respect all the participants and, with that respect, to engage their different perspectives. Using \href{https://www.engagingmultiples.com/the-5-rule/}{the 5\% rule} as an approach may give some level of comfort to the protector that things will not get out of hand. That same approach may allow for an alter that is enraged to express a small piece of anger at a time and feel safer doing it that way.

The healing journey is actually quicker and deeper when one goes small step by small step. Anything more runs the risk of retraumatization. The protector is likely aware of and concerned about the risk of potential betrayal and/or abandonment. The risk of retraumatization is something perhaps the protector is also aware of and concerned about. Both are important functions of protectors. For healing, creating a path that protects from the retraumatization while allowing for engagement is best.

The second part of the question flowed from the first. It concerned the experience of alters in despair specifically because they would leave therapy sessions feeling that they had not presented truly or as they needed to. The result is that they leave feeling worse than when they came to therapy, feeling once again that they had failed. I think that this is also not an uncommon experience.

Making sure that everyone -- all of the alters whether they are presenting externally or not -- is invited to listen is an important first step. This can be done at the beginning of each session. Then, at the end of each session, everyone should definitely be thanked for listening, whether they actively participated out loud or not.

It is important to acknowledge the bravery of alters that are willing to show up even if they are as yet unable to express what they need to say. By inviting them at the beginning and always thanking them at the end, you demonstrate your willingness to let them decide if and when they feel safe enough to participate directly. By doing so, you demonstrate your appreciation for their desire to heal.

The act of acknowledging alters is a powerful method of validating them because they have never been acknowledged before. Often, therapists are mistaken in thinking that alters should disappear, because they are seen as something pathological to be eliminated. This is a mistaken view.

This acknowledgment is critical. When a very hostile alter feels \emph{acknowledged and understood}, something is going to shift. Sometimes it can be like defusing a bomb, and the DID system knows this. Remember, behind anger there is always deep hurt.

When one alter is able to engage in therapy, using the 5\% Rule or otherwise, other alters will begin to feel the benefit. As one heals, the others will begin to feel safer and eventually participate in the healing process. It is a rippling effect, which often happens in DID therapy. When an alter presents and wishes to participate directly in therapy, they will do so if they are invited with genuine warmth and empathy.

Many alters will heal by witnessing the therapeutic process of other alters as they go through it. As one alter is healed, others may feel the therapeutic effect. Because of this, each alter does not need separate therapeutic intervention.

So be kind to everyone inside, be patient with them as you engage. With that kindness, with that patience, healing can take place.

\hypertarget{empathy-for-therapists-part-1}{%
\section{Empathy For Therapists: Part 1}\label{empathy-for-therapists-part-1}}

\emph{Posted on May 8, 2018}

Empathy is something mental health professionals are assumed to have in abundance. We normally take for granted that anyone wishing to be a therapist would have that fundamental quality as it is the cornerstone of proper mental health assessment and treatment. But, while many therapists have sympathy, empathy is not quite so common, particularly in treating individuals with DID.

It is important to understand the differences between sympathy and empathy. Both are necessary to engender and cultivate a therapeutic alliance but they have separate functions and impacts on both therapists and patients. Sympathy is a feeling that engenders warmth in a connection while empathy is something far more active that provokes a much more personal and deeper understanding.

Sympathy is feeling compassion for the hardships that another person has encountered or is currently experiencing. It doesn't require that you actually understand or can share in some way that person's experience. It is more like you feel bad that they have had to experience something distressing.

Empathy is actually imagining yourself in the shoes of another person, getting a sense of what their pain might really be by imagining yourself in their circumstances. It is a deeper understanding because, to a greater or lesser extent, you are touching the feelings of another person -- not just witnessing them. Sympathy is like seeing the other person's experience from the outside whereas empathy is like touching the person's experience from the inside.

While a therapist cannot truly experience the early childhood abuse of their patient, the therapist can seek to truly imagine themselves in the circumstances of their patient \emph{at the time of the original traumas}. One has to consider as deeply as possible what the terror and pain was for the patient. To do this, you cannot imagine yourself now, as an adult, but rather imagine being a small child under attack by an abuser who is 10 times your size and controls every aspect of your being. Imagine that attacker threatening you or your siblings if you were to say anything about the abuse. Imagine that the attacker is the person who is supposed to be caring for you, the person everyone in the outside world assumes is protecting you. Imagining yourself like that, having only a small child's limited verbal and physical development, and in that set of circumstances, is one way to generate empathy, to appreciate the intensity of the traumatic experience of a patient.

Doing this on an ongoing basis is a way to cultivate direct empathy for the patient. It is critical to being able to develop the capacity to communicate safety and understanding to the patient in the present. It is this capacity that enables the patient to begin to trust the therapeutic alliance that is so necessary for effective treatment.

In practice, empathy involves sympathy and compassion. So, it is important to enhance those qualities as well. It is not possible to have true empathy for someone injured in a car accident without feeling sympathetic towards their pain as well as feeling the desire to lend a helping hand. Many people, therapists and otherwise, can relate to car accidents and injuries that result from them.

Not so many people, therapists and otherwise, can relate to the circumstances that result in DID -- which are much more terrifying. It is the terrifying nature of the abuse experience, happening in early childhood, that sometimes keeps therapists from being willing to fully empathize with their patients. For therapists, one has to be careful with these kinds of empathy exercises because there is a risk of vicarious trauma. I have discussed this further in Volume 2 of Engaging Multiple Personalities as I believe it is a real issue therapists must deal with in their own lives.

Remember that while empathy is the ability to understand another individual's experience by putting oneself into the other's place, the therapist must retain their own objectivity. Therapists must be introspective and assess their own reactions to what their patient may have survived. This includes being aware of the therapist's own fears of vicarious trauma and perhaps fears as to how they themselves might have reacted had they been subjected to that abuse.

As therapists, empathy is perhaps the most crucial quality needed in the establishment of rapport, of a therapeutic alliance. Deep empathy helps our patients to be open to experiencing the therapeutic milieu as safe, as trust-worthy and as having integrity. This is the prerequisite for effective helping relationships, enabling the patients to share their innermost concerns with their therapist to begin and continue in the process of healing.

\hypertarget{empathy-for-therapists-part-2}{%
\section{Empathy For Therapists: Part 2}\label{empathy-for-therapists-part-2}}

\emph{Posted on May 8, 2018}

The important role of empathy in a therapeutic alliance is seldom emphasized in training, particularly as the treatment focus has moved toward pharmaceutical intervention. Perhaps some teachers of psychotherapy feel it is self-evident and therefore there is no need to elaborate. However, in practice, this deficiency is often evident. It shows up immediately when there is the mistaken view that information can be gathered without paying attention to the unconscious currents displayed in how the patient presents during the initial interview, the initial contact where the therapist is gathering background data. This continues if there is the further mistaken view that therapy can be conducted in a detached, apparently scientific way, as if that appearance is, in itself, sufficient.

Novice therapists, particularly those whose training has focused on psycho-pharmaceuticals, sometimes are under the false impression that merely following a checklist will result in competent therapeutic intervention and guidance. This is foolish. To think that one can expect genuine healing of depression from merely a prescription of a drug, without awareness of the patient's personal milieu and social background, the past and present contexts of their life, is both ludicrous and dangerous.

In the case of a cold or ``too busy to listen'' attitude of the therapist, relevant information is often not communicated or, if it is communicated, it fails to be identified as important. Results of intake assessment interviews can be biased if they follow a pattern of questions and answers according to what is solely the interviewer's definition of essential data. A checklist style of interview presumes that one will end up with a complete and accurate sheet of information if only one asks the right questions.

Nothing is farther from the truth. That mode of interrogation may yield many false positive answers as well as many false negative answers. When there is a lack in empathy, communication often becomes meaningless. Novice therapists may miss the critical context of a response by becoming diverted over some minor detail. A simple and unfortunately accurate example is that missing a clue to early childhood sexual abuse is a mistake of vital significance in an assessment.

Empathy directs the therapist in the how, when and what to say in the history taking. The sensitive therapist will know when to keep silent, when to ask follow-up questions, and what to ask while remaining always tuned in to the emotional tone of the communication. This means that each and every intake assessment will be different, based on the presentation of the patient.

In other words, the therapist becomes sensitive to the voice of the individual's unconscious. Conscious data and words are seen as only part of the picture. True reliable and meaningful data of the interview are obtained only in a positive therapeutic relationship. The foundation of that lies in the therapist's empathic understanding. This highlights the fact that there is no clear line of demarcation between when an assessment ends and therapy begins.

Never forget that the patient is assessing the therapist during the entire assessment event. A patient that doesn't see empathy from the therapist is not going to trust that therapist enough to make the assessment accurate. Further, without empathy, there is a very real risk that the interviewee will not return to become a patient due to that lack of trust. In other words, an improperly conducted assessment, without empathy, is already heading to a therapeutic failure.

Personally, I suspect empathy can be nurtured and developed in most individuals. But, there is a prevailing tendency to denigrate the importance of empathy, because it is not seen as true science. According to Carl Rogers (1977), three attributes of the therapist form the core part of the therapeutic relationship -- congruence, unconditional positive regard and accurate empathic understanding. These are the only tools the therapist possesses, just as indispensable to the therapist as scalpels, anesthesia and the asepsis are to the surgeon.

Today, the individuals who are overly focused on psycho-pharmaceutical approaches may forget these critical attributes. In practice, some professionals are exclusively focused on accurate record-keeping Accurate record-keeping is extremely important for therapy, but is not so helpful if it is focused on primarily for the sake of practicing defensive psychiatry, the fear of litigation. A therapist with perfect record-keeping may have done everything in a legally impeccable way -- always prescribing in accord with the manufacturers' recommendations -- but without empathy may be unable to successfully treat their patients.

In the absence of a warm ``ready to listen'' clinical approach, case after case can easily get misdiagnosed. How can that happen to good therapists? It can happen quite easily when therapists are exhausted and overwhelmed by their caseload. When the caseload becomes too much, those who only pay lip service to genuine psychotherapy will limit their success in helping to those who will respond to antidepressants. The very real problem with this then is that the carpenter whose only tool is a hammer will see everything like a nail to strike.

How often do psychiatrists go home to their double martini to relieve the distress caused by vicarious trauma? The burned-out therapist often unwittingly chooses turning off empathy as a way of protecting themselves from the emotional cost of providing therapy. The fact is that everyone wants to avoid pain, even if the pain belongs to the other person. However, therapists do not have that choice if they wish to truly benefit their patients.

Empathy requires the ability to handle psychological conflicts, including that of the therapist. It is much easier to turn off empathy and do one's work mechanically, than to listen with empathy and feel the pain of the other person. But, the penalty for that is doing bad or useless psychiatry. Therapists need to protect themselves by caring for their own state of mind. In that way, they can expand their ability to care for their patients.

\hypertarget{on-mapping-systems}{%
\section{On Mapping Systems}\label{on-mapping-systems}}

\emph{Posted on June 27, 2018}

With respect to mapping one's DID system, if you find it beneficial, then by all means do so. In my psychiatric practice I neither encouraged nor discouraged my patients to map their systems.

With my patients, it was always important to return to the fundamental point of treatment of DID, which is to allow the system to process trauma. In my experience, this happens through engaging presenting alters in a genuine, empathic and trustworthy manner. Having a schematic of their systems was not necessary to do that.

Again, based only on the experience I had with my DID patients, mapping systems was not necessary to an efficient or focused therapeutic alliance. The problem is not particularly the mapping but rather that therapists who encourage mapping systems may infer, or sometimes outright claim, that you have to understand each and every part of the system before you can heal. Certainly, for systems with massive multiplicity, this runs the risk of turning therapy into a never-ending marathon.

Mapping also suggests that therapists need to have some detailed knowledge of the individual alters, almost like requiring a census of ``who is who'' including how they are grouped or related. Mistaking meticulousness for clarity, a therapist can be lured or distracted into trying to provide individual psychotherapy of each and every alter rather than simply engaging with alters as they present. In the case of Ruth, described in Chapter 5 in Volume 1 of Engaging Multiple Personalities, some alters' problems were taken care of as a by-product of other alters who engaged with me as well as by other alters who acted as co-therapists or ``preachers'' rather than by me as the psychiatrist.

Alters functioning as both co-therapists and preachers made perfect sense in Ruth's context as she had decided the way to solve her problem was to convert the ``evil'' non-believer alters into believers (of Christianity.) As her therapist, my task was to maintain my neutrality so as to enable the therapeutic alliance to be extended to all alters, whether they were presenting as non-believers or otherwise. This individual choice by Ruth was a very positive decision in her healing journey. And, as always, I was careful to not interfere in the system as to religious or other matters unless specifically invited to do so.

Ruth had about 100 known alters when she saw me, and continued to present many, many more over time. It was instructive to see how they often had quite separate handwriting styles that remained consistent throughout and then long after therapy had ended. Years later she told me she had hundreds of alters. I was never sure if the number had grown or that she had become more comfortable in recognizing their presence. If her healing was dependent on mapping an ever-expanding system, she never would have healed to the point of going beyond the need for ongoing therapy. The fact is that after a relatively short time in therapy, for all practical purposes, her self-harming activity ceased. She was able to live independently, care for her children once again, and make a fulfilling life for herself which continues to this day, some 20 years later.

Mapping is sometimes also used to encourage the idea that integration is the appropriate goal in DID therapy. It is as if a DID system is really like humpty-dumpty and mapping would allow the therapist and patient to find all the pieces so as to glue them all back together. Readers of my books and this blog already know that I don't believe that integration is or should be the goal of therapy. Why? Because under stress, the integrated personality will again split both out of habit and the need to protect itself from danger. In my opinion, it is far better and safer to focus on healing, on eliminating the intrusion of the past into the present while training to remain vigilant rather than hyper-vigilant. If integration takes place in whole or in part, that is fine. If not, that is fine too.

The goal is to heal from the trauma. To claim that healing from the trauma \emph{requires} mapping (or integration) is a false leap of logic. The point is to eliminate the power of the past to re-traumatize you in the present. That is not based on mapping or integration. It is based on engaging alters so as to allow them to process the trauma in which they are trapped, that they are repeatedly playing out, and that they likely continue to dissociate around.

My further concern is that focusing on mapping and/or integration runs the risk of driving some alters into resisting a genuine therapeutic alliance. This can undermine another goal of helping the alters function as a team with cooperation and finely tuned coordination. It is incredibly beneficial to shift from alters as a group of mutually antagonistic individual parts to parts working harmoniously together. So long as they are not in conflict, they can have a peaceful co-existence. Otherwise, time loss, competing for time out, or even self-harm, will continue to cause tremendous stress.

Here are some simple therapeutic guidelines:

\begin{enumerate}
\def\labelenumi{\arabic{enumi}.}
\item
  Symptoms can usually be traced to alters getting triggered by repeated intrusion of past trauma into the present. These are flash-backs which turn the patient's life upside down again and again -- just like the original repeated early childhood traumas. So, the first goal is to stabilize the situation, to do a kind of trouble-shooting based on what alters are presenting to the therapist. It is PTSD treatment for the early childhood trauma. Essentially, it is figuring out what to do therapeutically on a kitchen sink everyday level.
\item
  Once activated, alters assert their right to be, to exist, to communicate. They can take over and cause havoc in the ordinary life of the DID individual. For example, chunks of time loss can occur which are very disconcerting and often very frightening for the host. At the same time, the alters who take over during those periods of time-loss for the host hold critical keys to healing. The immediate goal in treatment is directed towards quickly negotiating some kind of cooperation among the alters. It is focusing on turning the chaotic conflicted group into a disciplined team-like group with the common goal of healing. That is the ideal. While it is far more easily said than done, that is the target.
\item
  Engage whatever alters present and work with them. Remember that all of the alters are around when you speak with one, and make sure you formally invite them to participate by listening, by watching and by speaking when they so wish. Many alters can heal as they touch in or simply follow a more principal alter's therapeutic journey. They do not always need to be called out or to be otherwise addressed directly. In fact, many just feel safer watching and listening. A corollary to this is that being mapped can be frightening to them. It might be seen as telling them they need to stop hiding when they are still not feeling safe enough to be identified. And frightening an alter can make them potentially uncertain about the therapist's motives. That uncertainty can be a recipe for therapeutic disaster.
\end{enumerate}

In my experience, the most important therapeutic tool is deep respectful listening. With that as the ground, inviting all alters to listen in, mapped or not, addressing their concerns and understanding them in their context becomes possible. Other tools I used were stillness on the part of the therapist, working with the practice of one safe breath at a time to connect them to the safety of each present moment, self-soothing techniques, grounding techniques and the 5\% rule. Medication, if used as an adjunct to psychotherapy rather than the principal therapeutic intervention, can have clear benefits to support the patient.

\hypertarget{considering-the-use-of-drugs-in-did-treatment-part-1}{%
\section{Considering the Use of Drugs in DID Treatment -- Part 1}\label{considering-the-use-of-drugs-in-did-treatment-part-1}}

\emph{Posted on November 7, 2018}

This is the first of a series of posts discussing pharmaceuticals and DID treatment. The purpose is to encourage those with DID to avoid psychiatrists that have already made any kind of diagnosis before they have established any safe rapport with you. Hopefully, it will also provide some clarification in the somewhat muddy field of psycho-active pharmacology and its place in treatment of mental health issues.

I am not against the use of all psychiatric medications. I am very grateful for what modern pharmaceutical science has achieved in relieving suffering, including medication for mental health issues. But I do not believe we will ever solve all mental health problems with pills alone.

My general advice to dissociative individuals, is not to blindly go along with pills alone. Medication alone, without actual psychotherapy, won't address underlying trauma. Pills may temporarily put out the surface fire so to speak, the symptoms, but they don't put out the embers burning underneath, which is the unprocessed trauma. Without a doubt, the trauma and the symptoms will reappear so long as the trauma itself is not treated.

If your mental health professional recommends taking an antidepressant, set agreed-upon boundaries for tracking its impact. For example, you might agree that it is being used tentatively. That way you can get a sense of what happens as a result and whether or not it is beneficial. It may indeed help you and often does function as a temporary fix. If it is helpful, use the stability that results so as to take the necessary steps in psycho-therapy to process the trauma. But, don't accept it as the exclusive approach for your psychiatric problem.

Instead, you have the right, and the responsibility to yourself -- including all parts of any DID system, to assess your therapist. A therapist should be interested in you as a person. A chemical cannot express an interest in you as a person.

Assessing your therapist is the first step toward establishing a genuine therapeutic alliance with that therapist. It is that therapeutic alliance that enables your therapist to help and guide you in processing trauma. You can make a therapeutic alliance with a person, you cannot make a therapeutic alliance with a drug.

\hypertarget{considering-the-use-of-drugs-in-did-treatment-part-2-depression-and-antidepressants}{%
\section{Considering the Use of Drugs in DID Treatment: Part 2 -- Depression and Antidepressants}\label{considering-the-use-of-drugs-in-did-treatment-part-2-depression-and-antidepressants}}

\emph{Posted on November 14, 2018}

We often find these 2 words, depression and antidepressants, spoken in the same breath. Why is this a problem? Because always coupling them together erroneously implies that depression is a disease and antidepressants are the cure. It is dangerous to see them together so often because they begin to appear to be naturally identified as a pathology and its treatment.

Depression is not necessarily a pathology. It can also refer to a very ordinary state of mind triggered by some kind of loss, whether it be material or emotional. Depression is often part of the ordinary ups and downs of life.

Depression is a term used when a patient expresses particular feelings. It can be used for a psychiatrist's observation in referring to the inner world of the patient. It is also the term used for a mental illness, a pathological disorder which is a clinical state.

It is easy for us to become sloppy with words. We use the same words in different circumstances without necessarily clarifying the different nuances we mean to communicate. We lump words together in ways that blur their meanings. These result in false logics that can do a great deal of harm. It has affected many people and created much suffering for patients.

In a casual conversation recently, a well-established psychiatrist shared the sad news of a mutual friend who lost a family member through suicide. He then commented that 1) young people today do indeed tragically commit suicide, and 2) they are notoriously resistant to taking antidepressants.

I was shocked. Why did he immediately associate the suicide to depression that would respond to drug treatment in such a linear way? The thought arose in my mind that the troubled young man perhaps might have been helped if he had someone to speak with at that difficult moment in his life, someone that would listen to him with understanding and empathy.

Medication would certainly not be the first thought that comes to my mind in such circumstances. To know a person is feeling badly and to then help him requires more than prescribing a pill. It is an inappropriate leap in logic to so completely associate prevention of suicide with a pill. Surely some mental health professionals are missing the point, the basic importance of listening deeply and always being kind.

\hypertarget{considering-the-use-of-drugs-in-did-treatment-part-3-the-widespread-prescribing-of-antidepressants}{%
\section{Considering the Use of Drugs in DID Treatment: Part 3 -- The Widespread Prescribing of Antidepressants}\label{considering-the-use-of-drugs-in-did-treatment-part-3-the-widespread-prescribing-of-antidepressants}}

\emph{Posted on November 21, 2018}

According to Paul W. Andrews, an assistant professor in the Department of Psychology, Neuroscience \& Behavior at McMaster University in Ontario, Canada:

\begin{quote}
Antidepressant medication is the most commonly prescribed treatment for people with depression. They are also commonly prescribed for other conditions, including bipolar depression, post-traumatic stress disorder, obsessive-compulsive disorder, chronic pain syndromes, substance abuse and anxiety and eating disorders. According to a 2011 report released by the US Centers for Disease Control and Prevention, \emph{about one out of every ten people (11\%) over the age of 12 in the US is on antidepressant medications} (italics added). Between 2005 and 2008, antidepressants were the third most common type of prescription drug taken by people of all ages. They were the most frequently used medication by people between the ages of 18 and 44. In other words, millions of people are prescribed antidepressants and are affected by them each year.
\end{quote}

This information is in keeping with most of the statistics I have read, which show that the percentage of adults using antidepressants in developed countries is extraordinary. It is alarmingly high to most everyone -- except for the companies that manufacture and profit from them. In short, this is a major alert. We need to re-think the rampant use of these drugs.

The narrative used to support this widespread use is simple: Suicide is the result of depression and depression is a disorder amenable to drug treatment. It is a simple but quite muddy thinking that is pushed out to both the medical and general population. It comes from misinformation coupled with aggressive advertising by drug companies to the public as well as professionals. They advertise directly to the public, and promote it through continuing medical education events for professionals. All of this is paid for and promoted by the very companies profiting from the sales. They tell the public to rely on the doctors, and they tell the doctors to rely on the pharmaceutical company sponsored literature along with other information that is not subject to outside or peer review.

Here are a few points to consider:

\begin{enumerate}
\def\labelenumi{\arabic{enumi}.}
\tightlist
\item
  Suicide attempts do not necessarily result solely from depression.
\end{enumerate}

For some time, it has been noted as a potential side effect that some antidepressants actually lead consumers to suicidal behavior. The term ``suicidality'' has been brought into somewhat common use. The U.S. Food and Drug Administration (FDA) proposed that makers of all antidepressant medications update the existing black box warning on their products' labeling to include warnings about increased risks of suicidality, suicidal thinking and behavior, in young adults ages 18 to 24 during the initial treatment. Initial treatment generally refers to the first one to two months of medication usage. The first question I have with this warning is whether the label is primarily for prospective litigation defense rather than for any other patient centered reason.

Suicide is a complex behavior that cannot be reduced to a pseudo-scientific term like suicidality. Not all depression leads to suicidal ideation. I believe suicidal behavior is a form of anger turned inwards. I have numerous examples of patients who harbored internalized rage. By turning and maintaining that intense anger inwards, the need to express that rage was translated into suicidal behavior.

Once, a suicidal patient was referred to me who was taking an overdose of drugs every other day. She would end up in the Hospital Emergency ward for weeks on end. Finally, some of the nurses in the ER suggested sending her to me because what her then-therapists were trying was obviously not working.

I saw her a few times. She told me that she was extremely angry at one of my colleagues, a psychiatrist who had a responsible position in the hospital. She was boiling in anger but had no way to complain about his conduct. Just listening to her and acknowledging her grievances was ventilated that smoldering anger.

The ritual of repeated hospital visits was her way of expressing her anger. The simple act of listening and acknowledging her with empathy abruptly ended her repeated ``suicidal overdoses.'' Someone with a psychiatry degree, me, bothered to listen to her. Listening and acknowledging her was all that was needed to change her behavior. She stopped coming to see me after a few sessions, and abruptly ended her pattern of overdoses and visits to the ER.

I was later asked what I had done to stop her suicidal behavior. I hadn't done much other than recognizing that her suicidal behavior was simply her way of protest. It was how she was trying to tell the world how angry she felt being trapped in that authority/helpless victim struggle with a perceived authoritarian psychiatrist with degrees and status. She was a single woman in her 50s feeling powerless. I was confident about the importance of listening and acknowledging her because 2 other patients had already complained to me about that psychiatrist's abrasive manner in their own encounters with him.

This was an example of a patient who perceived that their therapist was not interested in listening to her innermost concerns. Immediately, such a patient loses his/her faith in the therapeutic relationship. If the doctor's primary goal is choosing a pill as the mainstay of treatment, that is a direct message to the patient. That direct message is not one of empathy or compassion. The patient may and will likely feel rejected, ignored, helpless, and hopeless. Anger \emph{should} be an expected response. And anger \emph{will} often be redirected inward or outward.

If the patient loses hope, suicide is often seen both as a way out and a statement of protest. It is a red herring to coin a new word ``suicidality'', as if that is a reasonable scientific risk of chemical side effects. It is as deceptive as implying that depressed patients will most likely have their depression alleviated with magic chemicals labeled ``antidepressant'' and that there is a mix of chemicals/dosages that will make the problem disappear.

\begin{enumerate}
\def\labelenumi{\arabic{enumi}.}
\setcounter{enumi}{1}
\tightlist
\item
  Antidepressant use is not an accurate reflection of the prevalence of depression.
\end{enumerate}

The popularity of antidepressants in a given country is the result of a complicated mix of depression rates, stigma, wealth, health coverage, the degree of aggressive sales tactics of the pharmaceutical industry, the availability of treatment. It is also tied to the biological bias of the therapists toward chemical intervention rather than psychotherapy -- most of whom are trained and marketed to by the pharmaceutical sales representatives.

Again, I want to be completely clear that I am not against the appropriate use of antidepressants as an adjunct to psychotherapy. I have done exactly that with some of my patients. However, the mind is not simply a box of neural circuitry where wires can cross and be uncrossed, where chemical switches can simply be toggled on or off. We must not forget our humanity. Do not ignore its powerful effect in helping to transcend despair. We must not forget the power of empathy, of compassion, and of hope in healing and recovery.

\hypertarget{considering-the-use-of-drugs-in-did-treatment-part-4-understanding-the-clinical-presentation}{%
\section{Considering the Use of Drugs in DID Treatment: Part 4 -- Understanding the Clinical Presentation}\label{considering-the-use-of-drugs-in-did-treatment-part-4-understanding-the-clinical-presentation}}

\emph{Posted on November 30, 2018}

\begin{enumerate}
\def\labelenumi{\arabic{enumi}.}
\setcounter{enumi}{2}
\tightlist
\item
  When a patient has made repeated suicide attempts, that patient is often labeled with the diagnosis of depression as part of a Bipolar or DID diagnosis. As we have been discussing, the correct diagnosis is critical as there are medication protocols for treating bipolar whereas there are no medication protocols for treating DID.
\end{enumerate}

Bipolar Affective Disorder and DID are diagnoses based solely on their clinical presentation. Unlike malaria, they cannot be confirmed in a laboratory. In the past, before microscopes, malaria was also diagnosed by its clinical presentation, which is a specific fever pattern. But now, it is diagnosed using a microscope that enables the parasite to be seen in a blood smear.

There are no laboratory tests for these psychiatric disorders. The clinical presentation alone is used to make a diagnosis of DID or Bipolar Affective Disorder. And the evaluation of clinical presentations is subjective. It is based on an interpretation of what is behind the behavior, of what is causing it. There is a risk of the clinician's bias in that interpretation. If bias drives the decision, that can compound the risk of mistaking one diagnosis for another, perhaps correct one.

This is more common than is generally acknowledged because the same or similar clinical presentation can be seen as quite different illnesses. For example, one psychiatrist may identify something as a mood swing and decide this is a bipolar patient. Another psychiatrist might identify it instead as a dissociative event where a different alter is presenting.

Those who have difficulty in accepting the phenomenon of dissociation often choose the diagnosis of a Bipolar disorder to fit their patients into a pigeon hole with which they, the psychiatrists, are comfortable. These disorders often include depression and instability in mood states. With the identification of a behavior as a symptom, the correct diagnosis is critical because treatment is quite different for each of these disorders. A critical distinction in the diagnoses is that identifying the behavior as a symptom of bipolar legitimizes the use of drugs. This is because there are drugs approved for use in treating bipolar disorders while there is no drug approved for use in treating DID.

We do know that diagnoses having an approved drug for treatment mean short interviews with patients that are less emotionally taxing for the therapist. This means that there is a greatly diminished risk of vicarious trauma for the psychiatrist to go along with the convenience of a prescription based treatment rather than psychotherapy.

Despite the many papers published on brain amine metabolism and depression, we do not know exactly how these are truly related. Nevertheless, using drugs as the treatment means that instead of putting out the energy of empathy, and deeply listening to the patient, there is just the checklist of questions to ask. The questions are all versions of ``are you feeling better?''

The answers are then coupled with trying different kinds of antidepressants, dosages and combinations. A diagnosis that has an approved drug treatment guides the psychiatrist to focus on the relatively simple task of choosing the right pill rather than on psychological and social issues. But if that diagnosis is incorrect, the resulting treatment plan will not address the problem. It will cause more suffering to the patient and often further mask the correct diagnosis.

So, the correct diagnosis is critical.

Evaluating clinical presentations means that the symptoms and signs are documented by selecting and interpreting those presentations. The problem, to give one example, is that a psychiatrist who is biased towards a bipolar diagnosis will see a behavior as hypomania. The result is that he will give a patient the latest mood stabilizer as the first line treatment. If that psychiatrist ignores indications of early childhood trauma or even remaining open to that possibility, he will simply not identify the behavior for what it most likely is -- dissociation related to flashbacks of that early childhood trauma.

For that psychiatrist, a diagnosis of bipolar affective disorder and the use of a mood stabilizer will appear to be a sound clinical practice. The doctor and the drug manufacturer are protected legally from claims of negligence. It is the essential litigation insurance. It is difficult years later to prove that the doctor was wrong.

Identifying severe agitation or panic in a patient with a history of abuse as hypomania rather than recognizing it as an episode of flashback agitation is a mistake with real and difficult consequences. Flashbacks are not a feature of hypomanic behavior. But, I have seen it described as hypomanic behavior in patient files because of a clinician's bias favoring a diagnosis of bipolar.

\hypertarget{considering-the-use-of-drugs-in-did-treatment-part-5-diagnostic-bias-in-files}{%
\section{Considering the Use of Drugs in DID Treatment: Part 5 -- Diagnostic Bias in Files}\label{considering-the-use-of-drugs-in-did-treatment-part-5-diagnostic-bias-in-files}}

\emph{Posted on December 7, 2018}

Few people outside the psychiatric and pharmaceutical communities know how common the practice of stretching and bending the meaning of words is in medical files. That practice is influenced quite strongly by the bias of the clinician. I have personally had client files sent to me that clearly were based on a liberal and intentional misuse of words. This misuse served the purpose of identifying an otherwise understandable behavior into a symptom. I am confident in saying this because many of the files referred to me included dissociative behaviors and events. In fact, the files actually used the term dissociation but failed to include any primary or even secondary dissociative diagnosis. Further, those files usually indicated pharmaceutical treatment failures and no application of psychotherapy.

For example, patients were referred to me that were experiencing agitation related to a flashback of abuse. In the files, agitation was interpreted as ``a variant of hypomanic behavior.'' Such misuse of language completely shocked me. Those patients had often lost years on a wild goose chase, with therapists trying to find the right pharmaceutical agent for ``a variant of hypomanic behavior.'' The correct approach should all along have been trauma therapy as it was their trauma that was being displayed in the symptoms.

It is common to see patients that are kept on antidepressant for years yet remain depressed. Although they are labeled as suffering from ``treatment resistant depression'', it is more appropriate that they be labeled as suffering from Antidepressant-resistant depression!

If a patient on antidepressant(s) has not improved as expected, the correct procedure is to review the diagnosis, not just to persist in trying different dosages or a newer drug. There is no logical reason or peer reviewed study that would indicate that the depression symptom is part of a disorder that justifies the exclusive use of medication. In reality, that is the common practice -- to increase the dosage or change of antidepressants. Instead, try listening to the patient. Or, at least, continue with the medication and try listening to the patient.

In Volume 1 of Engaging Multiple Personalities, there are several examples of patients I had referred to me that were labeled as having treatment resistant depression who made progress in their healing journey through psychotherapy. With psychotherapy, those patients were treated. Their traumas were acknowledged and often successfully processed. During the psychotherapy, they were weaned away from antidepressants successfully and fairly quickly. I only remember a very few DID patients who required antidepressants as adjunct to being treated with psychotherapy.

\hypertarget{considering-the-use-of-drugs-in-did-treatment-part-6-determining-treatment-for-depression-in-did}{%
\section{Considering the Use of Drugs in DID Treatment: Part 6 Determining Treatment for Depression in DID}\label{considering-the-use-of-drugs-in-did-treatment-part-6-determining-treatment-for-depression-in-did}}

\emph{Posted on December 11, 2018}

\begin{enumerate}
\def\labelenumi{\arabic{enumi}.}
\setcounter{enumi}{3}
\tightlist
\item
  As I have said before, I am not rigidly against the use of antidepressants per se.
\end{enumerate}

Some of my depressed patients did indeed respond positively to treatments other than psychotherapy, often in ways that might be seen as miraculous. My disappointment and concern is that there remains no clear protocol that confirms what kind of depression will respond to which treatments. The result was that I used my own criteria when considering options for my patients, based primarily on my clinical experience.

I used psycho-pharmaceuticals in the past. I can attest to the fact that they do help some very severely depressed patients just as I can also attest to the fact that they do not help others. To this day, for me at least, there are no studies that satisfactorily define what kinds of depression respond to which chemical interventions.

It can be an assault on the patient to give them a small manufactured pill. How is that possible? Keeping a patient on antidepressants for years while ignoring psychological factors such as early childhood trauma, or recurrent ongoing trauma as the cause of the depression, is a chemical assault. Such an approach has the quality of trying to beat down the depression rather than cure its cause. Until we have an actual proven answer in identifying which depression would be responsive to which drug, we need to be extremely careful in using these approaches.

\begin{enumerate}
\def\labelenumi{\arabic{enumi}.}
\setcounter{enumi}{4}
\tightlist
\item
  The term ``Chemical Imbalance'' has no real meaning.
\end{enumerate}

It is a false assumption that antidepressants are generally both safe and effective. The truth is that all pharmaceuticals are substances foreign to our bodies, even when they are based on natural chemicals produced by plants for example. Pharmaceuticals are highly potent chemicals. They are specially designed to quickly alter our metabolism and interfere with it. In fact, psycho-active medications are designed to rapidly impact one's existing brain chemistry. They are far more potent than the plants they may be derived from.

The term ``Chemical Imbalance'' is somewhat a sales device. The identification of the numerous serotonin-receptors in the brain has helped some, but so far has not cured the pain and suffering of all or even most depressed patients. The truth is that psychiatry in the 21st century remains an inexact science.

After almost a century of sophisticated biochemistry research, we are still generally operating in a fog as to defining exactly what is the chemical imbalance in a brain that expresses pathological depression. I do not dispute that psychiatric medications have contributed to the treatment of certain psychoses. They have, in fact, led to a reduction of the number of institutionalized psychotic patients in developed countries. However, we must accept that there are some unavoidable limitations in the purely pharmaceutical approach to depression. It is a false hope that we can trade pills for genuine psychotherapy in the name of saving time and man-power.

\hypertarget{considering-the-use-of-drugs-in-did-treatment-part-7-go-slowly}{%
\section{Considering the Use of Drugs in DID Treatment: Part 7 -- Go Slowly}\label{considering-the-use-of-drugs-in-did-treatment-part-7-go-slowly}}

\emph{Posted on December 14, 2018}

\begin{enumerate}
\def\labelenumi{\arabic{enumi}.}
\setcounter{enumi}{5}
\tightlist
\item
  The best we can do is to humbly accept the limitations we have in striving for a more precise description of depression that may respond to medication.
\end{enumerate}

DSM 5 gives a clinical picture that defines a depressive condition that would be appropriate to treat with medication. In one example, this includes a somewhat arbitrary time limit: If grief in bereavement is prolonged more than a certain number of days, then we deem it a pathological state. And, with that diagnosis, comes the implied appropriateness of trying some pharmaceutical intervention.

If a genuine therapeutic alliance has been established with the patient, I would have a clearer sense of the likelihood of early childhood trauma, or an assessment of potential ongoing trauma in the patient's current life. Being able to identify trauma leads to one treatment path. Absence of trauma would lead to a different treatment path.

My approach is to look at a person's depression. If it is there most of the time, when he wakes up, when he does not get cheered up seeing his loved ones, when he is socially withdrawn, when he cannot shake it off, that would satisfy my criterion of a form of depression where I might try antidepressant. But, that would only be as an adjunct to psychotherapy.

Depression that responds to drugs usually has a different quality than depression connected to trauma. It is more like someone who has lost interest in things that used to generate a positive experience, a positive response. A common description is of a patient that no longer enjoys his favorite foods.

Psychiatric textbooks describe true depressive symptoms in different ways. The term ``True depressive symptoms'' refers to depression as a syndrome, a disorder; in other words a mental illness that prevents one from living one's life in a way that accommodates the ups and downs of ordinary existence.

\begin{enumerate}
\def\labelenumi{\arabic{enumi}.}
\setcounter{enumi}{6}
\tightlist
\item
  There is a dangerous pattern in psychiatry to quickly conclude that a depressed patient should be on medication.
\end{enumerate}

This kind of presumption is illogical, dangerous, and based on an inflated sense of one's insight. But, it is inflated by the promotional materials of the pharmaceutical industry and the money that flows from it. I have heard this kind of nonsense from the press as well as from many of my peers. What is missing? It is empathy that is missing. It is the warmth of genuine compassion that is missing. Both of those should be tested before anyone is given a license to be a therapist -- whether it is a license as a psychiatrist, a psychologist, a clinical social worker or perhaps even just an ordinary human being that deals with other human beings in trouble.

For the sake of billions of dollars of sales, pharmaceutical companies invest heavily in propaganda and brain-washing to promote the use of drugs as the exclusive means in solving the mental health problems.

Be aware of the erroneous assumption that depression is a disease curable by antidepressants. This is, at best, a half truth. We need to be alert to identify patients with depression that is amenable to psychotherapeutic intervention.

\hypertarget{considering-the-use-of-drugs-in-did-treatment-part-8-a-sales-channel-is-not-therapy}{%
\section{Considering the Use of Drugs in DID Treatment: Part 8 -- A Sales Channel Is Not Therapy}\label{considering-the-use-of-drugs-in-did-treatment-part-8-a-sales-channel-is-not-therapy}}

\emph{Posted on December 16, 2018}

\begin{enumerate}
\def\labelenumi{\arabic{enumi}.}
\setcounter{enumi}{7}
\tightlist
\item
  Drug companies use a particular sales technique known as ``off-label'' marketing to expand the sales potential market of their psychoactive medications.
\end{enumerate}

The technique of ``off-label'' marketing is selling the medication for a purpose that has \emph{not} been approved by, in the US, the Food and Drug Administration. With this kind of marketing approach, we are being led to participate in a cold, money orientated mental health systemc who need help dealing with trauma or other mental health issues.

For example, a drug may be approved for treating psychosis, but not dementia. However, that drug may be marketed for ``off-label'' use for individuals with dementia. This is not at all uncommon. This kind of marketing is often done at seminars that are sponsored by the pharmaceutical companies seeking to boost sales. It is based on anecdotal information they promote rather than peer reviewed studies. You can look at past drug litigation, such as around the use of Risperdal, to see the dangers in this technique. This kind of marketing made Risperdal a multi-billion dollar drug despite harming many children.

This same danger was recently highlighted in a study of Haldol, a drug that has been marketed for decades as an anti-anxiety drug but established an enormous off-label use. That use was for ``treating'' dementia related anxiety issues. According to the study, there were zero peer reviewed research papers indicating a positive impact of the drug for dementia patients. Here is a warning from 2007 that is instructive, given that the medication had been in used for dementia patients for decades at that point:

\begin{quote}
``Haloperidol (Haldol, Johnson \& Johnson) is approved for intramuscular use, off-label intravenous use of the drug is relatively common for treating severe agitation in intensive care units. However, due to a number of case reports of QT prolongation, torsades de pointes, and sudden death thought to be associated with this practice, the FDA has issued an alert to healthcare professionals. The prescribing information for Haldol, Haldol Decanoate, and Haldol Lactate has been revised to reflect the concern and potential risk when the drug is administered intravenously or at higher doses than recommended.''
\end{quote}

Note that this warning doesn't say you should \emph{not} continue to use it for dealing with agitation in dementia patients. It is merely an ``alert'' that was likely issued as a prospective litigation defense.

In short, beware of off-label use of psycho-active medications. Perhaps the anecdotal information promoted by the pharmaceutical industry is accurate, but perhaps it is not.

\begin{enumerate}
\def\labelenumi{\arabic{enumi}.}
\setcounter{enumi}{8}
\tightlist
\item
  Emotional difficulties have to be approached through first understanding the emotions involved.
\end{enumerate}

All aspects of the individual have to be considered in therapy; the biological, psychological, social and spiritual aspects. There is no substitution for this by prescription. Prescriptions are not time-saving if that is all you offer the patient. Why? It is because missing the underlying factors that generate the symptoms will only cause delay and suffering -- often for years -- as a result of the wrong treatment. Should the wrong treatment include psychoactive medication, there will likely be an even more difficult path of undoing the impact of that medication before being able to address the actual issues.

Years of pharmaceutical experiments will ensue for the patient. The hunt for another therapist will eventually follow, often many years after the original misdiagnosis and corresponding error in treatment. It breaks my heart that this is so common. It is why I continued to practice psychiatry into my 70s and why, in retirement, I wrote the Engaging Multiple Personalities Series.

The case histories in Engaging Multiple Personalities Volume 1 were all of people that came to see me after being treated to no avail by other therapists and psychiatrists for years. For those I was able to help, it was not that I was a particularly brilliant therapist. It was because I actually listened to them. Without exception, their prior therapists either did not believe in dissociation or were too callous to pay attention to the early childhood trauma these individuals experienced.

An effective therapist speaks to the heart, not to the brain. We must never forget the humanity of our patients, or our own.

\hypertarget{treating-did-a-brief-summary-of-key-points-part-1}{%
\section{Treating DID -- A Brief Summary of Key Points: Part 1}\label{treating-did-a-brief-summary-of-key-points-part-1}}

\emph{Posted on December 24, 2018}

\hypertarget{treating-did}{%
\subsection*{Treating DID}\label{treating-did}}
\addcontentsline{toc}{subsection}{Treating DID}

My three small volumes of ``Engaging Multiple Personalities'' were written with the intention of introducing to the public to Dissociative Identity Disorder, the often forgotten and neglected mental disorder arising from early childhood trauma. Since early childhood trauma is often ignored by professionals and the topic trauma/dissociation often misunderstood, there is unfortunately an enormous pool of individuals at large suffering from these conditions. Often, they remain misdiagnosed by therapists and bounced around within the mental health systems.

Many people erroneously regard this condition as rare. Others believe it to be a ``controversial'' diagnosis, which is actually saying that they don't believe it exists. Such misunderstandings continue to cause untold suffering in many individuals with DID, keep many therapists from considering such a diagnosis or caring for an individual who has been so diagnosed. In short, competent DID therapists are difficult to find.

Looking back on my career, I encountered these patients early in my practice but failed to recognize their plight. Even if I had recognized them at the time, I did not have the training or skill to help them -- despite my medicine degree and protracted training in psychiatry at some of the best centres in London, England. For the first decade of practising psychiatry, I remained ignorant as to how to recognize and help patients suffering from DID.

Eventually, I learned the hard way -- directly from my patients, from both my failures and successes. I learned from each one of them something of how to work with those suffering from DID. Eventually, I developed some skills in helping patients suffering from trauma and dissociation. I wished I had some guidance, a mentor, when I was struggling as a therapist to find ways to help the DID patients more than a decade after I was considered a DID specialist.

Although at this point in my life I cannot be a personal mentor to other psychiatrists/therapists, the Engaging Multiple Personalities series is an attempt to provide some guidance to those with DID, their therapists and their potential therapists.

Treatment of DID begins with the recognition and understanding of the psychopathology of trauma and dissociation. Digging deeper, one must recognize that trauma and dissociation can indeed begin at a very early age, a horrifyingly early age. Trauma like that can culminate in fracturing the mind of a child, resulting in the condition now called Dissociative Identity Disorder, formerly termed Multiple Identity Disorder. It is difficult to learn how to treat DID through reading textbooks. It would be somewhat like reading the Oxford dictionary to learn the English language. It is not completely impossible, but for most people, it is not a particularly helpful approach to learning a new language. Therapists dealing with DID patients must learn these key points. Otherwise, the therapist will be unprepared to handle the appearance of an alter in a patient suffering from DID. That lack of preparation will lead to a cruel failure in therapy and damage any potential therapeutic alliance.

Here is a summary of the guidelines I recommend in the treatment of DID:

\textbf{1. We can use empathy to understand.}

DID is a condition with an extreme form of dissociation, with the mind fractured into parts that are referred to as ``alters,'' or ``alternative identities.'' The host personality is usually the patient that initially comes into the office. But, the host personality is part of a system of alters that each experience themselves as individuals separate from the host. They have a separate sense of self, and display a separate personality. Based on their experience, the alters insist that they are individuals inside the patient that either remain inside or sometimes emerge to take over the body of the patient. When they emerge, they function for a period of time -- ranging from a few minutes to several months in my patients' experience -- like any other individual you might meet out in the world.

How does empathy help a therapist understand DID? First, know that the dissociation is a survival mechanism. It arises instantaneously so that the child can escape in some way from the experience of an insurmountable trauma. Without the dissociation, going through the traumatic experience as a whole, the child would have been overwhelmed and destroyed. Simply put, the immature developing ego has found a way to circumvent the trauma by dissociating from it. This manifests as the experience ``this is not happening to me.''

In short, an alter goes through the trauma while the remaining parts of the system -- other alters and perhaps the host -- experience the trauma quite differently, something like, ``I am hiding here safe and floating up towards the ceiling.'' This is a verbatim statement made by one of my patients describing the experience of being severely beaten by her sadistic father when she was an infant.

While I don't have the first hand experience of someone with DID, based on the communications I have had with my DID patients, this is how I envisage the way an alter is formed. Therapists with a limited capacity of empathy might think this is a theatrical way of exaggerating the suffering of an abused child.

We must consider the truly horrific nature of a helpless infant encountering repeated trauma to generate real empathy. Truly imagine yourself as an infant being beaten, again and again and again. There is no way to escape. If you genuinely listen to a patient's experience of early sexual abuse, repeatedly with no way to escape, how quickly could you ``get over it''? To presume that you could ever get over it without tremendous help and your own herculean effort, is an egregious and cruel lie.

\textbf{2. The slogan to remember is ``Engage the alters.}

The alters are not the pathology, so do not think of ignoring them to hope they will disappear. They have the primary functions of protecting and stabilizing the system. One must always remember to treat each alter with respect and to appreciate their important roles within the system.

There are 2 ways such extreme dissociation generally cause dysfunction in later adulthood.

A. Each alter may have their own issues that require therapeutic intervention. Many of them can be identified as suffering from PTSD. Those with self-harm or potentially violent acting out behaviour should receive priority treatment. The approach is simply determined by the urgency of the problem presented by the alters. Attend to each problem as presented by each alter, according to severity.

While each alter may have issues that might need therapeutic intervention, this does NOT mean that therapy requires directly working with each alter. It is not the case that the therapist needs to identify each and every alter, and seek to address each and every issue they may have. What has happened with my patients is that treatment of even one alter eased the difficulties of other alters who were watching, so to speak, from the sidelines. In other words, providing therapy to the presenting alter had a positive cascading effect on other alters. To seek to identify the trauma each alter may have, in the absence of a presentation by that alter, would likely lead to retraumatization rather than benefit.

Alter generally have some PTSD flashbacks as traumatic memory rises to the surface. However, once a therapeutic alliance was established, I was always amazed that there was much cooperation among the alters as well as a sense of urgency to work hard in the healing process. It is as if the system truly appreciates it when, finally, it has found hope that healing is possible. The system of alters, both individually and as a whole, becomes ever more approachable and ready for change when they are listened to with respect by the therapist. It is often the first time in their life that any outsider genuinely listened to them.

B. Many alters are secondary elaborations arising from the primary splitting. It is critical to understand that identifying them as secondary elaborations is absolutely not to diminish them in any way. They arise to perform their protective functions. They nevertheless can cause friction to a system by exerting each of their own individuality, which individuality likely has its own trauma triggers as well as its own quality of hyper-vigilance.

Seen in a narrow perspective, an alter may appear to be extremely angry, paranoid, mistrustful or controlling and dictatorial. They jealously guard their individuality, which makes sense in the context of their emergence in the midst of specific traumatic events as they are hyper-vigilant about the potential for similar trauma that might come up.

Most alters have never learned compromise or genuine cooperation. If X wants to go dancing, and Y wants to study, there may be an ongoing clash and confusion that impacts the entire system. In the early phase of therapy, many alters share varying degrees of co-consciousness. Consider how often a conflict or clash will occur in one single body holding several sets of will and desire. It is no wonder that a single choice may take an incredibly long time, whether shopping for a dress or choosing where to eat.

Treatment can be likened to negotiating for some harmony and cooperation among a group of different aged people housed in a single dormitory, who may be complete or partial strangers to one another. The therapist has to be resourceful, for example suggesting that the alters elect a director for shopping who makes the final decision of items being bought which director alter is required to ensure that all alters get their way occasionally. Therapists will be amazed that the alters do listen and appreciate help in this way.

\hypertarget{treating-did-a-brief-summary-of-key-points-part-2}{%
\section{Treating DID -- A Brief Summary of Key Points: Part 2}\label{treating-did-a-brief-summary-of-key-points-part-2}}

\emph{Posted on December 28, 2018}

\textbf{3. Treat the trauma, not the drama.}

While the presentation of DID may appear to be melodramatic or overly complicated to the therapist, common sense dictates. There is no need to treat every alter as a full fledged individual who needs individual psychotherapy. Generally speaking, they don't. The key is to just address the alter specific presenting problems in any session. Alters are extremely responsive to, and appreciate such individual attention. And despite their initial hesitation, they are usually highly changeable.

In my experience, alters were willing to take turns to have their problems addressed according to their severity. They can all listen in and learn from each other's sessions. This allows the healing process to spread throughout the system a little bit at a time without the need for continuous individual treatment. Remember, a therapeutic alliance gives them hope for help in dealing with burdens they have been shouldering all alone for many years. Burdens that have never been acknowledged by anyone outside, and in fact were often terrorized into keeping those burdens tied up inside.

Eventually, alters develop empathy -- some sooner than others. With gentle encouragement by the therapist, they will often try to start helping each other within the system. I was often astonished with the efficiency of the inner guide(s) or inner therapist(s), that develop to hasten the therapeutic process. I have attempted to encourage one alter helping another, or to even just be sympathetic to others in pain. Therapists have an important role in teaching alters empathy towards their fellow alters.

\textbf{4. Promote co-consciousness and communication.}

When talking to individual alters, the therapist must understand that it is like speaking in a classroom to one student but in the presence of the entire class. Such awareness will optimize therapeutic effect, good will, and planting multiple seeds of hopefulness into the system.

\textbf{5. Be prepared knowing that there will be both trusting and mistrustful alters remaining quietly in the background watching the on-going therapy.}

In extreme cases, hostile dictatorial alters may try to sabotage therapy. They take this position genuinely in the name of protecting the system from being hurt again. Given their history of trauma involving those with power over the patient, this is both reasonable and important to acknowledge.

Occasionally, such an alter may drop a note to the therapist warning them that she/he is watching, protecting the others from being fooled. Don't be insulted or be defensive and try to convince that alter that there is no need for their vigilance. Therapists should know that this is completely in keeping with that alter's protective function. I would always thank those alters and encourage them to continue watching me. This is a correct and polite response. While they didn't need the encouragement to keep watching, such responses generate more trust and good will.

\textbf{6. Empowerment is essential for successful therapy.}

Following such a ``client-centered'' approach gives the patient a sense of autonomy and empowerment. There is no better way to help a DID patient than empowering the patient during therapy. Always keep that in mind. Essentially, the trauma the patient has been dealing with all his/her life has been one of dis-empowerment, of being the victim. Abuse is ultimately a process of domination, of one person overpowering the other. If, in therapy, the therapist finds ways to enable the patient to reclaim their power as an individual, there is tremendous benefit in healing. And critically, that empowerment will begin to allow the patient to undermine the strength of flashbacks that otherwise re-traumatize the patient.

\hypertarget{treating-did-a-brief-summary-of-key-points-part-3}{%
\section{Treating DID -- A Brief Summary of Key Points: Part 3}\label{treating-did-a-brief-summary-of-key-points-part-3}}

\emph{Posted on January 2, 2019}

\textbf{7. Metaphorical hand-holding helps the frightened child who keeps reliving the trauma, helping them to process it in small digestible doses.}

Treating PTSD involves \emph{metaphorically} holding the hand of an injured and terrified child. It is comforting them so as to enable them to process the impact of the trauma in a way that protects them from being overwhelmed or re-traumatized. It is to enable them to process the trauma in small doses that are digestible and not overwhelming to the individual.

The therapist must resist the urge to learn the details of the abuse unless and until the patient wants to reveal details. And then, no follow-up interrogation of the patient. Avoid asking questions when they are based primarily on the curiosity of the therapist. All historical events of trauma must be seen as private to the patient. We only find out as much or as little as is required to get the patient over the distress. Remember we are not police detectives writing up police reports. The details of the trauma are of limited therapeutic relevance except to the extent that an alter needs to express it. The need to express, and to protect from re-traumatization, is of therapeutic relevance -- not the details that are expressed.

\textbf{8. ``The body keeps the score'' so help the patient connect with their body.}

The memory of the trauma is kept in the body. Therefore, a physical approach rather than an intellectual approach is at times more relevant in therapy. Teach ``grounding'' techniques. Spend time to teach how awareness of the breath can impart calmness as can physical exercise and movement. Patients can use that awareness to ground and so neutralize the panicky feelings.

Flash-backs are best understood as a combined physical and psychological event rather than simply a psychological event. Some alters have severe PTSD features in the form of flashbacks. In a flashback, the alter is essentially reacting bodily to the memory of the past trauma. In other words, the past trauma is intruding into the patient's present. He/she is in fact re-living a segment of the original trauma. The body is reacting/behaving as if it is actually facing that same trauma. Imagine if you had once been attacked by a man-eating tiger. The next time you see that kind of carnivorous animal, your body would no doubt flood itself with adrenaline. You might run as fast as possible in the opposite direction when you hear the roar -- even if this time you see that the animal roaring is caged in the zoo.

The individual is frightened and confused during a flashback because they are experiencing a massively hyperactive sympathetic branch of the autonomic nervous system that is not in accord with their actual perceptions. The affected individual is not in control of his/her hyper-reactive physical state. Even though their sense perceptions are giving them the same information we interpret as no big deal, their nervous system is screaming danger. In other words, PTSD is basically a disorder where an individual experiences flashbacks of trauma that take away control of the body. The body goes into panic mode when encountering a trigger, like encountering a sudden storm when you are traveling in a calm sea. For those not triggered, it seems like the individual is completely panicking at the drop of a hat.

Treatment is essentially teaching the individual to take back the control of his/her own body. When flashbacks happen in therapy, if the therapist remains calm, there is a powerful transmission of that calmness to the patient. Simply teaching the patient that attending to one's breath in the present moment can be an effective way of giving them the skills to handle the flash back. Self-induced calmness means empowerment. It means that one has found a way to overcome this distress though one's own effort. A self-generated sense of calmness is a skill that can be regained by the patient, the result of which is vastly superior to a tranquilized sensation induced by a pill.

Drug induced calmness, even as it works, maintains the patients in a dis-empowered helpless role. He/she is being trained to rely on the availability of the medication when the next panic attack or next symptom appears. This avoids addressing the real issue, which is the past trauma taking over the present experience. In other words, with medication, one remains in a helpless posture. Further, it is common to find the body needing a higher dose of medication, the next time panic or agitation arises. Exclusively administering drugs to treat PTSD symptoms is doomed to failure and runs the substantial risk of chemical dependency.

Treating DID is teaching an individual how to handle the result, the consequences, of having had tremendous overwhelming and repeated exposure to early trauma. The mind is fractured. What is left behind is a system of split and conflicting parts forced to live together in one body. Prior to appropriate DID therapy, each part likely has only varying degrees of awareness of the split. Each part has its own agenda.

How to bring about a fragmented selves to function in a cooperative way is the task of the therapist. How to deal with flashbacks is the key skill to teach through communication, cooperation, and compromise. In the wider world, we need to learn to live with our neighbours. Within their systems, those with DID need to learn to live with the divided parts to learn how to control impulses and delay gratification when necessary. Both the path and result of healing is that we have to do it ourselves, not through use of an external agent, like a pharmaceutical.

\textbf{9. EMDR or CBT (cognitive behavior therapy), are only tools to use in the treatment of symptoms in PTSD.}

If they are helpful to any particular patient, that is great. But, they are not exclusive tools for treatment. Therapists must know how to apply these tools, like surgeons know how to excise a malignant tumour. But, just as surgeons know that there are often other options for treatment than surgery, therapists must be familiar with other options as well. Tools can be used but their limitations must be recognized.

Using an antidepressant for someone with DID is like using a cough medicine in someone who has chest infection. There are cases where a patient may have a true brain disease that has a fair chance of responding to pharmaceutical intervention. But, so far there is no laboratory method to diagnose these cases, to separate them from depression that requires predominantly a psychological approach for its healing. We rely on subjectively identifying a group of symptoms to fit into a diagnostic label.

In PTSD, whether the result of early childhood, wartime or other trauma, the brain is set to a hyperactivity mode, like a thermostat that is set a few notches off the scale. So far, purely using a mechanistic approach, like chemical or physical methods, has failed miserably. Witness the poor track record of treating veterans with PTSD, returning from the Gulf war and from Afghanistan. The results have been very disappointing when pharmaceutical methods are used exclusively.

It is unlikely that there will be a magic pharmaceutical agent that can exclusively be used to heal the damage of early childhood trauma that results in DID. We must come to our senses to recognize that to fix the cause of a car accident, we cannot just focus on the mechanical parts of the car. We need to understand the whole car, driver, weather, and road conditions to actually understand what really happened. In that same way, we must look at the entire patient beyond a simple mechanistic view.

With empathy, compassion and a willingness to engage the alters, by both the therapist and the patient, healing is possible.

\hypertarget{religion-and-working-with-trauma-survivors-part-1}{%
\section{Religion and Working with Trauma Survivors -- Part 1}\label{religion-and-working-with-trauma-survivors-part-1}}

\emph{Posted on February 25, 2019}

I have been hesitant to write a post on religion because it is a highly charged topic in virtually every setting, but it comes up very often in DID treatment. While religious faith can have a great positive impact in therapy, it can just as easily have a great negative impact on a patient should it have been connected with the underlying trauma. As a psychiatrist, whether religion was brought up by my patients in a positive or negative light, I dealt with it based on that particular patient's preferences only. I avoided making generalizations of any kind because each patient is an individual, and therapy must be geared to that individual's experience.

I will restrict my comments to the impact of religion on the therapeutic approach one takes with patients. Appreciating its impact on each individual patient that brings it up is critical for establishing and maintaining the therapeutic alliance.

The therapist must not push back against a patient's view of religion, regardless of the therapist's own view. Otherwise, there is a serious risk of diverting therapy away from its primary obligation, which is helping the patient deal with trauma. Why is there such a risk? Remember that being told not to believe their own experience, their own perceptions, and the consequent feelings of being invalidated, are all common experiences of early childhood abuse survivors.

Should the therapist try to impose his/her own ideas about religion onto the patient, it can trigger distrust and retraumatization. It can become yet another replay of some terrible memory. To have any chance of a real therapeutic alliance, therapy cannot involve any demand by the therapist -- direct or indirect -- for the patient to have the same view of God or religion as the therapist.

In my work with DID patients that had specific views of religion, rule No.~1 was to respect the patient's perception or idea of God, including the idea that God does not exist. The therapist's own belief system does not apply here. I would never argue or disagree with whatever my patients' religious belief might be. The only time to question a patient's belief would have been if the belief was encouraging them to harm themselves or harm others.

For the DID patients I worked with, it was clear that harming themselves or others was tied to how they were dealing with the trauma and its aftermath, not to any religious view or lack thereof.

For patients that disparage and are frightened of religion, all that therapists who believe their own religious tradition need to consider in order to set aside their own belief system is the truth that throughout history people have performed sadistic horrors in the name of religion. They can remember that wars have been and continue to be fought in the name of religion. Critically important for those with early childhood trauma, abusers often hide behind the facade of religious piety. The fact is that people have hidden their commission of evil deeds behind many names and facades, religious and otherwise.

For patients that do have religious faith, therapists that disparage religion need to consider that faith, throughout history, has been a powerful source of strength that has sustained people as survivors. Faith can sustain people by nourishing their hope of survival and healing from their trauma.

It is important to maintain that open view so as to be able to consider both the negative and the positive experience of religion in patients. Why? Just as I have seen religion used to perpetuate early childhood abuse, I have also observed in some of my patients that faith can play an important role in helping heal those who have been severely traumatized. I have seen many patients whose therapists considered them ``too damaged'' to benefit from therapy. Nevertheless, they derived strength to fight successfully for recovery because of their religious faith. It was clear that their faith sustained them with hope, that most important element in the process of healing past trauma.

Confidence that it is possible to heal, that it is possible to be freed from the bonds of retraumatizing memories, is the key to healing. For some, abusers have twisted religious imagery and practice. These patients may find healing only in a life that is completely extricated from religion. For patients like that, a therapist might gingerly feel out whether it is safe for the patient to hear the view that one can have a spiritual view without any trappings of religion.

For others, even those whose abusers twisted religious imagery and practice as part of the abuse context, maintaining or even finding faith beyond those evil twists gave them the confidence needed for healing. Because that key of confidence is so important, it is inappropriate to judge another person's religious faith as right or wrong or superstitious. Instead, support them with the view that what gives them confidence in their healing journey is of benefit.

\hypertarget{religion-and-working-with-trauma-survivors-part-2}{%
\section{Religion and Working with Trauma Survivors -- Part 2}\label{religion-and-working-with-trauma-survivors-part-2}}

\emph{Posted on February 25, 2019}

For me, religion is a worldview that relates humanity to life's transcendental elements. My definition does not necessarily include or exclude an omnipotent deity. It includes all the religious traditions I have encountered in my life so far; including varities of Christianity, Judaism, Islam, Hinduism, and Buddhism as well as atheism.

For the therapist, when listening to a patient express their religious faith or lack thereof, the issue to consider is what enables this particular victim of horrendous repeated abuse to undertake the hard journey necessary for healing and restoration? Something is enabling them, or trying to enable them, or they would not be seeking help in therapy.

We know that resilience is the single most valuable attribute required in healing and recovery from past trauma. Accessing that source of resilience and protecting it from attacks is critical to successful therapy.

So, what is it that gives survivors the strength to persevere? For those who believe they have a personal connection to God, or an unnamed higher power, that connection can be used to their advantage in healing and recovery. Prayer can be extremely helpful and sustaining for those who believe that they can rely on that for their healing.

In such cases, irrespective of the therapist's personal belief, it is only appropriate that the patient be supported and encouraged to continue in their spiritual path so they can benefit from that faith. I have seen survivors in religious communities that are separate from where the abuse occurred. They have experienced a sense of sisterhood and brotherhood in those new communities that gives them powerful support in an otherwise lonely individual struggle.

For those taking a path separated from religion on their healing journey, their motivating force might be the need to bear witness as a survivor. This is why I often spoke to angry alters about how important they were to the survival of the system. I encouraged them to see that their rage could be turned into fuel for the journey of bearing witness as a survivor.

This idea of bearing witness is a traditional element in many religious traditions but is clearly something that exists beyond any religious structure. The founder of the logotherapy, Viktor Frankl, was a Viennese psychiatrist who survived the Auschwitz concentration camp in World War II. He was trapped in a terrifying place specially designed to crush the human spirit, subjecting prisoners to a completely dehumanized environment. However, he had and held on to his reason to survive. That reason was for him to survive so as to bear witness to the fact that a people or ``race'' had been assigned for elimination through modern assembly-line methods. He believed it was necessary to bear witness that common people would lose their minds and individual will to participate in that elimination, and that they would obey orders to carry it out. Based on his experience in the camps, he indeed bore witness and used his experience to develop logotherapy, as discussed in his book Man's Search for Meaning.

There are therapists who decry religion within and outside of therapy. Those therapists should consider the success of Alcoholic Anonymous, which focuses on connecting alcoholics to a ``higher power'' in order to heal from addiction. It would probably be more appropriate to use the word spirituality than religion to describe that program, which brings hope to those helplessly addicted to alcohol. Comparing AA to both conventional psychotherapy and drug therapy in helping people with alcohol dependency problems, AA's encouragement of working with a higher power has a well recognized success rate which is higher than either psychotherapy or medication. So therapists should not denigrate the power of spirituality of any kind in healing.

In the New Testament of the Christian Bible, St.~Paul brought up Faith, Hope and Charity (agape or love) together, likening them to a three legged stool. With three legs, it is stable even on uneven ground. Most Christians believe that being grounded in faith, hope, and charity, allows them to remain on solid footing even when the ground beneath them is bumpy. I point out that this view is consistent with the path of any successful trauma therapy. The healing journey is certainly traversed over uneven and bumpy ground. Having the patient's own connection to those three legs, within or outside of the Christian or any other religious tradition, is a most powerful resource.

Faith generates hope, and hope sustains us at difficult times. This is true whether you see faith through a religious lens or otherwise. Charity, again whether through a religious lens or otherwise, can be interpreted as being generous with love to those injured -- including generosity to ourselves. Within the DID system, charity can be seen in some alters being generous to others frightened as well as hostile alters. This is something to be encouraged whenever it arises.

For those patients who disparage religion, therapists can focus on the spirituality of a beautiful sunset, the earthiness of a moss covered rock, the intricacy of a bird's song, the nourishment of breathing in the forest after a brief rain. Regardless of the therapist's own belief system, you must be open to the possible paths of faith your patients can access -- even if that faith is limited to confidence that the earth will hold you up, that the sun will warm you.

In short, and this is the whole point: Abused individuals all have to be helped to give themselves a reason to wake up in the morning, to have a meaningful task to accomplish. Recovery from abuse is a deeply meaningful task.

This is something therapists can continue to remind their patients about. Begin with keeping the meaningful task of recovery split into quite small steps, like breathing in a warm sense of goodness even just once each day. In the context of patients who are religious, it is appropriate to encourage them to apply the tenets of their religious views of kindness and compassion to and between their alters. In the context of patients who wish to avoid religion, it is appropriate to remind them to be grounded in faith, hope, and love, and again to apply those qualities to and between their alters.

\hypertarget{correcting-misunderstandings-about-recovered-memory-part-1-of-3}{%
\section{Correcting Misunderstandings about Recovered Memory -- Part 1 of 3}\label{correcting-misunderstandings-about-recovered-memory-part-1-of-3}}

\emph{Posted on October 23, 2019}

There was a recent Facebook post concerning a statement of the American Psychological Association (APA) on recovered memory. That statement reflects a misunderstanding of the etiology of DID. It ends with a statement that many researchers say there is no empirical evidence for even the idea of dissociation sheltering memories from ordinary conscious access. That misunderstanding continues to guide therapists (and their patients) in the wrong direction.

The APA statement asserts that certain questions ``lie at the heart of the memory of childhood abuse issue.'' The first question noted is: ``Can a memory be forgotten and then remembered?'' This question presumes that a traumatic memory is actually forgotten. That presumption is a fundamental misunderstanding of dissociation resulting from early childhood abuse.

A more correct question is a bit longer and more to the point. It would be something along the following lines: ``Can memories be compartmentalized so as to be rendered inaccessible to the conscious mind so long as amnestic barriers created as a function of that compartmentalization persist?''

Why is this important? From the very beginning of psychiatry, it has been clear that there are many memories of events which are not readily accessible to the conscious mind. This is true whether you consider distinguishing between the conscious and subconscious mind or whether you are analyzing dissociative experiences involving alters.

This then puts the second posed question in its appropriate context: ``Can a memory be `suggested'' and then rendered as true?'' Without the above re-framing of the first question, this second question sets up the false inference that recovered memories are equivalent to hypnotic suggestion.

Once again, context is critical to understanding. Yes, there are similarities between hypnotic states and dissociative states. Should one take from those similarities that hypnotic suggestion and dissociation resulting from trauma are identical? No.~One should understand that human minds have the capacity to act in the world without those actions always being consciously accessible and controllable. Hypnotic suggestion is one way that can happen. Dissociation resulting from trauma is another.

Clarifying that this is a fundamental ability of mind should enable psychiatrists, therapists and others to understand why certain memories would be inaccessible for periods of time or only be accessible in particular situations. They are conventionally inaccessible, not forgotten. It should be clear that under the pressure of massive early childhood trauma, such a fundamental ability of mind would necessarily be used to allow a child to survive the abusive onslaught.

\hypertarget{correcting-misunderstandings-about-recovered-memory-part-2-of-3}{%
\section{Correcting Misunderstandings about Recovered Memory -- Part 2 of 3}\label{correcting-misunderstandings-about-recovered-memory-part-2-of-3}}

\emph{Posted on October 24, 2019}

The APA statement continues with the claim that experienced clinical psychologists view the phenomenon of a recovered memory as being rare. In support of that claim, it notes that one experienced practitioner reporting having a recovered memory arise only once in 20 years of practice. Again, such a statement needs to be put in context: In my 40 years of practice as a psychiatrist, I received many referrals, from other psychiatrists as well as from family doctors, of patients with noted dissociative symptoms including alters. None of those referrals included a dissociative diagnosis despite their identification of dissociative symptoms!

Why would a referral that included dissociative symptoms fail to include a primary or even a secondary diagnosis of dissociation? I think that the referring physicians didn't want to give such a diagnosis as there was no medication to prescribe for treatment. They didn't want to run the risk of having a long term patient with a difficult prognosis. More importantly, they didn't want, or they did not know how, to engage in proper psychotherapy.

As the article continues, it notes that memory researchers do not subject people to a traumatic event in order to test their memory of it. I understand that memory research usually takes place either in a laboratory or some everyday setting and harming participants is not part of any acceptable protocol. Further, DID arises when there is \emph{ongoing} early childhood trauma, not just a one-time event. One time events can result in PTSD but I am unaware of any information indicating that one-time events can result in DID. So, how to research these questions?

While I cannot advise traumatizing animals as a test model, there are plenty of traumatized animals that can be examined. If you go to any animal shelter, you will likely find traumatized cats, dogs and birds that get triggered by certain input. In fact, many of them experienced trauma on an ongoing basis from infancy. One might do a behavioral analysis of those animals and extrapolate from there.

I am aware of a rescue dog that wouldn't come out from under a bed for three weeks after he was adopted by a family. He gradually became a loving and positive addition to the household. Several years later, a grandparent visited. The dog had met this elderly man many times before without incident. But at this point in his life, something changed: the man now needed to use a cane to walk.

The moment the dog saw the old man with a cane, he ran under the bed and refused to come out while the man was there. One can assume with some confidence that somewhere the dog retained the memory of a man with a stick beating him. The cane triggered memories that overwhelmed memories of this specific grandparent he had been unafraid of \emph{until triggered by seeing the cane}.

\hypertarget{correcting-misunderstandings-about-recovered-memory-part-3-of-3}{%
\section{Correcting Misunderstandings about Recovered Memory -- Part 3 of 3}\label{correcting-misunderstandings-about-recovered-memory-part-3-of-3}}

\emph{Posted on October 25, 2019}

The article continues, saying that ``we can not know whether a memory of a traumatic event is encoded and stored differently from a memory of a non-traumatic event.'' This ignores the foundational history of psychotherapy.

This mistaken view is a product of the following kind of bias: ``If something like this ever happened to me, I am sure I would never forget it for the rest of my life.'' It assumes that everyone's experiences are equivalently encoded in memory. In many cases of traumatic events, the trauma is so overwhelming that the victim's survival drive results in accessing resources that overwhelm one's ordinary mental process in order to deal with the trauma, including dissociation. Trauma memory is both stored and accessed differently than ordinary memory, as discussed in the Engaging Multiple Personality series.

Here is an example from my own patient histories that is by no means rare in a therapist's practice: A successful professional woman came to me with the complaint that she thought she was losing her mind. She said she had been having hallucinations or delusions that her father had sexually abused her. She was certain it never happened. Therefore, it must be that she was losing her mind.

All I said to her at the time was that she must have had ``some bad experience in her past.'' I purposely gave her a vague and ambiguous answer. I said it in a reassuring and supportive way. It is important to give people in need both support and hope that an explanation and potential resolution was possible for difficulties. At the next session, an alter jumped out and confirmed that the abuse memory was true, that she (the alter) was the one who had been holding the memory in order to protect the other parts of the system. While alters usually take a lot longer to feel comfortable and trusting enough to appear in therapeutic sessions, this quick appearance was not unique.

Why am I confident that the memory was correct? In fact, the father had been dead many years. No third party witnesses were around to confirm or deny the events. So the question might be raised as to how can anyone prove that such a memory is true?

Again, context and definitions are critical. First, the notions of ``correct'' and ``true'' must be understood properly. In early childhood trauma, most details are irrelevant. Why? An infant or toddler, any very young person, will not focus or remember the details of most any event. What they do remember is the feeling they have; love, warmth, irritation, and so forth. The experience of ongoing abuse of a child is an overwhelming mass of fear, pain, confusion and panic. That is the key memory that one can consider to be correct and true.

The size of the room, what the abuser might have been wearing at the time, or other conventional perceptions are irrelevant to the truth of such a memory. Witnesses in court cases that are not dissociative often err on such details and their veracity is then attacked. Do not be deceived about what you need to evaluate as true and correct in cases of early childhood abuse.

In the case of this patient, the proof is that after suffering from years of suicidal depression, despite being unsuccessfully treated with anti-depressants for years, the patient recovered through psychotherapy. By engaging in dialogue with the alters in the DID system through the psychotherapy, she was able to process trauma that they were holding within amnestic barriers, she recovered. She was rapidly able to eliminate anti-depressants.

Further, the ongoing physical pain she complained about as a constant in her life eased tremendously. Instead, the roots of the pain were identified by the alters because that pain was connected to memories of the abuse, not to muscle strains, over-exertion, or any other external factor. Dealing with the trauma of abuse eliminated the physical pain. One can say, ``the proof is in the pudding.''

Why am I so focused on context and definitions, on asking the right questions? Just consider whether or not you would reveal a closely held personal secret to someone who has already said they won't believe that whatever you say could possibly be true.

Recovered memory is not rare for those with DID. If it is being held by alters in a DID system, it will not be revealed to therapists who deny, do not understand, or do not accept the phenomenon of dissociation. The gateway to healing those with DID is engaging the alters, not dismissing them.

\hypertarget{the-devastating-clinical-consequences-of-child-abuse-and-neglect}{%
\section{The Devastating Clinical Consequences of Child Abuse and Neglect}\label{the-devastating-clinical-consequences-of-child-abuse-and-neglect}}

\emph{Posted on February 4, 2020}

The subject of this post is a paper I just read online published in the American Journal of Psychiatry. I usually only glance at the subject lines of articles and dismiss them, because they are usually about psycho-active drugs. This time the title focused on the roots of mental illness. The link included an interview by Stephen M. Strakowski, MD. with the authors of the paper entitled:

\emph{The Devastating Clinical Consequences of Child Abuse and Neglect: Increased Disease Vulnerability and Poor Treatment Response in Mood Disorders}

For the DID community, and for society in general, this is a critically important topic. The title of the paper speaks for itself. I have been fighting for years to elicit recognition of this. As of the date of my reading, there were 52 comments at the end of the article, representing a fair cross-section of psychiatrists today.
As noted in my previous post regarding progress in the DID community, and the (mostly) lack of progress in the therapeutic community treating DID, it is of grave concern to me. I wanted to know if psychiatrists of this generation have moved on. Or, are they still dismissing the serious consequence of early childhood trauma and neglect like my contemporaries. Because I have been retired for more than a decade, I try to follow this issue in journals.

I remain disappointed. While the article, interview and research are spot on -- highlighting the deep and ongoing impact of trauma, many readers of the article are still harbouring, defending and promoting their old ignorance. They remain committed to their mis-understanding of psychological trauma, about the nature of traumatic memory, and fail to see the presence and impact of trauma in their daily clinical work. I have addressed the common questions, listed in quotations below, that readers of the article raised.

\begin{enumerate}
\def\labelenumi{\arabic{enumi}.}
\tightlist
\item
  ``Most importantly, just how accurate are these reports? What are the biases present? Considerable research has shown that human memory is notoriously faulty.''
\end{enumerate}

In assessing traumatic memory, we are concerned with the effect of the trauma on the patient's current functioning. The fact that exact details are not accessible precisely is of no significance. If a child has been raped, I don't care if her recall is not precisely accurate. The inability to accurately identify the culprit's height and weight, or the crime scene's exact detail. This is the nature of traumatic memory.

In my book series Engaging Multiple Personalities, I wrote, ``It is dangerous to use our own ability to access non-traumatic memories as a standard against which we judge a trauma victim's response.'' Clinicians should not be bogged down worrying about individual minor details of the event, but instead should focus on the clarity of the emotional memory. Otherwise, they will continue to ignore the effect of the past trauma on their patient's present functioning.

\begin{enumerate}
\def\labelenumi{\arabic{enumi}.}
\setcounter{enumi}{1}
\tightlist
\item
  ``Do those answering questionnaires often do so subconsciously wanting to please or support the expert asking the question? Are those suffering in other ways predisposed to emphasize past negative experiences?''
\end{enumerate}

There is this persistent charge that we help create false memories in our patients. It is no doubt left over from the 1980s which saw a sudden rise court cases of victims accusing their parents or care takers of sexual abuse. The pendulum does swing well over the median in any social phenomenon when it first arises, but that simply means we should examine our own biases as well as the statistical likelihood of abuse. We must maintain an appropriate index of suspicion -- particularly when encountering depression that is drug resistant.

\begin{enumerate}
\def\labelenumi{\arabic{enumi}.}
\setcounter{enumi}{2}
\tightlist
\item
  ``How often are accounts independently verified?''
\end{enumerate}

This fails to acknowledge that most abuse occurs behind closed doors where the only witnesses are the abuser(s) and the abused. The demand for independent verification ignores the fact that the trauma can be identified enough to know that something bad happened by its impact on a patient's current presentation.

\begin{enumerate}
\def\labelenumi{\arabic{enumi}.}
\setcounter{enumi}{3}
\tightlist
\item
  ``As the preceding comments show, there is always an abundance of anecdotes. What is needed is hard questioning scientific work and evidence closely scrutinized.''
\end{enumerate}

It is common in challenging psychiatrists by dismissing what they do when they report on a single case. They call it anecdotal rather than ``scientific'' evidence. Anecdotal simply means that it is based on personal experience rather than formal research. Formal research is fine, just as the article that provoked these responses was based on a large study. Nevertheless, there is so much anecdotal evidence that psychiatrists should not wait to adjust their index of suspicion when encountering patients who likely have trauma in their background. If a patient experiences multiple treatment failure by psychiatrists who only used pharmacological agents, and showed recovery or significant improvement following psychotherapy, surely any inquiring mind would seek to find out the reason. Common sense, empathy and compassion suggests that therapists should at the very least start questioning the lack of humanistic aspect in merely prescribing psycho-active medication as the sum total of the therapeutic engagement.

We must reconsider the error of seeing all mental illness as a brain disease. In medical training, we all were taught that when considering a diagnostic formulation, we take into account, biological, psychological, social and noetic elements. It is amazing that today, in the name of ``science'', psychiatrists have mostly turned into mechanistic pill pushers. This is science as defined by the pharmaceutical industry that has its own profit driven agenda -- hence all the ``off-label'' recommendations they promote in Continuing Medical Education conferences. They infer that psychiatrists should feel proud that their work is ``scientifically based'' because they are prescribing pills to correct a ``chemical imbalance.'' This logic allows them to ignore social, psychological or spiritual factors in a patient's life milieu. In fact, it is like prescribing insulin to pre-diabetic patients without asking whether or not they eat sugar saturated meals every day. There is only symptom management as the underlying cause is not being addressed. As a result, healing is not possible.

\begin{enumerate}
\def\labelenumi{\arabic{enumi}.}
\setcounter{enumi}{4}
\tightlist
\item
  ``Maybe the most distressing aspect of some of this is the arrogance of those who purport to know what really happened\ldots{} and the judgments laid on many families just trying to do their best.''
\end{enumerate}

This is not about corporal punishment by an overworked house wife or an over-strict father, following the Biblical admonishment of ``Spare the rod and spoil the child.'' This is about someone, not necessarily a parent, engaging in sexual molestation, physical abuse, neglect, and betrayal trauma. It is about abuse, not about ``spoiling'' a child.

With respect to corporal punishment used by parents to discourage certain unwanted behaviors, one should consider whether or not the child automatically learns a different lesson, that one should use force if someone disagrees with you.

The significant factor in analyzing corporal punishment that may actually be abuse is whether that harsh physical punishment is given in the absence of love. In the absence of other supportive and loving people in the environment, corporal punishment will leave a permanent injury to the victim. Alice Miller has written amply on effect of early child abuse and trauma. Her books are thoughtful and practical.

\begin{enumerate}
\def\labelenumi{\arabic{enumi}.}
\setcounter{enumi}{5}
\tightlist
\item
  ``There are millions of traumas a year, including those to children. Trauma is the common cold of psychiatry. Around 90\% of people feel bad for a week, then forget about the trauma. This is analogous to having a cold.''
\end{enumerate}

This reader should go back to the definition of psychological trauma, which means \emph{stress that overwhelms the system, leaving behind a gaping wound that refuses to heal by itself}. The common cold does not devastate the patient. It is healed by one's own healthy immune system. And yes, it is usually forgotten a few weeks later because its impact ends when your body finishes the healing process. The effects of child abuse and neglect last a life time. Even if it is not accessible to one's declarative memory at any given time, the body keeps the score because the damage has not been healed. It often emerges in the form of symptoms like depression, rage, or self-harm rather than an accessible declarative memory.

\begin{enumerate}
\def\labelenumi{\arabic{enumi}.}
\setcounter{enumi}{6}
\tightlist
\item
  ``Those who are affected by trauma have pre-existing conditions or genetic vulnerabilities to it.''
\end{enumerate}

This reminds me of what happens when patients are labeled as suffering from a personality disorder. This unfortunate and common practice implies that the patient has to live with the dysfunction or disability because of constitutional factors. Effectively, is it saying: ``You are born with an inability to handle distress. You may as well learn to live with it. Just get over it.''

Finding a pre-existing condition to explain a patient's vulnerabilities does not help. The main problem in understanding and accepting the connection of early abuse and neglect to their consequence of dysfunction in later life is the difficulty in finding a concise, easy to apply treatment -- such as a medication. But, there are no medications that heal early childhood trauma. Psychiatrists perhaps feel threatened and insecure when we face a case for which we have to employ full empathy, exercise compassion and be fully genuine when facing another human being who is experiencing this level of psychological pain. Pharmaceutical companies and their affiliated conferences/training programs promote simple clear cut mechanistic approaches, as if the human mind is like fixing a car or draining of an abscess.

Psychiatry is, or used to be, predicated on a deep understanding of the need to engage with empathy, a positive regard, and a genuine openness on the part of the therapist. Carl Rogers named the three essential attitudes necessary for a therapist to be of benefit: congruence (genuineness), unconditioned positive regard, and empathy. This comes with deep listening. It is far from a mechanistic cold surgical procedure or prescription pad.

I believe this view is much more important than EMDR (eye movement desensitization and reprocessing,) or CBT (cognitive behavior therapy), which are listed as the recognized treatment procedure for trauma based PTSD, for different kinds of dissociative disorder, and disabling emotion of depression, anxiety, panic disorder. In and of themselves, no doubt they are helpful for some patients, and more helpful in the hands of well trained and empathetic therapists. But, we should understand that EMDR and CBT are just tools. They are like scalpels: It is only in the hands of a skillful surgeon that a scalpel becomes a truly useful tool.

The real problem is the difficulty in finding therapists who understand the need to be grounded in empathy. Less important is the number of years of training or how many diplomas are in the office walls. Not enough attention is paid to the humanistic issues.

In medical schools we were all taught that when considering a clinical problem, we need to consider the biological, psychological, social and noetic roots. This has not changed and will not change in all worthwhile medical institutions wherever they are found. It is unfortunate that for many psychiatrists once graduated and licensed to practice, these considerations are soon forgotten. When such simple rules are forgotten, it is easy for a materialistic philosophy to take over. Financial consideration takes precedence and, as a result, one becomes more easily swayed by pharmaceutical company marketing.

Just consider a hypothetical child who is inattentive in school and gets a quick diagnosis of ADHD. If no one is interested in identifying his concern that his parents are fighting every night to the point of violence, is the critical diagnosis of ADHD all you need to come to?

We must do better than this. Real advances in psychiatry will require getting back to its roots of empathy and compassion. Let us all push ahead step-by-step in the right direction.

\hypertarget{society-and-did}{%
\chapter{Society and DID}\label{society-and-did}}

\hypertarget{reflections-on-responding-to-reports-of-abuse-by-public-figures}{%
\section{Reflections On Responding To Reports Of Abuse By Public Figures}\label{reflections-on-responding-to-reports-of-abuse-by-public-figures}}

\emph{Posted on March 12, 2015}

Reports of famous personalities being accused of sexually abusing young girls decades ago, such as Jimmy Savile (deceased English TV celebrity, knighted by the Queen) or Bill Cosby, appear regularly in the news. People without experience dealing with sexual trauma always ask, ``Why did it take so long for the accusations to come to the public attention?'' The question is asked in a way that is intended to challenge the credibility of the accusers. People with experience dealing with sexual trauma know that, invariably, abusers take advantage of their social position and power to make sure victims are intimidated, frightened, and therefore very reluctant to come forward to report the crime.

Often, when complaints are actually made, they are not taken seriously. They are blocked at the very beginning, by lower levels of administrators, celebrity handlers, and sometimes at the police level. The complaints almost never get to the right place even to be investigated. The abusers are usually not threatened with prosecution until decades have passed and, unfortunately, not until dozens of accusers come forward to break through the ``he-is-famous, that-cannot-be-true'' barrier.

Therapists may have the concern that they themselves will be sued by people in power who are accused of abuse. They may worry that they will be attacked on some kind of a claim that they were incompetently affirming a client's delusion and, in that way, threaten the therapist. We must remember that we practice psychotherapy for the benefit of patients that have been traumatized. Often it is the therapist that is the first individual to undermine the belief instilled by abusers that no one will take their claims of abuse seriously. We cannot help them heal if we do not communicate our confident belief to our client.

If a client told me that she had been abused by someone revered by the public, like Bill Cosby or Jimmy Savile, a critical question will then follow, implicitly or explicitly, ``Do you believe me?''

I would respond just as I would if they told me that someone not famous, perhaps their parent, had abused them. Experienced therapists usually have developed enough insight to determine whether the client is telling their truth or lying for some ulterior motive. If the client shows all the congruent body-language and demeanor of someone telling me of past trauma, I would have no difficulty recognizing that truth. Within that recognition, the truth I am concerned about is whether or not the patient has been traumatized. As I write in Engaging Multiple Personalities, the exact details are not important to the therapy. What is important is to recognize the truth of the trauma and proceed to support the patient in the healing process.

To show doubt about the traumatic memory, or to demand external checks on the accuracy of any memory, will likely be an experience of re-traumatization for the client. The key to understanding this is that abusers always impress upon their victims that no one will believe them, that they have no power to convince anyone that any abuse has taken place. This is why in the case of Sir Jimmy Savile, it took decades for these cases of child sexual abuse to come to the public awareness.

At the time of the crimes, victims were generally far too scared to tell anyone. Indeed, if they told someone, they were not believed. After all, Sir Jimmy was honored and knighted by Her Majesty the Queen. Her Majesty would never knight anyone who had done such an evil thing. How dare the victim suggest that! A similar logic is used against those accusing Bill Cosby of sexual abuse.

As a therapist, if your assessment is that the client has been traumatized, you need the courage to stand by your client, to support the truth of their painful history of abuse. If, in the unlikely situation the therapist is put on the stand in court, the therapist has every right to affirm and assert that:

\begin{enumerate}
\def\labelenumi{\arabic{enumi}.}
\item
  Yes, I believe the patient was telling me the truth of her abuse experience.
\item
  No, I did not seek external corroborating proof as no such proof was necessary to proceed with psychotherapy. Investigations are the responsibility of the police. Following those investigations, it is up to the lawyers and judges to argue about whether or not the burden of proof for criminal law purposes has been met -- which is a very different standard than a therapist needs to determine whether or not a patient has been traumatized. My expertise allows me to determine that the patient has indeed been traumatized, and that is all I need to provide therapy.
\item
  The attack by the defense lawyers will likely be based in the argument that the client's identification of the abuser to the therapist is hearsay. But, hearsay evidence \emph{is} permitted in court if you are stating it not for confirming the truth of the statement but rather for the purpose of confirming that the statement itself was made. Remain confident. You can clearly state that you are not accusing the public figure, your client is \emph{and you have no reason to doubt her}. In truth, the only reason for doubting the accusers of Jimmy Savile was his public persona. Again, this is the same argument people use for doubting the accusers of Bill Cosby. Any therapist who has dealt with trauma knows that the public persona of abusers is often quite different than their private conduct. The Catholic church is dealing with the repercussions of this dichotomy and their failure to protect innocent children for many decades.
\item
  All I need for doing therapy is the confidence that my client is telling me the truth of a past abuse experience, and I have no doubt, based on my training and experience, that she was abused. I am not interested in who abused her, except that in all abuse situations where there is a relationship between the abuser and the abused, the abuser is always someone in a position of power over the abused, that it was someone she could have trusted, and that individual took advantage of her. I do not need a lie-detector test or a police forensic report to confirm that abuse happened for providing therapeutic support to that client.
\end{enumerate}

\hypertarget{the-focus-in-documentaries}{%
\section{The Focus in Documentaries}\label{the-focus-in-documentaries}}

\emph{Posted on September 19, 2016}

I have been asked to participate in a public television documentary on DID in Hong Kong. While I think a locally produced documentary is an excellent idea for public education in a city of 7 millions in Asia, where DID is considered nonexistent by the mental health care system, I have reservations about it. There are already many MPD documentaries (and movies) in the public media, whether it be on television, YouTube or otherwise.

The effect in the past has been that the public sees DID as a curiosity, a circus show. So far, all the movies about DID continue to create the false impression that it is a rare condition. The movies and most documentaries portray it as a very curious and, for those not afflicted by it, entertaining illness. Marketing clips, for example, show an adult professional woman suddenly turning into a 4 year old girl so that the viewer will think, ``how extraordinary -- I must watch this!''

The result is that the public is impressed for the wrong reason. The DID community will never overcome the prejudicial idea that DID is very rare. The media focus is on the display of alters rather than the root cause of horrific early childhood abuse. That is where the (sometimes) bright light of documentary journalism needs to focus.

DID only appears to be rare. It is a hidden phenomenon, based on very private and confidential personal histories. It is not like a skin rash that someone on the outside immediately sees. Individuals with DID often include hosts that do not know their alters exist, or hosts that consider this kind of splitting as something private. They don't even want their doctors to know for fear of ridicule, disbelief or being insulted.

I personally know of psychiatrists who simply ``don't believe'' in DID, as if it were an issue of faith. For those psychiatrists, the sudden appearance of an alter as a 4 year old girl sitting on the floor in their consultation room is suppressed. It is met with ``Go back to your chair and behave like an adult. You are not four, you are thirty-four.'' But, DID is not an issue of faith. It is a diagnostic category that has been included for decades in multiple editions of the DSM.

There are several psychiatrists in apparent authority who promulgate their mistaken view. For example, there is a well-known authority, a university professor holding a chair in psychiatry, who proclaims that although he has been in his authoritative position for decades, he has never come across a genuine case of DID. Most laymen, and psychiatrists as well, do not challenge his view. They do not challenge that apparently authoritative statement. It can be scary to confront a so-called authority who has the power to belittle you, to attack you. No wonder society cannot get rid of the idea that DID is a very rare condition.

Most psychiatrists in Hong Kong have never seen even one case of DID. I would argue that the psychiatrists have almost definitely seen DID individuals because, statistically speaking, research shows it to be as common as schizophrenia -- which virtually all would acknowledge having seen. What is the argument one can use with them to help them understand that it isn't that they haven't seen DID, that they have simply failed to \emph{recognize} DID? It is to point out that statistics don't lie.

Busy psychiatrists looking for symptoms to pigeonhole a patient into a particular diagnostic box of depression, bipolar disorder or perhaps borderline personality disorder, will see how they can fit the patient into their familiar basket of diagnoses. In other words, their index of suspicion -- which excludes DID -- will lead them to what they are most comfortable identifying and treating. As a result, DID will not be recognized and therefore not get diagnosed. When that happens, the patient will likely decide that it would be no use to let such a psychiatrist know of the true nature of their affliction. The consequence is that there is once again no feedback and once again a psychiatrist fails to recognize the disorder.

There isn't much to be gained in showing an adult speaking as a child on the screen. It becomes another cycle of entertainment, rather than an expose of an extremely serious public health and social issue; the issue of early childhood trauma. This is the point to stress. It is critical that the public be educated about the widespread nature of such trauma along with its tremendous and wide-spread ramifications to the individuals traumatized and to society in general.

I understand that movies and television shows seek to show something impressive to grab viewers. Unfortunately, what they think is impressive (and more palatable to viewers) is the display of alters rather than the heart of the issue, which is abuse.

I have not yet confirmed my willingness to participate in the documentary as I am still pondering these points.

\hypertarget{disclosing-your-did-a-cautionary-note}{%
\section{Disclosing Your DID: A Cautionary Note}\label{disclosing-your-did-a-cautionary-note}}

\emph{Posted on October 16, 2015}

\emph{I am delighted and honored that Robert Oxnam, author of A Fractured Mind: My Life with Multiple Personality Disorder (Hyperion, 2005) has most kindly consented to be a guest blogger on this topic of disclosing one's DID to others. I am confident that his generosity in writing this piece will result in much benefit to the DID community. His ongoing willingness to share his experience with others is a tribute to the power of genuinely walking the path of healing.}

``Disclosing Your DID: A Cautionary Note''
From: Robert Oxnam
October 16, 2015

I've been asked by my good friend, David Yeung, to offer some advice about the wisdom and dangers of disclosing your DID condition to others beyond your family and a trusted circle of close friends. Having published my DID story a decade ago, he knew I had lots of experience with the ups and downs of openly revealing the disorder.

Looking back, I think my disclosure motivations were similar to many who have struggled privately with DID over many years. I wanted to be honest about who ``we'' are inside and how we've coped with a difficult life. I wanted to embrace my outer associates -- family, friends, workmates -- just as I had learned to embrace my inner identities. As one publisher said to me -- ``I think you're writing this book so you can own the rest of your life.''

And so, I blithely pushed ahead, wrote the book, and awaited the results, good or bad. In retrospect, I was very fortunate to have a relatively favorable outcome -- roughly 80\% of the responses were positive/very positive while 20\% very negative/outright vicious. Many in the media world embraced the book and, for a few weeks at least, it became a bestseller. I was deluged with supportive emails and letters, especially from mental health professionals and from fellow DIDs. But nasty anti-DID shrinks unloaded on me and some reviews were laced with haughty and mocking language. Some former friends and even family backed away; while others implied that I was making up the whole story. Just go to Amazon.com, check the reviews of A Fractured Mind, and you'll see the whole spectrum.

In retrospect, I've learned a great deal about the volatility that surrounds our disorder, and ``we'' have learned how to find inner strength to cope with the harsher realities of DID disclosure. Most of all, I have come to realize that my 80/20 breakdown was an outright miracle and it could have been much worse. I have also concluded that my relatively-positive experience with DID disclosure has been an exception that proves the rule. What rule? Don't go public unless you've thought it out carefully and are ready to face difficult consequences, short and long term. Remember, you'll live with ever-expanding circles of ``people who know and gossip'' for the rest of your life.

Why was my experience an ``exception that proves the rule''? I think there were four factors that prompted an 80/20 response rather than 50/50 or perhaps even 20/80.

\begin{enumerate}
\def\labelenumi{\arabic{enumi}.}
\item
  ``Inner Consensus.'' In 2005, when the book was published, ``we'' already had fifteen years of post-diagnosis DID under our collective belt. We had fully identified the whole raft of inner personalities, found ways to break down the walls that separated us, and gone through a long-term merging process. The remaining five identities committed ourselves to a cooperative framework called ``cohesive multiplicity.'' And then, ``we'' openly discussed the pros and cons of going public. Eventually, we reached a heartfelt decision that, for our own sakes, and for the potential good of others, it was essential to write the book. And, we also agreed that each of us would tell his/her story separately so that none of us felt left out or diminished by the experience. In short, we were all ready for the reactions, come hell or high water.
\item
  ``Controlling the Narrative.'' The book itself was ``our story'' in our own words. Before anyone might react to that story, they would presumably have read the book and thus encountered experiences and observations that we ourselves had revealed in context. So we were not just disclosing our DID, but also offering an orderly and positive framework for helping others understand DID. These are the messages in a nutshell: a) DID occurs because of vicious abuse inflicted on very young children, b) DID is an intelligent child's way of coping with horrible treatment and staying alive in physical and psychological terms, c) There are great therapists who can treat DID with patience and care, producing remarkable results, and d) In addition, those without DID can learn from the disorder about how multiplicity is embedded in all humankind. And maybe, we hoped, non-DIDs could learn how to deal creatively with those inner forces and perhaps even find their own way to ``cohesive multiplicity.''
\item
  ``Timing.'' When the book was published, I was 62 years old, at the end of a multi-faceted and successful career as a specialist in China and Asia. I was already pursuing other activities as a novelist, business consultant, and television journalist. I didn't know it at the time, but I was also poised to enter the creative world as an artist working in sculpture and photography. Yes, I suppose one might say that my career trajectory was as diverse as my inner psychology. But my key point is that the timing was right to take a disclosure leap without fearing that I would lose my job and livelihood in the process.
\item
  ``Highly Supportive and Admired Partner.'' Vishakha Desai, my wife, has been and remains a crucial factor in dealing with my DID and coping with ``going public.'' It is impossible to imagine the arduous process of DID therapy and then public disclosure without her at my side. Vishakha has not only helped me in a thousand ways, but she has also become a fervent advocate for DIDs and dissociation therapists. She makes the powerful point that ``DID denial'' is really ``the second abuse'' -- first the child is brutally abused and suffers severe dissociation, and then, many in the public and not a few shrinks deny that DID even exists. The fact that Vishakha is now a major figure in global education, culture, and business means that her insightful views are deeply respected. Many now see her as a role model for ``DID partners.''
\end{enumerate}

So my message is this . . . The desire for disclosing your DID is totally understandable, and even noble, but the potential dangers are substantial. You need to think out the strategies and consequences in great detail, producing a DID version of what the business community calls ``risk analysis'', and what professional athletes call a ``game plan.'' Without such forethought, it's particularly difficult to engage in a ``partial disclosure'': letting a few more people know, trusting they will keep it private, but this risks a rippling effect if someone breaks your confidence. On the other hand, if you, along with your therapist and current circle of supporters, can create a plan that works for your inner DID system, and for your social and professional situation, then it's worth considering disclosure.

When thinking about these issues, DIDs and our therapists are fortunate to have a wide array of communication routes, both online and at in-person conferences. One remarkable example is the annual ``Healing Together'' gathering expertly hosted by an organization with an appropriately-upbeat title -- ``An Infinite Mind.'' I have had the honor of keynoting those conferences several times in recent years and will do so again in February, 2016 in Orlando, Florida. The Healing Together conference offers a wonderful chance to meet with hundreds of other DIDs and therapists, allowing attendees to be who they are without apology or having to hide. The conference offers a rich array of speakers -- including several who are coping with dissociation themselves -- and ample opportunity to raise whatever questions and viewpoints in a totally confidential environment. Above all, the chance to talk with other DIDs is enormously important, sharing our experiences and escaping the burden of feeling trapped and helpless. It is always helpful when getting ready to walk through a minefield to get advice from those who have already traversed it and can point out the dangers.

I pray for the day when DID is universally seen as a treatable disorder, not caused by something you did, not posing threats to others, and deserving sympathy rather than suspicion. Then we can all reveal our disorder without fretting about unintended consequences.

\hypertarget{the-failure-to-acknowledge-comparing-abuse-in-the-military-and-childhood-trauma}{%
\section{The Failure to Acknowledge -- Comparing Abuse in the Military and Childhood Trauma}\label{the-failure-to-acknowledge-comparing-abuse-in-the-military-and-childhood-trauma}}

\emph{Posted on September 2, 2015}

In my books, Engaging Multiple Personalities Volume 1 and 2, I briefly discuss the fact that PTSD was not really acknowledged until the military was overwhelmed with veterans suffering from it. I pointed out some of the similarities between veterans with PTSD and DID patients whose trauma arose from very early childhood abuse. The key similarities are the inescapability of the danger and the resultant hyper-vigilance. The key difference in PTSD resulting from the battlefield is that a soldier has the support of other soldiers who understand the wartime environment whereas a child being abused is all alone, with no buddies, no peer group to support them or get them help.

\href{http://www.huffingtonpost.com/rep-niki-tsongas/support-male-survivors-of-sexual-assault_b_7832846.html}{There is now a report from the General Accountability Office of the US Federal government} on the ``staggering number of men in the military that have been sexually assaulted, and hinted at the underlying problem, writing: `DOD has recognized that a cultural change is needed to address sexual assaults but has not yet taken several key steps to further this change.' For all victims, male and female, the environment frequently acts as a deterrent rather than a support structure; but for men the effect appears to be more significant.''

It is a societal bias that the issue gets attention when it impacts men but not so much when it impacts women. This is simply wrong -- terribly wrong. It has been known, and not particularly seen as a ``staggering'' problem, that women have been similarly victimized. However, the fact that sexual assault in the military is now being scrutinized may have a positive impact on men outside the military -- particularly those abused as children -- who have been sexually attacked. Hopefully the changes that the military makes to protect its men will similarly protect its women. From my experience treating both men and women who have been sexually abused, I think it is quite possible that the finding that ``for men the effect appears to be more significant'' will be seen as wrong -- terribly wrong.

The information described in the article ties into my experience treating DID patients, where they were raised in an environment that was a deterrent to reporting and healing, where the risk of retaliation is stupendous, and where the assaulted individual has no safe option to confront their attacker(s). It is instructive that the language within the military report context talks about betrayal: ``Retaliation compounds the injustice and personal betrayal survivors experience and has been a lasting concern among survivors, advocates and those of us in Congress fighting to institute reform.'' Betrayal trauma is almost always a key component in child sexual abuse.

The SAPRO report acknowledges the high levels of retaliation, and in May a report conducted by Human Rights Watch drew similar conclusions. Human Rights Watch made the problem vividly clear by sharing candid stories from service members who experienced backlash firsthand.

It is interesting to note that many comments were made about this report questioning who was doing the assaulting. A specific concern was raised that, once again, those doing the assaulting were not being identified, called out or punished for their crimes. All of this is quite familiar to anyone with experience treating the trauma of early childhood abuse.

It is my hope that just as the military's concerns about veterans ended up mainstreaming the understanding of PTSD, this report and any follow-up work will clarify for the therapeutic community that betrayal trauma has a lasting deep impact and must be understood and addressed. This is true whether that betrayal affects an adult in the military or a child living in a domestic war zone.

I see it as an optimistic sign that finally there is an opening that may force psychiatry to face the issue of trauma and possible dissociation directly. It is something we can no longer ignore or keep silent with the prescription of a pill.

\hypertarget{trigger-warning-news-of-abuse-by-men-in-power-part-1}{%
\section{Trigger Warning: News of Abuse by Men in Power Part 1}\label{trigger-warning-news-of-abuse-by-men-in-power-part-1}}

\emph{Posted on November 18, 2017}

The year 2017 is becoming an eye opener for many people about the pervasiveness of sexual abuse. It is now, finally, being widely and publicly acknowledged that 1) sexual abuse is common; 2) sexual abuse is often covered up and under-reported; 3) abusers often hide under a veneer of respectability, and 4) the power dynamic at the center of abuse enables abusers to continue to abuse and suppresses victims from speaking out -- often for decades.

The recent cascade of reports from abuse survivors accusing men in power of taking advantage of women and children (of both genders) does not shock me, nor does the fact that the events in question often have been hidden far longer than most people can imagine. These dynamics are well known to survivors of early childhood abuse and to members of the DID community. As a psychiatrist who worked for many years on the impact on survivors of early childhood trauma, I feel compelled to comment in support of survivors speaking out. No matter how long ago the abuse happened, and no matter who the abuser was, these testimonies are critical for healing those attacked. They are critical for protecting others from harm right now as well as into the future.

Many of the same arguments used against these survivors of celebrity abuse have been made against early childhood abuse survivors with DID. Despite their lack of celebrity involvement, my clients' histories of abuse are quite similar to those now being made public. Unfortunately, very few people pay attention when the attackers are non-celebrities who might be parents, siblings, doctors, clergy and others in the community, just as very few people pay attention when the victims are not celebrities. As Jane Fonda pointed out, people are paying attention now because the victims are celebrity white women now coming forward.

The loudest and most common ways survivors are attacked are by assertions that because the abuse happened so long ago, the report is unreliable; because it is a case of he said/she said, the report is unreliable; and because the abuser is a well-regarded person in the community, the report is unreliable or even fraudulent. These statements are the marks of actual ignorance, self-serving intentional ignorance, and/or participatory enabling.

I cannot speak for everyone, but I can certainly speak as to the survivors that were my clients, dealing as adults with their early life unprocessed trauma. Here it is: I have zero doubt about the fact that my clients had been abused. Zero doubt.

\hypertarget{trigger-warning-news-of-abuse-by-men-in-power-part-2}{%
\section{Trigger Warning: News of Abuse by Men in Power Part 2}\label{trigger-warning-news-of-abuse-by-men-in-power-part-2}}

\emph{Posted on November 18, 2017}

People ask why it took so long for these celebrities to speak out, just as they ask of individuals abused as children -- with and without DID. Because the current reports involve adults for the most part, people are now beginning to accept that there is an impact of power dynamics. Where abuse happens in a setting involving a high differential of power between the abuser and the abused, the victim is usually completely intimidated. Whether it is a celebrity predator, an Olympic team doctor, a local clergyman, or a family member, these attacks often happened in one of two ways: 1) so unexpectedly that one is usually taken off-guard, rattled and confused, in which case the first thing that arises in conscious thought is: ``Who would believe me?'' or 2) following a period of grooming, a step-by-step dismantling of personal boundaries a little at a time, that appears to shift the abuse to some twisted and illusory appearance of consent. And, again, the conscious thought is: ``Who would believe me?''

Imagine being a very small child, the level of intimidation is life and death. The power differential with respect to adults, particularly adult family members, is incomprehensible.

In most cases, as survivors fear and sometimes learn, the accusations are readily brushed off for an extraordinarily long time. Survivors, DID and otherwise, often find out long after the fact that they were not the only target. Why? Because abusers rarely limit their abuse to a single target. It is part and parcel of the power dynamic that enables ongoing secrecy.

As we see in the current news reports, survivors are told how important and prominent the abuser is. They are told how the survivor's life would be destroyed if they speak out. In family abuse, the child is warned that the entire family will be destroyed. Further, reporting abuse usually requires going to or through someone whose responsibility is to screen the information and then to convey it to someone higher up. At any point in the chain, the report can be suppressed, dismissed or ignored. Protection is crafted for the abuser, not the survivor. The survivor's life, as they once knew it, is undermined and often destroyed.

How does it work? A scenario like this may help to illustrate it: An abusive man in charge of the local orphanage has dinner with the police chief of the city. The next day, the police receive a complaint about sexual abuse in the orphanage. It would be tempting for the chief to deem the report a lie -- after all, he just had such a nice dinner together with the purported abuser and nothing seemed amiss. If he thinks there might be even a smidgeon of possibility, it would be much easier to minimize the offence as a simple case of confusion that can simply be brushed off. Isn't it more important to protect the reputation of the purported abuser and the institution from such terrible claims? No, it is not.

You can substitute anyone in a position of power for the orphanage director in the above example. You can substitute anyone for the police chief in that example -- anyone in the money chain connected with the abuser. And remember, abusers are used to hiding their tracks. Even good people often miss the clues that their old friend, a pillar of the community, has an incredibly hidden dark side.

Evil deeds must be called out as quickly as possible -- even if ``as quickly as possible'' means decades later! Why? Because any time an evil deed is covered up, it will fester and grow, like a deep infection that periodically erupts to the surface. Old or new, whenever you identify an infection is the time to treat it.

It is quite understandable that the victim, having been caught by surprise, remains silent. The longer they remain silent, the harder it is to speak up. The more they see those who speak up be dismissed as crazy or as liars, the harder it is for them to summon the courage to speak out themselves. A perpetrator, however, takes that silence as encouragement. He is free to do it again because, after all, the victim(s) is (are) too cowed to speak up. The longer the silence, the less believable people will find their words when victims do speak out.

Perpetrators, notoriously, will re-offend. Like thieves, people almost never stop once they have gotten away the first time. The victim meanwhile has the unrealistic wish that this will not happen again to them. Why not listen to the warnings about speaking out, after all, this man is powerful---he could throw you out of the school, or ruin any opportunities which you worked so hard for years to approach. Can you stand up to a world class famous coach in hockey, swimming, or gymnastic coach who has taken you under his wing? Especially when you are representing your country at the Olympics and becoming world famous? Could you stand up so easily to a father who has been abusing you from as far back as your memory can go?

We should be shouting to the heavens our support and appreciation for the many celebrities as well as the ordinary men and women who have summoned the courage to speak out. It is only in this way that our children, our friends and our society can be protected from this scourge.

It remains extraordinary that in this day and age, we seem to accept the mind-set of a man that can speak about how he can ``grab a woman by her pussy'' and people will still elect him as a leader. The message this has sent continues to reverberate. But, now, that reverberation has resulted in the gathering of strength, of the coming together of survivors who are gaining power by exposing the abuse. Exposure is the disinfectant to protect ourselves and our children.

\hypertarget{trigger-warning-reports-of-abuse-by-men-in-power-part-3}{%
\section{Trigger Warning: Reports of abuse by men in power Part 3}\label{trigger-warning-reports-of-abuse-by-men-in-power-part-3}}

\emph{Posted on November 18, 2017}

What is happening now, people speaking out about having been abused, is incredibly important. For a single victim of abuse to think about speaking out is like thinking about going down a dark alley alone late at night. It is quite scary. From those initial victims speaking out despite the fearful consequences, speaking out is no longer going down a dark alley alone. Instead, it is going with dozens if not hundreds or thousands of friends, holding hands, protecting and cheering for each other. No longer so scary and not nearly so difficult.

Many men have been brainwashed to believe silence means consent, many women have been brainwashed to believe whatever they say or do will not make a difference in a male dominated authority structure. Don't accept that brainwashing. Your body knows the truth of its experience. Trust that.

Look at the impact of a few brave women speaking out. \#metoo, among other efforts, is forcing some of the abusers out in the light -- calling them to account for their heedless destruction of the lives of others. This call to account involving adult celebrities has led to starting to open the doors to acknowledge the evil impact that pedophilia have wrought on so many child stars, whose lives often fell apart completely under the pressure of their unresolved trauma.

It is spreading to the music and fashion industries as well as into politics and corporate executive offices. Let's continue that push, extend it to all survivors of early childhood abuse. Sunlight is the best disinfectant; natural, bright, and healing.

\hypertarget{why-ordinary-people-deny-testimony-of-abuse}{%
\section{Why Ordinary People Deny Testimony of Abuse}\label{why-ordinary-people-deny-testimony-of-abuse}}

\emph{Posted on October 10, 2018}

Weeks of heated debate concerning the appointment of a US supreme court judge, has come to an end and a decision was made -- a decision by the Republican majority that was not swayed by the bravery of Dr.~Blasey Ford. This hearing reminded people of a previous confirmation hearing in 1991, that also revolved around alleged sexual misconduct. The latest hearing went beyond simply bringing the issue of sexual harassment into the public eye. It went beyond that earlier hearing involving workplace sexual harassment, and forced the issue of sexual assault into the light of public scrutiny.

The \#MeToo movement has been critical in raising public awareness on the issue of sexual assault. But there remains, in general, serious misunderstandings on the question, function and qualities of traumatic memory. Until the public is better educated about how trauma impacts memory, victims' statements will always be doubted and misunderstood. In that way, victims of sexual assault will be retraumatized.

This dynamic played out in these hearings where many Senators affirmatively ignored the depth of research into how sexual abuse events are remembered by victims. The logic used by deniers of Dr.~Blasey Ford's testimony relied on their so-called ``common sense.'' Senators expressed their denial based on the questions people that have as anyone else might raise that has no genuine understanding of sexual trauma:

\begin{enumerate}
\def\labelenumi{\arabic{enumi}.}
\item
  If something so terrible as sexual assault or rape did happen, why does she not even remember the time/place/persons involved, with some clarity? The subtext of that question is barely hidden: \emph{If it was me, I would remember.}
\item
  If something so devastating did happen, why did she not make a formal complaint right afterwards? This also extends into the scenario pointing out, perhaps, that the victim repeatedly saw (or returned to) the abuser and acted as if nothing serious had happened. The subtext of that question is similarly barely hidden: \emph{She kept seeing him to wait in ambush to make a complaint decades later.}
\end{enumerate}

The public, and certainly elected officials, need to be better educated about the unique phenomenon of traumatic memory and behaviour. When judges, or other people in high position, fail to understand the nature of traumatic memory and phenomenon of victimization, all victims of sexual assault are subject to retraumatization. Unfortunately, one can simply refer to President Trump's mocking of Dr.~Ford -- attacking her memory of being sexually assaulted.

Once again, we must go back to distinguishing the different kinds of memory. We can easily access non-traumatic memory. This ordinary explicit memory, which is termed declarative memory, can be expressed in narrative form. An example of this is recalling what you had for lunch, when you had it and with whom you were eating; at around 1 pm, sitting at a corner of such and such eatery, hastily downing soup and a sandwich with my friend John. This is ordinary explicit, or declarative memory.

In contrast to explicit memory, there is implicit or non-declarative memory. This kind of memory is usually without verbal references. Generally speaking, it is vague, all jumbled up non-verbal memory. It often manifests in the body as somatic sensations and visual imageries.

It is in this kind of non-declarative memory that trauma is processed and stored. It is challenged and often disbelieved by people evaluating the memories of victims of abuse. Those who deny this kind of memory misjudge it. They make the mistake of comparing their own explicit memory to a victim's implicit memory. In other words, as was seen in the analysis of people denying Dr.~Ford's testimony, their erroneous logic is, ``If I can clearly remember what I had for lunch with John yesterday, why can't you remember clearly where, how and when XX attacked you?'' They make that error in judgment because implicit memory related to trauma and explicit memory related to everyday experience is processed very differently in our brain.

When an experience is encoded in fragmented, non-declarative memory, only raw emotions and physical sensations are accessible in one's consciousness. These may manifest in hyper-vigilance, sudden and overwhelming feelings of panic or dread. They usually include intense feelings of alienation, rage, and helplessness as well as terror at loss of control.

Instead of precisely expressive words, victims of assault (such as my patients when I was actively practicing psychiatry) may speak of ``wanting to throw up,'' or an intensely ``yucky feeling.'' Often they have intrusions of bizarre visual images. The inability to translate what is so strongly felt into something expressible in words leaves them frustrated, bewildered, angry, and hopeless. Their dilemma is perhaps best expressed by John Harvey (1990): ``Trauma victims have symptoms instead of memories.''

Working with patients in therapy, a psychiatrist must translate this body of knowledge into appropriate therapeutic processes. While therapy is quite a different process than a hearing involving an assault victim's statement of recollection, it does not excuse the misjudgment of those denying someone's traumatic memories.

The second issue raised in such misjudgments is why a victim would remain in some kind of relationship with an abuser or fail to make a complaint within the ``right'' time frame. Once again, one has to understand the dynamics of the victim/abuser relationship. Suffice to say that I have encountered a victim of incest who continued to allow the abuse to take place even after she reached the age of 30 and had gotten married.

Even for someone without experience in dealing with trauma, one should consider the following question: ``If I let someone abuse me and have complete power over me at age 3, how could I suddenly have the strength to rebel and stop the abuse at age 3 plus 1 day?'' It is easy to answer that question. Of course one wouldn't have that strength at 3 years plus 1 day. But if the power hierarchy is maintained for decades, when does one day finally become different from the previous day? If it happened at age 30, could I have the strength to stop it at age 30 plus 1 day?

When we make an informed judgment, we must make sure we understand the various dynamic factors. We should not jump to conclusions when we have insufficient information or, just as important, insufficient empathy. Empathy, the ability to put ourselves in the other person's shoes and try to think and feel as they might, is the real key.

The leadership in society should be willing to be educated in matters related to post-traumatic stress disorder (PTSD) and Complex PTSD. Many women suffered and continue to suffer in the prevailing culture of male entitlement. This cultural view accommodates men taking what they want, treating women as sex objects, and treating women simply as objects over which they can assert power.

This entitlement culture is simply wrong, it opens the door to potential horrors. We only need to open our eyes to cultures that advocate female circumcision, cultures where gang rape is the norm rather than an isolated incident, and cultures where families sell their young girls (and boys) for sex slaves. The President openly admitted to what he called ``locker room talk'', of how he could easily grope women's genitals. We have a long way to go before we can arrive at a better world -- for our daughters, sisters and our mothers. Such a world would be far better for our sons, brothers and fathers as well. Dr.~Ford's bravery is a poignant marker on this journey.

\hypertarget{public-virtue-and-private-abuse-part-1-of-2}{%
\section{Public Virtue and Private Abuse -- Part 1 of 2}\label{public-virtue-and-private-abuse-part-1-of-2}}

\emph{Posted on March 16, 2020}

In recent years we have encountered news reports of famous, well respected individuals being exposed as having a dark side in their lives; a very dark side of sexually abusing women or children. Famous philanthropists, spiritual leaders, musical conductors, and people of great wealth have been credibly accused, and now fortunately some of whom are being convicted, of utilizing their position to exploit people under their influence. The list does not exclude psychiatrists, therapists, or healers -- professional or otherwise.

We presume that those we revere conduct themselves in accord with high moral standards both in public and private. We then feel betrayed and at a loss to understand their (formerly) hidden heinous conduct that has now been exposed.

I am writing this as part of my personal response to the case of Jean Vanier. His life's work inspired many people, including me, in their attitude and service towards the mentally challenged. But Vanier, always characterized during his life as a devout Catholic, had ``manipulative and emotionally abusive'' sexual relationships with six women in France, between 1970 and 2005. \href{https://www.bbc.com/news/world-51596516}{This is according to a statement by L'Arche International}, the organization he founded that did and does so much to benefit the mentally challenged.

While I truly hope the recent charges that surfaced after his death does not harm the work of the L'Arche International, I have no doubt that it will. Here was a man who for all intents and purposes was an extremely good person in public but whose dark side was kept hidden as he violated women in private.

It is important to remember that his conduct did not just harm the women he abused. He knew or should have known that it would be revealed at some point and, as a result, that it would definitely harm the work of L'Arche. He put his own self-interests ahead of his care for the marginalized group that was the foundation for the power/charisma that he then abused.

There is a painful and extreme cognitive dissonance for me. I followed his public career with joy as he helped a very marginalized community. But the characterization of Vanier as a ``devout Catholic'' doesn't compute with the abuse charges confirmed by his own organization. I feel betrayed both by his public persona and by myself in my presumption that his public deeds were in keeping with what I \emph{imagined} were his private morals.

In trying to make sense of this, I thought about when I watched Cowboy movies as a young child. The first thing I did was try to identify who were the good guys and who were the bad guys. Until I did that, I couldn't settle in to watch the movie.

Just like me, most young children are told that people are clearly divided into good and bad. This is simply not true. In the real world, that presumption is not useful in navigating one's way because people are almost never 100\% good nor 100\% bad. There is no clear line of demarcation separating them.

For a child, it is their parent(s) that are responsible for protecting them from the bad people and bad circumstances that one encounters in life until they are old enough to have learned how to navigate this world of moral grays for themselves. For a very young child, they cannot possibly navigate the world unaided. For a child being abused by someone who others presume are indeed protecting that child, the level of betrayal is incredibly more horrific -- as detailed in the work of Dr.~Jennifer Freyd.

For such abused children, it is no surprise that abusers often appear to other adults as ordinary decent individuals while behind closed doors they are the exact opposite. A young child keys off of the attitude of other adults toward their abused, and so are often unable to understand what is safe, what is normal, and what is simply evil. But children try to bond with their primary caregiver, no matter the conduct of that caregiver, because that bonding to the primary caregiver is a biological imperative.

This episode of Jean Vanier is a painful opportunity for me; a very small echo of the betrayal experienced by an abused child. It remains only a very small echo because Vanier was not responsible for me: I never met him. He did nothing to me personally but it is an echo of sorts because I do feel deeply betrayed.
At the same time, it confirms the advice I gave my patients when I was practicing psychiatry 1) to never ignore the messages from internal system protectors; 2) to be very careful when engaging with anyone those protectors caution about; and 3) to completely avoid anyone those protectors are going full red alert about. \href{https://www.engagingmultiples.com/meaning-forgiveness-part-1/}{In particular, I recommend re-reading my posts on the issue of forgiveness}.

Virtually any human being is capable of great deeds of kindness as well as of evil deeds. The decision to do something virtuous or to do something immoral often happens in a split-second. This is obvious when considering the profound power of peer pressure when a situation presents itself, whether in the context of bullying at school, assisting the murder of countless people in concentration camps, or torturing an animal. It is not easy to resist the group momentum that seeks to carry you along with it. It is also obvious when considering the profound power and opportunity one may have in private over people in thrall to you.

The exertion of one's own internal moral authority to overcome such group or internal pressure is difficult, whether one is Gandhi, Martin Luther King, or anyone else willing to lay down his life in the service of protecting the most vulnerable in society. It is extraordinarily difficult for those of us who have not the spiritual strength and discipline to contest such overwhelming pressure and opportunity. Vanier clearly failed to stand up to the seduction of his power in private.

It is human nature that we can do either moral and immoral acts at any moment in our lives. It is not incompatible that one person can do both in their lives. Because our minds are not always stable, each opportunity brings that same choice to us. The important point is that we can also make the decision, again and again, to commit virtuous acts.

We humans are social creatures. That enables us to accommodate unacceptable behavior which to others or even to ourselves later on, is called rationalization. When we are tempted to commit a transgression of either society's or our own morals, we all can readily make up a reason to allow us to do so. A famous American public figure once said that he did ``it'', an immoral act, simply because it could be done.

I am not making excuses for anyone to commit immoral acts. Rather, I want to discourage cynicism and encourage hope. But it is important to acknowledge that our so-called power of reasoning is often weak and easily influenced by our own as well as other's strong emotions.

Many of us, in a moment of impulsivity, step on the gas just to experience the sensation of going way over the speed limit. We think that we are unlikely to be caught, that no one will ever know, that it is not really going to hurt anyone, and so on.

With sexual transgressions, the rationalization is usually that ``this is a special relationship, one that is high and above the usual mundane worldly liaison.'' People can delude themselves that this relationship is special and sublime, that no one is hurt, that it is intensely satisfactory to both parties concerned and that the other person will be ok with it because it is intensely satisfying to me. The bigger picture, including the risks and potential/likely terrible consequences is ignored.

Once this borderline of deluded rationalization is crossed, the second incident begins the habit of thinking that this can be done without a problem. Even the questioning of why or why not becomes weaker. Sexual offenders almost never do it just once. Having experienced violating someone without penalty, with the second and third time the conduct becomes a habit -- the beginning of.an entrenched pattern.

In most cases involving sexual transgression, the act is almost always predicated on a power differential between the parties. The perpetrator is usually of a higher social status and in a position of power. In the intimacy of therapy or spiritual counseling, it is easy to fall victim to the higher social status of the therapist, counselor or religious figure because they are bestowed in those dynamics with the seemingly magical power of a superior being, just as it is in child sexual abuse.

The perpetrator is always aware of this power differential. It feeds into his ego. In any moment, one can lose sight of and fall under the sway of pride, of greed. One gravitates toward the satisfaction of being admired and the possibility of sexual gratification. One can delude oneself with the rationalization that this time it is the rare experience of true love, a genuine meeting of souls. One ignores the fact that the rationalization is a delusion.

It is hard to be a saint in any public sphere. It is hard to resist the temptation of seeking confirmation of one's power and/or gratification of one's sexual desire in a private sphere. It is difficult to resist the temptation of a great sensual (sexual) experience in a life full of stresses and loneliness.

Most people are not able to let go of the cravings of the ego for sensual experiences. Take for instance, a friend of mine who has such strong craving for good food, that he would go to one restaurant for its soup, to another for its orange duck main course, and to a third for its dessert of Tiramisu. No kidding!

Giving in to this kind of craving, played out by choosing this or that item in different particular restaurants is completely different from giving in to a craving that leads to abuse. Moving between restaurants does not involve the traumatization of anyone, so there is no harm. This is a qualitatively different gratification of desire. But other gratification actions may result in severe traumatization of another individual. That is what defines for me what is acceptable and what is not.

My conclusion is that good people are capable of immoral acts just as bad people are capable of kind acts. We should well remember that according to the great book that serves as the foundation of the 3 principal monotheistic religions of Islam, Judaism and Christianity, God's chosen heroes were all imperfect specimens. Abraham, was despicable in that he offered his wife Sarah to the Pharaoh to save his own skin. King David coveted Bathsheba so much that he would send her husband to be killed in battle so that he could have her. Lot offered his daughters to be raped before they escaped from Sodom

It is unrealistic to separate people into all good and all bad. In short, we should hesitate to give anyone a blanket ``certificate of righteousness.'' Human beings are potentially good and potentially bad. Let us first examine each of our own opportunities, as they arise, to be virtuous and kind then act appropriately.

\hypertarget{coping-with-anxiety-in-the-pandemic}{%
\section{Coping With Anxiety in the Pandemic}\label{coping-with-anxiety-in-the-pandemic}}

\emph{Posted on March 25, 2020}

Since SARS visited Hong Kong in 2003, the world has had warning visits of Ebola, Swine Flu, Avian Flu and MERS-CoV. Now, the pandemic of COVID-19 is upon us. We have been watching it grow rapidly in China and aboard the Diamond Princess cruise ship. It has spread rapidly in Italy, Spain, South Korea, and is now severely threatening the US, Canada, and many other countries. Political leaders around the globe have reacted with varying degrees of alertness. We have witnessed country dependent degrees of control over its spread and growth.

As individuals, how do we handle the fear and anxiety we face with this viral threat? The question of dealing with fear and anxiety is one that is always present for those with DID. But now, we need to understand that just as individual trauma is processed individually, it can also happen in a society or, as now, sometimes on a global scale.

In general, fear has a protective function. Within the context of DID, fear about survival triggers the dissociative experience in a young child. Epidemiologists are characterizing this virus as a survival issue for many people, economists are characterizing it as a survival issue for commerce.

It is fear that is motivating us as a society to adopt drastic measures such as school and factory closures, the cancellation of public gatherings, cruise ship trips, unnecessary vacation flights, and even restaurant dining. These measures will curb the growth and spread of the viral infection. This is an example of vigilance heightened appropriate to the threat.

What is left is handling panic or the excessive fear. This is not unfamiliar to those accustomed to working with trauma and PTSD. One point that is emphasized in the Engaging Multiple Personality series is the need to slowly transform hair-trigger hyper-vigilance into ordinary protective vigilance. I always point out that we must maintain vigilance, not eliminate it. Why? Because there are dangers in the world and we need to remain alert.

However, the opposite of hyper-vigilance is also on display in this pandemic. There are countries where leader(s) have ignored science and discouraged vigilance (hyper or otherwise) against all apparent facts. We see news reports of some young people in particular that are affirmatively ignoring and denying the social distancing recommendations. They put themselves at risk because they do not believe it is a danger for them that outweighs their need to some immediate gratification -- like Spring Break.

So let's stay appropriately vigilant. In that light, I would like to focus on the following points as a guideline in facing this Corona virus Crisis.

\begin{enumerate}
\def\labelenumi{\arabic{enumi}.}
\item
  Be informed of what you fear. For most people, COVID-19 infection presents from an asymptomatic state to mild symptoms similar to a cold or flu. Some complain of a dry cough, abdominal discomfort, and breathing difficulty. Only a small percentage of those infected may need special care in a hospital setting; for intubation, oxygen in ``equipped beds'' and so forth.
\item
  You have a responsibility of protecting others if you are sick. If you suspect you have COVID-19, don't go to your doctor and potentially infect everyone in the waiting room. Instead, call the hospital for help and let them know you may have the viral infection. Note that several sitting US Senators self-isolated immediately upon suspicion of the virus while one who is actually a doctor was tested because of his high personal risk exposure. He continued to interact with colleagues for days without telling them he was a potential asymptomatic carrier. He was indeed infected. Again, an example of an inappropriate lack of vigilance putting a wide circle of people at risk.
\item
  Frequent hand washing is most important as well as avoiding touching door knobs and door handles with your uncovered hand. Wear disposable gloves or have a tissue/napkin to cover your hand when touching one of those surfaces. Wear face masks if you feel any symptom and avoid touching your own face as a general rule. In person, keep social distancing. Better yet, maintain physical isolation if possible but make sure to connect socially through the phone and/or internet so as to avoid emotional isolation.
\item
  Having put 1-3 into practice, pay attention to the psychological aspect of coping with the stress. Do grounding exercises throughout the day. Preferably start the exercises before you panic so that when anxiety erupts you are already habituated to grounding yourself. It is a good practice to do them when you get up in the morning, perhaps once mid-day, and before going to sleep at night.
\end{enumerate}

The following is my very personal view of the pandemic, so please take it with as many grains of salt as you think appropriate . . .

In the 14th Century, the plague killed one third of the population of Europe. Despite subsequent medical advances, the plague has recurred many times since then. The last time that I am aware of was in October 2017 when it hit Madagascar and killed 170 people. Ebola had been hitting certain African states regularly. In modern times, we have had repeated corona virus infections of different varieties that we have hardly any means of control.

These epidemics regularly hit different parts of the world. We generally don't hear about them because we (and our local media) tend to ignore them when they do not impact us directly. We usually have other bad news to worry about, such as fighting in Afghanistan, Iraq, Syria, Yemen and elsewhere.

I do not blame God or any extra-natural sources. It doesn't seem helpful to me. As humans, we must learn to survive in an increasingly complicated world. This has been the story of humanity throughout our history. A key to what is helpful is learning to appreciate that we are all interdependent. We have a long way to go to reach a world of peace and accord. Perhaps we can use this as an opportunity to move further toward that goal.

Life comes with needing to accept the fact there are periods of adversity and to work with them. For some experiences of adversity, we can learn to face them with insight, compassion for ourselves and others, and eventually with some equanimity. Indeed, it is time for us to rethink our place in the world and the universe. This is an opportunity for each of us, certainly for me, to question why are we here, what is our purpose today.

It goes back to the question I faced when beginning my retirement, what is the reason for me to get up in the morning? The answer began to dawn when going through my patient files as part of closing up my psychiatric practice. It seemed that my DID patients had gone through such profound trauma and taught me so much. It was difficult to be retired, unable to help any more. I began to try to organize my thoughts, which resulted in the Engaging Multiple Personality Series and blog posts.

I believe these writings have been helpful to many people, at least as indicated by emails and Facebook comments I have received from those with DID and their spouses/partners as well as sometimes psychiatrists and other therapists. In this way, the DID community has helped me process the vicarious trauma I experienced listening and trying to assist my patients. It continues to heal my heart.

I offer a quotation from Rabindranath Tagore which I have found inspiring:

\begin{quote}
``I slept and dreamt that life was joy. I awoke and saw that life was service. I acted and behold, service was joy.''
\end{quote}

In my retirement, and in this COVID-19 crisis, I am seeking to make contact with some people I know who might benefit from a warm phone call, one that might penetrate their loneliness and isolation. I read books that I have always wanted to read but never before had time. I listen to music that nourishes me. I try to keep my body active and supple, to prevent waking up with pain all over my stiff body. I also do my one-breath-meditation and walking meditation. This is the least I can do to be kind to myself. I also write my blog articles when the circumstances and spirit moves me.

With best wishes for your healing and strength.

\hypertarget{afterword}{%
\chapter*{Afterword}\label{afterword}}
\addcontentsline{toc}{chapter}{Afterword}

\bibliography{book.bib,packages.bib}


\end{document}
